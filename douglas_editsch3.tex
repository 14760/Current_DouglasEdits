OK chapter 3…

Possibly it was just lost in the text but the jump from ‘essential’ proteins in networks to why intelligence would be expected to correlate with centrality was not obvious. Perhaps refresh the core idea around 3.6.
If this was the whole human proteome you would predict centrality to correlate loosely with essential function but since this is a synaptic subset you are looking for synaptic phenotypes of which intelligence is one to correlate. Given that you have 25\% of the human proteome you would likely see both.

P95 Title is not very good. Suggest it needs a bit more maybe something like: Do centrality measures in the post synaptic proteome correlate with function?

DONE as the title is long I have retained centrality measures for the bit that goes across the top (\chaptermark{})


P95 Delete sentence one or move it down as until you have some definition of what centrality actually is then it does not mean anything.

DELETED
P95 “There is evidence that….mutations in hub” needs to be tighter, perhaps” There is evidence tat genes encoding proteins that are central in the proteome network are more likely to be essential. The most established finding being that null mutations in hub"

DONE


P96 typo over-repreented
Must have fixed not there now

P96 “However. the definition of variations in disease and disease genes varies” - clunky. Maybe this is a good place to raise the issue that context is important. At the level of a whole proteome “essential” is likely to mean viability in general. However something like a PSD “essential” is more likely to mean essential to elements of synaptic activity/function which may still be ultimately viable?

Now the tricky bit for Ch3.: It is too long and the key messages are somewhat obscured beyond the one that you have been pretty comprehensive about thi. There are three options really.
1. Disagree with me and leave it as it is (which is fine, it is just my opinion)
2. Remove most of the studies leaving only the most interesting ones and move the deleted ones to a supplementary chapter. Leave in CH3 a paragraph that says you are limiting discussion to the key findings, a more extensive review of the other main (list them) centrality measures is provided in supplementary chapter X.
3. include a statement that you are deliberately being comprehensive in your assessment and presenting all results but if the reader wants to focus on the core findings they should skip to sections 3.X and 3.Y and the summary in 3.14
Note: You could almost turn this into a tutorial paper on how to do this for any protein network and that might be a worthwhile activity at some point.

P97 3.1.1 node(s) typo

DONE
P98 Figure 3.1 General comment: Review each of the figure legends. Do they provide enough text to understand the basics of the figure. This one and one or two others are quite terse.
P99 3.2 see section ??  (a few examples of this through the text)
Does the dolphin story add much?
P106 3.2.5 significant typo
P114 3.4 subtitle is a bit messy
P114 3.4 and other discussions - technically they were all mutations but from memory a lot of these early studies were full gene/protein deletions so a bit different and maybe worth highlighting this difference.
P124 (USML)?

DONE  UMSL
P130 reference needs fixed dorogovtsev2006k


DONE
P133 tables SUPPLE ??
P134 table ?? X2
P138 Newman(section 3.6.4) needs a space
P148 conclusion - hmm its not much of a conclusion…
P149 (methods section 3.6.6) just use (see 3.6.6)
P150 table ??
P152 reported in section ??
P153 BH 0.026)3.28. (looks like a typo)
P157 3.10 general comment in this section. Make sure it is clear you are comparing the top/bottom 10% against all PSP as a background. It is in there but was easy to miss and
P157 “The ontology terms are similar across all centrality measures with some differences” I’m fairly sure this sentence has no meaning ;)
P169 section ??
P170 intolerant.The (missing space)
P171 including Parkinson’s disease see section 3.2.4. NOTE: 3.2.4 does not mention PD...
