
J Douglas ARMSTRONG
25 May 2021, 16:58 (4 days ago)
to me


I like it

P229 perhaps restate your hypothesis?
P229 In doing so, I will discuss..
P230 also identified a similar? group of proteins
P234 That communities may be topologically good and yet not represent a ground truth is an important point.
P234 A further limitation of the study is that gene based..


> On 25 May 2021, at 16:44, J Douglas ARMSTRONG <Douglas.Armstrong@ed.ac.uk> wrote:
>
> Chapter 4 (and 5)
> Structural suggestion: Split chapter 4 at page 200 (4.5) and merge the second half into chapter 5 so you have one on selection of clustering algorithms and one on GSA using clusters?
>
> P173 - The intro section to me feels backwards. Might be worth looking at again but maybe say:
> 1. GSA is limited by choice of gene sets
> 2. Network analysis using clustering provides data driven lists that could be used as geneses
> 3. Custer specific enrichment using GSA would be a good test of my main hypothesis.
> 4. Lots of ways to do clustering and so needs a careful evaluation and selection independent of the GSA hence this chapter…
>
> P174 - the text is hard to read, sentences are complex with various negatives in them. Might want to look again and re-write to simplify what you are trying to say.
> P175 - Do you need to mention Girvan-Newman here or should you just mention that some clustering algorithms are self optimising often using modularity as an objective function. Then go on to discuss modularity as you do?
> P176 Does the text from the top down to 4.2 say anything really? Could it be deleted?
> P179 - “reasonable size” and reasonable time are very vague. I guess you do discuss these later but maybe just say (discussed later section 4.x) for each maybe?
> P181 The Louvain method (messed up reference I think)
> P183 table??
> P190 Table ??
> P194 These results are in accord
> P200 New Chapter?
> P202 “In order to determine….section 2.3.6” this paragraph seems to repeat much of the one above
> P205 Fig 4.18 This needs a more descriptive legend.
> P224 Summary. The last two sentences are just kind of floating there, maybe integrate into the paragraph?
>
>
>> On 25 May 2021, at 16:23, Grant Robertson <gxrobertson@gmail.com> wrote:
>>
>> This email was sent to you by someone outside the University.
>> You should only click on links or attachments if you are certain that the email is genuine and the content is safe.
>>
>> That sounds good thanks I’ll do that. Just finishing the edits to one then I’m moving them to an overleaf with the university style
>>
>> Cheers
>>
>> Grant
>>
>>> On 25 May 2021, at 16:18, J Douglas ARMSTRONG <Douglas.Armstrong@ed.ac.uk> wrote:
>>>
>>> 
>>> One more structural suggestions.
>>>
>>> Split Chapter 4 at 4.5, merge the second half with chapter 5 so you have the assessment and selection of clustering algorithms in one coherent chapter then one chapter on using communities for GSA?
>>>
>>> Will write up notes on chapter 4 asap but just thought I’d run that past you.
>>>
>>> D
>>>
>>>
>>>> On 25 May 2021, at 15:57, Grant Robertson <gxrobertson@gmail.com> wrote:
>>>>
>>>> This email was sent to you by someone outside the University.
>>>> You should only click on links or attachments if you are certain that the email is genuine and the content is safe.
>>>> Hi Douglas,
>>>>
>>>> Sorry PS.
>>>> I have sent you a link to the overleaf I am editing with the updates. There is another overleaf folder with the university latex style class which I will move chapters to after editing (to make sure I don't include anything by accident)
>>>>
>>>> Cheers
>>>>
>>>> Grant
>>>>
>>>> On Tue, 25 May 2021 at 14:59, Grant Robertson <gxrobertson@gmail.com> wrote:
>>>> Hi Douglas,
>>>>
>>>> Thanks for the feedback. I agree 3 is too long.
>>>>
>>>> Suggest:
>>>> Keep Something like histogram of centrality measures and that they are correlated. Keep fig 3.8 and table 3.9
>>>>
>>>> Keep fig 3.14 has power law distribution
>>>>
>>>> Say shows scale free (cross ref to appendix) and power law distribution with exponent 2.51 (cross ref to supplementary)
>>>>
>>>> Remove 133-149 to supplemental
>>>>
>>>> Keep or remove assortativity gene score, murine measures  (no assortativity z score, mild assortativity murine genes)- not sure what do you think
>>>> p153- 156 remove to supplemental
>>>>
>>>> Keep 157-66 Disease enrichment and centrality then keep
>>>> pli and intelligence centrality and intelligence to summary
>>>>
>>>> Thanks
>>>>
>>>> Grant
>>>>
>>>> On Tue, 25 May 2021 at 14:13, J Douglas ARMSTRONG <Douglas.Armstrong@ed.ac.uk> wrote:
>>>> OK chapter 3…
>>>>
>>>> Possibly it was just lost in the text but the jump from ‘essential’ proteins in networks to why intelligence would be expected to correlate with centrality was not obvious. Perhaps refresh the core idea around 3.6.
>>>> If this was the whole human proteome you would predict centrality to correlate loosely with essential function but since this is a synaptic subset you are looking for synaptic phenotypes of which intelligence is one to correlate. Given that you have 25% of the human proteome you would likely see both.
>>>>
>>>> P95 Title is not very good. Suggest it needs a bit more maybe something like: Do centrality measures in the post synaptic proteome correlate with function?
>>>> P95 Delete sentence one or move it down as until you have some definition of what centrality actually is then it does not mean anything.
>>>> P95 “There is evidence that….mutations in hub” needs to be tighter, perhaps” There is evidence tat genes encoding proteins that are central in the proteome network are more likely to be essential. The most established finding being that null mutations in hub"
>>>> P96 typo over-repreented
>>>> P96 “However. the definition of variations in disease and disease genes varies” - clunky. Maybe this is a good place to raise the issue that context is important. At the level of a whole proteome “essential” is likely to mean viability in general. However something like a PSD “essential” is more likely to mean essential to elements of synaptic activity/function which may still be ultimately viable?
>>>>
>>>> Now the tricky bit for Ch3.: It is too long and the key messages are somewhat obscured beyond the one that you have been pretty comprehensive about thi. There are three options really.
>>>> 1. Disagree with me and leave it as it is (which is fine, it is just my opinion)
>>>> 2. Remove most of the studies leaving only the most interesting ones and move the deleted ones to a supplementary chapter. Leave in CH3 a paragraph that says you are limiting discussion to the key findings, a more extensive review of the other main (list them) centrality measures is provided in supplementary chapter X.
>>>> 3. include a statement that you are deliberately being comprehensive in your assessment and presenting all results but if the reader wants to focus on the core findings they should skip to sections 3.X and 3.Y and the summary in 3.14
>>>> Note: You could almost turn this into a tutorial paper on how to do this for any protein network and that might be a worthwhile activity at some point.
>>>>
>>>> P97 3.1.1 node(s) typo
>>>> P98 Figure 3.1 General comment: Review each of the figure legends. Do they provide enough text to understand the basics of the figure. This one and one or two others are quite terse.
>>>> P99 3.2 see section ??  (a few examples of this through the text)
>>>> Does the dolphin story add much?
>>>> P106 3.2.5 significant typo
>>>> P114 3.4 subtitle is a bit messy
>>>> P114 3.4 and other discussions - technically they were all mutations but from memory a lot of these early studies were full gene/protein deletions so a bit different and maybe worth highlighting this difference.
>>>> P124 (USML)?
>>>> P130 reference needs fixed dorogovtsev2006k
>>>> P133 tables SUPPLE ??
>>>> P134 table ?? X2
>>>> P138 Newman(section 3.6.4) needs a space
>>>> P148 conclusion - hmm its not much of a conclusion…
>>>> P149 (methods section 3.6.6) just use (see 3.6.6)
>>>> P150 table ??
>>>> P152 reported in section ??
>>>> P153 BH 0.026)3.28. (looks like a typo)
>>>> P157 3.10 general comment in this section. Make sure it is clear you are comparing the top/bottom 10% against all PSP as a background. It is in there but was easy to miss and
>>>> P157 “The ontology terms are similar across all centrality measures with some differences” I’m fairly sure this sentence has no meaning ;)
>>>> P169 section ??
>>>> P170 intolerant.The (missing space)
>>>> P171 including Parkinson’s disease see section 3.2.4. NOTE: 3.2.4 does not mention PD...
>>>>
>>>>
>>>>>> On 25 May 2021, at 13:13, Grant Robertson <gxrobertson@gmail.com> wrote:
>>>>>
>>>>> This email was sent to you by someone outside the University.
>>>>> You should only click on links or attachments if you are certain that the email is genuine and the content is safe.
>>>>>
>>>>> Great thanks.
>>>>>
>>>>>> On 25 May 2021, at 13:07, J Douglas ARMSTRONG <Douglas.Armstrong@ed.ac.uk> wrote:
>>>>>>
>>>>>> 
>>>>>> Hang off for now, I don’t think these will take you long to review and either ignore or deal with. I’m working through ch3 right now which is a bit of a monster but so far good.
>>>>>>
>>>>>> D
>>>>>>
>>>>>>> On 25 May 2021, at 13:05, Grant Robertson <gxrobertson@gmail.com> wrote:
>>>>>>>
>>>>>>> This email was sent to you by someone outside the University.
>>>>>>> You should only click on links or attachments if you are certain that the email is genuine and the content is safe.
>>>>>>> Hi Douglas,
>>>>>>>
>>>>>>> Thanks for the feedback, really appreciate it. I am just getting on with it as I have time today.
>>>>>>>
>>>>>>> Should I let Patrick know about progress (just in case there are any probs?)
>>>>>>>
>>>>>>> Cheers
>>>>>>>
>>>>>>> Grant
>>>>>>>
>>>>>>>
>>>>>>> On Tue, 25 May 2021 at 10:45, J Douglas ARMSTRONG <Douglas.Armstrong@ed.ac.uk> wrote:
>>>>>>> OK most of this is my personal thoughts on how I would do it rather than any significant problem.
>>>>>>>
>>>>>>> P1, C1.1, delete sentence This thesis…animals
>>>>>>> P1, C1.1 delete para This thesis..function
>>>>>>> P2 C1.1 delete “that will be discussed…chapters”
>>>>>>> P2 C1.1.1 (see figure 1.2) - just (figure 1.2) , just to be consistent with other fig references.
>>>>>>> P3 C1.1.1 para The term…[149] - suggest move this up a bit, its a historical introduction, maybe make it second paragraph in this section?
>>>>>>> P3 C1.1.2 para 1. Quite recently ->until recently
>>>>>>> P4 “high level functional the synapse” - again these two paragraphs “the synapse…between neurons” feel like good introduction and maybe needs to come before some of the text above. Again maybe second paragraph in this section?
>>>>>>> P4 Figure 1.2 “Picture credit” suggest chage this to either “Taken from:” or “Modified from:”
>>>>>>> P4 C1.2 “gap” - just call it a cleft since thats what you use thereafter?
>>>>>>> P5 1.2.1. Have a read through this again. I’m not sure the order of the sentences really works and might be better in a slightly different order to make more sense.
>>>>>>> P6 “They are relatively” - what is actually resistant? the protein or the network? Reads like the protein in current context….
>>>>>>> P6 1.2.3 Calcium ion…depression” delete
>>>>>>> P7 4 sub-units - four sub-units
>>>>>>> P7 notes to remove
>>>>>>> P10 1.2.6 - The first sentence does not make sense and actually could just be deleted. then The PSP has ancient origins: The proteins evolved around the time of the earliest…
>>>>>>> P11 1.2.9- I think this is written backwards. Don’t you need to introduce what a PPI is first before going into database quality? Also is it really the database quality or the interaction data quality?. The first paragraph os 1.2.10 would be a good introduction.
>>>>>>> P13 "To explain why…disorders” delete
>>>>>>> P14 question. I always put quotes in both italics and quotation marks so it is ultra clear.
>>>>>>> P16 1.3.4& general from now on. More X-refs. You mentions SNPs here for the first time but you have a good description later  in 1.4.1, stick in a cross reference here.
>>>>>>> P17 1.3.4 MAGMA (as above, cross ref to your section on MAGMA
>>>>>>> P19 1.3.6 “I will…educational attainment” - suggest this is edited down and moved to the end of chapter 1 as part of the hypothesis and how you are going to test it?
>>>>>>> P20-P28 - this section feels a bit methods but probably best where it is since chapter 2 has a different set of aims (aims in terms of your thesis)
>>>>>>> P28 figure 1.3 - is this figure useful or interesting?
>>>>>>> P30 Plan
>>>>>>> I would expand this a little. Your reader has just spent pages on GWAS, GSA, MAGMA etc. Bring them back to the idea that what you propose is that we can use synapse networks built from proteomics and PPIs and use clustingg to derive new communities which might be useful gene sets. Then just clearly state your hypothesis. Then briefly explain how you plan to test your hypothesis and maybe even explain what a positive and or negative result would look like.. Once you have done that you don’t need to introduce the following chapters, they just make sense.
>>>>>>>
>>>>>>> P31 general - you have “results” subheadings in multiple places - are these really results or are they more accurately  "validation of datasets and/or methods used"
>>>>>>> P31 - intro chapter doesn’t really add value as it is. Need to explain that you are going to explain where the datasets come from and included in that, in particular for the GWAs, demonstrate that the methodology you want to apply can replicate the previous findings - validating your methods and the data.. Essentially this is all due diligence since it is largely stuff from other people.
>>>>>>> P31 2.1 Chapter title(and others) are a bit terse. Network graph of what?
>>>>>>> P34 figure??
>>>>>>> P35 and elsewhere in this chapter. “The fine;l filtered list contained 325….” - where does the reader get a copy of this list?
>>>>>>> P36 “the original gal file” - again is this provided somewhere as supplementary material or on GitHub?
>>>>>>> On the whole chapter 2 is good, is it worth a mini conclusion at the end of the summary that on the basis of the checks performed you have the confidence to proceed in the next chapters which focus on your development and testing of the hypotheses from chapter 1?
>>>>>>>