\chapter{Do centrality measures in the post synaptic proteome correlate with function?}
\label{chap:Centrality measures}
\chaptermark{Centrality measures}

 \begin{displayquote}
 I asked a person of intelligence how many steps he thought it would take, and he said that it would require 100 intermediate persons, or more, to move from Nebraska to Sharon. \cite{milgram1967small}
 
 Stanley Milgram 1969
\end{displayquote}


 

Networks can be studied at three levels: the properties of the network as a whole, such as the small-world property (section~\ref{sec:Small world}), the properties of individual nodes, including their importance to the network, and the properties of mesoscale structures within the network such as communities (chapter~\ref{chap:community detection}).
  
  The principle node properties are measures of \textit{centrality}. The concept of centrality arose in the social sciences literature in the late 1940s, and its early evolution is reviewed in Freeman \cite{freeman1978centrality}. Centrality measures use network topology to determine which nodes are important \cite{newman2018networks} and where they are (central or peripheral) in the network \cite{freeman1978centrality}. 

Real-world networks often have a small number of nodes that are critical to network function, for example, a hub airport in an air transportation network\cite{borenstein1989hubs}. Centrality statistics measure different ways in which nodes can be important in a network allowing key nodes to be identified, ranked, and any association between centrality and network function assessed. 
 
There is evidence tat genes encoding proteins that are central in the proteome network are more likely to be essential. The most established finding being that null mutations in hub (high degree centrality) genes in the yeast proteome are more likely to be lethal (the \textit{centrality lethality} effect)\cite{jeong2001lethality}. Central genes have also been reported to be more commonly associated with diseases, most notably cancer\cite{vogelstein2000surfing}, \cite{albert2005scale},\cite{xu2006discovering} although the relationship is complex and recent results question even the previously accepted degree centrality effects in yeast (section~\ref{sec:biomedical significance of centrality measures}).

Before addressing the association between synaptic proteome centrality and cognition, it is necessary to introduce the centrality measures, briefly review their previous use in the biomedical literature and their effects on the function of the network. 

As the synapse is important in cognition, critical nodes in the synaptic proteome network may be over-represented in animal models of cognition or human differences in intelligence. There is evidence that central nodes affect numerous genes, and, in a protein network, impact \textit{pleiotropic} traits and diseases \cite{chavali2010network}. The genetic architecture of intelligence is subject to pleiotropic effects\cite{plomin2015genetics},\cite{visscher2016plethora}.  Conversely there may be no effect; genetic variation in critical nodes may be so deleterious as to be lethal, or cause levels of impairment that would absent sufferers from large scale genetic studies of cognitive ability. Central genes have been reported to be essential rather than associated with disease\cite{barabasi2011network}. However, the definition of variations in disease and disease genes varies (see section \ref{sec:assessment of communities}) in a way in which has not always been recognised in the literature (for example, see \cite{goh2007human} ).

 Therefore, what is needed is an empirical approach to study the effects of network topology on complex traits and diseases. This should include an appreciation that effects may be specific to only some traits or diseases. 
 
 
 
 

 

 


\section{Properties of `real world' networks}
\label{sec:Properties of empirical networks}
Real world  networks often have properties that would not be expected to arise by chance. These include: a power law degree distribution \cite{barabasi1999emergence},\cite{barabasi1999mean} (section~\ref{sec:scale_free}) and `small world' structure \cite{watts1998collective}. Network level properties in turn arise from the collective properties of individual nodes; for example a degree sequence (node level) gives rise to a power law distribution (network level) or individual local transitivity to global clustering coefficient\cite{watts1998collective}. In addition to centrality measures (section~\ref{sec:centrality measures}), nodes may have properties including  non-topological node data\cite{wang2016approach} or metadata\cite{peel2017ground} of which the gene level significance scores from GWAS studies are an example. Chapter~\ref{chap:community detection} will address meso-scale structures and community detection. 

Network-level statistics reflect how networks form and evolve, their patterns of information flow and how resistant they are to disruption. Not all naturally occurring networks share the same network-level properties; for example, road networks do not show a scale-free property\cite{xie2007measuring}\footnote{their degree distribution is similar to that found in a random graph}, although air travel networks do\cite{boccaletti2006complex}\footnote{to be more accurate we might say do not belong to a heavily right-skewed distribution}. Comparing network centrality statistics to a random model  (section \ref{sec:degree distribution}), helps show if they are of significance or interest.








\subsection{Small world property}
\label{sec:Small world}
% \textcolor{red}{this is not covered in introduction}
The `small world' property means that a short number of hops from any node along the network edges can reach any other node.  The maximum distance, the \textit{diameter D}, between any two nodes in the network is therefore small\cite{zhu2007getting}. Random networks are also small, but their nodes do not cluster together, unlike most real-world networks. 

 Milgram\cite{milgram1967small}\footnote{although it was previously described 
by Pool and Kochen, although published later \cite{de1978contacts}see Newman \cite{newman2018networks} p 62 } famously demonstrated this phenomenon in a 1967 experiment; he tried to find the number of steps (the average path length) it would take for a letter to reach someone unknown to the first letter holder if they and everyone else in the chain could only mail it to people they knew personally. The average number of links was six, which was less than expected\footnote{Milgram reports it as the median number of intermediaries as five, which is six edges between sender and recipient using the definition of path length}. 

Watts and Strogatz \cite{watts1998collective} reported small-world properties in naturally occurring networks: the neural network of the nematode \textit{Canenorhabditis elegans}; the power network of the western United States, and the collaboration graph of film actors generated from the Internet Movie Database (IMDb). These properties are a short average path length, with a high clustering coefficient in contrast to a random graph which has a short path length but low clustering coefficient. The average path length of small-world networks and random networks are proportional to the logarithm of the number of nodes in the network. Scale-free networks scale as loglogN\cite{cohen2003scale}.\footnote{scales as loglogN in scale-free with core-periphery \cite{boccaletti2006complex}}. 

Watts and Strogatz\cite{watts1998collective} proposed a generative model for this phenomenon in which  a circular lattice network has edges rewired randomly with probability $p$ in which interconnections are rewired to span the circle\cite{newman2018networks}\footnote{p312}. A small amount of rewiring allows a short path length and preserves the high clustering coefficient. Figure~\ref{fig:small_world} shows the generation of a small world graph from a circular lattice with increasing rewiring probability.

Only a small number of edges need to be rewired to produce the small-world property, and it has been argued that this is why it is so common in nature \cite{newman2018networks}. Small world networks in the Watts-Strogatz model (WS) are an interpolation between random networks and lattices, and some have suggested that it is the usefulness of the combination of clustering and connectivity\cite{latora2001efficient} that explains why they are ubiquitous.

The small-world property allows rapid information transfer within networks. It has recently been posited as a model for widespread genetic effects on traits (the omnigenic model)\cite{boyle2017expanded}\footnote{however see also \cite{wray2018common} for a non-network criticism and see discussion}, and as a developmental stage in the evolution of brain networks\cite{vertes2015annual}.




\begin{figure}
    \centering
    \includegraphics[width=\textwidth]{images/Rplot_corrected_small_world.png}
    \caption[Small world lattice]{Small world lattice $n$=20. Probability of rewiring (left to right)= 0, 0.1, 0.3.}
    
    \tiny\url{https://github.com/14760/thesis/blob/main/R/chapter2/smallworld/plot_small_world.R}
    %\textcolor{red}{ADD APL and transitivity below each plot}}
    \label{fig:small_world}
\end{figure}

%\todo{small world network either do here or put in introduction this goes with results }

 

\subsection{The configuration model}
\label{sec:configuration model}
To assess the significance of a network or a vertex statistic, we can measure how much they differ from their expected value given a random network. Simple random networks such as Erd\H{o}s-R\'enyi (ER) random graphs have already been introduced (see section~\ref{sec: PSP graph connected component and missing}).

Erd\H{o}s-R\'enyi models generate networks with a vertex degree distribution that is approximated by a Poisson distribution (see section~\ref{sec:degree distribution}), and hence the mode of the probability mass is centred around the mean degree $<k>$ of the network.
%\todo{move this to after degree distribution}.
Many real-world networks have more complicated degree structures, often a heavy-tailed distribution such as a power-law distribution (section~\ref{sec:scale_free}).

The configuration model generates a new random network that retains the degree sequence of the original network. Network properties can be compared between the existing network and a configuration model with an identical degree sequence.

 The process for generating the configuration model is as follows: break each edge into two stumps and then rewire each of the stumps at random to another stump with uniform probability.  Self-loops and multi edges can result, which do not occur in most real-world networks, but the number of these are low as the network size increases\cite{newman2018networks}.

There are also specific models to generate random power-law distribution networks such as the Barabasi-Albert model\cite{barabasi1999emergence} and earlier Price model\cite{price1965networks}, but if one is studying a \textit{particular} network, and one has the degree sequence, as is the case here, the configuration model is a good choice\cite{newman2018networks}.


\begin{figure}
  \begin{subfigure}{7cm}
    \centering\includegraphics[width=7cm]{images/chapter3/small_world_histogram/Rplot_average_path_length.png}
    \caption{Average path length}
    \end{subfigure}
  \begin{subfigure}{7cm}
    \centering\includegraphics[width=7cm]{images/chapter3/small_world_histogram/Rplot_transitivityWS_ER_graph3457.png}
    \caption{Transitivity Watt Strogatz.}
  \end{subfigure}

 
 \begin{subfigure}{7cm}
    \centering\includegraphics[width=7cm]{images/chapter3/small_world_histogram/Rplot_average_pathlmean.png}
    \caption{Average path length showing mean}
    \end{subfigure}
  \begin{subfigure}{7cm}
    \centering\includegraphics[width=7cm]{images/chapter3/small_world_histogram/Rplot_transitivity_mean.png}
    \caption{Transitivity Watt Strogatz.}
  \end{subfigure}
\caption{Figure shows average path length and transitivity (Watts Strogatz mean of local transitivity) for an Erd\H{o}s-R\'enyi graph of 3,457 nodes and 30,498 edges (i.e. the same as the PSP graph). The difference in values is small but the sampling distributions do not overlap. Upper panels show the distribution of these values over 1000 iterations \url{source('~/RProjects/chapter3/R/small_world/eda_small_world.R')}}
 
  \end{figure}

\section{Centrality measures}
\label{sec:centrality measures}
Centrality measures identify important, central nodes  using  network topology\cite{newman2018networks}\footnote{(p159)}. Node centrality also determines how information flows through a network \cite{borgatti2005centrality}. Several measures of importance have been described and can be used to rank, group or identify key nodes. All measures are derived from the adjacency matrix. 

% \begin{figure}
%   \begin{subfigure}{7cm}
%     \centering\includegraphics[width=7cm]{images/chapter3/dolphin/degree.png}
%     \caption{Degree}
%     \end{subfigure}
%   \begin{subfigure}{7cm}
%     \centering\includegraphics[width=7cm]{images/chapter3/dolphin/betweenness.png}
%     \caption{Betweenness}
%   \end{subfigure}
%   \begin{subfigure}{7cm}
%     \centering\includegraphics[width=7cm]{images/chapter3/dolphin/closeness.png}
%     \caption{Closeness}
%   \end{subfigure}
%   \begin{subfigure}{7cm}
%     \centering\includegraphics[width=7cm]{images/chapter3/dolphin/eig.png}
%     \caption{Eigenvector}
%   \end{subfigure}
%   \caption[\textbf{Illustration of centrality measures} in a dolphin social network.]{\small Illustration of centrality measures. Betweenness centrality, eigenvector centrality and local transitivity for a social network of bottle nose dolphin social relations data. Node size is proportional to centrality measure and node colour varies from pale yellow (low) to red (high centrality). Dolphin names are shown in the text and the font size is also proportional to centrality measure. Note that the betweenness shows two prominent nodes SN100 and Beescratch through whom many shortest paths flow although they have modest degree centrality and eigenvector centrality. Hook, is a neighbour of a number of high degree nodes but has a modest degree centrality shown in a). The importance of Hook's connections is shown in the high eigenvector centrality in panel (d). Data from Prof Newman's website\footnote{\url{http://www-personal.umich.edu/~mejn/netdata/dolphins.zip}} based on Lusseau et al. (2003) \cite{lusseau2003bottlenose}.Visualisation in Gephi with Force Atlas layout. }
%   \label{fig:dolphin_multipanel}
% \end{figure}




\begin{figure}
    \centering
    \includegraphics[width=0.9\textwidth]{images/chapter3/centality_plos/journalpone0220061g001.PNG}
    \caption{Centrality measures from Oldham et al. (2019)\cite{oldham2019consistency}. In panel A the red node has both high degree and closeness. In contrast in panel B the red node has high centrality but low degree. This image illustrates that although these measures can be correlated they can also represent different properties.}
    \label{fig:centrality measures from Plos}
\end{figure}

Degree centrality is the simplest and is the number of edges that connect to a node. Other vertex measures include eigenvector centrality, a measure of the importance of the node that includes the importance of its neighbours\cite{bonacich1987power}.  Some nodes are bottlenecks in the network. If a large number of the shortest (geodesic) paths pass through a node, it is a potential bottleneck, and its removal can disrupt the integrity of the network. The measure of a nodes propensity to be a bottleneck is the betweenness centrality\cite{freeman1977set}. Closeness centrality measures how close a vertex is on average to all other nodes in the network. Though not a classic centrality measure, transitivity can be seen as a local form of betweenness centrality \cite{newman2018networks} and is therefore included here along with kcoreness. 

 Centrality measures are frequently correlated with each other \cite{valente2008correlated}, although their correlation varies\cite{oldham2019consistency}. 
%  Figure~\ref{fig:dolphin_multipanel} shows vertex degree, betweenness centrality, closeness and eigenvector centrality in a real world network, a social network of bottle nose dolphins \footnote{\url{http://www-personal.umich.edu/~mejn/netdata/dolphins.zip}} \cite{lusseau2003bottlenose}. High values of these measures are shown in red. 

Figure~\ref{fig:centrality measures from Plos} shows the fact that centrality measures can be concordant (for example, high degree and closeness in panel a) or discordant (red node has high centrality and betweenness in panel b but low degree). 

% \begin{figure}
%     \centering
%     \includegraphics[width=0.9\textwidth]{images/centrality2002.png}
%     \caption{Display of centrality measures. Betweenness centrality, eigenvector centrality and local transitivity for network of bottle nose dolphin social relations data. \cite{lusseau2003bottlenose}}
%     \label{fig:dolphin}
% \end{figure}






Over 200 centrality measures have been described\cite{jalili2015centiserver}\footnote{references for what are 'classic' measures vary \cite{cadini2008using} has closeness, betweenness, information and degree but not eigenvector which most have \cite{brandes2016maintaining} has closeness, betweenness and degree joined later by eigenvector centrality}. This chapter will consider degree centrality, eigenvector centrality, betweenness, closeness and transitivity in the PSP and their association with animal models of learning and human cognitive ability. The six measures primarily used in this thesis have been those that are implemented in igraph and are sufficiently popular to appear in a general textbook\cite{newman2018networks}. Degree centrality and eigenvector centrality are neighbourhood based centrality measures, and closeness and betweenness shortest path-related\cite{kardos2020stability}. The simplest centrality measure, degree centrality, \cite{newman2018networks} is considered below. 

\subsection{Degree centrality}
\label{sec:degree centrality}

%from \url{/home/grant/Dropbox/PhD_latex_master}
The degree, $k$, of a node is the number of edges connected to it. A node connected to a large number of other nodes can influence more of the network. In a protein-protein interaction network, changes in a high degree protein may give rise to a variety of different phenotypes or, if the protein is sufficiently essential, give rise to severe disease.  In the PSP proteome network, the degree is the number of proteins a protein (represented by its encoding gene) directly interacts with.


All centrality measures can be derived from the adjacency matrix. The adjacency matrix $A_{i,j}$, represents an undirected, unweighted network such as the PSP,  where $A_{i,j}=1$  if there is an edge between nodes $i$ and $j$ and 0 otherwise, the degree of node $i$, $k_i$ is\cite{boccaletti2006complex}:
\begin{equation}
k_i = \sum_{j=1}^n A_{i,j}.
\label{Equation:Degree_from_adjacency}
\end{equation}

 Nodes with a high degree, often called `hubs'\cite{zhu2007getting}\footnote{In a directed network, hubs may refer to nodes that point to authorities nodes with important information. As the postsynaptic proteome network is undirected, hubs will be used in the first sense in this thesis\cite{kleinberg1999authoritative}} may have more influence in a network.  In a social network, individuals with a high degree centrality have many contacts and may be influential in the network. In a protein-protein interaction network, high degree proteins will have numerous interactions, take part in many biological processes and are likely to be necessary. 
 
 Disruption of high degree nodes also leads to disproportionate effects upon a network structure\cite{jeong2001lethality}, \cite{albert2000error}. Scale-free networks are relatively immune to disruption by removing random nodes but break apart with disruption of hubs. Hub effects on PPI networks are reviewed by Zotenko\cite{zotenko2008hubs}.


\subsection{Degree distribution}
\label{sec:degree distribution}
The degree distribution describes the probability, $p(k)$, that a  node has degree $k$. It is a probability distribution, and so

\begin{equation}
    \sum_k p(k)=1.
\end{equation}

For an empirical network $P(k)$ is the proportion of nodes with degree $k$. 

% An alternative representation for an emprical network is the degree sequence which is  node degree for each vertex in a network eg $1,1,2,3,5,7,\dots,n$. The properties of networks can be discovered by comparing these networks to random models (section~\ref{sec:configuration model}. It seemed reasonable that networks would have degree distributions similar to those found in Erdos Renyi graphs. It was only with the availability of information on large networks such as the internet and world wide web and computerised databases such as the IMDB that the fact that many natural networks have a power law distribution of node degree (section ~\ref{sec:scale_free}. Barabasi, Albert and Jeong found that the internet had a scale free distribution. A narrative of this discovery is provided in Network Science by Barabasi \cite{barabasi2016network}. 


If a network were to have edges distributed between nodes with a uniform probability, $p$, we would expect the degree distribution ($P(k)=k)$) to be distributed according to the binomial distribution.  

 The probability of a node having $k$ links is $p^k$, the probability of there being no links to the remaining $N-1-k$ nodes (-1 as we exclude self loops) is $(1-p)^{N-1-k}$, there are $\binom{N-1}{k}$ ways in which this arrangement of present and missing edges can be arranged, so the degree distribution follows the binomial distribution\cite{barabasi2016network}:

\begin{equation}
   \binom{N-1}{k}        p^k (1-p)^{N-1-k}
   \label{Equation:BinomialDistributionForDegreeProbability}
\end{equation}

In the limit, where the mean degree is much smaller than the size of the network, the degree distribution is closely approximated by a Poisson distribution\cite{barabasi2016network}. The Poisson distribution

\begin{equation}
    p(k) = \frac{\lambda^k e ^-\lambda}{!k},
\end{equation}

has only one parameter for an empirical network, $\lambda$ is $E[k]$.  The probability mass should be centred around the empirical mean  (17.6 in the case of the PSP). The variance of a Poisson distribution is $\lambda^{1/2}$, which is approximately 4.12. The twenty postsynaptic proteome genes with highest degree (table~\ref{tab:The 20 post synaptic proteome genes with highest degree in their cognate proteins.}) have a degree of 221 or greater. Under a Poisson model, these are around 100 standard deviations from the mean. 
\begin{table}[ht]
\centering
\setlength{\extrarowheight}{2pt}
{
\begin{tabular}{@{}llll@{}}
  \toprule
Entrez ID &  $k$ & Symbol & Description \\ 
\midrule
351 & 535 & APP & amyloid beta precursor protein \\ 
  2335 & 474 & FN1 & fibronectin 1\footnote{Fibronectin lethality} \\ 
  1994 & 473 & ELAVL1 & ELAV like RNA binding protein 1 \\ 
  1956 & 450 & EGFR & epidermal growth factor receptor \\ 
  7514 & 396 & XPO1 & exportin 1 \\ 
  10482 & 375 & NXF1 & nuclear RNA export factor 1 \\ 
  7316 & 363 & UBC & ubiquitin C \\ 
  26270 & 342 & FBXO6 & F-box protein 6 \\ 
  7412 & 327 & VCAM1 & vascular cell adhesion molecule 1 \\ 
  55832 & 316 & CAND1 & cullin associated and neddylation dissociated 1 \\ 
  8454 & 316 & CUL1 & cullin 1 \\ 
  2885 & 313 & GRB2 & growth factor receptor bound protein 2 \\ 
  7534 & 298 & YWHAZ & tyrosine 3-monooxygenase tryptophan 5-monooxygenase \\
  3320 & 266 & HSP90AA1 & heat shock protein 90 alpha family class A member 1 \\ 
  10075 & 263 & HUWE1 & HECT, UBA and WWE domain containing 1, E3 UPL \\ 
  4869 & 263 & NPM1 & nucleophosmin 1 \\ 
  7415 & 245 & VCP & valosin containing protein \\ 
  3178 & 227 & HNRNPA1 & heterogeneous nuclear ribonucleoprotein A1 \\ 
  22938 & 221 & SNW1 & SNW domain containing 1 \\ 
  8453 & 221 & CUL2 & cullin 2 \\ 
   \bottomrule
\end{tabular}
}
\caption{The twenty post synaptic proteome genes with highest degree in their cognate proteins. Degree = $k$. UPL = Ubiquitin protein ligase.}  
\tiny Code to generate table at \url{source('~/RProjects/gridsearch_gamma/R/degree_distribution.R')}
\label{tab:The 20 post synaptic proteome genes with highest degree in their cognate proteins.}
\end{table}

\subsection{Scale free}
\label{sec:scale_free}
% \textcolor{red}{need to give background on barabasi etc}
% \todo{background scale free}
The work of Barabasi, Albert and Jeong (1999) led to the recognition that real-world networks often have a scale-free degree distribution \cite{barabasi1999emergence},\cite{barabasi1999mean}\footnote{more recent work\cite{broido2019scale} has argued that scale-free networks are relatively uncommon and that a log-normal or other distribution may be a better statistical fit. When discussing the literature on scale-free networks, I have retained the use of scale-free or power-law from the original literature; however, this should be understood that this includes the possibility of another significantly right-skewed distribution}. In a scale-free network, the probability of a vertex having degree $k$, $p(k)$, is

\begin{equation}
    p(k) \sim k^{-\gamma},
\end{equation}
\label{eq:scale free}

where $\gamma$ is the degree exponent\cite{barabasi2016network}.

Power law distributions are `scale free' because increasing the degree, $k$, changes the probability of the degree by a common multiplicative factor. This pattern of change leads to a straight line graph on a log-log plot of degree $k$ and probability of degree $p(k)$, the signature of the scale-free distribution\cite{beltrami2013mathematical}. 

Many other things in nature follow a scale-free distribution (e.g. the size of craters on the moon, frequency of words and size of cities), and subsequent research showed that many other networks (the first being the wiring diagram of an IBM chip, the map of the power grid and the Hollywood actor database) are all scale free\cite{barabasi2016network}. \footnote{ p11 in \cite{barabasi2016network}. Chapter 0 of this book gives a detailed personal history of the discovery of the scale-free property in networks and its historical antecedents}

Albert, Jeong and Barabasi \cite{albert1999diameter} built a web robot to map out the links in the \url{nd.edu} domain of the world wide web. They found 325,729 documents and 1,469,680 links. As documents point to other documents on the world wide web, they were represented by a directed graph. The network they constructed displayed a power-law distribution with $\gamma_{out}=2.45$ and $\gamma_{in}$=2.1. At that time (1998), there was `no trace in the literature of a network with a power-law distribution'\cite{barabasi2016network}\footnote{p10} although  Price's network model, a generative model for power-law networks, was described in the 1960s\cite{price1965networks}. 

The Barab\'asi-Albert model for generating a scale-free network distribution requires network growth and preferential attachment. Preferential attachment means that as a network grows, new nodes are usually attached to high degree nodes\cite{barabasi1999emergence}.

The characterisation of power-law distributions remains an active area of research in statistics. Many networks display the power-law distribution only in the tails, and some seem to conform to log-normal distributions. Recent work has suggested that power-law distributions are comparatively rare compared to other right-skewed distributions\cite{broido2019scale}.  In any case, many real-world networks are right-skewed with heavy tails and deviate markedly from a Poisson distribution. 


\subsection{Betweenness centrality }
\label{sec:Betweeness centrality}
Betweenness centrality and closeness centrality (section~\ref{sec:closenesscentrality})  are both vertex centrality measures based on shortest paths between nodes\cite{newman2018networks}.  Betweenness centrality, popularised by Freeman (1977) \cite{freeman1977set}\footnote{Newman \cite{newman2018networks} points out that Freeman acknowledges an earlier unpublished account by Anthosine}is a measure of the number of geodesic (shortest) paths in the network passing through a particular node. 

 Following Newman's notation\cite{newman2018networks}, if $n_{st}^i=1$  if a shortest path between nodes $s$ and $t$ pass through node $i$, and  $g_{st}$ is the number of shortest paths between $s$ and $t$ (as there may be more than one shortest path), then the betweenness centrality of node $i$ is

\begin{equation}
    x_i = \sum_{st} \frac{n_{st}^i}{g_{st}}.
\end{equation}
\label{eq: Betweenness centrality}

There is a shortest path in both directions between nodes in an undirected network, e.g. from $s$ to $t$ and $t$ to $s$. This double-counting does not alter the rank of the nodes ordered by betweenness centrality and allows for easier application to directed networks\cite{newman2018networks}. The betweenness centrality can be normalised by dividing by the number of possible node pairs in the network $n^2$\cite{newman2018networks} \footnote{p176}.

Nodes with high betweenness centrality connect different areas on the network (i.e. paths between nodes in the network disproportionately flow through them) and are sites of increased flow (of information, for example). The description above assumes shortest paths are known and taken through the network\cite{borgatti2005centrality}  (random walk betweenness centrality measures betweenness when paths are random). Other nodes depend on nodes of high betweenness centrality to connect, and in the sociological literature, they are sometimes called  `brokers'\cite{newman2018networks}. 



\subsubsection{Edge betweenness centrality}
 %\todo{edge centrality}
 Edge betweenness centrality is the number of shortest paths that pass through a particular edge \cite{girvan2002community}. The removal of high edge betweenness centrality edges leads a network to break into communities (see section~\ref{sec:Newman and Girvan})

\subsection{Random walk betweenness}
\label{sec: random walk betweenness}
The betweenness centrality described above assumes that information takes the shortest path through the network although this may not always be true. Random walk betweenness is a betweenness metric dependent on the property of random walks in the network. The results are often similar and where they differ it is not clear if this is significant\cite{newman2018networks}. For this reason I have not included random walk betweenness. 

\subsection{Closeness centrality}
\label{sec:closenesscentrality}
Closeness centrality was described by Alex Bevelas in studies of information flows through social networks \cite{bavelas1948mathematical}. 

Closeness centrality measures how centrally located a node is in a network (i.e. closeness to all other nodes in the network). Closeness centrality is the inverse of the average length of the shortest path from a node to other vertices in the graph \cite{freeman1978centrality}. $l_i$ is the mean distance between node $i$ and every other node in the network:

\begin{equation}
    l_i = \frac{1}{n} \sum_j d_{ij}.
\end{equation}

 As a \textit{distance} measure it is small when nodes are close to all other nodes. The closeness \textit{centrality} is its inverse:


\begin{equation}
    C_i = \frac{1}{l_i}.
\end{equation}
\cite{newman2018networks}

While betweenness centrality measures the dependence of other nodes in the network on a particular node (as a bottleneck for information), the closeness centrality has been interpreted both as a measure of proximity and efficient access to other nodes, and alternatively as a measure of independence from other nodes \cite{brandes2016maintaining}.


\subsection{Eigenvector centrality}
\label{sec:explain eigenvcentrality}
Degree centrality assumes that all the connections between a node and others in the network are equally important.  However, adding a connection to a critical node may make a node more important than the increase in degree centrality by one would suggest. 

Eigenvector centrality, proposed by Bonacich \cite{bonacich1972factoring}, takes into account the importance of a node's neighbours in calculating centrality and does not consider all of the edges of a vertex to be equal. Instead, connecting to a critical node, results in a greater increase in importance. 

Using Newman's notation \cite{newman2018networks}, the importance of a node is defined as the sum of the importance of its neighbours if $x_i$ is the importance of node $i$

\begin{equation}
    x_i = \kappa^{-1} \sum_{j\in nei. i} x_j,
\end{equation}

where $\kappa$ is a constant of proportionality and $j$ denotes all neighbours of node $i$. This can be written using the adjacency matrix as

\begin{equation}
    x_i = \kappa^{-1} \sum_j A_{ij} x_j,
\end{equation}

so avoiding the introduction of $j$ as the neighbours of $x_i$, since the value of the adjacency matrix for nodes, $i,j$, that are not neighbours is 0.

In matrix form:
\begin{equation}
    \mathbf{x}=\kappa^{-1}\mathbf{Ax},
    \end{equation}
    
    so,
    
    \begin{equation}
         \kappa\mathbf{x}=\mathbf{Ax}.
    \end{equation}
  
  
  
  
This is an eigenvector problem and the vector of centralities is one of the eigenvectors of $\mathbf{A}$. The eigenvector will have as its $i$th entry the importance of the $i$th node. As all centralities should be positive, the appropriate eigenvector is associated with the largest eigenvalue (by the Perron-Frobenius theorem). 
  
     %\footnote{The eigenvector centrality values are not strict measures, but this is not a concern if we only use the order of eigenvector centrality, although they can, however, be normalised}. 
 \subsubsection{Related measures}
\label{sec:related centrality measures}
Katz centrality is a solution to the problem of calculating eigenvector centrality in directed networks. If a vertex is not in a strongly connected component, it will have an eigenvector centrality of 0. The Katz centrality includes a constant term $\beta$ which is added to the eigenvector centrality, and a term $\alpha$ that determines the balance between the eigenvector centrality and the constant term. It is not widely used in undirected networks\cite{newman2018networks}. 

Page rank is a variant of Katz centrality widely used in web-search. Here the benefit of an important neighbour is divided by their out-degree (or degree in an undirected network). If a node is linked to a very important neighbour, this is much less remarkable if the neighbour has a million other connections. A review is provided in Gleich \cite{gleich2015pagerank}. There are also constant terms $\alpha$ and $\beta$ analogous with Katz centrality\cite{newman2018networks}.
    
% Alpha centrality and bonacich centrality have parameters (bonacich is alpha with alpha preset per bonacich). Bonacich is power\_centrality in igraph. Power centrality with preset is symmetrical around 0. \cite{bonacich1987power}
% Nodes are important if their neighbours are important 
 

% \todo{? add pagerank}




\subsection{Transitivity}
\label{sec:transitivity}
Transitivity is a measure of how connected the neighbours of nodes are either globally or at the level of individual nodes. 

Transitivity is the phenomena where two nodes are more likely to be connected if they share common neighbours. Transitivity is measured by clustering indices, forms a basis of the small world effect and can be thought of as a centrality measure. There is some inconsistency in the literature over the use of the terms clustering, clustering coefficient and transitivity, and it is worth clarifying this before going on. 

A network's global transitivity measures the probability that two nodes, each connected to another node, will be connected (if someone has two friends, how likely are those people to be friends). The transitivity is the ratio of the number of these connections, \textit{closed triads}, occurring in a network to the maximum possible (range 0-1\footnote{[0-1]}).

Transitivity of relations, analogous to mathematical transitivity, has been used in the sociological literature since the 1940s (see Louch\cite{louch2000personal} for details). Heider writing in 1946 states: `A relation R is transitive if aRb and bRc imply aRc'\cite{heider1946attitudes}\footnote{p109}. It is natural, therefore, to extend the description to networks describing social relations. 

Watts and Strogatz described a \textit{global clustering coefficient} in 1989\cite{watts1998collective}, the arithmetic mean of local clustering coefficients. The local clustering coefficient measures transitivity (as described by Heider\cite{heider1946attitudes}), and the global measure is a global measure of transitivity. Watts and Strogatz do not use the term transitivity, but the global clustering coefficient is a mean of a measure of transitivity. Newman described a global measure of clustering (again a measure of transitivity) as an alternative formulation of the Watts and Strogatz global clustering Equation \ref{eq:Transitivity newman}\cite{newman2002assortative}. This was in the context of random graphs, and except in some circumstances (e.g. degree the same), these transitivities are not identical (see  Schank and Wagner\cite{schank2005approximating}).

There are, therefore \textit{two} global measures of global transitivity, one described by Newman and one due to Watts-Strogatz (most often referred to as the clustering coefficient). These can and do diverge. There is a local clustering coefficient (described in Watts and Strogatz\cite{watts1998collective}) which is sometimes referred to as local transitivity. I will refer to the transitivity due to Newman as $C$ and the Watts-Strogatz as $C_{WS}$. Local transitivity is $C_i$

Attempts have been made to clarify this nomenclature\cite{estrada2016local}, but these introduce more terms and definitions. A full exploration of this is beyond the scope of this thesis, but the interested reader will find all the relevant details Louch\cite{louch2000personal},  Schank and Wagner\cite{schank2005approximating} and  Newman \cite{newman2018networks}. 
 


\begin{table}[]
    \centering
    \begin{adjustbox}{width=\textwidth}
   
    \begin{tabular}{lllll}
    \toprule
       Name  & Global/Local & Measures & Symbol & Estrada\cite{estrada2016local} \\
 \midrule
        Global clustering coefficient & Global & Global transitivity & $C_{WS}$ & {the Watts–Strogatz clustering coefficient}  \\
        
        Global transitivity & Global & Global transitivity & $C$ & global transitivity index  \\
        Local clustering coefficient & Local & Local transitivity & $C_i$ & Watts–Strogatz clustering coefficientof a node i,\\
        \bottomrule
    \end{tabular}
     \end{adjustbox}
    \caption{Disambiguation of clustering and transitivity measures}
    \label{tab:disambiguation transitivity}
\end{table}


\subsubsection{Global transitivity $C$ (Newman)}
\label{sec:global transitivity newman}
A triangle of vertices,  all connected, forming a loop is a \textit{closed triad}  and is a \textit{path} of length two (as paths have distinct edges)\cite{newman2018networks}. The possible triads that could be formed by adding an edge are all of the paths of length two in the network, so the transitivity $C$ is:

\begin{equation}
    C = \frac{\textrm{number of closed paths of length two}}{\textrm{number of paths of length 2}}.
    \label{eq:Transitivity newman}
\end{equation}

Since six paths are present in a closed triad (counting once for each direction the path can take in an undirected network, e.g. $ab$ and $ba$ are distinct), this is:

\begin{equation}
    C = \frac{6 \times \textrm{number of triangles}}{\textrm{number of paths of length 2}}
\end{equation}.

Using \textit{connected triple} to refer to any connected three vertices (whether closed or not), the transitivity can be written,

\begin{equation}
    C = \frac{6 \times \textrm{number of triangles}}{2 \times \textrm{number of \textit{connected triples}}}.
      \label{eq:Transitivity newman - triples}
\end{equation}.

The average connectivity of two nodes joined by a neighbour for a network or graph is thus its transitivity\cite{newman2001structure},\cite{newman2018networks}. This is the form found in \cite{newman2002random} where it is said to be another form of the clustering coefficient of Watts and Strogatz (incorrectly, here I am citing  Schank and Wagner \cite{schank2005approximating}  \footnote{cited by 284 and pretty clear})

%\todo{expected value compare with random model}


\subsubsection{Global clustering coefficient $C_{WS}$}
\label{sec:Global clustering coefficient}
The local transitivity defined below (section \ref{sec:local clustering coefficient}), is also called the \textit{local clustering coefficient}. The global clustering coefficient is defined by Watts and Strogatz\cite{watts1998collective} as the arithmetic mean of the vertex local clustering coefficients\cite{estrada2016local},


\begin{equation}
    \bar{C}=\frac{1}{n}\sum_i^n C_i.
\end{equation}


The global clustering coefficient can become dominated by nodes of low degree\cite{newman2018networks}\footnote{p188}, and nodes with low degree tend to have high values of local clustering coefficient (section~\ref{sec:relationship of ck to k albert}). The transitivity above described by Newman\cite{newman2002random} (see equation \ref{eq:Transitivity newman - triples}) is referred to as the \textit{transitivity index}\cite{estrada2016local}. To avoid confusion I will follow Newman's \cite{newman2018networks} terminology and refer to $\hat{C}$ as $C_{WS}$.

\subsubsection{Local clustering coefficient $C_i$}
\label{sec:local clustering coefficient}
Local transitivity (also known as the \textit{local clustering coefficient}, or specifically by Estrada\cite{estrada2016local} as the \textit{Watts–Strogatz clustering coefficient} of a node $i$) is defined for a node with $k_v$ neighbours as: 

\begin{equation}
C = \frac{c_n}{c_{max}},
\label{eq:local_transitvity}
\end{equation}


where $c_n$ is the number of connections between a nodes neighbours, and $c_{max}$, the maximum number of connections is, 

\begin{equation}
c_{max} = \frac{k_v(k_v)-1}{2}.
\label{eq:local_transitivity_denominator}
\end{equation}

% Corrected above from erroneous

This \cite{newman2018networks} is precisely equivalent to the measure in Watts and Strogatz, which is the `fraction of allowable edges that exist` amongst the neighbour of a node where node v has $k_v$ neighbours is identical to \ref{eq:local_transitivity_denominator}.

The local clustering coefficient can be regarded as a local type of betweenness centrality and hence a type of centrality\cite{newman2018networks}\footnote{p187}. Local clustering measures the connectivity of nodes in the neighbourhood of a particular node. If the local clustering is low, then there exist disconnections between a nodes neighbours, which, in social networks, have been called structural holes \cite{burt2009structural}\footnote{and influence measured by redundancy}. The node connecting these neighbours has influence and importance or centrality similar to that measured by betweenness centrality but confined to the immediate neighbourhood of the node\footnote{In social network analysis, they have been proposed to be a source of social capital and innovation}. I have therefore included analysis of local transitivity with that of other centrality measures. 
%\todo{power}

It is clear from equation \ref{eq:local_transitvity} and \ref{eq:local_transitivity_denominator} that there is a problem for the local transitivity for nodes with degree 0 or 1. These \textit{isolates} have been treated either as having an undefined value or as zero by convention (although see \cite{schank2005approximating} where they are treated as 0 or 1). This has an effect both on local transitivity ( Schank and Wagner \cite{schank2005approximating}) and on the correlation of tranisitivity with other measures (see section~\ref{sec:relationship of ck to k albert}). See table~\ref{tab:disambiguation transitivity}
%\todo[inline]{do isoaltes zero or na}



  Schank and Wagner\cite{schank2005approximating} discuss isolates in-depth; they hold that the transitivity is only defined for nodes with a degree greater than or equal to two. For those with degree at most one, it is either zero or one \footnote{see (text under equation 3 in \cite{schank2005approximating}} and show that they get significantly different results using one rather than zero. 

\subsection{kcores}

kcores can be viewed as a centrality measure and a solution to finding a cohesive set of actors in a graph efficiently.

The kcore of a graph is the maximum induced subgraph of a graph, such that each vertex has a degree of at least $k$. Kcores can be nested, and cores are not necessarily connected subgraphs\cite{batagelj2003m}. 

H is a subset of g such that the degree of h is $>=k$

\begin{equation}
h \subseteq g \mid \delta'(h)>=k,
\end{equation}

where $g$ is the network, $h$ the induced subgraph and  $\delta'(g)$ the minimum of the degree of the vertices of $h$\cite{seidman1983network}.

For a vertex its kcoreness is the maximal $k$ core subgraph it is a member of (alternatively its shell index is $c$ if it belongs to the $c$ core but not the $c+1$ core)\cite{seidman1983network}\footnote{original description}, \cite{alvarez2006large}.K core is the connected graph left after components with degree less than k removed.  \footnote{This equation is actually from Seidman p 272}\cite{seidman1983network}.

The kcore algorithm has proved helpful in graph analysis as it runs in linear time and has been used as a measure of core-periphery structure in graphs\cite{newman2018networks}. 

 K coreness was used to study the hierarchy of biological networks in plants. High core metabolic nodes represented core pathways conserved across different plant species\cite{filho2018hierarchical} (figure~\ref{fig:kcore from Filho}). 

\begin{figure}
    \centering
    \includegraphics[width=\textwidth]{images/chapter3/from_plos/kcorejournalpone0195843g001.PNG}
    \caption{Image from Filho et al \cite{filho2018hierarchical} showing k cores arising in the metabolic network of plants. Panel d shows the k core dissection of the network for $k$ {1,6,9,13,18}}
    \label{fig:kcore from Filho}
\end{figure}

Wuchty and Almass carried out kcore dissection of the protein interactome of \textit{Saccharomyces cerevisiae} to better understand the association between lethality and degree. The maximal kcore in their network was nine, and kcore was higher for essential proteins and in those with a eukaryote ortholog\cite{wuchty2005peeling}.

Newman reports that the sociological literature finds high k core actors to be important but that the evidence for this is limited\cite{newman2018networks}. It has been viewed as a kind of centrality measure core-periphery \cite{newman2018networks}\footnote{p180}. Given its reported utility in identifying vital elements in biological networks and as a potential supplement to degree centrality lethality\cite{wuchty2005peeling} I have investigated kcoreness as a measure of centrality and report them along with other centrality measures.






\section{Assortativity and assortativity of measures}
\label{sec:assortativity}
%\textcolor{red}{? move this to the other network level statistics}

Assortativity is a measure of the connectivity pattern of the network. If like node links to like the network is assortative, if dissimilar nodes link then the network is disassortative. 

An example from a standard text is celebrities marrying one another more than would be expected by chance\cite{barabasi2016network}.

Assortativity was described and formalised by Newman as a network level measure of how likely nodes with similar properties are to be connected \cite{newman2002assortative}. It can be calculated for scalar or categorical vertex properties. For example if supporters of the same sports team are more likely to be friends their social network would be assortative for this categorical vertex property (team supported). If in a social network people are more likely to be friends if their total wealth is different then the network would be disassortative for this scalar property. The degree assortativity is an important form of scalar assortativity  with a significant impact on network structure and behaviour. We can use assortativity to see if significant nodes in GWA studies tend to be interconnected and if the same is true of animal models of cognition and LTP. 

For a categorical vertex property, the measure of assortativity is the fraction of edges between vertices that share the property compared with the number of edges expected by chance.

This measure is the modularity which we will see again in chapter~\ref{chap:community detection} where the groups are communities. 

The modularity ($Q$) is

\begin{equation}
    Q = \frac{1}{2m}\sum_{ij}(A_{ij}-\frac{k_ik_j}{2m})\delta_{g_ig_j},
    \label{eq:categorical assortativity}
\end{equation}

where $m$ is the number of edges in the network, $A$ is the adjacency matrix, $k_i$ is the degree of node $i$, $g_i$ is the group or category to which node $i$ belongs and $\delta$ is the Kroneker delta. 

The modularity lies in the range (-1,1)
%\todo{should this not be [-1,1]} 
and takes on a negative value for a disassortative network structure.

The measure of assortativity for scalar quantities is the covariance of the scalar quantity between connected nodes. The mean used in calculating the covariance is the mean value of the scalar quantity at the end of each edge. This is:

\begin{equation}
   \textrm{cov}(x_i,x_j)  = \frac{1}{2m}\sum_{i,j}(A_{i,j}-\frac{k_i,k_j}{2m})x_ix_j,
   \label{eq:assortativity covariance of xi and xj over edges}
\end{equation}

where $x_i$ is the value of the scalar property at node $i$. The \textit{assortativity coefficient}, $r$ is the covariance normalised by the maximum `value of the covariance in a perfectly mixed network'\cite{newman2018networks}
%\todo{review}

 \begin{equation}
     r = \frac{\sum_{ij}(A_{ij}-k_ik_j/2m)x_ix_j}{\sum_{ij}(A_{ij}-k_i\delta_{ij}/2m)x_ix_j},
 \end{equation}
 
 which is Pearson's correlation coefficient\cite{newman2018networks}.
 
We can use assortativity to determine if vertices associated with a high $z$ score in an intelligence GWAS (more associated with the trait) are more likely to be connected across the network. If there is widespread `guilt by association'\cite{oliver2000guilt} then the network would be assortative for that particular property. Similarly the categorical measure of assortativity (equation~\ref{eq:categorical assortativity}) can be used to determine if genes that are involved in animal models of cognition are connected. This provides an objective measure of guilt by association and the `disease module'\cite{baranzini2013network} hypothesis (or trait module).

\subsection{Degree assortativity}
\label{sec:degree assortativity}

The most commonly studied scalar property is degree assortativity \cite{newman2018networks}\footnote{p210}. The degree assortativity is the Pearson correlation coefficient between node degree and the mean degree of its neighbours (for connected vertices)\cite{noldus2015assortativity}. Assortative networks tend to form communities and are more robust to disruption by node removal \cite{newman2002assortative} and increased assortativity leads to increased speed of transmission of information through a graph \cite{noldus2015assortativity}.

Social networks tend to be assortative with a core structure of highly connected hubs linking to other hubs\cite{newman2018networks} however biological networks tend to have a hub and spoke structure.  



%%%%%%%%%%%%%%%%%%%%   CENTRALIY IN BIOLOGY AND MEDICINE %%%%%%%%%%%%%%%%%%%%%%%%%%%%%%%%%%%%%%%%%%%%%%%%%%%%%%%%%%%%%
\section{Centrality and essentialness}
\label{sec:Degree and essentialness}
Hub proteins/genes tend to be essential to life. This was the finding of an analysis of the in the proteome of \textit{Saccharomyces ceresvisiae} proteome) \cite{jeong2001lethality}  shortly after the discovery that scale free networks are vulnerable to disruption by the removal of hubs\cite{albert2000error}.
  The  \textit{Saccharomyces ceresvisiae} PPI network's (1,870 nodes, 2,240 edges) degree distribution followed a power-law distribution in its tail with 78\% of its proteins in the largest connected component\cite{jeong2001lethality}.  Elimination of the most connected nodes led to an increase in network diameter while random elimination did not, and mutations in high degree proteins were more likely to be lethal. Only 21\% of the genes encoding proteins with degree less than five (93\% of the proteome) had lethal effects from mutation, whereas although only 0.7\% of yeasts had more than fifteen links,  62\% of these had lethal phenotypes. The authors \cite{jeong2001lethality} concluded that the combination of centrality and lethality was due to the ``topological position of its protein product in the complex hierarchical web of molecular interactions''. The \textit{centrality-lethality} hypothesis holds that high degree hubs are more likely to be essential for life\cite{zotenko2008hubs}. 

He and Zhang (2006)\cite{he2006hubs} proposed an alternative mechanism: high degree node were more likely to be essential as they participated in more protein interactions, a proportion of which were essential and so had a higher probability of containing one of these essential links (rather than being integral to the network structure). Zotenko et al. \cite{zotenko2008hubs} rejected these hypotheses noting that Gene Ontology terms suggested that hub proteins were more likely to take part in essential gene processes and that these clustered in modules they called ECOBIMs ( Essential Complex Biological Modules - induced subgraphs around seed nodes representing gene ontology terms).  
Later proteomic work further complicated this picture: more complete networks constructed using high quality yeast interactomes did not show evidence of the centrality lethality effect\cite{milenkovic2011dominating},\cite{yu2008high}, \cite{ratmann2009evidence}. 

Estrada\cite{estrada2006virtual} used additional centrality measures to assess essentialness in the yeast protein interaction network (degree centrality, eigenvector centrality, information centrality, closeness centrality, betweenness centrality and subgraph centrality).  High centrality measures predicted essentialness better than random selection, and eigenvector centrality gave the best results and closeness and betweenness centrality the poorest.

Raman \cite{raman2014organisational} examined multiple centrality metrics (degree, betweenness centrality, closeness centrality and pairwise dis-connectivity index) and their association with gene essentiality for 20 organisms. Using the STRING database, filtered for high-quality interactions, they found a disassortative pattern of connection of essential genes  (section~\ref{sec:degree assortativity})  and that essential genes had a higher degree and betweenness centrality. The twenty organisms they studied ranged from bacteria and yeasts to the more complex \textit{Caenorhabditis elegans}.
% \todo{rephrase}


\section{Biomedical significance of centrality measures}
 \label{sec:biomedical significance of centrality measures}
\label{sec:biomedical review centrality}
Hub genes are reported to be more likely to be expressed ubiquitously\cite{goh2007human} and critical genes in cancer proteomic networks are often of high degree\cite{wachi2005interactome}. 

 High degree nodes were involved in more diseases in the Online Mendelian Inheritance in Man (OMIM) compendium of genetic phenotypes\cite{xu2006discovering}. Goh et al., however, found that disease genes were likely to be tissue-specific and were not more likely to be hubs \cite{goh2007human}.  High degree hub genes were found to be more commonly essential than genes associated with disease\cite{barabasi2011network} although all of these findings depend upon how disease-gene associations are defined. 

Lee et al.\cite{lee2013network} used the National Human Genome Research Institute (NHGRI)  catalogue of complex trait GWAS\footnote{now the NHGRI-EBI GWAS catalogue} and found that complex trait loci were more likely to be found in genes that encoded proteins that were hubs or bottlenecks. Here, hub and bottleneck are defined as the top 20\% of the degree centrality and betweenness centrality distributions\footnote{10 and 20\% are common choices for cutoffs for top genes and will be relevant later} in a network derived from the STRING protein-protein interaction database (14025 nodes and 492087 edges). The authors used their network findings to re-prioritise GWAS datasets composed of type 2 diabetes studies and Inflammatory Bowel Disease Genetics Consortium (IBDGC) and increased the number of significant genes found in independent sources such as OMIM\cite{hamosh2005online}.  The study found that SNP loci are commoner in genes that encode hubs or bottlenecks in the protein interaction network for at least these two polygenic disorders. 






Hahn and Kern\cite{hahn2005comparative} examined protein-protein interactions networks in fly, yeast and worm, measuring betweenness centrality, degree centrality and closeness centrality. They found that more centrally located proteins in all networks evolved more slowly. Bayes (2011)\cite{bayes2011characterization} found more sequence conservation in PSP hub proteins, and Hill found that SNPs associated with differences in intelligence are found in genomic regions that are evolutionarily conserved\cite{hill2016molecular}. These provide further reasons to investigate the association between centrality measures and genes associated with differences in intelligence.






Ghasemi et al. (2014) review centrality measures in biological networks and illustrate each centrality measure using Zachary's Karate network\cite{ghasemi2014centrality}. Some of the papers in this review are discussed below (see Ghasemi et al.\cite{ghasemi2014centrality} for full review). 


    Joy et al.\cite{joy2005high} found that in the yeast protein-protein interaction network, essential nodes had 77\% higher degree centrality and 82\% higher betweenness centrality compared with non-essential nodes. They found several proteins with high betweenness and low degree centrality (High Betweenness Low Connectivity - HBLC), which standard models of scale-free network generation did not predict (e.g. Barabsi Albert). Degree and betweenness correlated with age and essentialness, but there were too few high betweenness low degree nodes to state if they differed from nodes with a similar degree. They make the interesting point that betweenness is less robust to mutation as it is not protected from mutation by the constraint of maintining multiple interactions partners \footnote{ref Fraser evolutionary rate in the protein interaction network}
    



Yu et al.\cite{yu2007importance} found that `bottlenecks', nodes with high betweenness centrality, were more likely to be essential, particularly in directed and regulatory networks. They found that betweenness correlated with degree (Pearson correlation coefficient = 0.49). Non-hub bottlenecks,  were also associated with essentiality, but degree was a better predictor of essentiality. Bottleneck nodes had lower than average co-expression supporting the hypothesis that they connect different modules.

Ozgur et al.\cite{ozgur2008identifying} developed a curated gene interaction network and used centrality measures to attempt to predict the genes associated with prostatic cancer using the prostate gene database (PGDB). They measured degree, eigenvector, betweenness and closeness centrality and found that high scores on the first three measures predicted increased rank in the PDGB.

Milenkovic et al.\cite{milenkovic2011dominating} investigated whether biologically central genes are topologically central using dominating networks, an extension of the idea of a dominating set, to derive a centrality measure. The four categories of biologically essential were host-pathogen interactions, cancer, HIV and ageing. They also assessed betweenness centrality, subgraph centrality and graphlet centrality. They conclude that biologically essential genes are more enriched with all measures but most for the graphlet centrality, although the differences are modest\footnote{other than ageing genes figure p8 C}.  

Pandey et al.\cite{pandey2012epistasis} analysed GWAS results for bipolar affective disorder using eigenvector centrality. They used the WTCCC-BD GWAS (1,868 cases and 2,938 controls) as a discovery sample and an NIMH GWA study (1,001 cases; n=1,033 controls) of individuals of European ancestry as replication. The network was a genetic association network and limited by the lack of power provided by GWA study sample size.

Barrenas et al. \cite{barrenas2009network} found that fifty four complex diseases found at GWAS\cite{hindorff2009potential} were less central in the human interactome (using HPRD v7\cite{keshava2009human}) than monogenic diseases or essential genes. 

Authoritative reviews of the literature have concluded that disease genes are less likely to be essential\cite{barabasi2011network} and more peripheral than essential genes, while others have held that the literature supports their being central in the protein interaction network\cite{ideker2008protein}.


% \textcolor{red}{there is\subsection{Centrality and intelligence in PSP}
% We will see what the correlation is in population cohorts and what we might expect it to be given the effects of high degree genes in model animals. Essentially nobody knows \todo{rephrase}
% \todo{This is perhaps the bit to introduce the possible different hypothesise} a note betweennness and transitivity and betweeness and biological illegible maybe literature}
%   This should be some sort of bridging bit. \textcolor{red}{yes move the bit below here as an intro to methods and lose the bit from centrality and intelligence in PSP}
  
%   The evidence on centrality of proteins and biological function is somewhat heterogenous but there is a tendency towards highly important nodes being essential. Their roles in human disorders are unclear and are likely to vary between specific disorders. 
  
%   The methods will show how I will calculate the centrality for all proteins and compare them to GWAS of intelligence and measures of evolutionary pressure. I will see what the nature of the nodes that are of high degree is. I will report statistics that summarise the PSP network and allow its comparison to other networks and will investigate how significant genes are distributed through the PSP and whether this distribution is similar to or different to those of animal models of cognition and LTP. 
%%%%%%%%%%%%%%%%%%%%%%%%%%%    METHODS  %%%%%%%%%%%%%%%%%%%%%%%%%%%%%%%%%%%%%%%%%%%%%%%%%%%%%

%%%%%%%%%%%%%%%%%%%%%%%%%%%    METHODS  %%%%%%%%%%%%%%%%%%%%%%%%%%%%%%%%%%%%%%%%%%%%%%%%%%%%%

%%%%%%%%%%%%%%%%%%%%%%%%%%%    METHODS  %%%%%%%%%%%%%%%%%%%%%%%%%%%%%%%%%%%%%%%%%%%%%%%%%%%%%

%%%%%%%%%%%%%%%%%%%%%%%%%%%    METHODS  %%%%%%%%%%%%%%%%%%%%%%%%%%%%%%%%%%%%%%%%%%%%%%%%%%%%%

%%%%%%%%%%%%%%%%%%%%%%%%%%%    METHODS  %%%%%%%%%%%%%%%%%%%%%%%%%%%%%%%%%%%%%%%%%%%%%%%%%%%%%

%%%%%%%%%%%%%%%%%%%%%%%%%%%    METHODS  %%%%%%%%%%%%%%%%%%%%%%%%%%%%%%%%%%%%%%%%%%%%%%%%%%%%%

%%%%%%%%%%%%%%%%%%%%%%%%%%%    METHODS  %%%%%%%%%%%%%%%%%%%%%%%%%%%%%%%%%%%%%%%%%%%%%%%%%%%%%

\section{Methods}
The evidence of the effect of protein and gene centrality and biological function is heterogeneous but supports highly important nodes being essential. Their roles in specific human disorders are undetermined and may vary between disorders. 
  
 I will calculate the centrality for all proteins in the network and see if they correlate with the level of significance in GWAS of intelligence and educational attainment in the gene that encodes the cognate protein. To test if the centrality lethality effect holds in human populations, I will test the correlation with measures of genetic selection and determine the nature of these central nodes using Gene Ontology enrichment. I will determine if the PSP network has properties such as being scale free and will investigate how significant genes are distributed through the PSP, and whether this distribution is similar to or different to those of animal models of cognition and LTP. 
 
    \subsection{Calculation of centrality measures}

To test the hypothesis that genes encoding central proteins in the  PSP network were more likely to be associated with genetic variants associated with differences in intelligence or educational attainment I have calculated the following centrality measures for each gene in the PSP: degree, eigenvector, closeness, transitivity, kcoreness and betweenness, and tested their correlation with each other and with each gene's significance scores (GWGAS $p$ values in MAGMA) in the discovery and replication educational attainment and intelligence samples (section~\ref{sec:cohorts from paper section}). 


\subsection{Calculating centralities}
\label{sec:calculating centralities}
Centrality measures for the PSP network ( section~\ref{sec:network graph generation}) were calculated using the igraph package, version 1.2.6 for R. The functions used are shown in table~\ref{tab:calculating centrality metrics}\footnote{\url{source('~/RProjects/chapter3_mac/R/utils/get_centrality_df.R')}}. All nodes in the graph are included unless specifically reported otherwise. The centrality calculations are available as a csv file in the supplementary material.

The local transitivity was calculated using the transitivity method in \texttt{igraph} for R with the argument type being set to `localundirected'. Where isolates - nodes with degree zero or one - exist there arises the question if we are to treat them as undefined (NaN) and exclude them from calculations or as zero (section~\ref{sec:local clustering coefficient}; corresponding to arguments isolates=`NA' and isolates=`zero'). I will refer to transitivity with isolates=`NA' as transitivityNA ($C^{NaN}$) and isolates = `zero' as transitivity0 ($C^0$) and include both as outlined in section~\ref{sec:local clustering coefficient}\cite{schank2005approximating}.

\begin{table}[]
    \centering
    \begin{tabular}{llll}
    \toprule
        Centrality metric & method & Options   \\
        \midrule
         Degree & \texttt{degree} & default   \\
         Eigenvector centrality & \texttt{eigen\_centrality} & default (scale=TRUE) \\
         Betweenness centrality & \texttt{betweenness} & directed=FALSE  \\
         Closeness centrality & \texttt{closeness} & default \\
         TransitivityNA & \texttt{transitivity} & type=`localundirected', isolates=`NaN'   \\
         Transitivity0 & \texttt{transitivity} & type=`localundirected', isolates=`zero'\\
         K-core & \texttt{coreness} & default  \\
         \bottomrule
    \end{tabular}
    \caption[Methods for calculating centrality measures]{Calculating centrality metrics showing igraph methods for R. The local node level transitivity is synonymous with the Watts-Strogatz local clustering coefficient. The Watts-Strogatz clustering coefficient is transitivity0. The igraph default is to treat isolates as NA (transitivityNA)}
    \label{tab:calculating centrality metrics}
\end{table}

\subsection{Correlation between centralities and measures of genetic significance}
\label{Correlation between centralities}
 Centrality measures are often correlated\cite{valente2008correlated}. Few have distributions that appear normally or even linearly distributed(figure~\ref{fig:multiplot centrality histograms}). The correlation between centrality measures and the significance of genetic variants for intelligence and education were calculated using Spearman's rank correlation method (non-parametric) in R. This was also used to calculate the correlation between centrality measures and with other measures as is reported in other analyses of centrality in the literature \cite{oldham2019consistency}. The correlation was also measured with other numeric node metadata (such as \textit{loss of function intolerance}).  The -log10 transform of the cohort p values are reported in calculating the significance and degree of association between centrality and genetic significance in studies of intelligence and educational attainment. This means highly central nodes with highly significant results in GWA studies will have a positive correlation. 


\subsection{Characterising the degree distribution}
\label{sec: Methods degree distribution}
% We can calculate the $\gamma$ coefficient for the degree distribution using the \texttt{poweRlaw} package for R. The value of $\gamma$ will vary over the distribution depending on the lower limit of degree used to fit the power law distribution. This is shown graphically in figure \ref{fig:gamma} and table \ref{table:gamma}.
If a network follows a power law degree distribution this has implications for its growth, function and resilience (section~\ref{sec:degree distribution}). The characteristic feature of a power law degree distribution is a straight line in a log-log plot of the probability distribution $ P(k)$ against degree $k$.

 Calculating the gradient of this line results in a biased estimate of the exponent  \cite{newman2018networks}\footnote{p324}. The method of Clauset, Shalizi and Newman is widely used instead\cite{clauset2009power}.

The  degree distribution often conforms to a power law distribution only in its tail (high degree)\cite{newman2018networks}. There is therefore a minimum degree ($k_{min}$) above which the power law distribution holds. 
This value ($k_{min}$) is found such that the distance between the network degree distribution and the best fit power law distribution is minimised. The distance between distributions is measured by the Kolmogorov-Smirnov statistic following Clauset, Shalizi and Newman \cite{clauset2009power}\footnote{section 3.3 p 11-12}. The value of the power-law exponent $\gamma$ varies as a function of $k_{min}$ as shown in figure \ref{fig:gamma}.

The R package \texttt{poweRlaw} (version 0.70.6)\cite{gillespie2015fitting} was used to calculate $x_{min}$ and the exponent of the distribution\footnote{p.7 \url{https://cran.r-project.org/web/packages/poweRlaw/vignettes/a_introduction.pdf}} and has been widely used in other network analyses (for example\cite{wood2014defining} and \cite{miho2019large}). After $k_{min}$ is calculated by minimising the Kolmogorov-Smirnov statistic, the exponent is obtained by numerically optimising the log-likelihood of the fit initialised using the analytic maximum likelihood estimate (MLE)\cite{gillespie2015fitting}.

Recent research has argued that the terms `scale free distribution' and `power law distribution' are used inconstantly in the network literature and that power law distributions may be rarer in empirical networks than previously thought. For many networks a log-normal or other heavy tailed distribution may be a better statistical fit\cite{broido2019scale}. This remains controversial\cite{holme2019rare}, but what is clear is that even the log-normal is a significant deviation from the Poisson distribution found in an Erd{\"o}s-R{\'e}nyi random graph generated with uniform probability of an edge between nodes. 



The power law, log-normal, exponential and Poisson distribution were fitted to the empirical distribution following the instructions in the poweRlaw\cite{gillespie2015fitting} vignette.

A plot comparing potential models to the complementary cumulative distribution function (cCDF) of the empirical degree distribution of the PSP network was made to identify clearly implausible models. A bootstrap estimate $p$ of the plausibility of a power law distribution was made using the poweRlaw package\footnote{p 17 rule out powerlaw on bootstrap p if $p<=0.1$, p large when good fit}. The parameters and log-likelihood of each model were calculated using the $x_{min}$ of the power law distribution\footnote{(eg if $x_{min}=y$ for a power law distribution other distributions are fitted with $x_{min}=y$  and the log-likelihood reported for the range $x_{min}$ to maximum degree}. A likelihood ratio test was carried out (Vuong's closeness test) to compare models\cite{vuong1989likelihood}. The methods in the poweRlaw package\cite{gillespie2015fitting} implementing the methods of Clauset, Shalizi and Newman \cite{clauset2009power} are also those that were used to test for the presence of power law distributions in the recent paper questioning their frequency by Broido and Clauset \cite{broido2019scale}. The level of confidence in the existence of the power law distribution is reported using the scale of Broido and Clauset (not scale free, super weak, weakest, weak, strong and strongest)\cite{broido2019scale}.

\subsection{Calculating small world property}
\label{sec:small world methods}

The small world property is present if a network has a short average path length and high clustering coefficient compared to a random model. The average path length and global clustering coefficient of the PSP network were calculated using igraph. Both global clustering coefficients were calculated (section~\ref{sec:transitivity}); the proportion of connected triples described by Newman\cite{newman2002random} and the mean of the local transitivity due to Watts and Strogatz ($C$ and $C_{WS}$) (section~\ref{sec:transitivity})\cite{watts1998collective}. Given the ubiquity of the small world phenomena in complex networks\cite{newman2018networks}, a measure of small worldness was calculated,  the small world metric of Humphreys and Gurney\cite{humphries2008network}.

The global and local transitivity are calculated using the \texttt{transitivity} function in igraph for R. The global transitivity measure described by Newman ($C$)\cite{newman2002random} is calculated by specifying type as `globalundirected'. The Watts-Strogatz clustering coefficient (global transitivity -  $C_{WS}$)\cite{watts1998collective} using the argument type=``globalaverageundirected''. In calculating the local transitivity or Watts-Strogatz global clustering one has to decide how to deal with isolates, nodes with degree zero or one as this can have a significant effect on the calculation\cite{schoch2006molecular}. The type=``localaverageundirected'' global transitivity in igraph has as its default the mean of local transitivity with isolates removed (isolates=`NaN'). The implementation of local clustering in graph tool \footnote{\texttt{graph\_tool.clustering.local\_clustering}}, by contrast, sets isolates to zero\cite{peixoto_graph-tool_2014}, and global clustering \footnote{\texttt{graph\_tool.clustering.global\_clustering}} returns the transitivity defined by Newman\cite{newman2002random}. The Watts-Strogatz clustering coefficient ($C_{WS}$) is, by convention, the mean of local transitivity with isolates set to zero\cite{newman2018networks}. Treating isolates as zero for this and local transitivity seems to be the majority practice \cite{wang2017comparison},\cite{newman2018networks} and is better suited for analytic solutions. I will follow it here, although there are reasons to prefer treating these nodes as undefined (`NaN'), one is that it complicates the relation of the transitivity to other measures(section~\ref{sec:Global clustering coefficient}). How isolates are treated only affects the Watts-Strogatz \textit{global} transitivity which I will refer to as $C_{WS_0}$ if isolates are zero and $C_{WS_{NaN}}$ if isolates are excluded from the calculation of the mean. For local transitivity I will append 0 or NaN to the transitivity to show how isolates have been handled (e.g. local transitivity0) and as symbol $C_i^0$ or $C_i^{NaN}$.  

The PSP network was compared to two different Monte Carlo estimates of expected average path length and global clustering coefficient. These random graph models had the same number of nodes and edges as the PSP. The first, the Erd{\"o}s-R{\'e}nyi random graph, is a graph $g$ with parameters $n$ and $p$; $n$ the number of nodes and $p$ the probability of two nodes being connected by an edge. A network with the small world property has a high clustering coefficient compared to an Erd{\"o}s-R{\'e}nyi model and an average path length closer to a random graph than a lattice (section~\ref{sec:Small world}). 

The second is the degree corrected random model (configuration model), in which the degree sequence of the network is maintained by breaking all edges into two stubs which are then randomly rewired. The rewiring model has the potential for self loops and multiedges but the probability is small as the number of nodes increase. Eliminating the possibility of self edges is undesirable for technical reasons\cite{newman2018networks}. The configuration model is implemented in igraph for R in the function \texttt{degree.sequence.game} using default arguments\footnote{Three possible methods are available simple, simple avoiding multiedges and vi (Viger and Latapy)}. The Erd{\"o}s-R{\'e}nyi random graph using \texttt{erdos.renyi.game} . One thousand random graphs were generated, and the mean and standard deviation of the average path length and clustering coefficient calculated\footnote{The code that carries out these calculations is at \url{source('~/RProjects/chapter3/R/small_world/eda_small_world.R')}
using\texttt{erdos.renyi.game}, type gnm, 1000 iterations.}. As the two forms of global transitivity ($C_{WS}$ and $C$) often diverge I have reported both values\cite{estrada2016local},\cite{humphries2008network}. 

The small world metric $S$, was calculated where $S$, is

\begin{equation}
    S = \frac{C/C_r}{L/L_r},
    \label{eq:small worldness}
\end{equation}

if $C$ is the transitivity and $L$ is the average path length (see \cite{neal2017small} for other metrics). 
 
\subsection{Calculating assortativity}
\label{sec:Assortativity methods}
To determine whether genes in the PSP network associated with intelligence and educational attainment tend to disproportionately connect with one another on a network level\footnote{as opposed to areas where they interact as modules}, I have calculated the assortativity between nodes for the z score levels for gene significance in the GWA studies (section~\ref{sec:assortativity}). Calculations of assortativity were carried out using igraph. The degree assortativity was calculated and compared to other complex networks. I have also calculated the assortativity by the other centrality measures\footnote{ The kcore assortativity is of particular interest \url{source('~/RProjects/chapter3/R/assortativity_distribution/z_score/assortativity_z.R')}}. I have calculated whether the network is assortative by evolutionary pressure using essential gene status (nominal assortativity) and probability of loss of function intolerance (scalar)\footnote{  \url{source('~/RProjects/chapter3/R/assortativity_distribution/z_score/assortativity_z.R')} degree and kcore }(See section\ref{sec:assortativity}).

The degree assortativity of the PSP was compared with the mean and standard deviation of the degree assortativity of 1000 Erdos-Renyi random graphs with parameters matching the PSP\footnote{\url{source('~/RProjects/chapter3/R/assortativity_distribution/assortativity_er.R')}} and the mean and standard deviation of degree assortativity of 1000 degree sequence (configuration) models\footnote{\url{source('~/RProjects/chapter3/R/assortativity_distribution/assortativity_config.R')}},\footnote{See \url{source('~/RProjects/chapter3/R/assortativity_distribution/knn/knn.R')}}

Nominal assortativity for categorical variables was calculated by encoding the property (e.g. membership of gene set) as a numeric vector and using the nominal assortativity\footnote{\texttt{assortativity\_nominal}} function in igraph. The assortativity for murine models of LTP and related gene sets was calculated using the gene sets in table~\ref{tab:Nominal assortativity for mammalian phenotypes. p is one tailed estimate Monte Carlo estimate of assortativity of same number of random vertices. p is proportion of random $q>=Q$. Samples:10000}\footnote{\url{source('~/RProjects/chapter3/R/assortativity_distribution/nominal_murine/murine_LTP_assortativity.R')}}. The nominal assortativity (Q) is the modularity of the division of the PSP into two groups using this property (section~\ref{sec:assortativity}).

To determine if there was `guilt by association' amongst genes associated with intelligence and educational attainment I have calculated the network assortativity  for  gene z-scores in the population intelligence and educational attainment samples. This allows us to see if significant genes are more likely to be connected to one another without  specifying a cut-off level of significance. In calculating the assortativity with z score, PSP genes that do not have associated MAGMA genes scores had missing values  replaced by the median of the Z score of  the particular sample\footnote{the assortativity function does not handle missing values}.


\subsection{Gene Ontology Analysis of extreme centrality genes}
\label{sec:Methods gene ontology centrality}

Section~\ref{sec:Degree and essentialness}reviews evidence that that central nodes may be more essential because they are qualitatively different and more involved in core processes. 

To determine whether central nodes possess distinctive non topological properties I performed gene ontology enrichment analysis using ToppGene and as a background set all PSP genes (see section~\ref{sec: gene ontology analysis})\footnote{remove - link across chapters}. If there is no association between centrality and genes being associated with intelligence and educational attainment genes this would be unremarkable if genes with high or low centrality were not in some way distinct (one might expect all phenotypes to show no association). In addition  this may support or refute the variant of the centrality lethality hypothesis of Zotenko and colleagues \cite{zotenko2008hubs} as it applies to humans (section~\ref{sec:Degree and essentialness}).

In order to define a set of extremely central genes (which we need for GO enrichment analysis) we need to decide on a level beyond which values are considered extreme. The use of the top 10 or 20\% is common in the literature.  Lee et al.\cite{lee2013network} used the top 20\% of degree and betweenness centrality (section~\ref{sec:biomedical significance of centrality measures}).  Hsing et al.\cite{hsing2008use} report the use of gene ontology terms to predict if proteins are hubs using boosting trees. To create their training set they used the 90th centile of the degree distribution as it represented an inflection point in the cumulative distribution.


Given contradictory findings in the literature on the nature of central nodes in model organisms such as budding yeast, the nature and function of high degree or high centrality nodes in the synaptome is, to my knowledge, an open question. In addition low degree or centrality nodes have been neglected.
%todo{this is methods/intro}.






To calculate gene ontology enrichment I have used the ToppGene platform\cite{chen2009toppgene} using Entrez ID to identify nodes. Enrichment was carried out using the PSP as the background set of genes, as the central genes can only come from the PSP. This is in contrast to testing GO enrichment for significant intelligence and educational attainment GWA genes (where they were not constrained to be in the PSP - see section~\ref{sec:Methods gene ontology centrality}. The ToppGene report contains the results for all Gene Ontology clades and gene-ontology mappings resulting in easier and more accurate data storage and parsing. 

The methods for ToppGene have been described in section~\ref{sec:ToppGene GO enrichment}. The API does not support the use of a custom background set (the PSP)\, and so the web interface was used\footnote{\url{https://toppgene.cchmc.org/API/}}. 

The top and bottom 10\% of the centrality distributions were used for ontology enrichment. In some cases this results in more  than 10\% of the genes in the PSP (for example where over 10\% have degree 1). The choice of 10 or 20\% to represent extreme centralities is common in the literature\cite{oldham2019consistency}\cite{hsing2008use}. Gene ontology enrichment was carried out using the genes with the  highest and lowest 10\% of values for degree, betweenness centrality, eigenvector centrality, closeness centrality, transitivity and k-coreness. As several of the centralities have a right skewed distribution, low centrality measures may be less useful to discriminate between nodes and it may be most useful  in kcore, transitivity and closeness.

Local transitivity was calculated using NaN for isolates\footnote{as otherwise  there is no way to distinguish the very different topologies implied by the transitivity of a star graph or an isolated node}. Centrality measures were calculated using igraph for R (section~\ref{sec:calculating centralities}). The text output from ToppGene is available in the supplementary material.





\subsection{Disease term enrichment - DisGenNet}


To test whether central nodes are associated with specific disorders or groups of disorders I used DisGeNET v7.0 accessed using the disgenet2R package for R\cite{pinero2020disgenet}. DisGeNet is a comprehensive database of disease gene associations integrating data from a variety of sources\cite{pinero2016disgenet}. The R package remains under active development and changed in the course of writing the thesis. I have therefore used ToppGene to query the DisGeNet database with the PSP as background set as this platform is relatively stable.

Enrichment was carried out in the same manner as for gene ontology enrichment, using the top and bottom 10\% of genes in each centrality measure. The disease enrichment of all PSP genes is shown in table~\ref{tab:disgen_toppgene}.

Specific diseases were identified using Universal Medical Language System (UMLS) Concept Unique Identifiers (CUI)\cite{campbell1998representing}. Central nervous system and neuropsychiatric disorders were identified using the disease codes in the supplemental material to generate enrichment results for all diseases and neuropsychiatric/central nervous system only.



% latex table generated in R 3.6.3 by xtable 1.8-4 package
% Mon Mar  1 17:22:01 2021
\begin{table}[ht]
\centering
\setlength{\extrarowheight}{2pt}
\begin{tabular}{lllllll}
  \toprule
 & ID & Name & $p$ & $q$ FDR BH & PSP & Genome \\ 
  \midrule
1 & C0524851 & Neurodegenerative Disorders & $1.25 \times 10^{-54}$ & $2.47 \times 10^{-50}$ & 484 & 1480 \\ 
  2 & C3714756 & Intellectual Disability & $2.98 \times 10^{-50}$ & $2.94 \times 10^{-46}$ & 412 & 1219 \\ 
  3 & C0014544 & Epilepsy & $3.03 \times 10^{-46}$ & $1.99 \times 10^{-42}$ & 380 & 1124 \\ 
  4 & C0002736 & Amyotrophic Lateral Sclerosis & $9.16 \times 10^{-41}$ & $4.52 \times 10^{-37}$ & 354 & 1069 \\ 
  5 & C0030567 & Parkinson Disease & $2.80 \times 10^{-39}$ & $1.10 \times 10^{-35}$ & 548 & 1952 \\ 
  6 & C0338656 & Impaired cognition & $1.39 \times 10^{-37}$ & $4.59 \times 10^{-34}$ & 415 & 1368 \\ 
  7 & C0036572 & Seizures & $2.82 \times 10^{-36}$ & $7.96 \times 10^{-33}$ & 339 & 1052 \\ 
  8 & C0557874 & Global developmental delay & $2.37 \times 10^{-33}$ & $5.84 \times 10^{-30}$ & 224 & 608 \\ 
  9 & C0036341 & Schizophrenia & $5.08 \times 10^{-33}$ & $1.11 \times 10^{-29}$ & 291 & 883 \\ 
  10 & C0004352 & Autistic Disorder & $2.03 \times 10^{-32}$ & $4.01 \times 10^{-29}$ & 285 & 864 \\ 
   \bottomrule
\end{tabular}
\caption[Ten most enriched disease terms in the post synaptic proteome (PSP)]{Ten most enriched disease terms in the PSP using ToppGene PSP}
\tiny\url{source('~/RProjects/disgen2/R/toppgene_all_PSP/print_table_toppgene.R')}
\label{tab:disgen_toppgene}
\end{table}

\subsection{Methods Murine LTP and centrality}
\label{sec:methods murine LTP and centrality}
Long term potentiation (LTP) has been associated with learning in animals, and so following the genes to cognition approach determined whether murine genes associated with LTP had high or low centrality.



Murine genes associated with LTP were identified using Mammalian Phenotype (MP)\footnote{\url{https://www.ebi.ac.uk/ols/ontologies/mp}} identifiers\cite{bult2019mouse} and downloaded from ToppGene\cite{chen2009toppgene}. Mean centrality measures were calculated for murine LTP genes.

Potential Murine cognition gene sets were identified by manually searching the top 50 enriched mouse phenotypes for terms related to learning, potentiation or cognition. Ten particularly relevant gene sets were identified and are listed in table~\ref{tab:mouse_learning}\footnote{\url{https://hpo.jax.org/app/download/annotation}}.


\begin{table}[]
    \centering
    \setlength{\extrarowheight}{2pt}
    \begin{tabular}{lllll}
    \toprule
    Rank & ID & Name & Genes PSP & Genes total\\
    \midrule
        8 & MP:0002063 & abnormal learning/memory/conditioning & 304 & 742  \\
        10 & MP:0014114 & abnormal cognition & 304 & 744 \\
        11 & MP:0002207 & abnormal long term potentiation & 142 & 258 \\
        20 & MP:0001473 & reduced long term potentiation & 100 & 172 \\
        25 & MP:0001898 & abnormal long term depression & 66 & 102 \\
        29 & MP:0002062 & abnormal associative learning & 150 & 339 \\
        30 & MP:0001469 & abnormal contextual conditioning & 95 & 181 \\
        31 & MP:0001468 & abnormal temporal memory & 107 & 215\\
        44 & MP:0001463 & abnormal spatial learning & 121 & 272\\
        46 & MP:0004859 &abnormal synaptic plasticity & 50 & 77 \\
        \bottomrule
    \end{tabular}
    \caption[Mammalian phenotypes (MP) associated with learning and long term potentiation]{Mouse phenotypes associated with learning in the top 50 mouse phenotypes found on enrichment analysis of all PSP genes in order to identify potential animal cognitive gene sets for network analysis}
    \label{tab:mouse_learning}
\end{table}

\subsection{Centrality and Essential Genes}






\label{sec:centrality and essential genes}
To test if central genes in the PSP support the centrality-lethality hypothesis ( section~\ref{sec:Degree and essentialness}) in human populations,  and to investigate whether genes associated with differences in intelligence were under purifying selection (as described in \cite{hill2016molecular} section~\ref{sec:Degree and essentialness}), we used the database of essential genes (DEG)\cite{luo2021deg} and the Exac and the Gnomad databases to provide evidence of whether human genes were under selection presure.  

\subsubsection{Methods: Database of Essential Genes}
 \label{sec:Database of essential genes}
 
 %\paragraph{Most recent update}
 
 
 The Database of Essential Genes (DEG) hosted at Tianjin University Bioinformatics (Tubic) centre, records genes found in experiments to be essential for life in a variety of organisms along with details of those experiments. The database was first reported in 2004 and the most recent documented version for which a version number is available is 15.2\footnote{\url{ http://tubic.org/deg_bak/}}\cite{luo2014deg}.
 
An update became available from 1st September 2020 with an increase in the number of organisms in the database (9 to 12), the number of genes reported (34,590 to 43,294) and the number of experiments and publications cited (16 to 20). This is the version used in all analyses presented here\footnote{but also include the results in the supplementary section for the analysis using version 15.2 \url{ http://tubic.org/deg_bak/} DEG 15.2 updated Dec 18 2017\cite{luo2014deg}.}.
 
 In the course of writing this thesis the publication reporting the update became available\cite{luo2021deg} but at the time of writing the links on the website no longer work\footnote{presumably because the site remains under construction, which it has shown signs of since September. The older plain HTML site however no longer provides access either}. I have therefore stored the data in the supplementary electronic data store (SEDS). The database was downloaded on 18th October 2020\footnote{Strontium\url{/home/grant/RProjects/db_essential_genes/deg_annotation_e.csv} \url{-rw-r--r-- 1 grant grant 7937241 Oct 18 12:42 deg_annotation_e.csv}}.
 
Twelve organisms are found in the database, although the majority of genes are human (32,578 of total 43,294 - see table~\ref{tab:organisms_in_database_essential_genes}). A gene appears once in the record each time it appears in one of the twenty studies making up the database; the number of genes appearing in multiple studies is shown in figure~\ref{fig:num_studies_database_essential_genes}. 9,085 unique human genes are reported using gene symbols using the gprofiler conversion tool 9,001 unambiguous unique human genes can be assigned Entrez Ids. The number of genes is larger than the number typically regarded as being essential. 
 
 Recently Zhang et al.\cite{zhao2020misuse} used a combination of sequence and network data, with deep learning and graph embedding features, to predict essential human genes using the DEG. They found 8,256 human genes (identical to DEG 15.2) across 16 studies and used those reported in five or more experiments to accord with the general approximate 10\% estimate of the number of human genes that are essential\cite{wang2015identification}. 
 
%   I used the complete list of genes in the database of essential genes and the 2181 genes are in five or more studies and compared the numbers found in the PSP with those not in the PSP. For those in the PSP I compared the centrality measures for essential genes and non essential genes using both all genes and those appearing in five or more studies. 
 
 I calculated the number of essential genes found in the PSP compared to the rest of the genome (Genes with loci in NCBI 37.3) using Fisher's exact test\footnote{\url{source('~/RProjects/db_essential_genes/R/get_essential_genes/db_2020/findforfive/get_db_essential_greater_five_results_xtable.R')}\url{source('~/RProjects/db_essential_genes/R/get_essential_genes/db_2020/findforfive/get_db_essential_greater_five_results.R')}
 }. I compared the distribution of six centrality measures used in this section between essential genes and non essential genes using both parametric and non parametric tests of significance. I have carried out these calculations for all genes appearing in the essential list\footnote{\url{source('~/RProjects/db_essential_genes/R/get_essential_genes/db_2020/findforfive/get_db_essential_greater_five_results_centrality.R')}
 } and the 2,181 genes found in five or more\footnote{\url{source('~/RProjects/db_essential_genes/R/get_essential_genes/db_2020/findforfive/get_db_essential_greater_five_results_centrality.R'))}} studies\footnote{\url{source('~/RProjects/db_essential_genes/R/get_essential_genes/db_2020/get_db_essential_greater_five.R')}}. I have  tested the correlation between centrality scores and the number of experiments a gene has been shown to be essential in.  
 
 

 

 
%  t test essential non essential centrality measures
%  for five or more

 
%  for any in deg
 
 
 
%  up to date DEG 15 now available ref is \cite{luo2021deg} although links now broken

% comment from my git commit
% `infuriatingly you can't get access to the database or the previous one now (the website appears to work but all the links are broken. They have however managed to publish their paper. Pulled update from red machine'
%  [1] "-"                                        "Ergosterol biosynthesis"                  "Protein transport"                       
%  [4] "Amino acid biosynthesis"                  "Component of 90S preribosomal particles"  "Cellular metabolism"                     
%  [7] "Protein translation"                      "Ribosome biogenesis"                      "Cytoskeleton organization and biogenesis"
% [10] "Cell wall organization and biogenesis"    "RNA splicing"                             "Cell redox homeostasis"                  
% [13] "DNA replication"                          "Protein modification"                    
%  Data is available for 9 organisms:  Saccharomyces cerevisiae   ,Caenorhabditis elegans,Arabidopsis thaliana,Danio rerio,Mus musculus,Homo sapiens,
%  Drosophila melanogaster,Aspergillus fumigatus and Schizosaccharomyces pombe 972h-.
 
%  Data on 34590 genes are recorded. 
%  Data on 28286 \todo{is this correct this is more than the number of protein coding genes?} human genes are recorded. 8256 unique gene symbols are found for human genes. 8214 unique hgnc id numbers  , ENSG 12947 Human genes, 8168 unique proteins  are identified by UniProtKB  \footnote{\url{source('~/RProjects/paper_xls_latex/R/essentail_genes/load_essential_genes.R')}}.
 
%  The pattern of duplicated data was complex for example 118 HGNC id were unrecorded (recorded as "-", the next commonest was one with 15 duplicates). Most of the 
 
%  Entrez id were obtained using the module \url{convert_gmx_private}\footnote{ at \url{/home/grant/PycharmProjects/convert_gmx_private}}.
 
%  This uses downloaded gene info from entrez gene to find first the symbol and if that is missing to search through synonyms and supply the matching entrez id. It returns a list of items not found. 

 
 
 
 
 
 
%  43294 records. \footnote{\url{www.essentialgene.org}} 15 organismsThe organisms appearing in the database are shown in table~\ref{tab:organisms_in_database_essential_genes}.. Update 1st September 2020. I could not ascertain the version number the previous numbered version 15.2 was updated in 2017 and there are more organisms and genes in this entry (43294 vs 34590) 9 organisms in previous. Downloaded from \url{http://tubic.tju.edu.cn/deg_test/public/download/deg_annotation_e.csv.zip} where Tubic is Tianjin University Bioinformatics Centre \url{http://tubic.org/}. Their home page links to the 1st September update and gives a publication of an update to DEG 15 in Nucleic Acids Research 2020 which  is not yet found on PubMed. The current site has been obviously under construction in the course of September 2020. There appear to be valid links to data on the experiments making up the database and some data was not available at the beginning of September so I assume it has been under construction. 
 
%  I had previously carried out the analysis with \url{ http://tubic.org/deg_bak/} DEG 15.2 updated Dec 18 2017\cite{luo2014deg}. This entry had 9 organisms and 34590 human genes of which 8256 unique human genes could be identified. The previous version had 28286 identifiable human genes.
 
%  Recently Zhang et al. \cite{zhao2020misuse} used a combination of sequence and network features to predict using deep learning and graph embedding essential human genes using the DEG. They found 8256 human genes (identical to DEG 15.2) and used those reported in five or more experiments to accord with the general approximate 10\% figure of essentialness\cite{wang2015identification}. There were 16 essential gene datasets in their study.
 
%  Current version contains 20 studies including \todo{complete}.
 
%  I used the complete list of genes in the database of essential genes and the 2181 genes are in five or more studies and compared the numbers found in the PSP with those not in the PSP. The number of genes appearing in multiple studies is shown in figure~\ref{fig:num_studies_database_essential_genes}. For those in the PSP I compared the centrality measures for essential genes and non essential genes using both all genes and those appearing in five or more studies\footnote{\url{source('~/RProjects/db_essential_genes/R/get_essential_genes/db_2020/get_db_essential_greater_five.R')}}. 
 % latex table generated in R 3.6.3 by xtable 1.8-4 package
% Sat Oct 24 11:51:37 2020
\begin{table}[ht]
\centering
\begin{tabular}{rlc}
  \hline
 & Species & Number of Entries \\ 
  \hline
1 & Homo sapiens & 32578 \\ 
  2 & Plasmodium falciparum & 2680 \\ 
  3 & Mus musculus & 2524 \\ 
  4 & Schizosaccharomyces pombe 972h- & 1260 \\ 
  5 & Saccharomyces cerevisiae & 1110 \\ 
  6 & Bombyx mori & 1006 \\ 
  7 & Arabidopsis thaliana & 356 \\ 
  8 & Drosophila melanogaster & 339 \\ 
  9 & Caenorhabditis elegans & 338 \\ 
  10 & Danio rerio & 315 \\ 
  11 & Komagataella phaffii GS115 chromosome 1 & 214 \\ 
  12 & Komagataella phaffii GS115 chromosome 2 & 201 \\ 
  13 & Komagataella phaffii GS115 chromosome 3 & 192 \\ 
  14 & Komagataella phaffii GS115 chromosome 4 & 146 \\ 
  15 & Aspergillus fumigatus &  35 \\ 
   \hline
\end{tabular}
\caption[Organisms in the Database of Essential Genes (DEG) 2020 update]{Organisms represented in Database of Essential genes 2020 updates. Number of entries is not equal to number of unique genes. All four of the chromosomes of the yeast K.phaffii (Pichia Pastoris)\cite{bernauer2021komagataella} have separate species entries perhaps because of its centrality in pharmaceutical processes. There are therefore a total of twelve organisms in the DEG. \url{source('~/RProjects/db_essential_genes/R/get_essential_genes/db_2020/organism_table.R')}}
\label{tab:organisms_in_database_essential_genes}
\end{table}
 
\begin{figure}
    \centering
    \includegraphics[width=\textwidth]{images/chapter3/ggplot2/essential_genes/Rplot_num_studies_in_DEG_corrected.png}
    \caption{Number of datasets genes in Database of Essential Genes appear in. Twenty nine genes appearing in forty two studies for which consistent entrez ID could not be established were removed. }
    \label{fig:num_studies_database_essential_genes}
\end{figure}
\subsection{Loss of function and evolutionary pressure ExAC and gnomad}
\label{sec:methods exac and gnomad}
The Exome Aggregation Consortiuim (ExAC) database collected the data from the exomes of over 60,000 individuals to provide a measure of the tolerance of mutation in genes and genetic loci\cite{lek2016analysis}. Comparing the number of loss of function mutations to the the number of harmless synonymous mutations provides a measure of how intolerant to mutation a gene or genetic region is in a population. The consortium provide a statistic, the probability of loss of function intolerance (pLI), that measures how much purifying selection a gene is under\footnote{The data was downloaded from \url{"https://storage.googleapis.com/gnomad-public/legacy/exac_browser/forweb_cleaned_exac_r03_march16_z_data_pLI_CNV-final.txt.gz"} (valid on 16.06)}, taking into account confounding factors such as gene size. 

To test the hypothesis that central genes are more likely to be essential and, by extension, intolerant of mutation, I measured the correlation coefficient between pLI and centrality measures such as degree. Likewise, to determine whether genes associated with intelligence and educational attainment were more likely to be loss of function intolerant, I calculated the correlation between gene z score and the -log10 transform of the gene p value. 

 Very recently a greatly enlarged successor the Genome Aggregation Database (gnomAD) has become available and I have repeated the analysis using this which I report here. It aggregates 15,708 whole genomes and 125,748 exomes across 141,456 subjects. This increases the number of small genetic variations reported to 241 million compared with 7.4 million by ExAC and thus a more complete picture of genetic constraint can be obtained\cite{karczewski2020mutational}. Educational attainment (and schizophrenia) have been reported to be the most enriched traits with constraint\cite{karczewski2020mutational}\footnote{specifically decile of LOEUF `90\% upper bound of this confidence interval, which we term the loss-of-function observed/expected upper bound fraction (LOEUF) (Extended Data Fig. 7b, c),\cite{karczewski2020mutational}'} but the level of constraint in the PSP and the association between constraint and network features are unreported.  

Gnomad v2.1.1 loss of function metrics by gene were downloaded from the consortium's cloud storage url\footnote{\url{https://storage.googleapis.com/gnomad-public/release/2.1.1/constraint/gnomad.v2.1.1.lof_metrics.by_gene.txt.bgz}}. This is the 6th of March 2019 release and contains loss of function summary metrics by gene\footnote{gnomad v3 is available but does not have these summary data}.

The file contained 19,704 rows and 77 columns. Three gene identifiers are provided: Ensembl Gene ID, Ensembl transcript ID and HGNC Symbol with 19,658 unique Gene Symbols and 19,704 unique Ensembl IDs.

Translating Ensembl ID to Entrez ID does not always result in a one to one mapping. I tried Ensembl ID and Gene Symbol to determine which performed best and used both gprofiler2\footnote{\url{https://CRAN.R-project.org/package=gprofiler2 }} R interface (v0.2) to the g:Profiler toolset\cite{raudvere2019g} and the ToppGene API\footnote{See discussion I had by this stage largely stopped using flat files from EntrezGene and BioMart which I had used at the start because of the good performance of both gprofiler2 and the ToppGene API.}. The best result was obtained using Ensembl ID and gprofiler2 resulting in 18,782 unique entries, 46 ambiguous/duplicated which were removed, and 983 which could not be mapped to an Entrez Gene ID\footnote{ \url{source('~/RProjects/db_essential_genes/R/gnomad/download_gnomad.R')}}.

The ToppGene translation API using a custom python script\footnote{\url{/home/grant/RProjects/db_essential_genes/python/toppgene_api.ipynb}} resulted in 18,604 unambigous, unique Entrez ID terms.  Both translations are available in csv format in the supplementary material\footnote{Using Ensebml ID to translate resulted in 19881 rows and 1005 duplicated entrez id.Using Gene ID to translate resulted in 20040 1896 duplicated},\footnote{Using the ToppGene API you get 18604 terms \url{/home/grant/RProjects/db_essential_genes/python/toppgene_api.ipynb}}.

  1,185 Ensembl ID could not be translated to Entrez ID and were excluded. Forty eight duplicate Entrez ID were generated in the translation process and these too were excluded. The data cleaning is documented in an R markdown file\footnote{\url{~/RProjects/db_essential_genes/R/gnomad/mkd_download_gnomad.Rmd}. 18,717 genes have an Entrez ID associated with a Gnomad record. Of these 338 have NaN pLI values and were excluded resulting in 18,379 records. 3,407 PSP gnomad records are found (98.6\% of PSP). Translation was stored as RDS and plain text files. 

}.  % Using script download gnomad \url{source('~/RProjects/db_essential_genes/R/gnomad/download_gnomad.R')}. I used the probability of loss of function intolerance (pLI) to measure of selection pressure on a gene. pLI is a probability and has a range 0-1. A pLI of 1 corresponds to the gene being totally loss of function intolerant (high pLI - more conserved gene)

% See also section~\ref{sec: exac methods from chapter community detection}
% The original results using ExAC are in the supplementary material section~\ref{sec:Processing and cleaning Exac data for PSP}




% \footnote{g profiler using Ensembl ID
% 983 none
% 46 duplicated}.
% df\_out3 = 18782



% \subsection{Gnomad}

% \textcolor{red}{check this again}

% 19704 rows and 77 columns. (CHECK) Ensembl ID was translated to Entrez ID using gprofiler 2. 1185 Ensembl ID were not identified and excluded. 48 Duplicate entrez id were generated and excluded. 

% This resulted in 18551 genes with 3417 in the PSP (correct) with 4 missing values for pLI. (looks like it is 18782 and only 983 none).

% There are 3413 PSP genes with valid pLI and 15030 other genes. 

% 507 of the gnomad v2.1.1 pLI values were NA - 19197 valid pLI values

% 18443 valid pLI with translation to entrez (96.07\% of available data)

% 3413/3457 (98.7\%) of PSP data.


% \textcolor{red}{checking this again}

% \url{source('~/RProjects/db_essential_genes/R/gnomad/download_gnomad.R')}

% gnomad v2.1.1. lof metrics by gene file

% 19704 rows and 77 columns
% Gene ID column is ENSG
% Gene is HGNC symbol




% 19704 rows of downloaded document 77 columns

% after translation using gprofiler ENSG to entrez and removal of duplicated ids (ie ambigious and NaN) see
% markdown file \url{~/RProjects/db_essential_genes/R/gnomad/mkd_download_gnomad.Rmd}

% 18717 entrez ensg pairs
% inner join (dplyr) with no data loss (ie 18717 out) on ENSG

% translated table


% latex table generated in R 3.6.1 by xtable 1.8-4 package
% Sat Oct 17 16:31:15 2020




\subsection{Cohorts and samples}
\label{Centrality:cohorts and samples}

The population data used to assign significance values to the association of genes and intelligence and educational attainment and to test the association of theses genes with centrality measures are described in sections~\ref{sec:cohorts from paper section} and \ref{sec:samples from paper section}.

Gene level statistics were calculated using MAGMA as outlined in section~\ref{sec:MAGMA gene level methods}.




%%%%%%%%%%%%%%%%%%%%%%%%%%%    RESULTS  %%%%%%%%%%%%%%%%%%%%%%%%%%%%%%%%%%%%%%%%%%%%%%%%%%%%%

%%%%%%%%%%%%%%%%%%%%%%%%%%%    RESULTS  %%%%%%%%%%%%%%%%%%%%%%%%%%%%%%%%%%%%%%%%%%%%%%%%%%%%%

%%%%%%%%%%%%%%%%%%%%%%%%%%%    RESULTS  %%%%%%%%%%%%%%%%%%%%%%%%%%%%%%%%%%%%%%%%%%%%%%%%%%%%%

%%%%%%%%%%%%%%%%%%%%%%%%%%%    RESULTS  %%%%%%%%%%%%%%%%%%%%%%%%%%%%%%%%%%%%%%%%%%%%%%%%%%%%%

%%%%%%%%%%%%%%%%%%%%%%%%%%%    RESULTS  %%%%%%%%%%%%%%%%%%%%%%%%%%%%%%%%%%%%%%%%%%%%%%%%%%%%%

%%%%%%%%%%%%%%%%%%%%%%%%%%%    RESULTS  %%%%%%%%%%%%%%%%%%%%%%%%%%%%%%%%%%%%%%%%%%%%%%%%%%%%%

%%%%%%%%%%%%%%%%%%%%%%%%%%%    RESULTS  %%%%%%%%%%%%%%%%%%%%%%%%%%%%%%%%%%%%%%%%%%%%%%%%%%%%%





\clearpage

\section{ Results Summary of centrality measures}
     Degree, eigenvector centrality and betweenness centrality have  heavy right tails with the mean greater than the median. The mean degree for the PSP network, was 17.6 and its median 8. A histogram showing the distribution of centrality measures is shown in figure~\ref{fig:multiplot centrality histograms}. None of the distributions resemble a Gaussian distribution other than the  closeness centrality. The K core index has a more uniform distribution (figure~\ref{fig:Kcore_histogram}) with more nodes in the lowest kcores \cite{dorogovtsev2006k}. The U shaped kcore distribution of some complex networks is note seen\cite{burleson2020k} 
 . The transitivity has a marked peak at zero as discussed previously. 
    
Summary statistics for the centrality measures are shown in table \ref{Table:Summary of centrality measures1}. The local transitivity is reported first with isolates (nodes with degree $k=1$\footnote{there are no nodes with $k=0$}) set to zero ($C_i^0$ - transitivity0) and second  ($C_{i}^{NaN}$ - TransitivityNA) where nodes of degree one ($k=1$) have an undefined transitivity and are removed from caclulations of summary statistics or correlation\footnote{because for a node with degree=1 the denominator $\frac{k_v(k_v)-1}{2}$ in equation~\ref{eq:local_transitvity}  would be 0)}.  There were 304 nodes of degree one ($k=1$). 

% Below this is reported the alternate method also implemented in igraph (isolates="zero") of setting nodes with degree 1 to have a local transitivity of 0 ($C_i^0$). \textcolor{red}{no you are not just now} Where not explicitly stated in subsequent results I am using the default method (remove nodes with $k=0$ from the calculations).

The eigenvector centrality has been scaled to have a maximum value of one and local transitivity (local clustering coefficient) lies in the range 0-1. The betweenness is shown both as the number of shortest paths going through a node and  scaled in the range 0-1.
    Closeness centrality does not follow a power law distribution \cite{newman2018networks}\footnote{p331-332} as it has a limited distribution as the reciprocal of the shortest path length from a node to all other nodes.   
    


 
 
 \begin{figure}
     \centering
     \includegraphics[width=\textwidth]{images/Rplot01_kcore_hist.png}
     \caption{Histogram of shell index of vertices in calculation of kcore. Shell index is the maximum kcore a vertex is a member of and not a member of shell index n+1. }
     \label{fig:Kcore_histogram}
 \end{figure}
 

\subsection{Testing the normality of centrality measures}
% latex table generated in R 3.6.3 by xtable 1.8-4 package
% Sat Jun 13 15:54:02 2020

Inspection of the histogram of the distribution of centrality measures in figure~\ref{fig:multiplot centrality histograms} suggest  only the closeness centrality is approximately normal and degree, betweennness and eigenvector centrality are heavy tailed. 

% All of the centrality measures appear to vary from a normal distribution using the Shaprio-Wilk test in R and in accordance with their plots(table~\ref{tab:Test for normality (Shapiro-Wilk) for centrality measures of PSP}) . Spearman's test of correlation will therefore be used for correlation with centrality measures.



% \begin{figure}
%     \centering
%     \includegraphics[width=\textwidth]{images/Rplot_histogram_centralities.png}
%     \caption{Multiplot of histogram of centrality measures. Dashed line is median}
%     \label{fig:multiplot centralities}
% \end{figure}
To determine the most appropriate method to test the correlation of centrality measures the Shapiro-Wilk test was performed using R and is shown along with the results of the Kolmogorov-Smirnov test, calculating the statistic as the distance between the empirical distribution and the normal distribution with identical mean and standard deviation is shown in table~\ref{tab:Test for normality (Shapiro-Wilk) for centrality measures of PSPcombinedKS}.

These confirm that the centrality measures are not normally distributed, and a non parametric test of correlation (Spearman's rho) was used \footnote{\url{source('~/RProjects/paper_xls_latex/R/centrality_summary/centrality_summary_latex.R')}} in accordance with some of the recent literature \cite{oldham2019consistency}.


% latex table generated in R 3.6.2 by xtable 1.8-4 package
% Mon Feb 10 14:17:12 2020
% \begin{table}[ht]
% \centering
% \begin{tabular}{rrrrrrr}
%   \hline
%  & Minimum & 1st Quartile & Median & Mean & 3rd Quartile & Maximum \\ 
%   \hline
% Degree & $1 $ & $4 $ & $8 $ & $17.64$  & $19$ & $535$ \\ 
%   Eigenvector & $1.821 \times 10^{-6}$ & $8.829 \times 10^{-3}$ & $2.194 \times 10^{-2}$ & $4.793 \times 10^{-2}$ & $5.322 \times 10^{-2}$ & $1 $ \\ 
%   Betweenness & $0 $ & $43.29 $ & $317 $ & $3421.1$ & $1571.6$& $6.447 \times 10^{5}$ \\ 
%   Transitivity & $0 $ & $4.376 \times 10^{-2}$ & $1.290 \times 10^{-1}$ & $1.714 \times 10^{-1}$ & $2.381 \times 10^{-1}$ & $1 $ \\ 
%   Closeness & $5.702 \times 10^{-5}$ & $9.217 \times 10^{-5}$ & $9.879 \times 10^{-5}$ & $9.845 \times 10^{-5}$ & $1.058 \times 10^{-4}$ & $1.399 \times 10^{-4}$ \\ 
%   \hline
% \end{tabular}
% \caption{Summary of centrality measures} 
% \label{Table:Summary of centrality measures}
% \end{table}
% Table created by stargazer v.5.2.2 by Marek Hlavac, Harvard University. E-mail: hlavac at fas.harvard.edu
% Date and time: Tue, Nov 10, 2020 - 16:35:43





% latex table generated in R 3.6.3 by xtable 1.8-4 package
% Sun Mar  7 13:07:50 2021
\begin{table}[ht]
\centering

\begin{adjustbox}{width=\textwidth}
    \setlength{\extrarowheight}{2pt}
\begin{tabular}{rrrrrrrrr}
  \toprule
 & $k$ & Bet & Bet\textsubscript{0-1} & Eigen & Closeness & $C^0_i$ & $C^{NaN}_i$ & kcore \\ 
  \midrule
Min. & 1.0 & 0.0 & $0.0$ & $1.82 \times 10^{-6}$ & $5.70 \times 10^{-5}$ & 0.00 & 0.00 & 1.0 \\ 
  1Q & 4.0 & 43.3 & $7.25 \times 10^{-6}$ & $8.83 \times 10^{-3}$ & $9.22 \times 10^{-5}$ & 0.00 & 0.04 & 3.0 \\ 
  Median & 8.0 & 317.0 & $5.31 \times 10^{-5}$ & $2.19 \times 10^{-2}$ & $9.88 \times 10^{-5}$ & 0.11 & 0.13 & 8.0 \\ 
  Mean & 17.6 & 3421.1 & $5.73 \times 10^{-4}$ & $4.79 \times 10^{-2}$ & $9.85 \times 10^{-5}$ & 0.16 & 0.17 & 9.2 \\ 
  3Q. & 19.0 & 1571.6 & $2.63 \times 10^{-4}$ & $5.32 \times 10^{-2}$ & $1.06 \times 10^{-4}$ & 0.22 & 0.24 & 13.0 \\ 
  Max. & 535.0 & 644670.7 & $1.08 \times 10^{-1}$ & $1.0 $ & $1.40 \times 10^{-4}$ & 1.00 & 1.00 & 24.0 \\ 
   \bottomrule
\end{tabular}
\end{adjustbox}
\caption[Summary centrality statistics]{Summary statistics for centrality measures in the PSP. 1Q 3Q = first and third quartiles. Bet\textsubscript{0-1} is betweenness scaled by maximum possible betweenness\footnote{$\frac{(N-1)(N-2)}{2}$}. $C^{NaN}_i$= TransitivityNA, isolates set to NA. Number of NA: 304. The value of transitivityNA ($C_i^{NaN}$) is slightly greater than transitivity with isolates set to zero (Transitivity0 $C_i^0$)}
\tiny\url{https://github.com/14760/thesis/blob/main/R/chapter3/centrality/summary_centrality_range_bet.R}
\label{Table:Summary of centrality measures1}
\end{table}







%\todo{density plot of tranitivityNA and transitivity0 showing probability around 0}
\begin{figure}
    \centering
    %\includegraphics[width=\textwidth]{images/chapter3/ggplot2/Rplot_histogram+_centrality_multiplot.png}
     \includegraphics[width=\textwidth]{images/chapter3/ggplot2/theme/multiplot_centrality/centrality_histogram_multiplot.png}
    \caption[Multiplot histogram of centrality measures]{Multiplot histogram of centrality measures. Bin size kept constant at 50 other than barplot for kcoreness. Median is red dashed line, mean is blue dashed line. Plots are of (clockwise from top right: closeness centrality, local transitivity with isolates removed (transitivityNA), kcoreness, betweenness centrality, eigenvector centrality and degree centrality. Degree, eigenvector and betweenness centrality have particularly marked heavy tailed distributions suggestive of a power law. The closeness centrality looks similar to a normal distribution.}
    \tiny\url{https://github.com/14760/thesis/blob/main/R/chapter3/histogram_centralities/multiplot_centralities.R}
    %\url{source('~/RProjects/chapter3/R/essential_genes/histogram_centralities/multiplot_centralities.R')}
    \label{fig:multiplot centrality histograms}
\end{figure}

%\todo{Global transitivity and compare}
%\todo{query pull out the global measures eg degree assortativity, transitivity}

% % Shapiro Wilk table
% \begin{table}[h]
%     \centering
%     \begin{tabular}{c|c|c}
%       Centrality measure  &  W & p\\
%       \hline
       
%       Degree  & 0.426 & $<2.2 \times 10^{-16}$ \\
%       Eigenvector &0.561  & $<2.2 \times 10^{-16}$ \\
%       Betweenness &0.1228& $<2.2 \times 10^{-16}$ \\
%       Transitivity &0.782 & $<2.2 \times 10^{-16}$\\
%       Closeness &0.9953& $<4.686 \times 10^{-09}$\\ 
%     \end{tabular}
%     \caption{Test for normality (Shapiro-Wilk) for centrality measures of PSP\textcolor{red}{add kcore}}
%     \label{tab:Test for normality (Shapiro-Wilk) for centrality measures of PSP}
% \end{table}

% latex table generated in R 3.6.3 by xtable 1.8-4 package
% Sun Feb 21 09:31:11 2021

\begin{table}[ht]
\centering
\begin{tabular}{lllll}
  \toprule
 & W & p value (SW) &D & p value (KS) \\ 
  \midrule
Degree & 0.425 & $2.52 \times 10^{-74}$& 0.311 & $<2.2 \times 10^{-16}$  \\ 
 Betweenness & 0.123 & $2.76 \times 10^{-83}$  & 0.434  & $<2.2 \times 10^{-16}$ \\ 
 Eigenvector & 0.561 & $6.55 \times 10^{-69}$&   0.269&$<2.2 \times 10^{-16}$ \\ 
 Closeness & 0.995 & $4.69 \times 10^{-9}$ &  0.046&$7.174 \times 10^{-7}$  \\ 
 Transitivity0 & 0.775 & $2.65 \times 10^{-56}$&0.198 &$<2.2 \times 10^{-16}$ \\ 
 Kcoreness & 0.911 & $4.32 \times 10^{-41}$& 0.126 &$<2.2 \times 10^{-16}$ \\ 
   \bottomrule
\end{tabular}
\caption[Shapiro-Wilk and Kolmogorov-Smirnov tests for centrality measures in the PSP]{Test for normality (Shapiro-Wilk and Kolmogorov-Smirnov) for centrality measures of PSP.}
\label{tab:Test for normality (Shapiro-Wilk) for centrality measures of PSPcombinedKS}
\end{table}

\section{Correlation of centrality measures}
\label{sec:correlation of centrality measures results section}
The pairwise relation between centrality measures is shown in the scatterplot in figure~\ref{tab:Correlation of centrality measures 2 transitvities} and figure~\ref{fig:Correlation plot centrality performance analytics}.

%SUPPLE\ref{fig:scatter plot of multiple centralities gg ally}. 






\subsection{Correlation of centrality measures}

All of the centrality measures appear significantly correlated except between transitivity and betweenness centrality (table~\ref{tab:Correlation of centrality measures 2 transitvities}). The transitivity has been viewed as an inverse betweenness (section~\ref{sec:local clustering coefficient}) and if so it is more marked with isolates set to NA when the distribution is not as dominated by zero values. Spearman's rank correlation with pairwise complete observations included was used to calculate the correlation of centrality measures. Transitivity is most weakly correlated with other measures\footnote{ \url{source('~/RProjects/paper_xls_output/R/make_df_correlation_between_centrality_measures.R')}}.

The global positive correlation between transitivity and centrality measures is largely caused by nodes of transitivity 0 or 1 (see section~\ref{sec:results_global_clustering_coefficient}) particularly when isolates are treated as 0 ($C_{WS}0$) as the lowest degree nodes ($k=1$) have zero transitivity by definition.





% latex table generated in R 3.6.3 by xtable 1.8-4 package
% Tue Oct 20 16:40:12 2020
\begin{table}[ht]
\centering
\setlength{\extrarowheight}{2pt}
\begin{adjustbox}{width=\textwidth}
\begin{tabular}{rrrrrrrr}
  \toprule
 & degree & betweenness & eigenvector & closeness & transitivityNA & transitivity0 & kcoreness \\ 
  \midrule
degree & 1.000 & 0.902 & 0.879 & 0.851 & 0.196 & 0.375 & 0.986 \\ 
  betweenness & 0.902 & 1.000 & 0.762 & 0.782 & -0.002 & 0.209 & 0.854 \\ 
  eigenvector & 0.879 & 0.762 & 1.000 & 0.969 & 0.330 & 0.462 & 0.906 \\ 
  closeness & 0.851 & 0.782 & 0.969 & 1.000 & 0.292 & 0.430 & 0.872 \\ 
  transitivityNA & 0.196 & -0.002 & 0.330 & 0.292 & 1.000 & 1.000 & 0.265 \\ 
  transitivity0 & 0.375 & 0.209 & 0.462 & 0.430 & 1.000 & 1.000 & 0.428 \\ 
  kcoreness & 0.986 & 0.854 & 0.906 & 0.872 & 0.265 & 0.428 & 1.000 \\ 
   \bottomrule
 
   
\end{tabular}
\end{adjustbox}
\caption[Correlation of centrality measures]{Correlation of centrality measures. Spearman's rho including only pairwise complete. Details of measures in table~\ref{tab:calculating centrality metrics}.  All $p<2.2 \times 10^{-16}$ other than correlation between transitivity NA and betweenness $p$= 0.901 using package \texttt{Hmisc} and checking with \texttt{R} stats \texttt{cor.test}. } 
\tiny Code at \url{source('~/RProjects/db_essential_genes/R/get_corr/cor_centrality_measures.R')}
\tiny Code at same address in object t \url{source('~/RProjects/db_essential_genes/R/get_corr/cor_centrality_measures_performance.R')}
\label{tab:Correlation of centrality measures 2 transitvities}
\end{table}

% % latex table generated in R 3.6.1 by xtable 1.8-4 package
% % Sun Mar  7 16:10:40 2021
% \begin{table}[ht]
% \centering
% \begin{tabular}{rllr}
%   \hline
%  & Centrality 1 & Centrality 2 & Cor rho \\ 
%   \hline
% 1 & transitivity0 & transitivityNA & 1.000 \\ 
%   2 & kcoreness & degree & 0.986 \\ 
%   3 & closeness & eigenvector & 0.969 \\ 
%   4 & kcoreness & eigenvector & 0.906 \\ 
%   5 & betweenness & degree & 0.902 \\ 
%   6 & eigenvector & degree & 0.879 \\ 
%   7 & kcoreness & closeness & 0.872 \\ 
%   8 & kcoreness & betweenness & 0.854 \\ 
%   9 & closeness & degree & 0.851 \\ 
%   10 & closeness & betweenness & 0.782 \\ 
%   11 & eigenvector & betweenness & 0.762 \\ 
%   12 & transitivity0 & eigenvector & 0.462 \\ 
%   13 & transitivity0 & closeness & 0.430 \\ 
%   14 & kcoreness & transitivity0 & 0.428 \\ 
%   15 & transitivity0 & degree & 0.375 \\ 
%   16 & transitivityNA & eigenvector & 0.330 \\ 
%   17 & transitivityNA & closeness & 0.292 \\ 
%   18 & kcoreness & transitivityNA & 0.265 \\ 
%   19 & transitivity0 & betweenness & 0.209 \\ 
%   20 & transitivityNA & degree & 0.196 \\ 
%   21 & transitivityNA & betweenness & -0.002 \\ 
%   \hline
% \end{tabular}
% \caption{Correlation of centrality measures. Spearman. Ordered by correlation. Method for missing data include only pairwise complete. TransitivityNA treats isolates as NA and removes from calculation. Transitivity0 uses the option to set isolates to 0\url{source('~/RProjects/db_essential_genes/R/get_corr/cor_centrality_measures_ordered_cortable.R')}} 
% \end{table}



% % latex table generated in R 3.6.1 by xtable 1.8-4 package
% % Sun Mar  7 16:19:10 2021
% \begin{table}[ht]
% \centering
% \begin{tabular}{rrrrrrrr}
%   \hline
%  & degree & betweenness & eigenvector & closeness & transitivityNA & transitivity0 & kcoreness \\ 
%   \hline
% degree & 1.000 & 0.827 & 0.815 & 0.769 & -0.086 & -0.086 & 0.958 \\ 
%   betweenness & 0.827 & 1.000 & 0.598 & 0.650 & -0.315 & -0.315 & 0.707 \\ 
%   eigenvector & 0.815 & 0.598 & 1.000 & 0.938 & 0.174 & 0.174 & 0.881 \\ 
%   closeness & 0.769 & 0.650 & 0.938 & 1.000 & 0.120 & 0.120 & 0.821 \\ 
%   transitivityNA & -0.086 & -0.315 & 0.174 & 0.120 & 1.000 & 1.000 & 0.058 \\ 
%   transitivity0 & -0.086 & -0.315 & 0.174 & 0.120 & 1.000 & 1.000 & 0.058 \\ 
%   kcoreness & 0.958 & 0.707 & 0.881 & 0.821 & 0.058 & 0.058 & 1.000 \\ 
%   \hline
% \end{tabular}
% \caption{Correlation of centrality measures. For nodes with degree five or more Spearman. Method for missing data include only pairwise complete. TransitivityNA treats isolates as NA and removes from calculation. Transitivity0 uses the option to set isolates to 0} 
% \label{tab:Correlation of centrality measures 2 transitvities deg five or more}
% \end{table}






% % latex table generated in R 3.6.1 by xtable 1.8-4 package
% % Sun Mar  7 16:19:10 2021
% \begin{table}[ht]
% \centering
% \begin{tabular}{rllr}
%   \hline
%  & Centrality 1 & Centrality 2 & Cor rho \\ 
%   \hline
% 1 & transitivity0 & transitivityNA & 1.000 \\ 
%   2 & kcoreness & degree & 0.958 \\ 
%   3 & closeness & eigenvector & 0.938 \\ 
%   4 & kcoreness & eigenvector & 0.881 \\ 
%   5 & betweenness & degree & 0.827 \\ 
%   6 & kcoreness & closeness & 0.821 \\ 
%   7 & eigenvector & degree & 0.815 \\ 
%   8 & closeness & degree & 0.769 \\ 
%   9 & kcoreness & betweenness & 0.707 \\ 
%   10 & closeness & betweenness & 0.650 \\ 
%   11 & eigenvector & betweenness & 0.598 \\ 
%   12 & transitivityNA & eigenvector & 0.174 \\ 
%   13 & transitivity0 & eigenvector & 0.174 \\ 
%   14 & transitivityNA & closeness & 0.120 \\ 
%   15 & transitivity0 & closeness & 0.120 \\ 
%   16 & kcoreness & transitivityNA & 0.058 \\ 
%   17 & kcoreness & transitivity0 & 0.058 \\ 
%   18 & transitivityNA & degree & -0.086 \\ 
%   19 & transitivity0 & degree & -0.086 \\ 
%   20 & transitivityNA & betweenness & -0.315 \\ 
%   21 & transitivity0 & betweenness & -0.315 \\ 
%   \hline
% \end{tabular}
% \caption{Correlation of centrality measures. Spearman. Degree five or more. Ordered by correlation. Method for missing data include only pairwise complete. TransitivityNA treats isolates as NA and removes from calculation. Transitivity0 uses the option to set isolates to 0. With degree 5 and over the negative correlation between degree and transitivity is cllearly seen and the role of the betweenness as a form of inverse the role of the betweenness as a form of inverse transitivity\url{source('~/RProjects/db_essential_genes/R/get_corr/cor_centrality_measures_ordered_cortable.R')}} 
% \end{table}



% latex table generated in R 3.6.1 by xtable 1.8-4 package
% Sun Mar  7 16:39:36 2021
% \begin{table}[ht]
% \centering
% \begin{tabular}{rllrrr}
%   \hline
%  & Centrality 1 & Centrality 2 & Cor rho & Cor rho deg gt 5 & Rank 5 \\ 
%   \hline
% 1 & transitivity0 & transitivityNA & 1.000 & 1.000 & 1 \\ 
%   2 & kcoreness & degree & 0.986 & 0.958 & 2 \\ 
%   3 & closeness & eigenvector & 0.969 & 0.938 & 3 \\ 
%   4 & kcoreness & eigenvector & 0.906 & 0.881 & 4 \\ 
%   5 & betweenness & degree & 0.902 & 0.827 & 5 \\ 
%   6 & kcoreness & closeness & 0.872 & 0.821 & 6 \\ 
%   7 & eigenvector & degree & 0.879 & 0.815 & 7 \\ 
%   8 & closeness & degree & 0.851 & 0.769 & 8 \\ 
%   9 & kcoreness & betweenness & 0.854 & 0.707 & 9 \\ 
%   10 & closeness & betweenness & 0.782 & 0.650 & 10 \\ 
%   11 & eigenvector & betweenness & 0.762 & 0.598 & 11 \\ 
%   12 & transitivity0 & eigenvector & 0.462 & 0.174 & 13 \\ 
%   13 & transitivityNA & eigenvector & 0.330 & 0.174 & 12 \\ 
%   14 & transitivity0 & closeness & 0.430 & 0.120 & 15 \\ 
%   15 & transitivityNA & closeness & 0.292 & 0.120 & 14 \\ 
%   16 & kcoreness & transitivity0 & 0.428 & 0.058 & 17 \\ 
%   17 & kcoreness & transitivityNA & 0.265 & 0.058 & 16 \\ 
%   18 & transitivity0 & degree & 0.375 & -0.086 & 19 \\ 
%   19 & transitivityNA & degree & 0.196 & -0.086 & 18 \\ 
%   20 & transitivity0 & betweenness & 0.209 & -0.315 & 21 \\ 
%   21 & transitivityNA & betweenness & -0.002 & -0.315 & 20 \\ 
%   \hline
% \end{tabular}
% \caption{} 
% \end{table}

% latex table generated in R 3.6.1 by xtable 1.8-4 package
% Sun Mar  7 16:48:58 2021



\begin{figure}
    \centering
    \includegraphics[width=\textwidth]{images/chapter3/cor_pairs/analytics/Rplot_analytics_spearman_cor.png}
    \caption{Correlation plot of centrality measures using package Performance Analytics. Deg = degree, eig= eigenvector centrality, clo = closeness centrality, bet = betweenness centrality, tra = local transitivity, kcore = kcoreness.}
    \tiny\url{source('~/RProjects/db_essential_genes/R/get_corr/cor_centrality_measures_performance.R')}
    \label{fig:Correlation plot centrality performance analytics}
\end{figure}





% (I feel there is a stronger argument when there are a large number of degree zero nodes in the network and the largest component is smaller - ? move to discussion)\textcolor{green}{all of that was about transitivity}. 


  This is at variance with the view of local transitivity as a local betweenness where a node is central if transitivity is \textit{low} (see section~\ref{sec:local clustering coefficient}) and can be seen clearly if we test correlation for nodes of degree five or greater (table~\ref{tab:correlation and rank centrality five or more}).



\begin{table}[ht]
\centering
   \setlength{\extrarowheight}{2pt}
\begin{tabular}{llllll}
  \toprule
Rank  & Centrality 1 & Centrality 2 &$\rho$ & $\rho_{k>=5}$  & Rank $k>=5$ \\ 
  \midrule
1 & transitivity0 & transitivityNA & 1.00 & 1.00 & 1 \\ 
  2 & kcoreness & degree & 0.99 & 0.96 & 2 \\ 
  3 & closeness & eigenvector & 0.97 & 0.94 & 3 \\ 
  4 & kcoreness & eigenvector & 0.91 & 0.88 & 4 \\ 
  5 & betweenness & degree & 0.90 & 0.83 & 5 \\ 
  6 & kcoreness & closeness & 0.87 & 0.82 & 6 \\ 
  7 & eigenvector & degree & 0.88 & 0.81 & 7 \\ 
  8 & closeness & degree & 0.85 & 0.77 & 8 \\ 
  9 & kcoreness & betweenness & 0.85 & 0.71 & 9 \\ 
  10 & closeness & betweenness & 0.78 & 0.65 & 10 \\ 
  11 & eigenvector & betweenness & 0.76 & 0.60 & 11 \\ 
  12 & transitivity0 & eigenvector & 0.46 & 0.17 & 13 \\ 
  13 & transitivityNA & eigenvector & 0.33 & 0.17 & 12 \\ 
  14 & transitivity0 & closeness & 0.43 & 0.12 & 15 \\ 
  15 & transitivityNA & closeness & 0.29 & 0.12 & 14 \\ 
  16 & kcoreness & transitivity0 & 0.43 & 0.06 & 17 \\ 
  17 & kcoreness & transitivityNA & 0.26 & 0.06 & 16 \\ 
  18 & transitivity0 & degree & 0.38 & -0.09 & 19 \\ 
  19 & transitivityNA & degree & 0.20 & -0.09 & 18 \\ 
  20 & transitivity0 & betweenness & 0.21 & -0.31 & 21 \\ 
  21 & transitivityNA & betweenness & -0.00 & -0.31 & 20 \\ 
   \bottomrule
\end{tabular}
\caption[Comparison of correlation coefficients for all nodes and nodes with degree greater than or equal to five]{Correlation of centrality measures. Spearman. Degree five or more. Ordered by correlation. Method for missing data include only pairwise complete. TransitivityNA treats isolates as NA and removes from calculation. Transitivity0 uses the option to set isolates to 0. There is minimal perturbation of the rank order but the relationship between transitivity and betweenness and degree become negative.}
\tiny\url{source('~/RProjects/db_essential_genes/R/get_corr/cor_centrality_measures_ordered_cortable_single_long_table.R')}
\label{tab:correlation and rank centrality five or more}
\end{table}

In addition it further obscures the relationship between transitivity and degree ($C(k)$ and $k$ - section~\ref{sec:relationship of ck to k albert}) reported in biological networks, where high degree nodes have lower local clustering
%\todo{this is the second time for saying this now}
\cite{albert2005scale}. Using $C_i^{NaN}$ has little effect on the correlation with other centrality measures or with gene statistics (table~\ref{tab:Correlation of centrality measures 2 transitvities}). 


The most correlated centrality measures were degree and kcore (0.986). Excluding transitivity and kcore which are not amongst the canonical centrality measures \cite{okamoto2008ranking}\footnote{Four measures of centrality that are widely used in network analysis are degree centrality, betweenness centrality,closeness centrality,and eigenvector centrality } the most correlated were eigenvector and closeness (0.969). This is higher than that reported in a collection of social science networks (0.63 in Valente et al\cite{valente2008correlated} - see table~\ref{tab:Correlation of centrality valente et al} and \ref{tab:Correlation of centrality valente et al compared with PSP}) 

See Oldham\cite{oldham2019consistency} for recent work on correlation across networks. Table~\ref{tab:Correlation of centrality valente et al} shows the results of the correlation between degree, eigenvector centrality, closeness and betweenness found by Valente et al.\cite{valente2008correlated} for 58 sociometric network datasets. Centrality measures appear more correlated in the PSP than in those reported by Valente\footnote{although Valente\cite{valente2008correlated} uses Pearson's product moment correlation} particularly closeness and betweenness (0.44 vs 0.782) and degree and closeness (0.66 vs 0.85; see table~\ref{tab:Correlation of centrality valente et al compared with PSP}) and persists even if lower degree nodes are removed from the calculation (table~\ref{tab:correlation and rank centrality five or more}) .

Whether the higher correlation of closeness and other centrality measures may be a feature found in biological networks compared with social networks (as for example assortativity) is unclear.

\begin{table}[]
    \centering   \setlength{\extrarowheight}{2pt}
    \begin{tabular}{lllll}
    \toprule
          & Degree & Eigenvector centrality & Closeness & Betweenness \\
         \midrule
    Degree  & 1  & 0.92  & 0.66 & 0.85\\
    Eigenvector centrality & 0.92 & 1 & 0.63 & 0.72  \\
    Closeness & 0.66 & 0.63 & 1 & 0.44  \\
    Betweenness& 0.85 & 0.72 & 0.44 & 1 \\
    \bottomrule
    \end{tabular}
    \caption[Correlation of centrality measures for sociometric networks]{Correlation of centrality measures for 58 sociometric network datasets from Valente et al \cite{valente2008correlated}. Pearson correlation coefficient. Values are for symmetrised closeness and betweenness centrality. Eigenvector centrality is necessarily symmetrised. Range of average network size 45-83. Some studies had maximal nominations for degree (i.e there was an artificial upper bound using a survey).}
    \label{tab:Correlation of centrality valente et al}
\end{table}



\begin{table}[]
    \centering
    \begin{tabular}{lllll}
    \toprule
          & Degree & Eigenvector centrality & Closeness & Betweenness \\
         \midrule
    Degree  & 1  & 0.92 (0.88)  & 0.66 (0.851)*  & 0.85 (0.902)*\\
    Eigenvector centrality & 0.92 (0.88)  & 1 & 0.63 (0.969)* & 0.72 (0.762)  \\
    Closeness & 0.66 (0.851)  & 0.63 (0.969) & 1 & 0.44 (0.782)* \\
    Betweenness& 0.8 (0.902) & 0.72 (0.762) & 0.44 (0.782) & 1 \\
    \bottomrule
    \end{tabular}
    \caption[Correlation of sociometric centrality measures compared with correlation of measures in PSP]{Correlation of centrality measures for 58 sociometric network datasets from Valente et al.\cite{valente2008correlated}. Pearson correlation coefficient. Values are for symmetrised closeness and betweenness centrality. Eigenvector centrality is necessarily symmetrised. Range of average network size 45-83. Some studies had maximal nominations for degree (i.e there was an artificial upper bound using a survey). Copied from notes.tex. Comparison added with PSP * shows where PSP notably higher correlation.}
    \label{tab:Correlation of centrality valente et al compared with PSP}
\end{table}

\subsubsection{Principal component analysis of centrality measures}

Principal component analysis was carried out to create a low dimensional representation of the relationship between centrality measures and in constructing a linear model of their effect upon intelligence and educational attainment gene scores. 

Principal component analysis of centrality variables was carried out using the R command \texttt{prcomps}\footnote{which uses SVD} with the centralities scaled to unit variance and all other arguments default. All centrality measures (six) other than transitivity NA were included (this was removed to avoid NA handling). The majority of the variance is explained by the first three principal components (93.4\%) (figure~\ref{fig:PCA_scree} and table~\ref{tab:Eigenvalue of dimensions of PCA for centrality measures with variance}). Transitivity and eigenvector centrality appear almost orthogonal in a plot of the first two dimensions (figure~\ref{fig:PCA_contributions}).





\begin{figure}
    \centering
    \includegraphics[width=\textwidth]{images/chapter3/centrality_pca_factoextra/Rplot_screeplot.png}
    \caption{Screeplot of the principal components of six centrality measures ( table~\ref{tab:calculating centrality metrics}). TransitivityNA has been omitted to avoid dealing with NA.} \tiny\url{source('~/RProjects/chapter3/R/centrality_ch3/prcomps/3_plotPCA.R')}
    \label{fig:PCA_scree}
\end{figure}

% latex table generated in R 3.6.3 by xtable 1.8-4 package
% Sat Feb 20 17:09:15 2021
\begin{table}[ht]
\centering
\begin{tabular}{rrrr}
  \toprule
Dimension & Eigenvalue & Variance (\%) & Cumulative variance (\%) \\ 
  \midrule
Dim 1 & 3.63 & 60.46 & 60.46 \\ 
  Dim 2 & 1.18 & 19.66 & 80.12 \\ 
  Dim 3 & 0.80 & 13.26 & 93.38 \\ 
  Dim 4 & 0.22 & 3.71 & 97.09 \\ 
  Dim 5 & 0.14 & 2.31 & 99.40 \\ 
  Dim 6 & 0.04 & 0.60 & 100.00 \\ 
   \bottomrule
\end{tabular}
\caption[Variance explained by PCA components of centrality measures]{Eigenvalue of dimensions of PCA for centrality measures with variance.}
\tiny\url{source('~/RProjects/chapter3/R/centrality_ch3/prcomps/5_PCA_eigval.R')}
\label{tab:Eigenvalue of dimensions of PCA for centrality measures with variance}
\end{table}


\begin{figure}
    \centering
    \includegraphics[width=\textwidth]{images/chapter3/centrality_pca_factoextra/Rplot_smaller_pca.png}
    \caption{Contribution of different variables. deg = degree, eig = eigenvector centrality, kco = kcore index, tra0 = transitivity with isolates set to zero, clo = closeness, bet = betweenness centrality.} 
    \tiny\url{source('~/RProjects/chapter3/R/centrality_ch3/prcomps/3_plotPCA.R')}
    \label{fig:PCA_contributions}
\end{figure}



\clearpage
\section{Global network measures}




\subsection{Results Degree distribution}
\label{sec:results degree distribution}
The degree distribution of the PSP is right-skewed. The mean degree is greater than the median (17.6 vs 8), and the maximum degree (535) is over thirty times greater than the average degree. 
 The degree distribution $P(K=k)$ against degree $k$ is shown in figure~\ref{fig:Degree distribution of post synaptic proteome. Log10 - log10 scale1} on a doubly logarithmic scale. The plot follows a relatively straight line in the tail of the distribution and would usually be considered to be consistent with a power law in the tail of the distribution.

\begin{figure}
    \includegraphics[width=12cm]{images/chapter3/poweRlaw/Rplot_degree_distribution_log10pktheme.png}
    \caption{Degree distribution of post synaptic proteome. Log10 - log10 scale.}
    \tiny Code at \url{source('~/RProjects/chapter3/R/fit_power_law/graphing/plot_degree_distribution_log10_theme_red_dotted_xmin.R')}\url{source('~/RProjects/gridsearch_gamma/R/plot_log_degreedist.R') }
    \label{fig:Degree distribution of post synaptic proteome. Log10 - log10 scale1}
\end{figure}


Using the method described by Clauset, Shalizi and Newman (section~\ref{sec: Methods degree distribution}) the minimum degree value ($x_{min}$) for which the power law distribution holds is $x_{min}$ is 24 and the power law coefficient $\gamma$ is 2.51 (table~\ref{tab:Fit of distributions for PSP})\footnote{, GOF 0.19 and  $n$tail (the number of nodes in the tail of the power law distribution i.e. greater than or equal to $x_{min}$) is 68 so there are a number of low degree  nodes outside the power law portion of the degree distribution\footnote{\url{source('~/RProjects/group_size_distribution/R/calulate_cdf.R')}}}. A bivariate plot of $x_{min}$ and $\gamma$ figure~\ref{fig:Plots of degree distribution}) shows that the power law exponent increases as the value of $x_{min}$ increases.

The plot of log-degree against the log of the cumulative distribution function (CDF) of the degree distribution is shown in figure \ref{fig:log_degree_distribution} and  \ref{fig:poweRlaw plot ggplot2}. The graph forms a straight line for most of the distribution with degree greater than $x_{min}$
%\todo{?change plot or add vline for xmin}
\footnote{the gradient of the CDF is $\gamma$-1}. In figure~\ref{fig:poweRlaw plot ggplot2} the fit of the power law distribution is shown in red. 

The lines of fit of a power-law, log-normal, exponential and Poisson distribution to the data are shown figure~\ref{fig:poweRlaw plot ggplot2} over the whole distribution (i.e. $x_{min}$ is calculated for each distribution as being the value for the fit that minimises the KS distance rather than $x_{min}$ equal to the power-law $x_{min}$ when comparing the log-likelihood of the distribution). This shows that the log-normal is a good statistical model of the whole distribution.

An estimate of the plausibility of a power law distribution was made by calculating the bootstrap $p$ value (100 sims) for the power law distribution is 0.583 with goodness of fit KS 0.19 \footnote{\url{source('~/RProjects/group_size_distribution/R/calulate_cdf.R')}}. A bootstrap $p$ value greater than 0.1 indicates that a power law distribution cannot be ruled out (see \cite{gillespie2015fitting} and \cite{clauset2009power}).



To determine the best fitting distribution, I calculated the log-likelihood of the power law, lognormal, exponential and Poisson distribution fitted to the data. The log-likelihoods were calculated for each distribution from $x_{min}$ determined for the power-law distribution $x_{min} = 24$ (see\cite{clauset2009power}) to allow direct comparison of the models. The values of the parameters are in table~\ref{tab:Fit of distributions for PSP} along with the bootstrap $p$ value for each distribution. The plot of the distributions using $x_{min}=24$ is shown in figure~\ref{fig:models_set_xmin}.


\paragraph{Fit over xmin}
Over the portion for which a power law distribution holds ($x_{min}>24$) the log-likelihood of the Poisson distribution is substantially lower than the exponential, log-normal\footnote{\url{https://www.sciencedirect.com/topics/agricultural-and-biological-sciences/lognormal-distribution}} or power law. 

\begin{figure}
    \centering
    \includegraphics[width=\textwidth]{images/chapter3/poweRlaw/Rplot_models_xmin.png}
    \caption{CDF of degree distribution and lines of best fit for power law (red), log-normal(blue) exponential(green) and poisson (purple). Poisson distribution is not visible as the probability is too low, the exponential is a poorer fit in the tail of the distribution than either the powerlaw or log-normal . x axis degree log 10 scaled, y axis CDF log 10 scaled. Exponential appears a good fit but is fitted to a much smaller part of the distribution than the power law and log-normal. Poisson distribution is clearly not a good fit.}
    \small\url{source('~/RProjects/chapter3/R/fit_power_law/compare_distributions.R')}
    \label{fig:models_set_xmin}
\end{figure}

The exponential distribution is clearly inferior to log-normal and power law using goodness of fit bootstrap p or the two sided Vuong test (p=$3.455\times10^{-7}$). The one sided Vuong test p = $1.72\times10^{-7}$ shows that the power law distribution is a better fit than the exponential as is clear in figure~\ref{fig:models_set_xmin}. 

There is no significant difference between the power law and log-normal using Vuong's two sided test (p=0.187). The one sided test favours the log-normal over power law (p=0.093) but was not significantly different\footnote{power law v log-normal One sided p=0.906}.  

The level of confidence in the power law distribution according to the  criteria of Broido and Clauset\cite{broido2019scale} for the PSP network is strong. The network is plausible by bootstrap $p$ (therefore stronger than weakest), $n_{tail}$ is greater than 50 (668 - at least weak) and meets the strong criteria by satisfying the previous criteria and having an exponent between two and three (2.51). 10\% of the networks examined by Broido and Clauset had this characteristic. It does not meet strongest as the likelihood ratio test is inconclusive and does not favour either the log-normal or power law ($<$ 4\% of studied networks)




The distribution could be characterised either as a power law distribution or as a log-normal distribution, given recent results it could be argued it is likely to possess the features of robustness of a power-law distribution\cite{serafino2021true}.






\begin{figure}
    \centering
    \includegraphics[width=\linewidth]{images/chapter3/ggplot2/theme/powerlaw/Rplot01_chapter3_cdf_and_powerlaw.png}
    \caption[Plot of power law fit to degree distribution of PSP]{Plot of fit of power law to degree distribution for the PSP. X axis is log 10 degree, Y axis is log 10 inverse cdf. Red line fit of power law distribution $\gamma= 2.51$ and $x_{min}=24$ \url{source('~/RProjects/group_size_distribution/R/calulate_cdf.R')}}
    \label{fig:poweRlaw plot ggplot2}
\end{figure}



\begin{figure}
    \centering
    \begin{subfigure}[t]{0.45\textwidth}
        \centering
        \includegraphics[width=\linewidth]{images/chapter3/poweRlaw/Rplot_gamma_theme.png} 
     \caption[$\gamma$ as function of $x_{min}$]{$\gamma$ as a function of $x_{min}$. }
    \tiny\url{source('~/RProjects/PhD_graphs/calculate_gamma_theme.R')}
        \label{fig:gamma}
    \end{subfigure}
    \hfill
    \begin{subfigure}[t]{0.45\textwidth}
        \centering
        \includegraphics[width=\linewidth]{images/chapter3/ggplot2/theme/powerlaw/Rplot01_chapter3_cdf_and_powerlaw.png} 
        \caption{The plot of log cumulative cdf of degree distribution log 10 scale x and y.} \label{fig:log_degree_distribution}
    \end{subfigure}
    \caption{Plots of degree distribution}
    \label{fig:Plots of degree distribution}
\end{figure}
% latex table generated in R 3.4.4 by xtable 1.8-4 package
% Sat Oct  5 16:31:0\begin{document}



%Bootstrap p: power law 0.575



% latex table generated in R 3.6.3 by xtable 1.8-4 package
% Sat Apr 17 12:56:22 2021
\begin{table}[ht]
\centering
\setlength{\extrarowheight}{2pt}
\begin{adjustbox}{width=\textwidth}

\begin{tabular}{lllllllll}
  \toprule
model & $LL$ & $p_{bs}$ & g.o.f & pars\textsubscript{1} & pars\textsubscript{2} & Vuong\textsubscript{statistic} & Vuong $p$\textsubscript{1} & Vuong $p$\textsubscript{2} \\ 
  \midrule
Power-law & -2943.61 & 0.57 & 0.02 & 2.511 &  & -1.32 & 0.906 & 0.0187 \\ 
  log-normal & -2941.56 & 0.38 & 0.01 & -2.345 & 2.11 & 1.32 & 0.094* & 0.0187 \\ 
  exponential & -3044.66 & $<0.001$ & 0.09 & 0.029 &  & 5.10 & $1.73 \times 10^{-7}$ & $3.46 \times 10^{-7}$ \\ 
  poisson & -14376.28 & $<0.001$ & 0.09 & 58.590 &  & 7.31 & $1.29 \times 10^{-13}$ & $2.58 \times 10^{-13}$ \\ 
   \bottomrule
\end{tabular}
\end{adjustbox}
\caption{Fit of distributions for PSP. Vuong test power law versus other test other than log-normal where one sided is log-normal versus power law marked with*. One sided test favours log-normal but not significantly so.} 
\tiny\url{source('~/RProjects/chapter3/R/fit_power_law/fit_pl_setxmin.R')}
\label{tab:Fit of distributions for PSP}
\end{table}


% latex table generated in R 3.6.3 by xtable 1.8-4 package
% Sat Apr 17 13:07:57 2021




% The log-normal when not constrained over the 




% Power law LL
% -2943.606

% -2943.606

% Log normal LL
% -2941.556

% exp ll
% -3044.662

% poisson ll
% -180155.6


% Vuong 
% power law v log normal One sided p=0.906

% log normal v power law One sided p = 0.093

%     p two sided both above 0.187

% Compare power law and exponential one sided p = $1.72\times10^{-7}$
% power law and exponential two sided $3.455\times10^{-7}$


 
  
\subsection{Global measures: Relationship of $C(k)$ to $k$}%\todo{is this needed?}
\label{sec:relationship of ck to k albert}
%\todo{add an introduction to this result and significance}
% You might want to start with another observed relationship is $C(k)$ and $k$ and I will expand on how $C_i$ depends on $k_{min}$ below.

In section~\ref{sec:correlation of centrality measures results section} all centrality measures are positively correlated yet mean local transitivity for nodes of degree $k$: $C(k)$, constant in random networks, is reported to decrease with increasing degree\cite{yook2004functional},\cite{albert2005scale} as,
\begin{equation}
            C(k) = \frac{B}{k^{\beta}},
            \label{eq:C(k) function average transitivity and degree}
\end{equation}


 where $B$ is a constant and $\beta$ is typically between one and two\cite{albert2005scale},\cite{yook2004functional} in networks with hierarchical organisation. High degree nodes should have a lower mean transitivity and the relationship be roughly linear relationship when both axes are logarithmic. The relationship between $C(k)$ and degree $k$ in the PSP network  is shown in figure~\ref{fig:log_transitivity_degree}\footnote{Code \url{source('~/RProjects/graph2community/R/transitivity_degree/Cluster_nodes.R')}}. Initially $C(k)$ does not change with increasing degree but eventually $C(k)$ clearly decreases as $k$ increases. An almost exactly identical pattern is seen in\cite{yook2004functional}  (figure 2b) reporting the relationship in Eq(\ref{eq:C(k) function average transitivity and degree}). 

The overall correlation coefficient between degree and transitivity\textsubscript{NaN} is positive ($\rho= 0.196$; $p < 2.2 \times 10^{-16}$) and is higher when isolates are zero ($\rho$=0.375 for transitivity\textsubscript{0}). The reason for the difference between the positive correlation across the entire distribution and the relationship in figure~\ref{fig:log_transitivity_degree}) is the effect of nodes of degree one ($k=1$) and hence transitivity 0 - having very low degree and minimal transitivity and so being highly correlated(figure~\ref{fig:Scatter plot of the relationship between degree and local transitivity for the PSP nodes}). The transitivity is also low for nodes of degree two.


\begin{figure}
    \centering
    \begin{subfigure}[t]{0.45\textwidth}
        \centering
        \includegraphics[width=\textwidth]{images/chapter3/ggplot2/degree_and_transitivity/Rplot_scatterplot_c_and_degree_new_format.png}
\caption{Scatter plot of the relationship between degree and local transitivity for the PSP nodes. Nodes with degree $k=1$ have transitivity zero. Plot appears to show a negative correlation between degree and transitivity. The opacity of points has been decreased to reduce overplotting and show the number of points of low degree. }
 \tiny\url{source('~/RProjects/chapter3/R/transitivity/transitivity_and_degree/mean_deg_seq.R')} plot p
    \label{fig:Scatter plot of the relationship between degree and local transitivity for the PSP nodes}
    \end{subfigure}
    \hfill
    \begin{subfigure}[t]{0.45\textwidth}
        \centering
        \includegraphics[width=\textwidth]{images/chapter3/ggplot2/degree_and_transitivity/Rplot_rho_and_starting_degree_transitivity.png}
        \caption{Plot of minumum degree included in the calculation of Spearmans rank correlation between degree and $\rho$. Although the overall coefficient is positive the correlation for nodes above degree 5 are clearly negative and this becomes more apparent at high degree.}.
    \tiny\url{source('~/RProjects/chapter3/R/transitivity/transitivity_and_degree/plot/plot_start_degree.R')}
    \label{fig:Plot of minumum degree included in the calculation of Spearmans rank correlation between degree and rho}
    \end{subfigure}
    \caption{$C(k)$ and $k_{min}$}
    \label{fig:Plots of degree distribution}
\end{figure}


For nodes with degree $k>=2$, $\rho$=0.196; $k>=3$ , $\rho$=0.056 and by $k>=5$ the correlation is negative ($\rho$=-0.085). The median degree is eight and five is the 0.369 quantile. The positive correlation between degree and transitivity found across all nodes is the result of the effect of the third of nodes with smallest degree (histogram of transitivity  see fig~\ref{fig:transitivity}). Plotting $\rho$ against the minimum degree included in calculating the correlation coefficient($k_{min})$ clearly shows the switch from positive to negative correlation between low degree and high degree regions (figure~\ref{fig:Plot of minumum degree included in the calculation of Spearmans rank correlation between degree and rho}).

The scaling described as being typical of hierarchical organisation is therefore seen clearly only when degree is above 5 and the relationship changes from a positive correlation in the low degree regime to the previously described scaling. In networks with large numbers of low degree node the correlation coefficient will not show this relationship and setting isolates to zero makes this positive correlation stronger\footnote{ \url{source('~/RProjects/centrality/R/transitivity/calculate_correlation_min_transitivity_and_plot.R', }}.



\subsection{Results Small world}

The small world effect is a combination of a high clustering coefficient and a short average path length (section~\ref{sec:small world methods})
\subsubsection{Global clustering coefficient}
\label{sec:results_global_clustering_coefficient}

The global clustering coefficient defined by Watts and Strogatz ($C_{WS}^0$),  is 0.156 when isolates are set to zero\cite{watts1998collective} and 0.171 if the isolates are undefined and excluded ($C_{WS}^{NaN}$ - the default in igraph) is 0.171 (see above section~\ref{sec:Global clustering coefficient}). A histogram of local transitivity values is seen in figure~\ref{fig:transitivity}, although bounded in the region 0-1 the probability mass is concentrated on the left of the graph due to the number of nodes with low transitivity.

 The value of the global transitivity described by Newman ($C$ section~\ref{sec:global transitivity newman}) is 0.070. The different global transitivity measures are reviewed in table~\ref{tab:Different global transitivities in igraph}.


% latex table generated in R 3.6.3 by xtable 1.8-4 package
% Sat Feb 20 12:04:47 2021
\begin{table}[ht]
\centering

\begin{adjustbox}{width=\textwidth}
\begin{tabular}{llllc}
  \toprule
Name & Symbol &  type & isolates & Global Transitivity \\ 
  \midrule
Watts-Strogatz & $C_{WS}^{0}$& `localaverageundirected'
 &isolates=`zero' & 0.156\vspace{1mm} \\ 

Watts-Strogatz & $C_{WS}^{NaN}$ &`localaverageundirected'*&isolates=`NaN' & 0.171\vspace{1mm} \\ 
  Newman &$C$&`globalundirected' &isolates=`zero'& 0.070\vspace{1mm} \\ 
  Newman &$C$&`globalundirected'&isolates=`NaN'& 0.070 \\ 
   \bottomrule
\end{tabular}
\end{adjustbox}
\caption[Different global transitivities in igraph]{Different global transitivities in igraph. * or mean of `localundirected' with isolates `NaN' 
\tiny\url{source('~/RProjects/chapter3/R/transitivity/global_transitivity/global_transitivity.R')} 
\tiny same project \url{R/transitivity/global_transitivity/mkdown/global_transitivity_markdown.html}} 
\label{tab:Different global transitivities in igraph}
\end{table}


% % latex table generated in R 3.6.1 by xtable 1.8-4 package
% % Wed Mar 10 19:35:16 2021
% \begin{table}[ht]
% \centering
% \begin{tabular}{rrrrrrrrrrr}
%   \hline
%  & Min. & 1st Qu. & Median & Mean & 3rd Qu. & Max. & PSP & difference & sd & n\_sd \\ 
%   \hline
% C Newman & 0.005 & 0.005 & 0.005 & 0.005 & 0.005 & 0.006 & 0.070 & 0.065 & 1.712E-04 & 377.5 \\ 
%   C\_WS\_NAN & 0.005 & 0.005 & 0.005 & 0.005 & 0.005 & 0.006 & 0.171 & 0.166 & 1.816E-04 & 915.4 \\ 
%   C\_WS\_0 & 0.005 & 0.005 & 0.005 & 0.005 & 0.005 & 0.006 & 0.156 & 0.151 & 1.816E-04 & 832.4 \\ 
%   \hline
% \end{tabular}
% \caption{Global transitivity Monte Carlo Erdos Renyi\url{source('~/RProjects/chapter3/R/transitivity/global_transitivity/monte_carlo_er_global_transitivity.R')}}
% \end{table}
% \paragraph{This bit is all about local transitivity}
% The expected value of mean local clustering given a random rewiring with fixed degree distribution is 0.07479691 sd (0.00183) removing NA and 0.068 setting NA to 0 (100 iterations) and 0.0682 seeting NA to zero with sd 0.00181 \footnote{Code  \url{source('~/RProjects/centrality/R/transitivity/calculate_local_transitivity_and_simulate_mean.R')} }


% latex table generated in R 3.6.1 by xtable 1.8-4 package
% Wed Mar 10 19:39:04 2021
% \begin{table}[ht]
% \centering
% \begin{tabular}{rrrrrrrrrrr}
%   \hline
%  & Min. & 1st Qu. & Median & Mean & 3rd Qu. & Max. & PSP & difference & sd & n\_sd \\ 
%   \hline
% C Newman & 0.055 & 0.058 & 0.058 & 0.058 & 0.059 & 0.061 & 0.070 & 0.011 & 8.032E-04 & 14.2 \\ 
%   C\_WS\_NAN & 0.069 & 0.073 & 0.075 & 0.075 & 0.076 & 0.081 & 0.171 & 0.097 & 1.981E-03 & 48.8 \\ 
%   C\_WS\_0 & 0.063 & 0.067 & 0.068 & 0.068 & 0.069 & 0.074 & 0.156 & 0.088 & 1.807E-03 & 48.8 \\ 
%   \hline
% \end{tabular}
% \caption{Global transitivity Monte Carlo configuration model \url{source('~/RProjects/chapter3/R/transitivity/global_transitivity/monte_carlo_er_global_transitivity.R')}}
% \end{table}

% latex table generated in R 3.6.3 by xtable 1.8-4 package
% Thu Mar 11 17:41:34 2021
The global transitivity values compared to Monte Carlo estimates using Erdos Renyi random graphs with the same number of edges and nodes as the PSP (table~\ref{tab:MC estimate of ER transitivity}). The Watts-Strogatz transitivity is larger than the Newman transitivity (0.156 vs 0.70). It is not clear what the significance of the divergence in values is \cite{newman2018networks} but it is absent in the random graphs and appears to be a feature of the network. The absolute size of the difference between the PSP and ER transitivity is small but is of the order of several standard deviations (377.5 for Newman Transitivity) and the samples of the ER transitivity appear normally distributed (see KS in table~\ref{tab:MC estimate of ER transitivity}).


\begin{table}[ht]
\centering
\begin{adjustbox}{width=\textwidth}
\begin{tabular}{llllllllll}
  \toprule
 & Median ER & Mean ER & PSP & difference & sd & $n$ sd & ks $d$ & ks $p$ & $p$ \\ 
  \midrule
$C$ & $5.096 \times 10^{-3}$ & $5.106 \times 10^{-3}$ & 0.070 & 0.065 & $1.712 \times 10^{-4}$ & 377.5 & 0.026 & 0.51 & 0 \vspace{1mm} \\ 
  $C_{WS}^{NaN}$ & $5.095 \times 10^{-3}$ & $5.104 \times 10^{-3}$ & 0.171 & 0.166 & $1.816 \times 10^{-4}$ & 915.4 & 0.024 & 0.63 & 0 \vspace{1mm} \\ 
  $C_{WS}^0$ & $5.095 \times 10^{-3}$ & $5.104 \times 10^{-3}$ & 0.156 & 0.151 & $1.816 \times 10^{-4}$ & 832.4 & 0.024 & 0.63 & 0 \vspace{1mm}\\ 
   \bottomrule
\end{tabular}
\end{adjustbox}
\caption[Global transitivity of PSP and Monte Carlo estimates - Erdos-Renyi]{The global transitivity measures of the PSP are compared with Monte Carlo estimates of the transitivity measures in Erdos Renyi random graphs with matching numbers of vertices and edges. 1000 iterations. difference  = PSP - mean Monte Carlo Estimate $n$ sd number of times difference greater than standard deviation of MC estimate mean. ks $d$ = Kolmogorov-Smirnov statistic distance,ks $p$ Kolmogorov-Smirnov test statistic comparing Monte Carlo estimate to normal distribution with identical mean and standard deviation,  $p$ probability of result as extreme as PSP using mean and standard deviation of MC estimates.} \tiny\url{source('~/RProjects/chapter3/R/transitivity/global_transitivity/monte_carlo_er_global_transitivity.R')} 
\label{tab:MC estimate of ER transitivity}
\end{table}


Monte Carlo estimates of the transitivity in the degree constant configuration model are shown in table~\ref{tab: MC estimate of config transitivity}. There is less difference between the PSP transitivity and the Configuration model than between the PSP and the ER graphs in both absolute terms or in the number of standard distributions it represents. The Watts-Strogatz global transitivity is again differs most from the random model. There is a difference in the value of the mean transitivity between the different types of transitivity with their rank order being the same as the PSP (Watts-Strogatz is greatest). I would hypothesise some but not all of the divergence in these values (WS and Newman) are due to degree structure but the divergence appears to be a feature of real world graphs. 

The global undirected transitivity of the PSP network is 0.0697. This is similar to the value reported by Newman\footnote{p305} \cite{newman2018networks} in metabolic networks (0.090) and PPI (0.072).
% latex table generated in R 3.6.3 by xtable 1.8-4 package
% Sat Mar 13 14:48:46 2021
\begin{table}[ht]
\centering
\begin{adjustbox}{width=\textwidth}

\setlength{\extrarowheight}{2pt}
\begin{tabular}{llllllllll}
  \toprule
 & Median Config & Mean & PSP & difference & sd & $n$ sd & ks $d$ & ks $p$ & $p$ \\ 
  \midrule
$C$  & $5.835 \times 10^{-2}$ & $5.836 \times 10^{-2}$ & 0.070 & 0.011 & $8.032 \times 10^{-4}$ & 14.2 & 0.016 & 0.95 & 0 \\ 
   $C_{WS}^{NaN}$  & $7.475 \times 10^{-2}$ & $7.480 \times 10^{-2}$ & 0.171 & 0.097 & $1.981 \times 10^{-3}$ & 48.8 & 0.024 & 0.63 & 0 \\ 
   $C_{WS}^0$ & $6.818 \times 10^{-2}$ & $6.822 \times 10^{-2}$ & 0.156 & 0.088 & $1.807 \times 10^{-3}$ & 48.8 & 0.024 & 0.63 & 0 \\ 
   \bottomrule
\end{tabular}
\end{adjustbox}
\caption[Transitivity PSP compared with configuration model]{PSP vs 1000 Configuration model Monte Carlo. 1000 iterations. difference  = PSP - mean Monte Carlo Estimate $n$ sd number of times difference greater than standard deviation of MC estimate mean. ks $d$ = Kolmogorov-Smirnov statistic distance,ks $p$ Kolmogorov-Smirnov test statistic comparing Monte Carlo estimate to normal distribution with identical mean and standard deviation,  $p$ probability of result as extreme as PSP using mean and standard deviation of MC estimates. }
\tiny\url{source('~/RProjects/chapter3/R/transitivity/global_transitivity/monte_carlo_config2_global_transitivity.R')}
\label{tab: MC estimate of config transitivity}
\end{table}





    
\begin{figure}
    \centering
    \begin{subfigure}[t]{0.45\textwidth}
        \centering
        \includegraphics[width=\linewidth]{images/Rplot_transitivity.png} 
        \caption{Histogram of transitivity} \label{fig:transitivity}
    \end{subfigure}
    \hfill
    \begin{subfigure}[t]{0.45\textwidth}
        \centering
        \includegraphics[width=\linewidth]{images/chapter3/ggplot2/c(k)/Rplot_C_k_new_formatted.png}
        %{images/Rplot01_logdegree_log_transitivity.png} 
        \caption{Plot of degree with mean local transitivity. Log10-log10 scale. Omitting degree $k=1$ as transitivity $C$ will be undefined.}
        \label{fig:log_transitivity_degree}
    \end{subfigure}
    \caption{Plots of transitivity.}
\end{figure}


\subsubsection{Results: Average path length and transitivity}
\label{sec:Results average path length and transitivity}

Having dealt with the results for the global clustering coefficient we also need to determine the average path length in order to investigate the small world phenomenon. The average path length of the PSP was again compared Monte Carlo estimates of the average path length for the Erdos Renyi models and configuration models.

The average path length, calculated using the \texttt{mean\_distance} function in igraph, was shortest for the PSP (2.98) and longest for the Erdos Renyi graph with the Configuration Model again in the middle. The value for one sample t tests of the difference between the Monte Carlo estimates and the PSP\footnote{\url{source('~/RProjects/chapter3/R/transitivity/global_transitivity/t_test/compare_er_config.R')}} indicate that the PSP has a significantly shorter APL \ref{tab:t test apl er}. 


\begin{table}[]
    \centering
    \begin{tabular}{llllll}
    \toprule
    Graph     &  APL & sd & $C_{WS}$& $C_{WS}0$ & $C$ \\
    \midrule
    PSP     & 2.98 & NA & 0.171 & 0.156 & 0.0697 \\
   
    Config  & 3.0069 &0.0053& 0.0683 & 0.0748 & 0.0584\\
     ER      & 3.118 & 0.00047 & 0.0051 & 0.0051 & 0.0051 \\
    \bottomrule
    \end{tabular}
    \caption{Alternative way of presenting the data take transitivity from PSP vs 1000 Configuration model Monte Carlo
    \tiny\url{source('~/RProjects/chapter3/R/transitivity/global_transitivity/monte_carlo_config2_global_transitivity.R')}}
    \label{tab:small world apl and c}
\end{table}





%For the degree corrected configuration model see table~\ref{tab:cmtransitivity_configuration_model} 
For the degree corrected configuration model see table~\ref{tab:small world apl and c}. %\textcolor{red}{cross ref with albert should have shorter average path length}.
The average path length is shorter than in the Erdos Renyi model likely due to the presence of hub nodes\cite{albert2005scale}\footnote{ \url{source('~/RProjects/graph_sw/R/small_world_random/degree_corrected_config_random_ws.R')}}.


This is similar to the L of 2.65 and C of 0.28 of C elegans, where L is the ``shortest path between two vertices, averaged over all pairs of vertices and C is the clustering coefficient'' as defined by Watts and Strogatz \cite{watts1998collective}. The random rewiring with a uniform probability leads to a L of 3.11. For comparison the path length and global clustering from \cite{watts1998collective} is shown in table~\ref{tab:clustering and path length Watts and Strogatz}


\begin{table}[h]
    \centering
    \begin{tabular}{lllll}
    \toprule
    Network      & L\textsubscript{actual} & L\textsubscript{random} & C\textsubscript{actual} & C\textsubscript{random} \\
    \midrule
     Film actors    & 3.65 & 2.99 & 0.79 & 0.00027\\
     Power grid & 18.7 & 12.4 & 0.080 & 0.005 \\
     \textit{C elegans} & 2.65 & 2.25 & 0.28 & 0.05 \\
     \bottomrule
    \end{tabular}
    \caption{Clustering coefficient and average path length of three networks compared to random Erdos Renyi graph with same number of nodes and edges - from Watts and Strogatz\cite{watts1998collective}. $L$ is average path length, $C$ is global transitivity of type Watts-Strogatz}
    \label{tab:clustering and path length Watts and Strogatz}
\end{table}

%todo{add omnigenic bit}%see code source('~/RProjects/graph_transitivity/R/transitivity.R')

% latex table generated in R 3.6.3 by xtable 1.8-4 package
% Sat Mar 13 15:45:33 2021
\begin{table}[ht]
\centering
\begin{adjustbox}{width=\textwidth}

\begin{tabular}{rrrrrrrrrr}
  \toprule
 & Median & Mean & PSP & difference & sd & n\_sd & ks\_d & ks\_p & p \\ 
  \midrule
APL & 3.01 & 3.01 & 2.980 & -0.027 & $5.304 \times 10^{-3}$ & -5.2 & 0.017 & 0.93 & 1 \\ 

   \bottomrule
\end{tabular}
\end{adjustbox}
\caption{PSP vs 1000 Configuration model Monte Carlo} 
\end{table}


% latex table generated in R 3.6.3 by xtable 1.8-4 package
% Sat Mar 13 16:16:30 2021
\begin{table}[ht]
\centering
\begin{adjustbox}{width=\textwidth}
\begin{tabular}{lllllllllll}
  \toprule
 & Median & Mean & PSP& CI & difference & sd & n\_sd & ks\_d & ks\_p & p \\ 
  \midrule
APL & 3.12 & 3.12 & 2.980 &3.1179-3.1180 & -0.138 & $4.875 \times 10^{-4}$ & -283.5 & 0.028 & 0.43 & $<2.2 \times 10^{-16}$ \\ 
 
   \bottomrule
\end{tabular}
\end{adjustbox}
\caption{PSP vs 1000 ER model Monte Carlo} 
\end{table}




 
 
 % latex table generated in R 3.6.3 by xtable 1.8-4 package
% Sat Mar 13 16:59:18 2021
\begin{table}[ht]
\centering
\begin{adjustbox}{width=\textwidth}
\begin{tabular}{lrrrrrrll}
  \toprule
 & estimate & t-statistic & p.value & d.f & conf.low & conf.high & method & alternative \\ 
  \midrule
t-test APL ER & 3.118 & 8964.21 &    0 & 999 & 3.1180 & 3.1180 & One Sample t-test & two.sided \\ 
   \bottomrule
   
\end{tabular}
\end{adjustbox}
\caption{T test APL ER\url{source('~/RProjects/chapter3/R/small_world/average_path_length/monte_carlo_er_apl.R')}}
\label{tab:t test apl er}
\end{table}


\subsubsection{Distance distribution}

An alternative test presence of the small world effect (\cite{barabasi2016network}\footnote{p143}) is to compare the distance distribution (pairwise distances between all nodes) with the Erdos-Renyi graph and the configuration model. This is essentially using all of the path lengths that make up the average path length to have a distribution of values for the PSP to compare to a random model.

The results show that the PSP has  a significantly shorter mean path length  compared to the other two. The configuration model has a shorter path length than the ER model but longer than the PSP. This is in keeping with our other results - tables~\ref{tab:ANOVA_distance_distribution} and table~\ref{tab:Tukey post hoc}. Infinite values have been removed from the random graphs (distance values are infinite when the graph is disconnected).

The results support the PSP having a small world configuration. 

% latex table generated in R 3.6.1 by xtable 1.8-4 package
% Mon Mar 22 16:22:45 2021
\begin{table}[ht]
\centering
\begin{tabular}{lrrrrr}
  \toprule
 & Df & Sum Sq & Mean Sq & F value & Pr($>$F) \\ 
  \midrule
Graph types & 2 & 64211.02 & 32105.51 & 81460.66 & 0.0000 \vspace{1mm}\\ 
  Residuals & 17921085 & 7063109.61 & 0.39 &  &  \vspace{1mm}\\ 
   \bottomrule
\end{tabular}
\caption{ANOVA (aov in R) of distance distribution of PSP, Erdos Renyi Graph and Configuration model. Erdos-Renyi with n,m matching PSP. Configuration uses degree distribution of PSP}
\tiny\url{source('~/RProjects/chapter3/R/small_world/distance_distribution/barabasi_distance_distribution.R')}
\label{tab:ANOVA_distance_distribution}
\end{table}
% latex table generated in R 3.6.1 by xtable 1.8-4 package
% Mon Mar 22 15:04:39 2021
\begin{table}[ht]
\centering
\begin{tabular}{rrrrr}
  \toprule  
 & diff & lwr & upr & p adj \\ 
  \midrule
ER-PSP & 0.139 & 0.138 & 0.140 & $<0.001$ \\ 
  Config-PSP & 0.024 & 0.023 & 0.025 & $<0.001$ \\ 
  Config-ER & -0.115 & -0.116 & -0.114 & $<0.001$ \\ 
   \bottomrule
\end{tabular}
\caption[Tukey Distance Distribution]{Post hoc  Tukey multiple comparisons of means
    95\% family-wise confidence level Distance Distribution. p is 0. Diff= difference in means. Mean pairwise distance $n$=5,973,696 for PSP, Erdos Renyi gnm graph with identical $n$ vertices and edges to PSP and configuration random graph using PSP degree distribution. }
\tiny\url{source('~/RProjects/chapter3/R/small_world/distance_distribution/barabasi_distance_distribution.R')}
\label{tab:Tukey post hoc}
\end{table}



\paragraph{Small worldness measure}
The small world measure $S$ where a graph is small world if $S>1$ \cite{humphries2008network} is 14.3\footnote{\url{source('~/RProjects/chapter3/R/small_world/calculate_humphries_small_world_metric.R')}}.
Using the definition of small worldness from Humphries and Gurney.

 $S$ is $S>1$ for small world networks:

\begin{equation}
    \gamma_g = \frac{C_g}{C_{random}}
\end{equation}

and

\begin{equation}
    \lambda_g = \frac{C_g}{C_{random}}
\end{equation}


then

\begin{equation}
    S = \frac{\gamma}{\lambda}
\end{equation}

S using the global measure of transitivity ($C$) was calculated to be 14.3\footnote{\url{source('~/RProjects/chapter3/R/small_world/calculate_humphries_small_world_metric.R')}} and $S$ using the Watts-Strogatz global clustering coefficient $C_{ws}$= 35.08\footnote{\url{source('~/RProjects/chapter3/R/small_world/calculate_humphries_small_world_metricWS.R')}}

A value greater than one is defined as denoting the small world property\cite{humphries2008network}. Thirty three real world networks showed a range of values from   1.34 (student relationships) to  approximately 11,660 for math coauthorship ($S$ $C$)  and for WS from S=0.27 student relationship (not small world) to 61,967 for biology co-authorship n=11,803,064. They found that $S$ scaled near linearly with $n$ the number of vertices in the network.  .

The predicted value for the PSP is is 57.39 ($0.023n^{0.96}$=57.39), $\Delta$ ie not WS) and for $S$ using WS $S_{ws}$=$0.012n^{1.11}$ = 101.65 



\paragraph{Conclusion} The network shows the small world property. It is less than theoretical maximum small world and behaves similarly to other networks in its scaling with n. There is a wide variation to the small worldness

\clearpage

\subsection{ Assortativity }
\subsubsection{Degree Assortativity}


 The PSP network is disassortative (section~\ref{sec:Assortativity methods}) with regard to degree ($\rho_D$=-0.18)\footnote{using the notation in Noldus\cite{noldus2015assortativity}},. 
 This value  %\todo{three sf}
 is similar to those reported in biological networks including yeast protein-protein interaction networks (-0.156) and neural networks (-0.163) \cite{newman2002assortative}  and summarised in table~\ref{Table:DegreeAssortativityNewman}. Amongst non biological networks it is most similar in magnitude to that of the internet in 2002\cite{newman2002assortative} ($r=$-0.189). 

 A Monte Carlo estimate of the expected value of the degree assortativity for the configuration model with identical degree sequence is very close to zero: -0.00003 (10,000 iterations ; min. -0.022, max.  0.026). The degree assortativity is far from what would be expected by chance  $p_{MC}<0.0001$\footnote{\url{source('~/RProjects/chapter3/R/assortativity_distribution/assortativity_config.R')}}.

The degree assortativity is negative in most real world networks other than social networks (table \ref{Table:DegreeAssortativityNewman})\cite{newman2002assortative}. The random model is known to have zero correlation although low values ($r<0.02$) may be due to chance)\cite{noldus2015assortativity}). A standard (Barabsi-Albert) generative model for scale free networks is  mildly disassortative\cite{noldus2015assortativity} when small but also shows no degree assortativity in the limit of large $n$\cite{noldus2015assortativity}. 
%\todo{this is the bit where the pseudohub bit would fit}. 
%\todo{? add positive $r$ to table}.

The assortativity suggests that the Barabasi-Albert model is `incomplete' as a model of the internet \cite{newman2002assortative} and by extension the PSP protein-protein interaction network and would need to be accounted for in construction probabilistic models of the network. 
The Callaway model of network generation ('grown graph' - newman) shows the positive degree correlation seen in social networks but is inaccurate as a model of other complex networks (including online social networks). 


\begin{table}[]
    \centering
    \begin{tabular}{lll}
    \toprule
       network  &N& $r$  \\
       \midrule
       internet & 10697&-0.189\\
       world wide web &269504 & -0.065\\
       post synaptic proteome & 3457 & -0.18\\
       protein interactions & 2115 & -0.156\\
       neural network & 307 & -0.163\\
       food web & 92 & -0.276 \\
       \bottomrule
    \end{tabular}
    \caption{Degree assortativity coefficients of networks adapted from Newman 2002 \cite{newman2002assortative} including post synaptic proteome}
    \label{Table:DegreeAssortativityNewman}
\end{table}

%\subsubsection{Degree correlation}
 
%\todo{Include \url{R_plot_degree_correlation.png} commented out}

Clustering and interconnection of communities tends to lead to higher assortativity by degree and branching negative\cite{estrada2011combinatorial},\cite{noldus2015assortativity}. Higher assortativity in social networks leads to higher modularity and clustering coefficient\cite{noldus2015assortativity}. Community formation is more remarkable therefore in overall disassortative networks. 

% This is possibly the bit where Keith's bit would be good to put in.  
 %\begin{figure}]h]
 % \includegraphics[width=\linewidth]{Rplot_degree_correlation.png}
 % \caption{Plot of degree with k nearest neighbour mean of degree}
 % \label{fig:knn_degree}
%\end{figure}
%
%
%
\subsubsection{Assortativity other centrality measures}

% latex table generated in R 3.6.3 by xtable 1.8-4 package
% Sat Feb 27 12:16:57 2021

Table~\ref{tab:assortativity by centrality} shows that the network is most disassortativty by degree. By contrast kcoreness and closeness are assortative suggesting a central area of genes (core) that are more highly connected\footnote{\url{source('~/RProjects/chapter3/R/assortativity_distribution/centrality/assortativity_by_centrality.R')}}. An accurate probabilistic generative model would also need to take this into consideration.  

\begin{table}[ht]
\centering
\begin{tabular}{lr}
  \toprule
Centrality measure & assortativity ($\rho$) \\ 
  \midrule
kcoreness & 0.089 \\ 
  Closeness & 0.072 \\ 
  Eigenvector & -0.033 \\ 
  Transitivity0 & -0.036 \\ 
  Betweenness & -0.093 \\ 
  Degree & -0.178 \\ 
   \bottomrule
\end{tabular}
\caption{Assortativity by centrality measure.}
\tiny\url{source('~/RProjects/chapter3/R/assortativity_distribution/centrality/assortativity_by_centrality.R')} 
\label{tab:assortativity by centrality}
\end{table}


\subsubsection{Assortativity and gene score}

 The guilt by association hypothesis (section~\ref{sec:Assortativity methods}) suggests that similar genes are linked. We can test whether genes associated with intelligence or educational attainment in the sample GWAS ore more likely to be linked considering the network as a whole by calculating the assortativity of the network by study z score provided by MAGMA (as we are interested in correlation rather than the absolute values - see discussion). 
 
 Some genes in the PSP do not have SNPs within them that appear in the study samples and therefore  not have a  z-score assigned by MAGMA. These missing values are set to zero or the median value of the z-score (to ensure there is no significant difference between either strategy) as the assortativity function does not accept missing values\footnote{code \url{source('~/RProjects/centrality/R/get_gene_scores/assortativity_with_z_scores.R')} },
\footnote{Redo \url{source('~/RProjects/chapter3/R/assortativity_distribution/z_score/assortativity_z.R', echo=TRUE)}}. 

\subsubsection{Assortativity with z score results}
 
 There is no evidence that across the PSP nodes are assortative or disassortative by the level of association of gene Z scores for studies of intelligence or educational attainment. 
 
 

\begin{table}[]
     \centering
     \begin{tabular}{llllll}
     \toprule
         Study & Missing values  & Assortativity\textsubscript{0} & Assortativity\textsubscript{median} & p\textsubscript{BS 0} & p\textsubscript{BS median}\\
         \midrule
         Intelligence\textsubscript{Replication} & 148 & 0.003 & 0.0032 & 0.78 & 0.79\\
         Education\textsubscript{Replication} & 172 & -0.006 &  -0.0064&0.20 & 0.18 \\ 
         Education\textsubscript{Discovery} & 156 & 0.001 & -0.0002&0.67 & 0.59\\
         Intelligence\textsubscript{Discovery} & 156 & -0.004 & -0.0038&0.31 & 0.33\\
         \bottomrule
     \end{tabular}
     \caption{Assortativity of PSP graph vertices by study Z score. PSP graph members without corresponding Z score marked as missing values had z set to either 0 (Assortativity\textsubscript{0} or the median z score Assortativity\textsubscript{median}for missing set to zero \url{source('~/RProjects/centrality/R/get_gene_scores/assortativity_with_z_scores.R')} . All values not significantly different ($p>0.05$) than bootstrap estimate of assortativity (\url{source('~/RProjects/chapter3/R/assortativity_distribution/bootstrap/bootstrap_zscore_assortativity_studies.R')}}
     \label{tab:Assortativity of PSP graph and z scores1}
 \end{table}

 
 


% latex table generated in R 3.6.3 by xtable 1.8-4 package
% Sun Mar 14 14:30:44 2021

The results are shown in table~\ref{tab:Assortativity of PSP graph and z scores1}, missing values have been assigned value 0 (Assortativity\textsubscript{0}) or the median z score (Assortativity\textsubscript{median}) as the nominal assortativity function in igraph requires values for all vertices. Assortativity scores tend to me modest (as score of approximately 0.15 is often found for degree assortativity).  These results are close to zero and do not differ from what could be expected by chance 

This suggests that nodes with high or low z scores in the population cohorts are no more likely to have links to one another \textit{globally} across the network. It does not exclude areas within the graph of high assortativity nor distinct areas that containing high scoring genes connected by link nodes.

\subsubsection{Assortativity with murine LTP}

% % latex table generated in R 3.6.3 by xtable 1.8-4 package
% % Mon Feb 22 17:39:30 2021
% \begin{table}[ht]
% \centering
% \begin{tabular}{llr}
%   \hline
% MP & Name & Q \\ 
%   \hline
% MP0001463 & abnormal spatial learning & 0.045 \\ 
%   MP0001468 & abnormal temporal memory & 0.029 \\ 
%   MP0001469 & abn context condition & 0.030 \\ 
%   MP0001473 & reduced long term potentiation & 0.025 \\ 
%   MP0001898 & abnormal long term depression & 0.063 \\ 
%   MP0002062 & abn assoc learning & 0.042 \\ 
%   MP0002062 & abormal associative learning & 0.042 \\ 
%   MP0002063 & abnormal learning memory conditioning & 0.053 \\ 
%   MP0004859 & abnormal synaptic plasticity & 0.049 \\ 
%   MP0014114 & abnormal cognition & 0.053 \\ 
%   \hline
% \end{tabular}
% \caption{Nominal assortativity for mammalian phenotypes. and $n$ and $ci$ Source \url{source('~/RProjects/chapter3/R/assortativity_distribution/nominal_murine/2_Feb_murine_LTP_assortativity.R')}} 
% \label{tab:Nominal assortativity for mammalian phenotypes}
% \end{table}


\begin{table}[ht]
\centering
\setlength{\extrarowheight}{2pt}
\begin{adjustbox}{width=\textwidth}


\begin{tabular}{lllllll}
  \toprule
MP & Name & n & $Q$ & max Q\textsubscript{MC} & $\mu_Q$ MC & $p$ \\ 
  \midrule
MP0001463 & abnormal spatial learning & 272 & 0.045 & 0.022 & -0.001 & $<0.0001$ \\ 
  MP0001468 & abnormal temporal memory & 215 & 0.029 & 0.021 & -0.001 & $<0.0001$ \\ 
  MP0001469 & abn context condition & 181 & 0.030 & 0.020 & -0.001 & $<0.0001$ \\ 
  MP0001473 & reduced long term potentiation & 172 & 0.025 & 0.021 & -0.001 & $<0.0001$ \\ 
  MP0001898 & abnormal LTDep & 102 & 0.063 & 0.024 & -0.001 & $<0.0001$ \\ 
  MP0002062 & abn assoc learning & 339 & 0.042 & 0.021 & -0.001 & $<0.0001$ \\ 
  MP0002062 & abormal associative learning & 339 & 0.042 & 0.018 & -0.001 & $<0.0001$ \\ 
  MP0002063 & abnormal learning memory conditioning & 742 & 0.053 & 0.020 & -0.001 & $<0.0001$ \\ 
  MP0004859 & abnormal synaptic plasticity & 77 & 0.049 & 0.025 & -0.001 & $<0.0001$ \\ 
  MP0014114 & abnormal cognition & 744 & 0.053 & 0.020 & -0.001 & $<0.0001$ \\ 
   \bottomrule
\end{tabular}
\end{adjustbox}
\caption[Nominal assortativity of gene sets of mammalian phenotypes associated with LTP or learning]{Nominal assortativity for mammalian phenotypes associated with LTP or learning. p is one tailed estimate Monte Carlo estimate of assortativity of same number of random vertices. p is proportion of random $q>=Q$. $Q$ is modularity as a measure of nominal assortativity. max Q\textsubscript{MC} is the maximum modularity found on Monte Carlo estimates (Samples:10000.),$\mu_Q$ MC is the sample mean of modularity for the 10,000 Monte Carlo estimates. }
\tiny\url{source('~/RProjects/chapter3/R/assortativity_distribution/nominal_murine/2_Feb_murine_LTP_assortativity_CI.R')} 
\label{tab:Nominal assortativity for mammalian phenotypes. p is one tailed estimate Monte Carlo estimate of assortativity of same number of random vertices. p is proportion of random $q>=Q$. Samples:10000}
\end{table}

    The ten genes sets of mammalian phenotypes associated with LTP or learning (section~\ref{sec:methods murine LTP and centrality}) show modest but significant assortative behaviour compared to a null model in contrast to the assortativity for population gene scores in the educational attainment and intelligence studies (table~\ref{tab:Assortativity of PSP graph and z scores1}). As nominal categories the modularity ($Q$) is reported in table~\ref{tab:Nominal assortativity for mammalian phenotypes. p is one tailed estimate Monte Carlo estimate of assortativity of same number of random vertices. p is proportion of random $q>=Q$. Samples:10000}. A Monte Carlo estimate of the modularity of 10,000 random samples of vertices of the same size as the gene set is used to calculate a one tailed p value. For 10,000 iterations none of the Monte Carlo values of $Q$ were greater than those in the mouse phenotypes. This suggests that the genes implicated in mammalian models of learning and LTP tend to link to one another more than would be expected by chance while the significant genes in the intelligence and educational attainment GWAS do not connect more than would be expected by chance. 
 
 Essential genes are assortative. Nominal assortativity is greater for genes having five or more appearances in the DEG ($Q=$0.118) than when testing using any appearance in the DEG ($Q=$0.0445)\footnote{\url{RProjects/chapter3/R/assortativity_distribution/DEG_assortatiivity/assortativity_DEG.R}}.
 
 Assortativity (scalar) using pLI is $r=0.081$\footnote{\url{RProjects/chapter3/R/assortativity_distribution/DEG_assortatiivity/assortativity_DEG}}. The assortativity is more remarkable considering that high degree nodes are associated with high pLI (table~\ref{tab:correlation pli vertex statistics mar}) but are disassortative by degree (table~\ref{tab:assortativity by centrality}).

 
 
 \clearpage
 
 
\subsection{Centrality and murine LTP}
\label{sec:Centrality and murine LTP}
The mean centrality measures for gene sets associated with murine models of learning were compared to the PSP (Wilcoxon rank test) with correction for multiple comparisons. 

No statistically significant difference from the full PSP was seen for degree (figure~\ref{fig:murine_ltp_centrality_boxplot_degree1}). Betweenness centrality was  increased in these gene sets (figure~\ref{fig:murine_ltp_centrality_boxplot_betweenness}) and was significantly elevated (table~\ref{tab:betweenness murine phenotypes. Wilcoxon test compared with full PSP. P adjusted using B.H}), after correction for multiple comparisons, in gene sets representing abnormal spatial learning (p=0.002, p adjusted BH 0.02) and  abnormal cognition (p=0.003, p adjusted BH 0.026) . Closeness centrality was generally reduced  (figure~\ref{fig:murine_ltp_centrality_boxplot_closeness}, table~\ref{tab:closeness murine phenotypes. Wilcoxon test compared with full PSP.}) but abnormal long term depression was the only significant phenotype after correction for multiple comparisons ($p=0.002$, $p_{BH}$=0.018). Eigenvector centrality was also reduced (figure~\ref{fig:murine_ltp_centrality_boxplot_eigenvector}), the difference between mean eigenvector centrality for the gene set abnormal synaptic plasticity ($p=0.002$;$p BH$=0.021) and abnormal long term depression ($p=0.002$; $p BH$=0.021) were significant after correction for multiple comparisons. There was no significant difference for kcoreness or transitivity.

The overall picture is of a reduction in closeness centrality and eigenvector centrality with an increase in betweenness centrality. This would be consistent with vertices which tend to be bottlenecks despite being towards the periphery of the network. 






\begin{figure}
    \centering
    \includegraphics[width=\textwidth]{images/chapter3/ggplot2/murine_centrality_boxplot/add_theme/addLTP/Rplot_murine_degree_ltp.png}
    \caption{Box plot of Degree centrality distribution for differenet gene sets affecting Murine LTP or learning. Log 10 scale for degree. No statistically significant difference other than minor change in long term depression.} 
    \tiny\url{source('~/RProjects/chapter3/R/murine_centrality/general_murine_boxplot/by_measure/degree_centrality_boxplot_murine_ltp_general_degree_add_table.R')}
    \label{fig:murine_ltp_centrality_boxplot_degree1}
\end{figure}





\begin{figure}
    \centering
    \includegraphics[width=\textwidth]{images/chapter3/ggplot2/murine_centrality_boxplot/add_theme/addLTP/Rplot_betweenness_edit.png}
    \caption{Box plot of betweenness centrality distribution for differenet gene sets affecting Murine LTP or learning. Log 10 scale for betweenness} 
    \tiny\url{source('~/RProjects/chapter3/R/murine_centrality/general_murine_boxplot/rough_centrality_boxplot_murine_ltp_general_closeness.R'}
    \label{fig:murine_ltp_centrality_boxplot_betweenness}
\end{figure}

% latex table generated in R 3.6.3 by xtable 1.8-4 package
% Sat Mar 27 14:40:21 2021
% latex table generated in R 3.6.3 by xtable 1.8-4 package
% Sat Mar 27 14:45:29 2021
% latex table generated in R 3.6.3 by xtable 1.8-4 package
% Sat Mar 27 15:47:08 2021
\begin{table}[ht]
\centering
\begin{tabular}{rlrrll}
  \toprule
 & phenotype & $p$ & $p$ adj & $p$ signif & method \\ 
  \midrule
1 & abnormal spatial learning & 0.002 & 0.020 & ** & Wilcoxon \\ 
  2 & abnormal temporal memory & 0.042 & 0.340 & * & Wilcoxon \\ 
  3 & abn context condition & 0.114 & 0.570 & ns & Wilcoxon \\ 
  4 & reduced long term potentiation & 0.363 & 1.000 & ns & Wilcoxon \\ 
  5 & abnormal LTDep & 0.750 & 1.000 & ns & Wilcoxon \\ 
  6 & abn assoc learning & 0.058 & 0.410 & ns & Wilcoxon \\ 
  7 & abnormal associative learning & 0.058 & 0.410 & ns & Wilcoxon \\ 
  8 & abnormal LTP & 0.271 & 1.000 & ns & Wilcoxon \\ 
  9 & abnormal synaptic plasticity & 0.946 & 1.000 & ns & Wilcoxon \\ 
  10 & abnormal cognition & 0.003 & 0.026 & ** & Wilcoxon \\ 
   \bottomrule
\end{tabular}
\caption{Mean betweenness centrality in gene sets with murine cognitive phenotypes. Comparison of mean betweenness Wilcoxon test compared with full PSP. P adjusted using Benjamini-Hochberg procedure} 
\label{tab:betweenness murine phenotypes. Wilcoxon test compared with full PSP. P adjusted using B.H}
\end{table}


\begin{figure}
    \centering
    \includegraphics[width=\textwidth]{images/chapter3/ggplot2/murine_centrality_boxplot/add_theme/addLTP/Rplot_closeness_editnolog10.png}
    \caption{Box plot of Closeness centrality distribution for differenet gene sets affecting Murine LTP or learning. Closeness not logarithimic} 
   \tiny\url{source('~/RProjects/chapter3/R/murine_centrality/general_murine_boxplot/by_measure/degree_centrality_boxplot_murine_ltp_general_closeness_add_table.R')}
    \label{fig:murine_ltp_centrality_boxplot_closeness}
\end{figure}

% latex table generated in R 3.6.3 by xtable 1.8-4 package
% Sat Mar 27 15:39:06 2021
\begin{table}[ht]
\centering
\begin{tabular}{rlrrll}
  \toprule
 & phenotype & p & p adj & p signif & method \\ 
  \midrule
1 & abnormal spatial learning & 0.608 & 1.000 & ns & Wilcoxon \\ 
  2 & abnormal temporal memory & 0.539 & 1.000 & ns & Wilcoxon \\ 
  3 & abn context condition & 0.273 & 1.000 & ns & Wilcoxon \\ 
  4 & reduced long term potentiation & 0.088 & 0.610 & ns & Wilcoxon \\ 
  5 & abnormal LTDep & 0.002 & 0.018 & ** & Wilcoxon \\ 
  6 & abn assoc learning & 0.233 & 1.000 & ns & Wilcoxon \\ 
  7 & abnormal associative learning & 0.233 & 1.000 & ns & Wilcoxon \\ 
  8 & abnormal LTP & 0.047 & 0.370 & * & Wilcoxon \\ 
  9 & abnormal synaptic plasticity & 0.008 & 0.070 & ** & Wilcoxon \\ 
  10 & abnormal cognition & 0.565 & 1.000 & ns & Wilcoxon \\ 
   \bottomrule
\end{tabular}
\caption{Closeness murine phenotypes. Wilcoxon test of mean compared with all PSP genes. P adjusted using Benjamini-Hochberg procedure} 
\label{tab:closeness murine phenotypes. Wilcoxon test compared with full PSP.}
\end{table}

% \begin{figure}
%     \centering
%     \includegraphics[width=\textwidth]{images/chapter3/ggplot2/murine_centrality_boxplot/add_theme/addLTP/Rplot_theme_add_ltp_corrected_spelling_betweenness.png}
%     \caption{Box plot of Betweenness distribution for differenet gene sets affecting Murine LTP or learning. Log 10 scale for degree.} 
%     \tiny\url{source('~/RProjects/chapter3/R/murine_centrality/general_murine_boxplot/rough_centrality_boxplot_murine_ltp_general_betweenness_add_table.R')}
%     \label{fig:murine_ltp_centrality_boxplot_centrality}
% \end{figure}

% latex table generated in R 3.6.3 by xtable 1.8-4 package
% Fri Mar 26 19:13:36 2021




\begin{figure}
    \centering
    \includegraphics[width=\textwidth]{images/chapter3/ggplot2/murine_centrality_boxplot/add_theme/addLTP/Rplot_eigen_edit.png}
    \caption{Box plot of Eigenvector centrality distribution for differenet gene sets affecting Murine LTP or learning. Log10} 
   \tiny\url{source('~/RProjects/chapter3/R/murine_centrality/general_murine_boxplot/by_measure/degree_centrality_boxplot_murine_ltp_general_eigenvector_add_table.R')}
    \label{fig:murine_ltp_centrality_boxplot_eigenvector}
\end{figure}


% latex table generated in R 3.6.3 by xtable 1.8-4 package
% Sat Mar 27 15:56:24 2021
\begin{table}[ht]
\centering
\begin{tabular}{rlrrll}
  \toprule
 & phenotype & p & p.adj & p.signif & method \\ 
  \midrule
1 & abnormal spatial learning & 0.926 & 1.000 & ns & Wilcoxon \\ 
  2 & abnormal temporal memory & 0.592 & 1.000 & ns & Wilcoxon \\ 
  3 & abn context condition & 0.321 & 1.000 & ns & Wilcoxon \\ 
  4 & reduced long term potentiation & 0.053 & 0.370 & ns & Wilcoxon \\ 
  5 & abnormal LTDep & 0.002 & 0.021 & ** & Wilcoxon \\ 
  6 & abn assoc learning & 0.229 & 1.000 & ns & Wilcoxon \\ 
  7 & abnormal associative learning & 0.229 & 1.000 & ns & Wilcoxon \\ 
  8 & abnormal LTP & 0.012 & 0.097 & * & Wilcoxon \\ 
  9 & abnormal synaptic plasticity & 0.002 & 0.021 & ** & Wilcoxon \\ 
  10 & abnormal cognition & 0.338 & 1.000 & ns & Wilcoxon \\ 
   \bottomrule
\end{tabular}
\caption{Mean eigenvector centrality for murine phenotypes. Wilcoxon test compared with full PSP. P adjusted using Benjamini-Hochberg Procedure} 
\label{tab:eigencentrality murine phenotypes. Wilcoxon test compared with full PSP}
\end{table}


\clearpage



\section{Characterisation of central genes}

To investigate whether central genes share common properties I carried out gene ontology enrichment of the top and bottom 10\% of genes for each centrality measures. Gene ontology enrichment was carried out using the 3,457 genes in the PSP as background so the results show enrichment \textit{compared to the rest of the PSP}. 


The most central genes shared a number of common features regardless of the centrality measure. The ten most enriched terms for high degree nodes in terms of degree are shown for biological process table~\ref{tab:ToppGENE GO: Biological Process. 90 centile cwpsp.txtp = p value; q FDR B H = q adjusted significance level False Discovery Rate using Benjamini and Hochberg adjustment; n= n genes annotated in test group; n PSP= n genes annotated in PSP. n significant in category 1584} , cellular component table~\ref{tab:ToppGENE GO: Cellular Component. 90 centile cwpsp.txtp = p value; q FDR B H = q adjusted significance level False Discovery Rate using Benjamini and Hochberg adjustment; n= n genes annotated in test group; n PSP= n genes annotated in PSP. n significant in category 159} and molecular function (table~\ref{tab:ToppGENE GO: Cellular Component. 90 centile cwpsp.txtp = p value; q FDR B H = q adjusted significance level False Discovery Rate using Benjamini and Hochberg adjustment; n= n genes annotated in test group; n PSP= n genes annotated in PSP. n significant in category 159}). 

Enriched biological processes, table~\ref{tab:ToppGENE GO: Biological Process. 90 centile cwpsp.txtp = p value; q FDR B H = q adjusted significance level False Discovery Rate using Benjamini and Hochberg adjustment; n= n genes annotated in test group; n PSP= n genes annotated in PSP. n significant in category 1584}, include terms related to the cell cycle associated with low FDR $q$ values. Molecular function terms include RNA binding and those related to transcription (table~\ref{tab:ToppGENE GO: Molecular Function. 90 centile cwpsp.txtp = p value; q FDR B H = q adjusted significance level False Discovery Rate using Benjamini and Hochberg adjustment; n= n genes annotated in test group; n PSP= n genes annotated in PSP. n significant in category 186}). Cellular component terms, table~\ref{tab:ToppGENE GO: Cellular Component. 90 centile cwpsp.txtp = p value; q FDR B H = q adjusted significance level False Discovery Rate using Benjamini and Hochberg adjustment; n= n genes annotated in test group; n PSP= n genes annotated in PSP. n significant in category 159}, shows enrichment for focal adhesion, cell junction terms, ribonucleoprotein and cytoskeletal fibre. 



\begin{table}[ht]
\centering
\begin{adjustbox}{width=\textwidth}
\setlength{\extrarowheight}{2pt}
\begin{tabular}{@{}clllcl@{}}
  \toprule
  ID & Biological Process & $n$ & $n$ PSP & $p$ & $q$ FDR B H \\ 

  \midrule
GO:0007049 & cell cycle & 139 & 489 & $1.98 \times 10^{-36}$ & $1.25 \times 10^{-32}$ \\ 
  GO:0022402 & cell cycle process & 119 & 414 & $7.10 \times 10^{-31}$ & $2.25 \times 10^{-27}$ \\ 
  GO:0000280 & nuclear division & 101 & 319 & $1.43 \times 10^{-29}$ & $3.02 \times 10^{-26}$ \\ 
  GO:0048285 & organelle fission & 103 & 337 & $9.65 \times 10^{-29}$ & $1.53 \times 10^{-25}$ \\ 
  GO:0000278 & mitotic cell cycle & 95 & 296 & $3.44 \times 10^{-28}$ & $3.10 \times 10^{-25}$ \\ 
  GO:0140014 & mitotic nuclear division & 95 & 296 & $3.44 \times 10^{-28}$ & $3.10 \times 10^{-25}$ \\ 
  GO:1903047 & mitotic cell cycle process & 95 & 296 & $3.44 \times 10^{-28}$ & $3.10 \times 10^{-25}$ \\ 
  GO:0044403 & symbiotic process & 106 & 364 & $1.23 \times 10^{-27}$ & $9.33 \times 10^{-25}$ \\ 
  GO:0051726 & regulation of cell cycle & 100 & 329 & $1.33 \times 10^{-27}$ & $9.33 \times 10^{-25}$ \\ 
  GO:0044419 & interspecies interaction between organisms & 106 & 368 & $3.48 \times 10^{-27}$ & $2.20 \times 10^{-24}$ \\ 
   \bottomrule
\end{tabular}
\end{adjustbox}
\caption[Gene ontology enrichment Biological Process of genes above 90th centile of degree distribution]{\textbf{High degree genes. Biological Process - Gene ontology enrichment analysis} using the ToppGene web client and all PSP genes as the background set.  Biological process tested gene set is 90th centile of degree distribution.  p value; q FDR B H = q adjusted significance level False Discovery Rate using Benjamini and Hochberg adjustment; $n$= $n$ genes annotated in test group; $n$ PSP= n genes annotated in PSP. $n$ ontology terms significantly (q FDR$>=0.05$) enriched for this centrality measure and ontology category (e.g. MF,CC,BP): 1584} 

\label{tab:ToppGENE GO: Biological Process. 90 centile cwpsp.txtp = p value; q FDR B H = q adjusted significance level False Discovery Rate using Benjamini and Hochberg adjustment; n= n genes annotated in test group; n PSP= n genes annotated in PSP. n significant in category 1584}
\end{table}


\begin{table}[ht]
\centering
\begin{adjustbox}{width=\textwidth}
\setlength{\extrarowheight}{2pt}

\begin{tabular}{@{}clllcl@{}}
  \toprule
ID & Molecular Function & $n$ & $n$ PSP & $p$ & $q$ FDR B H \\ 
  \midrule
GO:0031625 & ubiquitin protein ligase binding & 51 & 117 & $8.81 \times 10^{-22}$ & $9.12 \times 10^{-19}$ \\ 
  GO:0044389 & ubiquitin-like protein ligase binding & 52 & 123 & $1.88 \times 10^{-21}$ & $9.74 \times 10^{-19}$ \\ 
  GO:0008134 & transcription factor binding & 58 & 156 & $1.53 \times 10^{-20}$ & $5.26 \times 10^{-18}$ \\ 
  GO:0019904 & protein domain specific binding & 87 & 329 & $3.13 \times 10^{-19}$ & $8.11 \times 10^{-17}$ \\ 
  GO:0003723 & RNA binding & 120 & 554 & $5.07 \times 10^{-19}$ & $1.05 \times 10^{-16}$ \\ 
  GO:0044877 & protein-containing complex binding & 111 & 513 & $2.27 \times 10^{-17}$ & $3.91 \times 10^{-15}$ \\ 
  GO:0003720 & telomerase activity & 21 & 30 & $5.04 \times 10^{-15}$ & $7.00 \times 10^{-13}$ \\ 
  GO:0003964 & RNA-directed DNA polymerase activity & 22 & 33 & $5.41 \times 10^{-15}$ & $7.00 \times 10^{-13}$ \\ 
  GO:0034061 & DNA polymerase activity & 23 & 37 & $1.23 \times 10^{-14}$ & $1.41 \times 10^{-12}$ \\ 
  GO:0019900 & kinase binding & 78 & 325 & $1.50 \times 10^{-14}$ & $1.55 \times 10^{-12}$ \\ 
   \bottomrule
\end{tabular}
\end{adjustbox}
\caption[Gene ontology Molecular Function enrichment of genes above 90th centile of degree distribution]{\textbf{High degree genes. Molecular function - Gene ontology enrichment analysis} using the ToppGene web client and all PSP genes as the background set.  Molecular Function tested gene set is 90th centile of degree distribution.  p value; q FDR B H = q adjusted significance level False Discovery Rate using Benjamini and Hochberg adjustment; $n$= $n$ genes annotated in test group; $n$ PSP= n genes annotated in PSP. $n$ ontology terms significantly (q FDR$>=0.05$) enriched for this centrality measure and ontology category (e.g. MF,CC,BP): 186} 
\label{tab:ToppGENE GO: Molecular Function. 90 centile cwpsp.txtp = p value; q FDR B H = q adjusted significance level False Discovery Rate using Benjamini and Hochberg adjustment; n= n genes annotated in test group; n PSP= n genes annotated in PSP. n significant in category 186}
\end{table}

% Sun Oct  4 15:33:48 2020
\begin{table}[ht]
\centering
\begin{adjustbox}{width=\textwidth}
\setlength{\extrarowheight}{2pt}
\begin{tabular}{@{}clllcl@{}}
  \toprule
  ID & Cellular Component & $n$ & $n$ PSP & $p$ & $q$ FDR B H \\ 

  \midrule

GO:1990904 & ribonucleoprotein complex & 81 & 303 & $2.75 \times 10^{-18}$ & $2.36 \times 10^{-15}$ \\ 
  GO:0015630 & microtubule cytoskeleton & 102 & 447 & $1.01 \times 10^{-17}$ & $4.33 \times 10^{-15}$ \\ 
  GO:0036464 & cytoplasmic ribonucleoprotein granule & 33 & 74 & $8.92 \times 10^{-15}$ & $2.55 \times 10^{-12}$ \\ 
  GO:0035770 & ribonucleoprotein granule & 33 & 78 & $5.84 \times 10^{-14}$ & $1.25 \times 10^{-11}$ \\ 
  GO:1902494 & catalytic complex & 86 & 392 & $1.07 \times 10^{-13}$ & $1.84 \times 10^{-11}$ \\ 
  GO:0030055 & cell-substrate junction & 62 & 244 & $7.12 \times 10^{-13}$ & $1.02 \times 10^{-10}$ \\ 
  GO:0005925 & focal adhesion & 61 & 239 & $9.13 \times 10^{-13}$ & $1.12 \times 10^{-10}$ \\ 
  GO:0000974 & Prp19 complex & 19 & 31 & $3.66 \times 10^{-12}$ & $3.92 \times 10^{-10}$ \\ 
  GO:0099513 & polymeric cytoskeletal fiber & 95 & 482 & $4.16 \times 10^{-12}$ & $3.96 \times 10^{-10}$ \\ 
  GO:0005815 & microtubule organizing center & 60 & 246 & $1.21 \times 10^{-11}$ & $1.04 \times 10^{-9}$ \\ 
   \hline
\end{tabular}
\end{adjustbox}
\caption[Gene ontology enrichment Cellular Component of genes above 90th centile of degree distribution]{\textbf{High degree genes. Cellular Component - Gene ontology enrichment analysis} using the ToppGene web client and all PSP genes as the background set.  Biological process tested gene set is 90th centile of degree distribution.  p value; q FDR B H = q adjusted significance level False Discovery Rate using Benjamini and Hochberg adjustment; $n$= $n$ genes annotated in test group; $n$ PSP= n genes annotated in PSP. $n$ ontology terms significantly (q FDR$>=0.05$) enriched for this centrality measure and ontology category (e.g. MF,CC,BP): 159} 
\label{tab:ToppGENE GO: Cellular Component. 90 centile cwpsp.txtp = p value; q FDR B H = q adjusted significance level False Discovery Rate using Benjamini and Hochberg adjustment; n= n genes annotated in test group; n PSP= n genes annotated in PSP. n significant in category 159}
\end{table}


 Eigenvector centrality (table~\ref{tab:ToppGENE GO: Biological Process. 90 centile cw psp eig.txtp = p value; q FDR B H = q adjusted significance level False Discovery Rate using Benjamini and Hochberg adjustment; n= n genes annotated in test group; n PSP= n genes annotated in PSP} results in particularly strong enrichment for RNA binding (Molecular Function GO:0003723, $q$ FDR B H 7.01$\times10^{-76}$) as does kcoreness (table~\ref{tab:ToppGENE GO: Molecular Function. kco 90 centile cwpsp.txtp = p value; q FDR B H = q adjusted significance level False Discovery Rate using Benjamini and Hochberg adjustment; n= n genes annotated in test group; n PSP= n genes annotated in PSP. n significant in category 180} $q$ FDR=2.68$\times10^{-55}$). Transitivity shows far less enrichment than the other centrality terms possible reflecting its weaker correlation. It is interesting to note however that it seems quite specific with a large \textit{proportion} of the genes involved in mitochondrial translation and ribosomal function identified (Molecular function table~\ref{tab:ToppGENE GO: Molecular Function. tra 90 centile NA narm cwpsp.txtp = p value; q FDR B H = q adjusted significance level False Discovery Rate using Benjamini and Hochberg adjustment; n= n genes annotated in test group; n PSP= n genes annotated in PSP. n significant in category 2}; 29 of 103 genes in PSP identified, fold change 9.4). 


Gene ontology enrichment was carried out using the ToppGene package and the PSP as a background set. These are therefore sets of genes that differ from the background enrichment of the PSP. Given that fact it is remarkable how many terms are enriched even after correcting for multiple comparisons (186 for example in the top 10\% of degree for molecular function table~\ref{tab:ToppGENE GO: Molecular Function. 90 centile cwpsp.txtp = p value; q FDR B H = q adjusted significance level False Discovery Rate using Benjamini and Hochberg adjustment; n= n genes annotated in test group; n PSP= n genes annotated in PSP. n significant in category 186}) and 1,548 in Biological Process table~\ref{tab:ToppGENE GO: Biological Process. 90 centile cwpsp.txtp = p value; q FDR B H = q adjusted significance level False Discovery Rate using Benjamini and Hochberg adjustment; n= n genes annotated in test group; n PSP= n genes annotated in PSP. n significant in category 1584}.

By contrast fewer terms are enriched in low centrality gene sets. For degree only two terms in cellular component were enriched (table~\ref{tab:ToppGENE GO: Cellular Component. 10 centile cwpsp.txtp = p value; q FDR B H = q adjusted significance level False Discovery Rate using Benjamini and Hochberg adjustment; n= n genes annotated in test group; n PSP= n genes annotated in PSP}). In part this is due to the fact that the distribution of centrality functions such as degree, betweenness and eigenvector centrality is right skewed. Closeness which is bounded in the region 0-1 and the most approximately normally distributed provides more enrichment (table~\ref{tab:ToppGENE GO: Cellular Component. clo 10 centile cwpsp.txtp = p value; q FDR B H = q adjusted significance level False Discovery Rate using Benjamini and Hochberg adjustment; n= n genes annotated in test group; n PSP= n genes annotated in PSP} and table~\ref{tab:ToppGENE GO: Molecular Function. clo 10 centile cwpsp.txtp = p value; q FDR B H = q adjusted significance level False Discovery Rate using Benjamini and Hochberg adjustment; n= n genes annotated in test group; n PSP= n genes annotated in PSP}). These are genes one would expect on the periphery of the network and the synapse, those of the receptors themselves, intrinsic components of the plasma membrane and components of the extra cellular matrix. Kcoreness which has a relatively uniform distribution over a small range also has greater enrichment of low centrality terms(table~\ref{tab:ToppGENE Pathway. kco 10 centile cwpsp.txtp = p value; q FDR B H = q adjusted significance level False Discovery Rate using Benjamini and Hochberg adjustment; n= n genes annotated in test group; n PSP= n genes annotated in PSP. n significant in category 8}). Low eigenvector centrality genes also have common themes related to the plasma and post synaptic membrane (table~\ref{tab:ToppGENE GO: Molecular Function. 10 centile cw psp eig.txtp = p value; q FDR B H = q adjusted significance level False Discovery Rate using Benjamini and Hochberg adjustment; n= n genes annotated in test group; n PSP= n genes annotated in PSP} and table~\ref{tab:ToppGENE GO: Cellular Component. 10 centile cw psp eig.txtp = p value; q FDR B H = q adjusted significance level False Discovery Rate using Benjamini and Hochberg adjustment; n= n genes annotated in test group; n PSP= n genes annotated in PSP})

\begin{table}[ht]
\centering
\begin{adjustbox}{width=\textwidth}
\setlength{\extrarowheight}{2pt}
\begin{tabular}{@{}clllcl@{}}
  \toprule
  ID & Pathway & $n$ & $n$ PSP & $p$ & $q$ FDR B H \\ 

  \midrule
M5889 & \makecell{Ensemble of genes encoding extracellular matrix\\ and extracellular matrix-associated proteins}  & 37 & 91 & $1.50 \times 10^{-10}$ & $2.24 \times 10^{-7}$ \\ 
  M5884 & \makecell{Ensemble of genes encoding core extracellular matrix\\ including ECM glycoproteins, collagens and proteoglycans} & 17 & 31 & $8.53 \times 10^{-8}$ & $6.38 \times 10^{-5}$ \\ 
  M3008 & Genes encoding structural ECM glycoproteins & 13 & 25 & $6.70 \times 10^{-6}$ & $3.34 \times 10^{-3}$ \\ 
  132956 & Metabolic pathways & 74 & 353 & $6.99 \times 10^{-5}$ & $2.05 \times 10^{-2}$ \\ 
  82989 & Glycerophospholipid metabolism & 12 & 26 & $7.07 \times 10^{-5}$ & $2.05 \times 10^{-2}$ \\ 
  M5885 & \makecell{Ensemble of genes encoding ECM-associated proteins including\\ ECM-affilaited proteins, ECM regulators and secreted factors} & 20 & 60 & $9.51 \times 10^{-5}$ & $2.05 \times 10^{-2}$ \\ 
  M9131 & Glycerophospholipid metabolism & 11 & 23 & $9.57 \times 10^{-5}$ & $2.05 \times 10^{-2}$ \\ 
  1268710 & Post-translational modification: synthesis of GPI-anchored proteins & 7 & 11 & $1.96 \times 10^{-4}$ & $3.66 \times 10^{-2}$ \\ 
   \hline
\end{tabular}
\end{adjustbox}
\caption[Gene ontology enrichment Low kcoreness genes Pathway of genes above 90th centile of distribution]{\textbf{Low kcoreness genes. Pathway - Gene ontology enrichment analysis} using the ToppGene web client and all PSP genes as the background set.  Pathway tested gene set is 10th centile of degree distribution.  p value; q FDR B H = q adjusted significance level False Discovery Rate using Benjamini and Hochberg adjustment; $n$= $n$ genes annotated in test group; $n$ PSP= n genes annotated in PSP. $n$ ontology terms significantly (q FDR$>=0.05$) enriched for this centrality measure and ontology category (e.g. MF,CC,BP): 8} 
% \caption{ToppGene Pathway. kco 10 centile cwpsp.txtp = p value; q FDR B H = q adjusted significance level False Discovery Rate using Benjamini and Hochberg adjustment; n= n genes annotated in test group; n PSP= n genes annotated in PSP. n significant in category 8} 
\label{tab:ToppGENE Pathway. kco 10 centile cwpsp.txtp = p value; q FDR B H = q adjusted significance level False Discovery Rate using Benjamini and Hochberg adjustment; n= n genes annotated in test group; n PSP= n genes annotated in PSP. n significant in category 8}
\end{table}



\subsubsection{Low degree}
The only gene ontology class enriched was cellular component (table~\ref{tab:ToppGENE GO: Cellular Component. 10 centile cwpsp.txtp = p value; q FDR B H = q adjusted significance level False Discovery Rate using Benjamini and Hochberg adjustment; n= n genes annotated in test group; n PSP= n genes annotated in PSP}) but the enriched terms are common in other low vaules of centrality measures in that it is enriched for components of the plasma membrane. Pathway enrichment shows enrichment of the extracellular matrix and ECM proteins which is found with other low centrality measure genes. The results for high and low degree genes appear qualitatively different even compared to other post synaptic protein genes (ie cell cycle processes verus ECM). The network architecture seems to be related to function.



% latex table generated in R 3.6.3 by xtable 1.8-4 package 
% Sat Oct  3 17:17:17 2020
\begin{table}[ht]
\centering

\begin{adjustbox}{width=\textwidth}
\setlength{\extrarowheight}{2pt}
\begin{tabular}{@{}clllcl@{}}
  \toprule
  ID & Cellular Component & $n$ & $n$ PSP &$p$ & $q$ FDR B H   \\ 
  \midrule
GO:0031226 & intrinsic component of plasma membrane & 51 & 331 & $1.564 \times 10^{-5}$ & $7.462 \times 10^{-3}$ \\ 
  GO:0005887 & integral component of plasma membrane  & 47 & 307& $4.228 \times 10^{-5}$ & $1.008 \times 10^{-2}$ \\ 
   \bottomrule
\end{tabular}
\end{adjustbox}
\caption{ToppGene GO: Cellular Component. 10 centile cwpsp.txtp = p value; q FDR B H = q adjusted significance level False Discovery Rate using Benjamini and Hochberg adjustment; n= n genes annotated in test group; n PSP= n genes annotated in PSP. } 
\label{tab:ToppGENE GO: Cellular Component. 10 centile cwpsp.txtp = p value; q FDR B H = q adjusted significance level False Discovery Rate using Benjamini and Hochberg adjustment; n= n genes annotated in test group; n PSP= n genes annotated in PSP}
\end{table}


Pathway table~\ref{tab:ToppGENE Pathway. 10 centile cwpsp.txtp = p value; q FDR B H = q adjusted significance level False Discovery Rate using Benjamini and Hochberg adjustment; n= n genes annotated in test group; n PSP= n genes annotated in PSP} 

% latex table generated in R 3.6.3 by xtable 1.8-4 package
% Sat Oct  3 17:21:16 2020
\begin{table}[ht]
\centering
\begin{adjustbox}{width=\textwidth}
\setlength{\extrarowheight}{2pt}
\begin{tabular}{@{}clllcl@{}}

  \toprule
ID & Pathway & p & q FDR B H & n & n PSP \\ 
  \midrule
M5889 & \makecell{Ensemble of genes encoding extracellular matrix and\\ extracellular matrix-associated proteins}  & $5.689 \times 10^{-6}$ & $5.831 \times 10^{-3}$ & 19 & 91 \\ 
  1269590 & Adenylate cyclase activating pathway & $9.603 \times 10^{-5}$ & $4.798 \times 10^{-2}$ & 4 & 5 \\ 
  1269923 &\makecell{ Transport of glucose and other sugars, bile salts and organic acids,\\ metal ions and amine compounds} & $2.551 \times 10^{-4}$ & $4.798 \times 10^{-2}$ & 5 & 10 \\ 
  1272485 & Aldosterone synthesis and secretion & $2.896 \times 10^{-4}$ & $4.798 \times 10^{-2}$ & 9 & 34 \\ 
  1269930 & Metal ion SLC transporters & $3.277 \times 10^{-4}$ & $4.798 \times 10^{-2}$ & 3 & 3 \\ 
   \bottomrule
\end{tabular}
\end{adjustbox}
\caption{ToppGene Pathway. 10 centile cwpsp.txtp = p value; q FDR B H = q adjusted significance level False Discovery Rate using Benjamini and Hochberg adjustment; n= n genes annotated in test group; n PSP= n genes annotated in PSP} 
\tiny\url{source('~/RProjects/chapter3/R/sig_magma_genes/central_genes_toppgene/betweenness/5_0_clip_centrality_bet_hig_mf.R')}

\label{tab:ToppGENE Pathway. 10 centile cwpsp.txtp = p value; q FDR B H = q adjusted significance level False Discovery Rate using Benjamini and Hochberg adjustment; n= n genes annotated in test group; n PSP= n genes annotated in PSP}
\end{table}


 \section{Essentialness and centrality}
\subsubsection{The PSP is over-represented in essential genes}
\label{sec:results over representation PSP in DEG genes}
Genes in the post synaptic proteome are over-represented amongst essential genes as defined in the DEG (section~\ref{sec:Database of essential genes}). The contingency table~\ref{tab:DEG any appearance contingency table PSP vs not PSP} shows whether genes are in the PSP and if they appear once in the DEG. The PSP had 2,328 genes that appeared at least once in the DEG (    $n$ genes in PSP = 3457 ; 67.3\%) compared with 6,664 non PSP genes (41.7\% of 15,970 non synaptic genes appearing in NCBI 37).








\begin{table}
\centering
\begin{tabular}{cccc}
\toprule
& \multicolumn{2}{c}{Essential} & \\
\cmidrule{2-3}
    PSP & Essential &  Non-Essential & Total by PSP\vspace{1mm} \\
\midrule 
PSP      &   2,328     &     1,129 & 3,457\vspace{1mm}\\
not PSP    &  6,664      &    9,306  & 15,970\vspace{1mm}\\
\midrule
 Total by essential & 8,992 & 10,435 &  19,427\\ 

 
\bottomrule
\end{tabular}
\caption{Contingency table showing essential genes where essentialness is appearing once in the Database of Essential Genes (DEG). Total genes (19427).  Fisher's exact test p$<2.2\times10^{-16}$. OR 2.88 95\% CI 2.66 3.12; OR 2.92. Results checked Mar}
\tiny\url{source('~/RProjects/db_essential_genes/R/get_essential_genes/db_2020/findforfive/get_db_essential_greater_five_results_xtable_transpose.R')}  
\label{tab:DEG any appearance contingency table PSP vs not PSP}
\end{table}


Using the stricter definition of essentialness, appearing in five or more DEG studies (which aims to represent the estimated 10\% of human genes that are essential; section~\ref{sec:Database of essential genes}) $n$ = 728 (21.1\%) of PSP genes appear in five or more experiments in the DEG. By contrast the figure for non PSP genes is $n$=1,450 (9.99\%) (OR 2.67 Fisher's exact test table~\ref{tab:DEG five or more appearances in contingency table PSP vs not PSP}). 



   \begin{table}
\centering
\begin{tabular}{cccc}
\toprule
& \multicolumn{2}{c}{Essential} & \\
\cmidrule{2-3}
    PSP & Essential &  Non-Essential & Total by PSP\vspace{1mm} \\
\midrule 
PSP      &   728     &     2,729 & 3,457\vspace{1mm}\\
not PSP    &  1,450      &    14,520  & 15,970\vspace{1mm}\\
\midrule
 Total by essential & 2,178 & 10,435 &  19,427\\ 

 
\bottomrule
\end{tabular}
\caption{Contingency table showing essential genes where essentialness is appearing five or more times  in the Database of Essential Genes (DEG). Total genes (19427).  Fisher's exact test p$<2.2\times10^{-16}$. OR 2.67 ;95\% CI 2.42 - 2.95; OR 2.67. Results checked Mar}
\tiny\url{source('~/RProjects/db_essential_genes/R/get_essential_genes/db_2020/findforfive/get_db_essential_greater_five_results_xtable_transpose.R')}  
\label{tab:DEG five or more appearances in contingency table PSP vs not PSP}
\end{table}
    
Fisher's exact test  for over representation in synaptic genes for any appearance in the DEG (OR 2.879; 2.663-3.115,p$<2.2\times10^{-16}$,table~\ref{tab:DEG five or more appearances in contingency table PSP vs not PSP}) and more than five appearances in the DEG (OR 2.671; 2.418-2.949) are shown in table (table~\ref{tab:Fishers exact results DEG}).

PSP genes appear to be significantly more likely to be essential (table~\ref{tab:Fishers exact results DEG}).

% latex table generated in R 3.6.3 by xtable 1.8-4 package
% Thu Mar 18 14:19:31 2021
\begin{table}[ht]
\centering
\begin{tabular}{lllll}

  \toprule
Study & Odds ratio & 95\% CI Lower & 95\%CI Upper & p \\ 
  \midrule
Any DEG vs none & 2.879 & 2.663 & 3.115 & $1.968 \times 10^{-166}$\vspace{1mm} \\ 
  Greater than 5 DEG v none & 2.671 & 2.418 & 2.949 & $6.162 \times 10^{-79}$\vspace{1mm} \\ 
   \bottomrule
\end{tabular}
\caption{Fishers exact test for over representation of synaptic genes in DEG (database of essential genes). First is any appearance in DEG vs genes not in DEG. Second is those in five or more (see methods)Confidence intervals and significance for over representation of PSP proteins in Database of Essential Genes (DEG)}
\tiny\textsuperscript{source=} \url{source('~/RProjects/db_essential_genes/R/get_essential_genes/db_2020/findforfive/get_db_essential_greater_five_results_xtable_transpose.R')}
\label{tab:Fishers exact results DEG}
\end{table}





The PSP genes with higher centrality measures are more likely to be essential, evidence in humans for centrality lethality effects (section~\ref{sec:Degree and essentialness}) . Genes in five or more studies in the DEG have the highest centrality measures, then those appearing in one or more and the lowest mean centrality is seen in those PSP genes that are not in the DEG  as can be seen in the boxplots in figure~\ref{sec:multiplot of boxplot of centralities and DEG}. Degree, eigenvector centrality and betweenness centrality are shown with a log10 scale. The difference in the median values is clear even in the logarithmic plot. 


The correlation (Spearman's $\rho$) between node centrality and the number of studies in the DEG that the gene appears in is shown in table~\ref{tab:Correlation centrality and num studies DEG} . Higher centrality genes tend to appear in more of the DEG studies, the relationship is strongest for eigenvector centrality ($\rho=0.398$) and weakest for transitivity $\rho=0.207$. The betweenness centrality is the second smallest (transitivity often has weak effects compared to other centrality measures) $\rho=0.243$.




% above from https://tex.stackexchange.com/questions/84527/include-centered-full-page-figure-with-no-margins

\begin{figure}[p]
    \vspace*{-2cm}
    \makebox[\linewidth]{
        \includegraphics[width=1.3\linewidth, height=20cm]{images/chapter3/ggplot2/db_essential_genes/theme/Rplot_multiplot_with_theme_simple.png}}
    \caption{Multiplot of boxplot of centralities and DEG. On the x axis categories are non essential, Up to 5 appearances in DEG, 5 or more appearances in DEG }
    \label{sec:multiplot of boxplot of centralities and DEG}
    \tiny\url{ source('~/RProjects/db_essential_genes/R/get_essential_genes/db_2020/theme/get_db_essential_threecategoriesboxplot_multiplot_no_x_lab_essential_centrality_theme.R')}
\end{figure}

% latex table generated in R 3.6.3 by xtable 1.8-4 package
% Fri Feb 26 14:53:35 2021
\begin{table}[ht]
\centering
\begin{tabular}{lcrc}
  \toprule
Centrality Measure & S & rho & p \\ 
  \midrule
 Degree  & $1.401 \times 10^{9}$ & 0.334 & $1.265 \times 10^{-61}$ \\ 
 Eigenvector   & $1.266 \times 10^{9}$ & 0.398 & $3.479 \times 10^{-89}$ \\ 
 Betweenness centrality   & $1.591 \times 10^{9}$ & 0.243 & $9.620 \times 10^{-33}$ \\ 
   Closeness   & $1.341 \times 10^{9}$ & 0.362 & $4.825 \times 10^{-73}$ \\ 
  TransitivityNA   & $1.378 \times 10^{9}$ & 0.207 & $1.233 \times 10^{-22}$ \\ 
  kcoreness   & $1.336 \times 10^{9}$ & 0.365 & $4.081 \times 10^{-74}$ \\ 
   \bottomrule
\end{tabular}
\caption{Correlation of centrality measures and number of studies (max 20) found in DEG. Correlation type spearman.}
\tiny\url{source('~/RProjects/db_essential_genes/R/get_essential_genes/db_2020/findforfive/clean_up/get_db_essential_five_or_more_ttests_essential_centrality_correlation_count_centrality.R')}
\label{tab:Correlation centrality and num studies DEG}
\end{table}


\section{Probability of loss of function intolerance (pLI)}
\label{sec: results pLI}


The probability of loss of function intolerance is associated with measures of vertex centrality in the post synaptic proteome graph. The correlation is greatest for degree ($\rho=0.230$) and lowest for transitivity ($\rho=0.086$). Betweenness and eigenvector centrality have similar correlation to loss of function intolerance as degree (see table \ref{tab:correlation pli vertex statistics mar})

\subsection{Distribution of pLI}
\label{sec:results distribution of pli}
The distribution of the probability of loss of function intolerance based on Gnomad v2.1.1 (section~\ref{sec:methods exac and gnomad} is shown as a stacked histogram in  figure~\ref{fig:hist_pliPSP_theme} and as a density plot in figure~\ref{fig:densityPLI_theme}. Both plots show a bimodal distribution of probability of loss of function intolerance with the probability mass centred at extreme high and low values and then a low level of probability mass between these extremes. The PSP genes (green in figure~\ref{fig:hist_pliPSP_theme} and the yellow-brown line in figure~\ref{fig:densityPLI_theme}) have a roughly symmetrical distribution with similar density at high and low probability of loss of function intolerance while the peak at low levels of probability of loss of function intolerance is much higher in the non PSP genes. This reflects a greater evolutionary constraint on the PSP genes and is consistent with the findings using the database of essential genes (section~\ref{sec:results over representation PSP in DEG genes}) but with the additional benefit of a \textit{measure} of selection pressure. 


The median pLI is higher for the PSP genes than non PSP (0.246 vs 0.001). The median PSP is greater than the third quartile of non PSP genes (table~\ref{tab:Probability of loss of function intolerance in genes in PSP versus those not in PSP}) and it is clear that the means are significantly different (Wilcoxon rank sum test with continuity correction W=17,893,688 $p<2.2\times10^{-16}$) \footnote{\url{source('~/RProjects/db_essential_genes/R/gnomad/histogram_pli_psp/statistics/summary_pLI_PSP.R')}}.


\begin{figure}
    \centering
    \includegraphics[width=\textwidth]{images/chapter3/ggplot2/gnomad/theme/Rplot_histogram_pLI_theme.png}
    \caption[Histogram of pLI comparing PSP and non PSP genes]{Stacked histogram showing the distribution of pLI in PSP and non PSP genes. The distribution is bimodal with most of the probability mass at extreme low and high values of pLI however the PSP genes have a more symmetrical distribution with a relatively higher peak in the high probability of loss of function intolerance range.}
    \tiny\url{source('~/RProjects/db_essential_genes/R/gnomad/histogram_pli_psp/theme/histogram_pli_theme.R')}
    
    \label{fig:hist_pliPSP_theme}
\end{figure}



\begin{figure}
    \centering
    \includegraphics[width=\textwidth]{images/chapter3/ggplot2/gnomad/theme/Rplot_density_Mar_gnomad_theme.png}
    \caption[Density plot pLI for PSP and non PSP genes]{Density plot of pLI for PSP and non PSP. The distribution of probability loss intolerance is bimodal for both the PSP and non PSP genes with peaks at extreme low and high PSP values. The plot is more symmetrical for the PSP genes which shows a similar density of very high and very low probability of loss intolerance (pLI) values compared with the non PSP genes which, if they have extreme values are more likely to have low pLI}
    \tiny\url{source('~/RProjects/db_essential_genes/R/gnomad/histogram_pli_psp/theme/density_pli_theme.R')}
    
    \label{fig:densityPLI_theme}
\end{figure}






% latex table generated in R 3.6.3 by xtable 1.8-4 package
% Sun Mar 28 17:46:00 2021
\begin{table}[ht]
\centering
\setlength{\extrarowheight}{2pt}
\begin{adjustbox}{width=\textwidth}
\begin{tabular}{lllllllllll}
  \toprule
 & mean & median & sd & IQR & Q1 & Q3 & min & max & $n$ \\ 
  \midrule
PSP & 0.45 & 0.246 & 0.45 & 0.99 & $3.118 \times 10^{-5}$ & 0.988 & $1.861 \times 10^{-164}$ & 1 & 3407  \\ 
  non PSP & 0.20 & 0.001 & 0.35 & 0.23 & $2.068 \times 10^{-8}$ & 0.231 & $8.222 \times 10^{-118}$ & 1 & 14972 \\ 
   \bottomrule
\end{tabular}
\end{adjustbox}
\caption[Summary statistics PSP versus non PSP genes pLI]{Probability of loss of function intolerance in genes in PSP versus those not in PSP. Wilcoxon rank sum test with continuity correction W=17,893,688,  $p<2.2\times10^{-16}$. No missing values. Q1 = first quartile, Q3 = third quartile, sd = standard deviation, IQR = interquartile range} 
\tiny\url{source('~/RProjects/db_essential_genes/R/gnomad/histogram_pli_psp/statistics/summary_pLI_PSP.R')}
\label{tab:Probability of loss of function intolerance in genes in PSP versus those not in PSP}
\end{table}



\section{Disease enrichment for centrality measures}
\label{sec:Disease term enrichment centrality measures}



\subsection{Disease enrichment}
Disease enrichment is carried out using ToppGene to query the DisGenNet database. Table 

% latex table generated in R 3.6.3 by xtable 1.8-4 package
% Tue Mar 16 12:09:01 2021
\begin{table}[ht]
\centering

\setlength{\extrarowheight}{2pt}
\begin{adjustbox}{width=\textwidth}

\begin{tabular}{lllllll}
  \toprule
 & ID & Name & n PSP & n set & $p$ & q FDR B.H \\ 
  \midrule
1 & C0042769 & Virus Diseases & 332 & 95 & $1.434 \times 10^{-22}$ & $1.124 \times 10^{-18}$ \\ 
  2 & C0019163 & Hepatitis B & 290 & 82 & $6.045 \times 10^{-19}$ & $2.370 \times 10^{-15}$ \\ 
  3 & C2931822 & Nasopharyngeal carcinoma & 294 & 80 & $2.383 \times 10^{-17}$ & $6.228 \times 10^{-14}$ \\ 
  4 & C0026764 & Multiple Myeloma & 341 & 85 & $7.091 \times 10^{-16}$ & $1.390 \times 10^{-12}$ \\ 
  5 & C0024299 & Lymphoma & 300 & 77 & $4.178 \times 10^{-15}$ & $5.720 \times 10^{-12}$ \\ 
  6 & C0302592 & Cervix carcinoma & 352 & 85 & $5.658 \times 10^{-15}$ & $5.720 \times 10^{-12}$ \\ 
  7 & C0033578 & Prostatic Neoplasms & 175 & 55 & $6.894 \times 10^{-15}$ & $5.720 \times 10^{-12}$ \\ 
  8 & C0007847 & Malignant tumor of cervix & 315 & 79 & $6.963 \times 10^{-15}$ & $5.720 \times 10^{-12}$ \\ 
  9 & C0002736 & Amyotrophic Lateral Sclerosis & 354 & 85 & $8.152 \times 10^{-15}$ & $5.720 \times 10^{-12}$ \\ 
  10 & C1332977 & Childhood Leukemia & 367 & 87 & $8.194 \times 10^{-15}$ & $5.720 \times 10^{-12}$ \\ 
  11 & C0079731 & B-Cell Lymphomas & 273 & 72 & $8.656 \times 10^{-15}$ & $5.720 \times 10^{-12}$ \\ 
  12 & C0023418 & leukemia & 407 & 93 & $8.755 \times 10^{-15}$ & $5.720 \times 10^{-12}$ \\ 
  13 & C1832661 & Anopthalmia and pulmonary hyp. & 298 & 76 & $9.684 \times 10^{-15}$ & $5.840 \times 10^{-12}$ \\ 
  14 & C4048328 & cervical cancer & 340 & 82 & $2.147 \times 10^{-14}$ & $1.202 \times 10^{-11}$ \\ 
  15 & C0040136 & Thyroid Neoplasm & 231 & 64 & $2.877 \times 10^{-14}$ & $1.503 \times 10^{-11}$ \\ 
  16 & C3539878 & Triple Negative Breast Neoplasms & 351 & 82 & $1.510 \times 10^{-13}$ & $7.399 \times 10^{-11}$ \\ 
  17 & C0476089 & Endometrial Carcinoma & 283 & 71 & $2.256 \times 10^{-13}$ & $1.040 \times 10^{-10}$ \\ 
  18 & C0007103 & Malignant neoplasm of endometrium & 229 & 62 & $2.523 \times 10^{-13}$ & $1.099 \times 10^{-10}$ \\ 
  19 & C0007115 & Malignant neoplasm of thyroid & 195 & 56 & $2.950 \times 10^{-13}$ & $1.141 \times 10^{-10}$ \\ 
  20 & C0019196 & Hepatitis C & 362 & 83 & $3.195 \times 10^{-13}$ & $1.141 \times 10^{-10}$ \\ 
   \hline
\end{tabular}
\end{adjustbox}
\caption{Top 20 all disorders 90th centile degree} 
\label{tab:overrepresentation all disorders by degree}
\tiny\url{source('~/RProjects/chapter3/R/toppgene_disgen/xtable/output_high_degree.R')}
\end{table}

% latex table generated in R 3.6.3 by xtable 1.8-4 package
% Tue Mar 16 12:09:01 2021
\begin{table}[ht]
\centering

\setlength{\extrarowheight}{2pt}
\begin{adjustbox}{width=\textwidth}

\begin{tabular}{lllllll}
  \toprule
 & ID & Name & $n$ PSP & $n$ set & $p$ & q FDR B.H \\ 
  \midrule
1 & C0002736 & Amyotrophic Lateral Sclerosis & 354 & 85 & $8.152 \times 10^{-15}$ & $5.720 \times 10^{-12}$ \\ 
  2 & C0338451 & Frontotemporal dementia & 119 & 40 & $4.413 \times 10^{-12}$ & $1.081 \times 10^{-9}$ \\ 
  3 & C0236642 & Pick Disease of the Brain & 89 & 32 & $9.642 \times 10^{-11}$ & $1.429 \times 10^{-8}$ \\ 
  4 & C0949664 & Tauopathies & 91 & 31 & $9.233 \times 10^{-10}$ & $1.114 \times 10^{-7}$ \\ 
  5 & C1843013 & Alzheimer disease, familial, type 3 & 57 & 22 & $2.097 \times 10^{-8}$ & $1.536 \times 10^{-6}$ \\ 
  6 & C0524851 & Neurodegenerative Disorders & 484 & 88 & $3.194 \times 10^{-8}$ & $2.276 \times 10^{-6}$ \\ 
  7 & C0014057 & Japanese Encephalitis & 30 & 15 & $6.448 \times 10^{-8}$ & $4.284 \times 10^{-6}$ \\ 
  8 & C0751072 & Frontotemporal Lobar Degeneration & 81 & 26 & $9.075 \times 10^{-8}$ & $5.692 \times 10^{-6}$ \\ 
  9 & C0040997 & Trigeminal Neuralgia & 17 & 11 & $1.147 \times 10^{-7}$ & $7.139 \times 10^{-6}$ \\ 
  10 & C0004114 & Astrocytoma & 255 & 54 & $1.805 \times 10^{-7}$ & $1.040 \times 10^{-5}$ \\ 
  11 & C0220756 & Niemann-Pick Disease, Type C & 45 & 18 & $2.235 \times 10^{-7}$ & $1.231 \times 10^{-5}$ \\ 
  12 & C0020179 & Huntington Disease & 271 & 56 & $2.491 \times 10^{-7}$ & $1.347 \times 10^{-5}$ \\ 
  13 & C0006118 & Brain Neoplasms & 245 & 52 & $2.930 \times 10^{-7}$ & $1.552 \times 10^{-5}$ \\ 
  14 & C4551993 & Amyotrophic Lateral Sclerosis, Familial & 37 & 16 & $2.982 \times 10^{-7}$ & $1.566 \times 10^{-5}$ \\ 
  15 & C4024896 & Motor neuron atrophy & 55 & 20 & $2.996 \times 10^{-7}$ & $1.566 \times 10^{-5}$ \\ 
  16 & C0027765 & nervous system disorder & 289 & 56 & $2.308 \times 10^{-6}$ & $8.785 \times 10^{-5}$ \\ 
  17 & C0002622 & Amnesia & 49 & 17 & $5.056 \times 10^{-6}$ & $1.680 \times 10^{-4}$ \\ 
  18 & C0009241 & Cognition Disorders & 173 & 38 & $5.876 \times 10^{-6}$ & $1.927 \times 10^{-4}$ \\ 
  19 & C0026837 & Muscle Rigidity & 65 & 20 & $6.135 \times 10^{-6}$ & $2.004 \times 10^{-4}$ \\ 
  20 & C4552000 & Episodic Kinesigenic Dyskinesia 1 & 53 & 17 & $1.682 \times 10^{-5}$ & $4.866 \times 10^{-4}$ \\ 
   \bottomrule
\end{tabular}
\end{adjustbox}
\caption{Disease enrichment for neurological disorders using the 90th centile of degree distribution compared with a background set of all PSP genes. Correction for multiple comparisons by Benjamini and Hochberg method (q FDR BH) and is corrected using full list of diseases (a filtered list is presented). Twenty most enriched gene sets shown ordered by $q$ FDR. The number of genes annotated to the disease term in the PSP is $n$ PSP and the number in the top 90th centile of genes by degree in PSP is $n$ set } 
\tiny\url{source('~/RProjects/chapter3/R/toppgene_disgen/xtable/output_high_degree.R')}
\label{tab:neurological disorders by high degree}
\end{table}




Significant diseases numerous include neoplasms and viral processes. The most significant neuropsychiatric disorder is the 9th term Amyotrophic Lateral Sclerosis  (C0002736;  ; $n$ in high degree 85 and $n$ in PSP 354 p $8.15 \times 10^{-15}$ FDR q $5.72 \times 10^{-12}$) (table~\ref{tab:overrepresentation all disorders by degree}). Amyotrophic lateral sclerosis is a neurodegenerative disorder characterised by muscular weakness. It is part of a common disease spectrum with fronto-temporal dementia  which is the second most enriched neurological disorder in table~\ref{tab:neurological disorders by high degree} (C0338451 $q$ FDR BH $1.081 \times 10^{-9}$)). The q table~\ref{tab:neurological disorders by high degree} is corrected by the total number of all diseases in the database (i.e not just the neurological disorders).

\clearpage





\subsection{Distribution of pLI and intelligence}

\label{sec:pLI and intelligence and educational attainment}


%pLI z Spearman
% latex table generated in R 3.6.3 by xtable 1.8-4 package
% Tue Mar 23 15:38:29 2021
\begin{table}[ht]
\centering
\setlength{\extrarowheight}{2pt}
\begin{tabular}{llll}
  \toprule
Sample & $S$ & rho & $p$ \\ 
  \midrule
Intelligence\textsubscript{Replication} & $8.41 \times 10^{11}$ & 0.031 & $3.542 \times 10^{-5}$ \\ 
  Education\textsubscript{Replication} & $7.95 \times 10^{11}$ & 0.046 & $2.054 \times 10^{-9}$ \\ 
  Intelligence\textsubscript{Discovery} & $8.18 \times 10^{11}$ & 0.057 & $8.756 \times 10^{-14}$ \\ 
  Education\textsubscript{Discovery} & $8.01 \times 10^{11}$ & 0.077 & $6.338 \times 10^{-24}$ \\ 
   \bottomrule
\end{tabular}
\caption{Correlation pLI and z score Spearman } 
\tiny\url{source('~/RProjects/chapter3/R/magma_gnomad/compare_Mar/0_load_magma_and_load_gnomad.R')}
\label{tab: Correlation pLI and z score Spearman}
\end{table}


\begin{table}[ht]
\centering
\setlength{\extrarowheight}{2pt}
\begin{tabular}{lllllll}
  \toprule
Sample & $t$ & df & low & high & $r$ & $p$ \\ 
  \midrule
Intelligence\textsubscript{Replication} & 7.526 & 17334 & 0.042 & 0.072 & 0.057 & $5.470 \times 10^{-14}$ \\ 
  Education\textsubscript{Replication} & 9.234 & 17098 & 0.056 & 0.085 & 0.070 & $2.893 \times 10^{-20}$ \\ 
  Intelligence\textsubscript{Discovery} & 10.043 & 17324 & 0.061 & 0.091 & 0.076 & $1.150 \times 10^{-23}$ \\ 
  Education\textsubscript{Discovery} & 14.230 & 17324 & 0.093 & 0.122 & 0.107 & $1.078 \times 10^{-45}$ \\ 
   \bottomrule
\end{tabular}
\caption{Correlation pLI and z score Pearson } 
\tiny\url{source('~/RProjects/chapter3/R/magma_gnomad/compare_Mar/0_load_magma_and_load_gnomad.R')}
\label{tab: Correlation pLI and z score Pearson}
\end{table}



There was a weak association between probability of loss of function intolerance and genes having variants associated with differences in intelligence or educational attainment. In table \ref{tab: Correlation pLI and z score Spearman} there is a weak but significant correlation between $z$ score and probability of loss of function intolerance. Genes that were more likely to be significant in these samples were more likely to be loss of function intolerant. Table~\ref{tab: Correlation pLI and neg log10 Spearman} shows the correlation using the negative log10 transform of the p value which is commonly quoted in GWA studies. As these are in the same rank order as the $z$ score these are identical and show that more significant genes tend to be are more loss of function intolerant. The relationship is strongest in the Education\textsubscript{Discovery} sample ($\rho$=0.077). Results showing the difference in relationship between synaptic and non synaptic genes in these cohorts are shown in table ~\ref{tab: Correlation pLI and neg log10 SpearmanPSP, nonPSP and all}  \footnote{\url{source('~/RProjects/paper_xls_output/R/make_df_correlation_pLI_genescore.R')}} the relationship is stronger for synaptic genes than for non synaptic genes although the effect is small and synaptic genes as shown below have a higher probability of loss of function intolerance.

The distribution of pLI is bi-modal and a non-parametric test of association is most appropriate. For comparison the Pearson correlation between pLI and gene z score is shown and highest for Education\textsubscript{Discovery} (r=0.107) and lowest for Intelligence\textsubscript{Replication}(r=0.057) (table~\ref{tab: Correlation pLI and z score Pearson}). Using the negative log 10 transform of the MAGMA p values the highest Pearson correlation was with Education\textsubscript{Discovery} r=0.095 (table~\ref{tab: Correlation pLI and neg log10 Pearson}). The Education\textsubscript{Discovery} sample shows the highest correlation between log10 p and pLI ($\rho$=0.077) (table~\ref{tab: Correlation pLI and neg log10 Spearman})


% pLI z Pearson
% latex table generated in R 3.6.3 by xtable 1.8-4 package
% Tue Mar 23 15:35:23 2021




% pLI log10 Pearson

% latex table generated in R 3.6.3 by xtable 1.8-4 package
% Tue Mar 23 15:41:18 2021
\begin{table}[ht]
\centering
\setlength{\extrarowheight}{2pt}
\begin{tabular}{lllllll}
  \toprule
Sample & $t$ & df & low & high & $r$ & $p$ \\ 
\midrule
Intelligence\textsubscript{Replication} & 8.161 & 17334 & 0.047 & 0.077 & 0.062 & $3.546 \times 10^{-16}$ \\ 
  Education\textsubscript{Replication} & 10.534 & 17098 & 0.065 & 0.095 & 0.080 & $7.214 \times 10^{-26}$ \\ 
  Intelligence\textsubscript{Discovery} & 9.735 & 17324 & 0.059 & 0.089 & 0.074 & $2.440 \times 10^{-22}$ \\ 
  Education\textsubscript{Discovery} & 12.544 & 17324 & 0.080 & 0.110 & 0.095 & $6.141 \times 10^{-36}$ \\ 
   \bottomrule
\end{tabular}
\caption{Correlation pLI and neg log10 Pearson } 
\tiny\url{source('~/RProjects/chapter3/R/magma_gnomad/compare_Mar/0_load_magma_and_load_gnomad.R')}
\label{tab: Correlation pLI and neg log10 Pearson}
\end{table}



% PLI log10 Spearman

% latex table generated in R 3.6.3 by xtable 1.8-4 package
% Tue Mar 23 15:42:11 2021
\begin{table}[ht]
\centering
\setlength{\extrarowheight}{2pt}
\begin{tabular}{llll}
  \toprule
Sample & $S$ & $\rho$ & $p$ \\ 
  \midrule
Intelligence\textsubscript{Replication} & $8.41 \times 10^{11}$ & 0.031 & $3.542 \times 10^{-5}$ \\ 
  Education\textsubscript{Replication} & $7.95 \times 10^{11}$ & 0.046 & $2.054 \times 10^{-9}$ \\ 
  Intelligence\textsubscript{Discovery} & $8.18 \times 10^{11}$ & 0.057 & $8.754 \times 10^{-14}$ \\ 
  Education\textsubscript{Discovery} & $8.01 \times 10^{11}$ & 0.077 & $6.340 \times 10^{-24}$ \\ 
   \bottomrule
\end{tabular}
\caption{Correlation pLI and neg log10 Spearman } 
\tiny\url{source('~/RProjects/chapter3/R/magma_gnomad/compare_Mar/0_load_magma_and_load_gnomad.R')}
\label{tab: Correlation pLI and neg log10 Spearman}
\end{table}




\paragraph{Subsets}


The correlation between pLI and zscore is shown for subsets of the PSP, non PSP and all genes (table~\ref{tab: Correlation pLI and neg log10 SpearmanPSP, nonPSP and all}). The highest correlation is again in the Education\textsubscript{Discovery} sample ($\rho$=0.077 all genes, p=$6.34 \times 10^{-24}$) and correlation is highest for genes in the PSP ($\rho$=0.106;p= $1.284 \times 10^{-9}$ for Education\textsubscript{Discovery} ) 
The correlation is also shown between pLI and the negative log 10 transform of the gene p value (table~\ref{tab: Correlation pLI and neg log10 SpearmanPSP, nonPSP and all}) which have identical coefficients ($\rho$) as the correlation with the $z$ score as they will be in the same rank order. 









% pLI and neg log10
% latex table generated in R 3.6.3 by xtable 1.8-4 package
% Tue Mar 23 16:30:45 2021
\begin{table}[ht]
\centering
\setlength{\extrarowheight}{2pt}
\begin{tabular}{llllll}
  \toprule
Sample & subset & $n$ &$S$ & $\rho$ & $p$ \\ 
  \midrule
Intelligence\textsubscript{Replication} & PSP & 3276 & $5.59 \times 10^{9}$ & 0.047 & $7.362 \times 10^{-3}$ \\ 
 & non-PSP & 14060 & $4.55 \times 10^{11}$ & 0.018 & $3.601 \times 10^{-2}$ \\ 
 &  & 17336 & $8.41 \times 10^{11}$ & 0.031 & $3.542 \times 10^{-5}$  \vspace{9pt}\\ 

  Education\textsubscript{Replication} & PSP & 3254 & $5.43 \times 10^{9}$ & 0.055 & $1.698 \times 10^{-3}$ \\ 
 & non-PSP & 13846 & $4.25 \times 10^{11}$ & 0.039 & $5.707 \times 10^{-6}$ \\ 
 &  & 17100 & $7.95 \times 10^{11}$ & 0.046 & $2.054 \times 10^{-9}$\vspace{9pt} \\ 
  
  Intelligence\textsubscript{Discovery} & PSP & 3270 & $5.36 \times 10^{9}$ & 0.080 & $4.931 \times 10^{-6}$ \\ 
 & non-PSP & 14056 & $4.45 \times 10^{11}$ & 0.038 & $6.317 \times 10^{-6}$ \\ 
 &  & 17326 & $8.18 \times 10^{11}$ & 0.057 & $8.754 \times 10^{-14}$ \vspace{9pt}\\ 
  
  Education\textsubscript{Discovery} & PSP & 3270 & $5.21 \times 10^{9}$ & 0.106 & $1.284 \times 10^{-9}$ \\ 
  & non-PSP & 14056 & $4.38 \times 10^{11}$ & 0.054 & $1.528 \times 10^{-10}$ \\ 
 &  & 17326 & $8.01 \times 10^{11}$ & 0.077 & $6.340 \times 10^{-24}$ \\ 
   \bottomrule
\end{tabular}
\caption{Correlation pLI and neg log10 Spearman PSP, nonPSP and all} 
\tiny Edited from \url{source('~/RProjects/chapter3/R/magma_gnomad/compare_Mar/0_load_magma_and_load_gnomad_subset.R')} 
\label{tab: Correlation pLI and neg log10 SpearmanPSP, nonPSP and all}
\end{table}



 % latex table generated in R 3.6.3 by xtable 1.8-4 package
% Sun Mar 28 14:05:59 2021
\begin{table}[ht]
\centering
\setlength{\extrarowheight}{2pt}
\begin{tabular}{lrrr}
  \toprule
Centrality & $S$ & $\rho$ & p \\ 
  \midrule
kcoreness & $5.07 \times 10^{9}$ & 0.230 & $3.30 \times 10^{-42}$ \\ 
  Closeness & $5.11 \times 10^{9}$ & 0.225 & $2.83 \times 10^{-40}$ \\ 
  Degree & $5.11 \times 10^{9}$ & 0.224 & $4.01 \times 10^{-40}$ \\ 
  Eigenvector & $5.19 \times 10^{9}$ & 0.213 & $3.32 \times 10^{-36}$ \\ 
  Betweenness & $5.27 \times 10^{9}$ & 0.201 & $2.38 \times 10^{-32}$ \\ 
  Transitivity\textsubscript{0} & $5.79 \times 10^{9}$ & 0.121 & $1.26 \times 10^{-12}$ \\ 
  Transitivity\textsubscript{NaN} & $4.51 \times 10^{9}$ & 0.098 & $4.68 \times 10^{-8}$ \\ 
   \bottomrule
\end{tabular}
\caption{Correlation of pLI and vertex statistics. Spearman rank correlation using for Transitivity\textsubscript{NaN} pairwise complete observations.  Ordered by correlation coefficient}
\tiny\url{source('~/RProjects/chapter3/R/centrality_ch3/centrality_and_gwastudies/0_load_magma_spearman_centrality.R')}
\label{tab:correlation pli vertex statistics mar}
\end{table}

\paragraph{Centrality and pLI}
There is a significant correlation between all centrality measures and probability of loss of function intolerance (table~\ref{tab:correlation pli vertex statistics mar}). The relationship is strongest for kcore and weakest for the transitivity measures. Given the right skew of some of the measures it was possible that lower values of centrality did little to distinguish between vertices therefore I calculated the effect of removing the bottom 0.2 quantile but there was very little effect other than removing any effect from transitivity\textsubscript{NaN}\footnote{\url{source('~/RProjects/chapter3/R/centrality_ch3/centrality_and_gwastudies/0_load_magma_spearman_centrality_subset.R')}}


\subsection{Correlation centrality measures and study p}
\label{sec:correlation centrality measures and study p}
There is no significant correlation between any of the centrality scores and the results of the Gene level MAGMA analysis for the intelligence samples (table~\ref{tab:Intelligence Spearman Correlation}) or the educational attainment samples (table~\ref{tab:Education Spearman Correlation}). A linear model using the principal components of the centrality measures and allowing interaction between all principal components (maximal model) showed no significant effect (supplemental section~\ref{sec:PCA regression})



% latex table generated in R 3.6.3 by xtable 1.8-4 package
% Mon Mar 29 13:54:35 2021

\begin{table}[ht]
\centering
\setlength{\extrarowheight}{2pt}
\begin{tabular}{l@{\hskip 20pt}lll@{\hskip 20pt}lll}
  \toprule
  &  \multicolumn{3}{c}{\textit{Discovery}} & \multicolumn{3}{c}{\textit{Replication}} \\
  \cmidrule{2-7}
Centrality & $S$ & $\rho$ & $p$ & $S$ & $\rho$ & $p$ \\ 
  \midrule
Degree & $6.20 \times 10^{9}$ & -0.03 & 0.04 & $5.91 \times 10^{9}$ & 0.02 & 0.20 \\ 
  Betweenness & $6.20 \times 10^{9}$ & -0.03 & 0.05 & $5.91 \times 10^{9}$ & 0.02 & 0.21 \\ 
  Eigenvector & $6.16 \times 10^{9}$ & -0.03 & 0.11 & $6.05 \times 10^{9}$ & -0.00 & 0.93 \\ 
  Closeness & $6.14 \times 10^{9}$ & -0.02 & 0.17 & $6.08 \times 10^{9}$ & -0.01 & 0.68 \\ 
  Transitivity\textsubscript{NaN} & $4.55 \times 10^{9}$ & 0.00 & 0.85 & $4.68 \times 10^{9}$ & -0.02 & 0.34 \\ 
  Transitivity\textsubscript{0} & $5.96 \times 10^{9}$ & 0.01 & 0.75 & $6.08 \times 10^{9}$ & -0.01 & 0.67 \\ 
  kcoreness & $6.18 \times 10^{9}$ & -0.03 & 0.07 & $5.92 \times 10^{9}$ & 0.02 & 0.27 \\ 
   \bottomrule
\end{tabular}
\caption{Intelligence Spearman Correlation with negative log 10 p } 
\tiny\url{source('~/RProjects/chapter3/R/centrality_ch3/centrality_and_gwas_p_val/combined_tables/spearman/intelligence_centrality_correlation.R')}
\label{tab:Intelligence Spearman Correlation}
\end{table}

% latex table generated in R 3.6.3 by xtable 1.8-4 package
% Mon Mar 29 14:29:30 2021
\begin{table}[ht]
\centering
\setlength{\extrarowheight}{2pt}
\begin{tabular}{l@{\hskip 20pt}lll@{\hskip 20pt}lll}

  \toprule
  &  \multicolumn{3}{c}{\textit{Discovery}} & \multicolumn{3}{c}{\textit{Replication}} \\
  \cmidrule{2-7}
Centrality & $S$ & $\rho$ & $p$ & $S$ & $\rho$ & $p$ \\ 
  \midrule
Degree & $5.84 \times 10^{9}$ & 0.01 & 0.53 & $6.04 \times 10^{9}$ & -0.01 & 0.69 \\ 
  Betweenness & $5.82 \times 10^{9}$ & 0.02 & 0.39 & $6.01 \times 10^{9}$ & -0.00 & 0.92 \\ 
  Eigenvector & $5.95 \times 10^{9}$ & -0.01 & 0.65 & $6.18 \times 10^{9}$ & -0.03 & 0.07 \\ 
  Closeness & $5.96 \times 10^{9}$ & -0.01 & 0.65 & $6.19 \times 10^{9}$ & -0.03 & 0.06 \\ 
  Transitivity\textsubscript{NaN} & $4.63 \times 10^{9}$ & -0.03 & 0.10 & $4.59 \times 10^{9}$ & -0.01 & 0.77 \\ 
  Transitivity\textsubscript{0} & $6.01 \times 10^{9}$ & -0.02 & 0.31 & $6.05 \times 10^{9}$ & -0.01 & 0.62 \\ 
  kcoreness & $5.88 \times 10^{9}$ & 0.00 & 0.79 & $6.06 \times 10^{9}$ & -0.01 & 0.52 \\ 
   \bottomrule
\end{tabular}
\caption[Correlation of centrality measures with educational attainment]{Correlation of centrality measures with Educational attainment Spearman Correlation with negative log 10 p}
\tiny\url{source('~/RProjects/chapter3/R/centrality_ch3/centrality_and_gwas_p_val/combined_tables/spearman/education_centrality_correlation.R')}
\label{tab:Education Spearman Correlation}
\end{table}

A maximal linear model of interactions of the first three principal components showed no significant variables impacting upon the  z-score of Education Discovery or Intelligence discovery \footnote{\url{source('~/RProjects/chapter3/R/centrality_ch3/prcomps/6_cortest_lm.R')}}.

\subsection{Summary}
Summary central genes are associated with loss of function intolerance. Significant genes for intelligence and cognitive ability are slightly more likely to be loss of function intolerant.The relationship is strongest for the studies with the greatest power (ie the discovery cohorts).  There is no association between centrality and significance in intelligence of cognitive ability














\clearpage

\section{Summary of chapter}

We show that in the PSP nodes of very high centrality are associated with essentialness. The PSP has a high degree of essentialness compared to the rest of the genome. 

High degree nodes are associated with specific phenotypes and specific gene ontology terms however the are not correlated with genetic differences that are enriched for differences in intelligence. This is not obvious before the fact and is different from other traits eg essentialness. One might hypothesis that the reason is that there is either no effect (ie it is irrelevant) or that in population studies genes with high centrality are more likely to be essential and hence have less genetic variation. We find also in one of the principle animal models of learning murine long term potentiation that genes of high importance were no more likely or less likely to be involved in long term potentiation.

We used data from the Gnomad project and DEG to further investigate whether high importance nodes were subject to selection pressure such that they proved more intolerant of non synonymous genetic changes. Nodes with high degree are associated with an increased probability of being intolerant to changes. This is seen in other measures of centrality and provides evidence from human population studies of the effects of centrality

Nodes with high betweenness centrality were involved in neuro-degenerative disorders including Parkinson's disease see section~\ref{sec:Betweeness centrality}. They were also associated with severe phenotypes in murine models including phenotypes incompatibly with viability. Low betweenness centrality was associated with abnormalities in synaptic transmission in murine models. 

High eigenvector centrality is again associated with severe murine phenotypes and enrichment for RNA and cadherin binding. Disease enrichment is of neoplasms and neurodegenerative disorders. 

Low eigenvector centrality (10\%) were enriched for human phenotypes of seizures and ontology terms of transmitter, mouse phenotype was of abnormal synaptic transmission. The top five enriched disorders in DisGeNET were seizure related. 

Despite this there was no association between node centrality measures and the significance of genetic variants for differences in intelligence and educational attainment.

There is a complex relationship between transitivity and degree for nodes with degree 3 or less the correlation is positive and results in a positive overall correlation despite a plot of the relationship appearing clearly negative. This is due to the large number of low degree nodes. When the correlation is calculated for higher degree the relationship previously commented upon in the literature \cite{newman2018networks}\footnote{p 335} is seen. Caution should therefore be used in determining linear correlations involving transitivity especially in smaller networks and it may be prudent to plot the correlation against minimum degree.

Gene sets from animal models of cognition have modest but significantly increased assortativity. 





 
% \paragraph{Degree and local transitivity}
% \label{sec:discuss degree and local transitivity - summary just now}
% Reviewing the previous literature on the correlation between degree and local clustering Ravasz and Barabasi in a highly cited work \cite{ravasz2002hierarchical} review a generative model of high clustering coefficient and scale free degree distribution and real world networks. The real world networks include the actor network (392,340 nodes), the language network based on synonyms in Merriams English dictionary (182,853 nodes), the world wide web from mapping out from the \url{www.nd.edu} domain with 325,729 nodes, the internet at the router level (260,657 nodes) and the power grid of the Western United states 4941 nodes 13188 edges. \textcolor{red}{end move to discussion}

% These networks are all large and in the logarithmic graphs of clustering coefficient and degree the minumum degree starts around 10. It is interesting to note in figure 4 \textcolor{red}{of what pressumably Ravasz and Barabasi?} showing the internet at router level and power network there are lower degree nodes but the slope of the line representing the correlation coefficient is dominated by the effect of high degree nodes and the distribution of clustering coefficient in the lower degree domain is quite different. These matters may seem small as for large degree the overall pattern holds but it is a useful practical fact to be aware of if we are testing the association of clustering coefficient with another variable in smaller networks such as the PSP which is not readily apparent from the literature.

% (Although in measuring the correlation of transitivty with another measure you are still measuring its correlation but a high transitivity low degree region may have different behaviour in the network to the majority degree greater than 10 negative correlation with transitivity)\\todo{correlation both domains}

% \section{???tmp???}
%  \subsection{***Results from database of essential genes}
%  \todo{difference plI and DEG}
%  Most up to date data:
% 43294 entries in all

% 32578 human 

% 8973 genes

% number of genes in NCBI37.3 8973
% 8973/19427 = 46.1\%

% 15987 genes NCBI non synaptic
% 6616

% 2328 genes in PSP 67.3\%

% 6616/15987 41.4\% of genes not in PSP
%\todo[inline]{Look in findplace\_chapters.tex for temp removed add some more GO enrichment}





% %https://tex.stackexchange.com/questions/508546/help-fixing-a-contingency-table?noredirect=1&lq=1

% % latex table generated in R 3.6.3 by xtable 1.8-4 package
% % Wed Mar 17 18:22:08 2021



   


% \paragraph{Previous data}
% 2238 genes in the PSP are essential (64.7\%)

% This compares with 37.\% of the genome as a whole (assuming 20000 genes \todo{get non essential number from database}. 36.5\% (6043/20000) non PSP genes are essential. 

% \subsubsection{Degree and essential genes}

% The summary statistics for the degree of essential PSP genes and non essential PSP genes are shown in table~\ref{tab:degree distribution essential and non essential genes PSP}. The mean degree is higher in essential (21.29) vs non essential genes (10.94) $p<2.2 \time 10^{-16}   W=1777916$ Wilcoxon rank sum test with continuity correction in R.

% The boxplot of log degree for essential and non essential genes in the PSP is shown in figure~\ref{fig:boxplot_log degree essential genes}
% \footnote{code \url{source('~/RProjects/paper_xls_latex/R/essentail_genes/lbxplot_and_latex_essential_genes.R')}}.

% The assortativity of essential nodes in the PSP is 0.0438 (ie very mild assortative although they tend to have high degree and PSP degree is disassortative).


% \begin{table}[ht]
% \centering
% \begin{tabular}{rlrrrrrrr}
%   \hline
%  & name & n & Min. & 1st Qu. & Median & Mean & 3rd Qu. & Max. \\ 
%   \hline
% 1 & Essential & 2238 & 1.00 & 5.00 & 11.00 & 21.29 & 23.00 & 474.00 \\ 
%   2 & Non essential & 1219 & 1.00 & 2.50 & 5.00 & 10.94 & 12.00 & 535.00 \\ 
%   3 & All & 3457 & 1.00 & 4.00 & 8.00 & 17.64 & 19.00 & 535.00 \\ 
%   \hline
% \end{tabular}
% \caption{Degree distribution of essential and non essential genes in PSP}
% \label{tab:degree distribution essential and non essential genes PSP}
% \end{table}




% \begin{figure}
%     \centering
%     \includegraphics[width=\textwidth]{images/Rplot_essential_non_essential_gene_degree.png}
%     \caption{Degree distribution for essential and non essential PSP genes using DEG 15.2. 2238 Essential genes, 1219 Non essential}
%     \label{fig:boxplot_log degree essential genes}
% \end{figure}



% \subsection{Redo}
% Have done degree and all centrality measures with essential genes \url{source('~/RProjects/chapter3/R/essential_genes/load_essential_genes.R')}

% Closeness and transitivity not log
% \subsection{***Feb redo: Synaptic and essential genes}

%      Finding genes in five or more studies  
     
%      Total genes 8992  
     
%      Number of studies: 20.              20 according to website in most recent update  
     
%      Number of genes appearing in 5 or more studies 2178  
     
%      Number of synaptic genes in DEG 2328  
     
%      Number of synaptic genes in 5 or more studies 728  
     
%       Number of non synaptic ie NCBI 37 genes 15987
      
%      Number of non synaptic ie NCBI 37 genes (so eligible for study)             in DEG 6620  
     
%      Number of non synaptic ie NCBI 37 genes (so eligible for study)             in five or more studies 1444  
     
%      Synaptic genes in NCBI 37 3440  
     
%      Percentage of synaptic genes in DEG in five or more 31.27 \%  
     
%      Percentage of non synaptic genes in DEG in five or more 21.81 \% 
     
%      Percent of NCBI synaptic 17.71 \%  
     
%      Percent of DEG synaptic 25.89 \%  
     
%      Percent of greater than 5 synaptic 33.43 \% 
     
%      \footnote{source for above \url{source('~/RProjects/db_essential_genes/R/get_essential_genes/db_2020/findforfive/get_db_essential_greater_five_results_xtable.R')}}
  
  
  
%  \subsubsection{***Degree distribution of PSP genes - murine model}
 
%  We can use the mouse phenotype ontology to identify essential murine genes. 
%  We want to know if high degree PSP nodes are more commonly associated with essentiality in murine models. 
 
%  The 90\% centile for PSP gene degree is $k=37$. 349 genes in the PSP network model have degree $k>=37$.
 
%  Murine phenotype MP:008763 is embryonic lethality. 117 of the genes with degree $k>=37$ are essential (33.5\% of 349 genes). 458 genes are essential in the remaining PSP nodes ($k < 37$, 458 of 3108 14.7\%). p $3.13 \times 10^{-11}$ Fisher's exact test for count data R. Odds ratio 2.27 95\% CI 1.79-2.88. \footnote{code:\url{/home/grant/RProjects/phenome_and_network/R/get_degree_90_quantile_PSP.R}} \todo{Calculate difference in degree for essential versus not essential} Comment \todo{one reason for testing the top 10\% is that the degree distribution is of a few hubs (cite Jeong) with different properties from the rest given the scale free structure of the network}
 
%  \begin{table}[h]
%      \centering
%      \begin{tabular}{llllll}
%           Title & $k$& n essential & n & percentage essential & p   \\
%           \hline
%           Top 10\% & $>=37$ & 117 & 349 & 33.5 & $3.13 \times 10^{-11}$\\
%           Rest & $<37$ & 458 & 3108 & 14.7 & \\
%      \end{tabular}
%      \caption{Association of high degree nodes with being essential in murine models. Genes with degree in top 10\% of PSP compared with the rest. P Fisher's Exact Test for Count data.}
%      \label{Table:Degree and murine essentialness PSP}
%  \end{table}
 

% \todo{GO enrichment with background PSP for high degree are in \url{/home/grant/Dropbox/stront_share/data/GO\_high\_degree_psp_background}}


% \subsubsection{Centrality and pLI}
% The current results suggest that the relationship between the variables is as shown in the factor graph in figure \ref{Figure:Factor graph for relationship between pLI, vertex statistic and study p values}

% \begin{figure}
%     \centering
%     \includegraphics[width=0.9\textwidth]{images/Rplot03_boxplot_pLI_PSP_non_PSP.png}
%     \caption{Box plot of probability of loss of function intolerance (pLI) between PSP genes (n=) and non PSP genes (rest of genome n=14990 ). Diamond marker indicates mean PSP value. }
%     \label{fig:Boxplot of pli}
% \end{figure}

% \begin{figure}
%     \centering
%     \includegraphics[width=\textwidth]{IMG_6979.JPG}
%     \caption{Factor graph for relationship between pLI, vertex statistic and study p values}
%     \label{Figure:Factor graph for relationship between pLI, vertex statistic and study p values}
% \end{figure}
% \todo{Make into table}

 
%  The march redo is in table~\ref{tab:correlation pli vertex statistics mar}
 
  

% \paragraph{Non march}
% There was a positive correlation between loss of function intolerance and the gene Z stat. To be explicit the Z stat is high and positive when the gene has a more significant association with the phenotype and hence a lower p value. There is a weak association between high loss of function intolerance and significant (low p value genes)

% \todo{there is something wrong here as Jan we find correlation} 
%  In testing correlation spearman’s rank correlation was used as the distribution of the probability of loss of function intolerance statistic is very non linear. It has a bi-modal distribution (see figure~\ref{fig:density estimate pLi}) with a concentration of probability mass around 0 and 1 .


% The distribution of probabilitity of loss of function intolerance in non PSP genes is still bimodal but the probability mass around 1 is much less peaked and there is greater mass at the bottom of the distribution (see figure~\ref{fig:density estimate pLi two panel}. The boxplot in figure ~\ref{fig:Boxplot of pli} shows the median and means of the distributions\footnote{ Source for graphs \url{source('~/RProjects/paper_xls_output/R/plot_pli_PSP_v_restgenome.R')}}. 


% \subsection{***}

% \subsection{PCA regression}
% \ref{sec:PCA regression}

% \begin{table}[ht]
% \centering
% \begin{tabular}{rrrrr}
%   \hline
%  & Estimate & Std. Error & t value & Pr($>$$|$t$|$) \\ 
%   \hline
% (Intercept) & 0.8696 & 0.0252 & 34.55 & 0.0000 \\ 
%   Dim.1 & 0.0189 & 0.0297 & 0.64 & 0.5240 \\ 
%   Dim.2 & 0.0202 & 0.0376 & 0.54 & 0.5904 \\ 
%   Dim.3 & -0.0524 & 0.0627 & -0.83 & 0.4039 \\ 
%   Dim.1:Dim.2 & -0.0130 & 0.0235 & -0.55 & 0.5821 \\ 
%   Dim.1:Dim.3 & -0.0255 & 0.0287 & -0.89 & 0.3742 \\ 
%   Dim.2:Dim.3 & -0.0210 & 0.0319 & -0.66 & 0.5089 \\ 
%   Dim.1:Dim.2:Dim.3 & -0.0001 & 0.0010 & -0.15 & 0.8820 \\ 
%   \hline
% \end{tabular}
% \caption{Linear model with z score for education discovery cohort}
% \end{table}


% % latex table generated in R 3.6.3 by xtable 1.8-4 package
% % Sun Feb 21 09:11:16 2021
% \begin{table}[ht]
% \centering
% \begin{tabular}{rrrrr}
%   \hline
%  & Estimate & Std. Error & t value & Pr($>$$|$t$|$) \\ 
%   \hline
% (Intercept) & 0.7255 & 0.0248 & 29.28 & 0.0000 \\ 
%   Dim.1 & 0.0318 & 0.0292 & 1.09 & 0.2756 \\ 
%   Dim.2 & 0.0076 & 0.0370 & 0.21 & 0.8373 \\ 
%   Dim.3 & -0.0642 & 0.0618 & -1.04 & 0.2988 \\ 
%   Dim.1:Dim.2 & -0.0385 & 0.0232 & -1.66 & 0.0968 \\ 
%   Dim.1:Dim.3 & -0.0542 & 0.0282 & -1.92 & 0.0550 \\ 
%   Dim.2:Dim.3 & -0.0435 & 0.0314 & -1.39 & 0.1660 \\ 
%   Dim.1:Dim.2:Dim.3 & -0.0010 & 0.0010 & -0.97 & 0.3344 \\ 
%   \hline
% \end{tabular}
% \caption{Linear model with z score for Intelligence Discovery Cohort}
% \end{table}




% \paragraph{*** ? RETURN Linear model of pLI and vertex statistics}    
% A linear regression model for centrality measures and pLI was constructed \footnote{Code \url{source('~/RProjects/paper_xls_output/R/make_pLI_correlation.R')}}. Although the correlation may be similarly linear the eigenvector centrality may have more effect on the gradient of the line of the relationship. 

% The eigenvector centrality seems to predict pLI best. Taking the 3rd quartile of the eigenvector centrality then the following are the summary statistics for the pLI

% The genes with higher eigenvector centrality tend to have higher probability of loss of function intolerance see table~\ref{tab:pli and eigenvector centrality quartiles}

% \begin{table}[]
%     \centering
%     \begin{tabular}{llllllll}
%     \toprule
%       Sample &  Min. &1st Qu.&  Median &   Mean& 3rd Qu.&    Max. &   NA's     \\
%       \midrule
%      Eigenvector 3rd quartile  &0& 0.349& 0.9315& 0.689& 0.998& 1&     48  \\ 
%      PSP &0&0.007&0.620&0.522&0.987&  1 & 222\\
%      Eigenvector less than 3rd quartile &0&0.002&0.386&0.466&0.974&1&    174\\
%      \bottomrule
%     \end{tabular}
%     \caption{Probability of loss of function intolerance for different values of eigenvector centrality and for PSP as whole}
%     \label{tab:pli and eigenvector centrality quartiles}
% \end{table}
% %   Min. 1st Qu.  Median    Mean 3rd Qu.    Max.    NA's 
% % 0.0000  0.3485  0.9315  0.6892  0.9983  1.0000      48 

% %For the PSP as a whole this is	
% %  Min. 1st Qu.  Median    Mean 3rd Qu.    Max.    NA's 
% %0.00000 0.00719 0.61952 0.52215 0.98691 1.00000     222 

% %and for the eigenvector centrality below the third quartile it is

% %  Min. 1st Qu.  Median    Mean 3rd Qu.    Max.    NA's 
% %0.00000 0.00165 0.38598 0.46570 0.97380 1.00000     174

% Wilcoxon test for difference in means for all synaptic pLI and those with eigenvector centrality above the third quartile
%         Wilcoxon rank sum test with continuity correction

% data:  df\$pLI and joined\_df\_synaptic\$pLI
% W = 1611226, p-value $< 2.2e-16$
% alternative hypothesis: true location shift is not equal to 0



% \subsection{Results Centrality and murine models of long term potentiation}
% \label{sec:results centrality and murine models of long term potentiation}
%     The centrality measures of genes marked as murine models of LTP were considered as model animal models of intelligence. \footnote{\url{source('~/RProjects/centrality/R/model_ltp/model ltp.R')}}.
%     256 genes were identified as a
%     ssociated with murine LTD from \textcolor{red}{add Mouse ontology}\todo{add mouse ontology}. 141 of these genes were found in the PSP network model (55.1\%).
    
%     There was no statistically significant difference in association between the centrality measures of those genes associated with LTP and the rest of the PSP. Tested degree (0.809), eigenvector centrality (0.630), betweenness (0.895) and transitivity (0.898) Wilcoxon rank sum test with continuity correction. 
    
%     A logistic regression model of these factors and LTP PSP as outcome showed that eigenvector centrality and degree were significant factors (4.04 $\times 10^{-6}$ and $1.39 \times 10^{-6}$. The effects were very minor however with McFadden's pseudo R2 \cite{mcfadden1973conditional} calculated using the pscl package was  0.031 \cite{jackman2017package}.


% \paragraph{Centrality and MP2207}

% Centrality measures almost identical other than higher betweenness

% 142 in PSP 258 in Genome
% same degree lower k core higher betweenness

% \subsubsection{Other notes}


% enrichment of genes found in the synapse with orthologs in yeast reveal a large number for exosome and mitochondrial elements




% \textbf{Articulation points}
% 11 of 68 glutamate receptor binding are articulation points however Gene ontology enrichment analysis ofarticulation points with the entire PSP as background yileds no significant over-representation of terms. They appear to be under represented in ID (3 genes of 203.	 	


% \paragraph{Redo articulation}
% ToppGene

% Plasma membrane and signalling receptor v nasty phenotypes
% haematology
% 202 genes
% \url{source('~/RProjects/disgen2/R/disgen/background_ents/artic.R')}



 
% % \section{XXXXX Supplemental Gene Ontology Enrichment}
% % \subsection{Betweenness ToppGene}
% % \subsubsection{High betweenness}
% % \paragraph{Molecular function}
% % Molecular function see table~\ref{tab:ToppGENE GO: Molecular Function. bet 90 centile cwpsp.txtp = p value; q FDR B H = q adjusted significance level False Discovery Rate using Benjamini and Hochberg adjustment; n= n genes annotated in test group; n PSP= n genes annotated in PSP}


% % % latex table generated in R 3.6.3 by xtable 1.8-4 package
% % % Sun Oct  4 11:46:04 2020
% % \begin{table}[ht]
% % \centering
% % \begin{adjustbox}{width=\textwidth}
% % \setlength{\extrarowheight}{2pt}
% % \begin{tabular}{@{}clllcl@{}}
% %   \toprule
% %   ID & Molecular Function & $n$ & $n$ PSP & $p$ & $q$ FDR B H \\ 

% %   \midrule
% % GO:0044877 & protein-containing complex binding & 120 & 513 & $1.74 \times 10^{-22}$ & $1.83 \times 10^{-19}$ \\ 
% %   GO:0031625 & ubiquitin protein ligase binding & 51 & 117 & $5.84 \times 10^{-22}$ & $3.07 \times 10^{-19}$ \\ 
% %   GO:0044389 & ubiquitin-like protein ligase binding & 52 & 123 & $1.24 \times 10^{-21}$ & $4.34 \times 10^{-19}$ \\ 
% %   GO:0019904 & protein domain specific binding & 89 & 329 & $1.02 \times 10^{-20}$ & $2.69 \times 10^{-18}$ \\ 
% %   GO:0019900 & kinase binding & 76 & 325 & $9.81 \times 10^{-14}$ & $2.06 \times 10^{-11}$ \\ 
% %   GO:0008234 & cysteine-type peptidase activity & 54 & 189 & $1.31 \times 10^{-13}$ & $2.29 \times 10^{-11}$ \\ 
% %   GO:0008134 & transcription factor binding & 48 & 156 & $1.64 \times 10^{-13}$ & $2.45 \times 10^{-11}$ \\ 
% %   GO:0005102 & signaling receptor binding & 87 & 405 & $2.09 \times 10^{-13}$ & $2.74 \times 10^{-11}$ \\ 
% %   GO:0050839 & cell adhesion molecule binding & 69 & 288 & $5.05 \times 10^{-13}$ & $5.88 \times 10^{-11}$ \\ 
% %   GO:0019901 & protein kinase binding & 69 & 295 & $1.77 \times 10^{-12}$ & $1.86 \times 10^{-10}$ \\ 
% %   \bottomrule
% % \end{tabular}
% % \end{adjustbox}
% % \caption[Gene ontology enrichment High betweenness genes Molecular Function of genes above 90th centile of distribution]{\textbf{High betweenness genes. Cellular Component - Gene ontology enrichment analysis} using the ToppGene web client and all PSP genes as the background set.  Biological process tested gene set is 90th centile of degree distribution.  p value; q FDR B H = q adjusted significance level False Discovery Rate using Benjamini and Hochberg adjustment; $n$= $n$ genes annotated in test group; $n$ PSP= n genes annotated in PSP. $n$ ontology terms significantly (q FDR$>=0.05$) enriched for this centrality measure and ontology category (e.g. MF,CC,BP): \textcolor{red}{n}} 

% % \label{tab:ToppGENE GO: Molecular Function. bet 90 centile cwpsp.txtp = p value; q FDR B H = q adjusted significance level False Discovery Rate using Benjamini and Hochberg adjustment; n= n genes annotated in test group; n PSP= n genes annotated in PSP}
% % \end{table}

% % \paragraph{Biological process}
% % Biological process table~\ref{tab:ToppGENE GO: Biological Process. bet 90 centile cwpsp.txtp = p value; q FDR B H = q adjusted significance level False Discovery Rate using Benjamini and Hochberg adjustment; n= n genes annotated in test group; n PSP= n genes annotated in PSP}
% % % latex table generated in R 3.6.3 by xtable 1.8-4 package
% % % Sun Oct  4 11:49:29 2020
% %   \begin{table}[ht]
% % \centering
% % \begin{adjustbox}{width=\textwidth}
% % \setlength{\extrarowheight}{2pt}
% % \begin{tabular}{@{}clllcl@{}}
% %   \toprule
% %   ID & Biological Process & $n$ & $n$ PSP & $p$ & $q$ FDR B H \\ 

% %   \midrule
% % GO:0007049 & cell cycle & 124 & 489 & $5.10 \times 10^{-27}$ & $3.33 \times 10^{-23}$ \\ 
% %   GO:0051247 & positive regulation of protein metabolic process & 117 & 463 & $4.11 \times 10^{-25}$ & $1.34 \times 10^{-21}$ \\ 
% %   GO:0042981 & regulation of apoptotic process & 108 & 416 & $5.38 \times 10^{-24}$ & $1.17 \times 10^{-20}$ \\ 
% %   GO:0043067 & regulation of programmed cell death & 108 & 423 & $2.44 \times 10^{-23}$ & $3.99 \times 10^{-20}$ \\ 
% %   GO:0009894 & regulation of catabolic process & 95 & 350 & $2.16 \times 10^{-22}$ & $2.82 \times 10^{-19}$ \\ 
% %   GO:0006508 & proteolysis & 105 & 417 & $4.31 \times 10^{-22}$ & $4.29 \times 10^{-19}$ \\ 
% %   GO:0010941 & regulation of cell death & 112 & 464 & $4.60 \times 10^{-22}$ & $4.29 \times 10^{-19}$ \\ 
% %   GO:0032270 & positive regulation of cellular protein metabolic process & 106 & 433 & $2.89 \times 10^{-21}$ & $2.36 \times 10^{-18}$ \\ 
% %   GO:0009057 & macromolecule catabolic process & 106 & 443 & $2.02 \times 10^{-20}$ & $1.47 \times 10^{-17}$ \\ 
% %   GO:0043066 & negative regulation of apoptotic process & 73 & 240 & $3.94 \times 10^{-20}$ & $2.36 \times 10^{-17}$ \\ 
% %   \bottomrule
% % \end{tabular}
% % \end{adjustbox}
% % \caption[Gene ontology enrichment High betweenness genes Biological Process of genes above 90th centile of distribution]{\textbf{High betweenness genes. Biological Process - Gene ontology enrichment analysis} using the ToppGene web client and all PSP genes as the background set.  Biological process tested gene set is 90th centile of degree distribution.  p value; q FDR B H = q adjusted significance level False Discovery Rate using Benjamini and Hochberg adjustment; $n$= $n$ genes annotated in test group; $n$ PSP= n genes annotated in PSP. $n$ ontology terms significantly (q FDR$>=0.05$) enriched for this centrality measure and ontology category (e.g. MF,CC,BP): \textcolor{red}{n}\url{source('~/RProjects/chapter3/R/sig_magma_genes/central_genes_toppgene/betweenness/5_0_clip_centrality_bet_hig_bp.R')}} 
% % \label{tab:ToppGENE GO: Biological Process. bet 90 centile cwpsp.txtp = p value; q FDR B H = q adjusted significance level False Discovery Rate using Benjamini and Hochberg adjustment; n= n genes annotated in test group; n PSP= n genes annotated in PSP}
% % \end{table}

% % \paragraph{Cellular component}
% % Cellular component table~\ref{tab:ToppGENE GO: Cellular Component. bet 90 centile cwpsp.txtp = p value; q FDR B H = q adjusted significance level False Discovery Rate using Benjamini and Hochberg adjustment; n= n genes annotated in test group; n PSP= n genes annotated in PSP}
% % % latex table generated in R 3.6.3 by xtable 1.8-4 package
% % % Sun Oct  4 11:50:08 2020

% %     \begin{table}[ht]
% % \centering
% % \begin{adjustbox}{width=\textwidth}
% % \setlength{\extrarowheight}{2pt}
% % \begin{tabular}{@{}clllcl@{}}
% %   \toprule
% %   ID & Cellular Component & $n$ & $n$ PSP & $p$ & $q$ FDR B H \\ 

% %   \midrule
% % GO:0015630 & microtubule cytoskeleton & 98 & 447 & $4.56 \times 10^{-16}$ & $4.02 \times 10^{-13}$ \\ 
% %   GO:0005814 & centriole & 55 & 194 & $7.79 \times 10^{-14}$ & $3.43 \times 10^{-11}$ \\ 
% %   GO:0005813 & centrosome & 52 & 182 & $2.92 \times 10^{-13}$ & $8.59 \times 10^{-11}$ \\ 
% %   GO:0005815 & microtubule organizing center & 62 & 246 & $6.14 \times 10^{-13}$ & $1.35 \times 10^{-10}$ \\ 
% %   GO:0030990 & intraciliary transport particle & 51 & 212 & $5.95 \times 10^{-10}$ & $8.73 \times 10^{-8}$ \\ 
% %   GO:0005929 & cilium & 51 & 212 & $5.95 \times 10^{-10}$ & $8.73 \times 10^{-8}$ \\ 
% %   GO:0098589 & membrane region & 42 & 160 & $1.37 \times 10^{-9}$ & $1.72 \times 10^{-7}$ \\ 
% %   GO:0005819 & spindle & 35 & 120 & $1.74 \times 10^{-9}$ & $1.92 \times 10^{-7}$ \\ 
% %   GO:0099513 & polymeric cytoskeletal fiber & 87 & 482 & $3.13 \times 10^{-9}$ & $2.46 \times 10^{-7}$ \\ 
% %   GO:0045121 & membrane raft & 41 & 158 & $3.20 \times 10^{-9}$ & $2.46 \times 10^{-7}$ \\ 
% %   \hline
% % \end{tabular}
% % \end{adjustbox}
% % \caption[Gene ontology enrichment High betweenness genes Cellular Component of genes above 90th centile of distribution]{\textbf{High betweenness genes. Cellular Component - Gene ontology enrichment analysis} using the ToppGene web client and all PSP genes as the background set.  Biological process tested gene set is 90th centile of degree distribution.  p value; q FDR B H = q adjusted significance level False Discovery Rate using Benjamini and Hochberg adjustment; $n$= $n$ genes annotated in test group; $n$ PSP= n genes annotated in PSP. $n$ ontology terms significantly (q FDR$>=0.05$) enriched for this centrality measure and ontology category (e.g. MF,CC,BP): \textcolor{red}{n}}
% % \label{tab:ToppGENE GO: Cellular Component. bet 90 centile cwpsp.txtp = p value; q FDR B H = q adjusted significance level False Discovery Rate using Benjamini and Hochberg adjustment; n= n genes annotated in test group; n PSP= n genes annotated in PSP}
% % \end{table}





% % \clearpage

% % \subsubsection{Low betweenness}

% % \paragraph{Cellular component}

% % Cellular component table~\ref{tab:ToppGENE GO: Cellular Component. bet 10 centile cwpsp.txtp = p value; q FDR B H = q adjusted significance level False Discovery Rate using Benjamini and Hochberg adjustment; n= n genes annotated in test group; n PSP= n genes annotated in PSP}


% % % latex table generated in R 3.6.3 by xtable 1.8-4 package
% % % Sun Oct  4 11:58:29 2020
% % \begin{table}[ht]
% % \centering
% % \begin{adjustbox}{width=\textwidth}
% % \setlength{\extrarowheight}{2pt}
% % \begin{tabular}{@{}clllcl@{}}
% %   \toprule
% %   ID & Cellular Component & $n$ & $n$ PSP & $p$ & $q$ FDR B H \\ 

% %   \midrule
% % GO:0031226 & intrinsic component of plasma membrane & 61 & 331 & $1.85 \times 10^{-6}$ & $9.76 \times 10^{-4}$ \\ 
% %   GO:0005887 & integral component of plasma membrane & 56 & 307 & $7.36 \times 10^{-6}$ & $1.94 \times 10^{-3}$ \\ 
% %   \bottomrule
% % \end{tabular}
% % \end{adjustbox}
% % \caption[Gene ontology enrichment Low betweenness genes Cellular Component of genes below 10th centile of distribution]{\textbf{Low betweenness genes. Cellular Component - Gene ontology enrichment analysis} using the ToppGene web client and all PSP genes as the background set.  Biological process tested gene set is below 10th centile of degree distribution.  p value; q FDR B H = q adjusted significance level False Discovery Rate using Benjamini and Hochberg adjustment; $n$= $n$ genes annotated in test group; $n$ PSP= n genes annotated in PSP. $n$ ontology terms significantly (q FDR$>=0.05$) enriched for this centrality measure and ontology category (e.g. MF,CC,BP): \textcolor{red}{n}}
% % \label{tab:ToppGENE GO: Cellular Component. bet 10 centile cwpsp.txtp = p value; q FDR B H = q adjusted significance level False Discovery Rate using Benjamini and Hochberg adjustment; n= n genes annotated in test group; n PSP= n genes annotated in PSP}
% % \end{table}

% % \paragraph{Pathway}

% % Extra cellular matrix table~\ref{tab:ToppGENE Pathway. bet 10 centile cwpsp.txtp = p value; q FDR B H = q adjusted significance level False Discovery Rate using Benjamini and Hochberg adjustment; n= n genes annotated in test group; n PSP= n genes annotated in PSP}. Glycosylphosphatidylinositol proteins (GPI proteins) are directed to the endoplasmic reticulum. 




% % % latex table generated in R 3.6.3 by xtable 1.8-4 package
% % % Sun Oct  4 12:01:20 2020
% % \begin{table}[ht]
% % \centering
% % \begin{adjustbox}{width=\textwidth}
% % \begin{tabular}{lcllcl}
% %   \hline
% % ID & Pathway & n & n PSP & p & q FDR B H \\ 
% %   \hline
% % M5889 &\makecell{ Ensemble of genes encoding extracellular matrix\\ and extracellular matrix-associated proteins}  & 21 & 91 & $7.23 \times 10^{-6}$ & $8.28 \times 10^{-3}$ \\ 
% %   M1268710 & \makecell{Post-translational modification: synthesis of GPI-anchored proteins} & 6 & 11 & $9.13 \times 10^{-5}$ & $4.00 \times 10^{-2}$ \\ 
% %   M5884 & \makecell{Ensemble of genes encoding core extracellular matrix including\\ ECM glycoproteins, collagens and proteoglycans} & 10 & 31 & $1.05 \times 10^{-4}$ & $4.00 \times 10^{-2}$ \\ 
% %   \hline
% % \end{tabular}
% % \end{adjustbox}
% % \caption{ToppGene Pathway. bet 10 centile cwpsp.txtp = p value; q FDR B H = q adjusted significance level False Discovery Rate using Benjamini and Hochberg adjustment; n= n genes annotated in test group; n PSP= n genes annotated in PSP} 
% % \label{tab:ToppGENE Pathway. bet 10 centile cwpsp.txtp = p value; q FDR B H = q adjusted significance level False Discovery Rate using Benjamini and Hochberg adjustment; n= n genes annotated in test group; n PSP= n genes annotated in PSP}
% % \end{table}

% % \clearpage
% % \subsection{Closeness Centrality ToppGene}



% % \paragraph{Molecular function}

% % Molecular function Gene Ontology Enrichment High Closeness centrality in table \ref{tab:ToppGENE GO: Molecular Function. clo 90 centile cwpsp.txtp = p value; q FDR B H = q adjusted significance level False Discovery Rate using Benjamini and Hochberg adjustment; n= n genes annotated in test group; n PSP= n genes annotated in PSP}

% % % latex table generated in R 3.6.3 by xtable 1.8-4 package
% % % Sun Oct  4 12:10:43 2020

% %   \begin{table}[ht]
% % \centering
% % \begin{adjustbox}{width=\textwidth}
% % \setlength{\extrarowheight}{2pt}
% % \begin{tabular}{@{}clllcl@{}}
% %   \toprule
% %   ID & Molecular Function & $n$ & $n$ PSP & $p$ & $q$ FDR B H \\ 

% %   \midrule
% % GO:0003723 & RNA binding & 167 & 554 & $2.11 \times 10^{-50}$ & $2.14 \times 10^{-47}$ \\ 
% %   GO:0031625 & ubiquitin protein ligase binding & 57 & 117 & $1.38 \times 10^{-27}$ & $7.00 \times 10^{-25}$ \\ 
% %   GO:0044389 & ubiquitin-like protein ligase binding & 58 & 123 & $4.10 \times 10^{-27}$ & $1.39 \times 10^{-24}$ \\ 
% %   GO:0051082 & unfolded protein binding & 31 & 56 & $2.20 \times 10^{-17}$ & $5.57 \times 10^{-15}$ \\ 
% %   GO:0044877 & protein-containing complex binding & 110 & 513 & $4.36 \times 10^{-17}$ & $8.84 \times 10^{-15}$ \\ 
% %   GO:0019904 & protein domain specific binding & 81 & 329 & $5.59 \times 10^{-16}$ & $9.46 \times 10^{-14}$ \\ 
% %   GO:0003729 & mRNA binding & 39 & 100 & $6.55 \times 10^{-15}$ & $9.50 \times 10^{-13}$ \\ 
% %   GO:0008134 & transcription factor binding & 50 & 156 & $8.56 \times 10^{-15}$ & $1.09 \times 10^{-12}$ \\ 
% %   GO:0005102 & signaling receptor binding & 86 & 405 & $7.31 \times 10^{-13}$ & $8.25 \times 10^{-11}$ \\ 
% %   GO:0140097 & catalytic activity, acting on DNA & 31 & 76 & $1.13 \times 10^{-12}$ & $1.15 \times 10^{-10}$ \\ 
% %   \bottomrule
% % \end{tabular}
% % \end{adjustbox}
% % \caption[Gene ontology enrichment High Closeness genes Molecular Function of genes above 90th centile of distribution]{\textbf{High Closeness genes. Molecular Function - Gene ontology enrichment analysis} using the ToppGene web client and all PSP genes as the background set.  Molecular function tested gene set is 90th centile of degree distribution.  p value; q FDR B H = q adjusted significance level False Discovery Rate using Benjamini and Hochberg adjustment; $n$= $n$ genes annotated in test group; $n$ PSP= n genes annotated in PSP. $n$ ontology terms significantly (q FDR$>=0.05$) enriched for this centrality measure and ontology category (e.g. MF,CC,BP): \textcolor{red}{n}} 
% % \label{tab:ToppGENE GO: Molecular Function. clo 90 centile cwpsp.txtp = p value; q FDR B H = q adjusted significance level False Discovery Rate using Benjamini and Hochberg adjustment; n= n genes annotated in test group; n PSP= n genes annotated in PSP}
% % \end{table}

% % \paragraph{Biological Process}
% % Biological process table~\ref{tab:ToppGENE GO: Biological Process. clo 90 centile cwpsp.txtp = p value; q FDR B H = q adjusted significance level False Discovery Rate using Benjamini and Hochberg adjustment; n= n genes annotated in test group; n PSP= n genes annotated in PSP}

% % % latex table generated in R 3.6.3 by xtable 1.8-4 package
% % % Sun Oct  4 12:13:59 2020
% % \begin{table}[ht]
% % \centering
% % \begin{adjustbox}{width=\textwidth}
% % \setlength{\extrarowheight}{2pt}
% % \begin{tabular}{@{}clllcl@{}}
% %   \toprule
% %   ID & Biological Process & $n$ & $n$ PSP & $p$ & $q$ FDR B H \\ 

% %   \midrule
% % GO:0044265 & cellular macromolecule catabolic process & 131 & 372 & $4.77 \times 10^{-46}$ & $2.98 \times 10^{-42}$ \\ 
% %   GO:0009057 & macromolecule catabolic process & 143 & 443 & $1.34 \times 10^{-45}$ & $4.18 \times 10^{-42}$ \\ 
% %   GO:0016071 & mRNA metabolic process & 106 & 260 & $2.87 \times 10^{-43}$ & $5.96 \times 10^{-40}$ \\ 
% %   GO:0044403 & symbiotic process & 125 & 364 & $3.12 \times 10^{-42}$ & $4.86 \times 10^{-39}$ \\ 
% %   GO:0044419 & interspecies interaction between organisms & 125 & 368 & $1.23 \times 10^{-41}$ & $1.53 \times 10^{-38}$ \\ 
% %   GO:0016032 & viral process & 118 & 348 & $6.85 \times 10^{-39}$ & $7.13 \times 10^{-36}$ \\ 
% %   GO:0006402 & mRNA catabolic process & 75 & 180 & $1.13 \times 10^{-30}$ & $1.00 \times 10^{-27}$ \\ 
% %   GO:0006401 & RNA catabolic process & 76 & 186 & $2.10 \times 10^{-30}$ & $1.63 \times 10^{-27}$ \\ 
% %   GO:0010608 & posttranscriptional regulation of gene expression & 82 & 225 & $1.11 \times 10^{-28}$ & $7.69 \times 10^{-26}$ \\ 
% %   GO:0034655 & nucleobase-containing compound catabolic process & 80 & 217 & $2.37 \times 10^{-28}$ & $1.48 \times 10^{-25}$ \\ 
% %   \bottomrule
% % \end{tabular}
% % \end{adjustbox}
% % \caption[Gene ontology enrichment High Closeness genes Biological Process of genes above 90th centile of distribution]{\textbf{High Closeness genes. Biological Process - Gene ontology enrichment analysis} using the ToppGene web client and all PSP genes as the background set.  Biological process tested gene set is 90th centile of degree distribution.  p value; q FDR B H = q adjusted significance level False Discovery Rate using Benjamini and Hochberg adjustment; $n$= $n$ genes annotated in test group; $n$ PSP= n genes annotated in PSP. $n$ ontology terms significantly (q FDR$>=0.05$) enriched for this centrality measure and ontology category (e.g. MF,CC,BP): \textcolor{red}{n}} 

% % \label{tab:ToppGENE GO: Biological Process. clo 90 centile cwpsp.txtp = p value; q FDR B H = q adjusted significance level False Discovery Rate using Benjamini and Hochberg adjustment; n= n genes annotated in test group; n PSP= n genes annotated in PSP}
% % \end{table}





% % \paragraph{Cellular component}
% % Cellular component table~\ref{tab:ToppGENE GO: Cellular Component. clo 90 centile cwpsp.txtp = p value; q FDR B H = q adjusted significance level False Discovery Rate using Benjamini and Hochberg adjustment; n= n genes annotated in test group; n PSP= n genes annotated in PSP}

% % % latex table generated in R 3.6.3 by xtable 1.8-4 package
% % % Sun Oct  4 12:15:50 2020
% % \begin{table}[ht]
% % \centering
% % \begin{adjustbox}{width=\textwidth}
% % \setlength{\extrarowheight}{2pt}
% % \begin{tabular}{@{}clllcl@{}}
% %   \toprule
% %   ID & Cellular Component & $n$ & $n$ PSP & $p$ & $q$ FDR B H \\ 

% %   \midrule
% % GO:1990904 & ribonucleoprotein complex & 102 & 303 & $5.76 \times 10^{-33}$ & $4.89 \times 10^{-30}$ \\ 
% %   GO:0005925 & focal adhesion & 77 & 239 & $4.94 \times 10^{-23}$ & $2.10 \times 10^{-20}$ \\ 
% %   GO:0030055 & cell-substrate junction & 77 & 244 & $2.18 \times 10^{-22}$ & $6.15 \times 10^{-20}$ \\ 
% %   GO:0005912 & adherens junction & 85 & 314 & $8.51 \times 10^{-20}$ & $1.80 \times 10^{-17}$ \\ 
% %   GO:0070161 & anchoring junction & 85 & 319 & $2.64 \times 10^{-19}$ & $4.47 \times 10^{-17}$ \\ 
% %   GO:0035770 & ribonucleoprotein granule & 36 & 78 & $1.03 \times 10^{-16}$ & $1.30 \times 10^{-14}$ \\ 
% %   GO:0036464 & cytoplasmic ribonucleoprotein granule & 35 & 74 & $1.07 \times 10^{-16}$ & $1.30 \times 10^{-14}$ \\ 
% %   GO:0000974 & Prp19 complex & 20 & 31 & $2.08 \times 10^{-13}$ & $2.21 \times 10^{-11}$ \\ 
% %   GO:0101031 & chaperone complex & 16 & 21 & $1.04 \times 10^{-12}$ & $9.79 \times 10^{-11}$ \\ 
% %   GO:0005681 & spliceosomal complex & 22 & 40 & $1.30 \times 10^{-12}$ & $1.11 \times 10^{-10}$ \\ 
% %   \bottomrule
% % \end{tabular}
% % \end{adjustbox}
% % \caption[Gene ontology enrichment High Closeness genes Cellular Component of genes above 90th centile of distribution]{\textbf{High Closeness genes. Cellular Component - Gene ontology enrichment analysis} using the ToppGene web client and all PSP genes as the background set.  Biological process tested gene set is 90th centile of degree distribution.  p value; q FDR B H = q adjusted significance level False Discovery Rate using Benjamini and Hochberg adjustment; $n$= $n$ genes annotated in test group; $n$ PSP= n genes annotated in PSP. $n$ ontology terms significantly (q FDR$>=0.05$) enriched for this centrality measure and ontology category (e.g. MF,CC,BP): \textcolor{red}{n}} 

% % \label{tab:ToppGENE GO: Cellular Component. clo 90 centile cwpsp.txtp = p value; q FDR B H = q adjusted significance level False Discovery Rate using Benjamini and Hochberg adjustment; n= n genes annotated in test group; n PSP= n genes annotated in PSP}
% % \end{table}






% % \subsubsection{Low closeness}

% % Note human phenotype epilepsy and disease enriched\footnote{\url{source('~/RProjects/chapter3/R/sig_magma_genes/central_genes_toppgene/closeness/low_closeness/5_0_clip_centrality_clo_low_cc_n_10.R'}}
% % \paragraph{Molecular function}
% % Molecular function table~\ref{tab:ToppGENE GO: Molecular Function. clo 10 centile cwpsp.txtp = p value; q FDR B H = q adjusted significance level False Discovery Rate using Benjamini and Hochberg adjustment; n= n genes annotated in test group; n PSP= n genes annotated in PSP}. Closeness is more normally distributed it appears the low part of its distribution is also more discriminating than those with power law distribution that have a number of genes separated by small absolute values at the lower end of the distribution. Genes encoding proteins with low closeness centrality are enriched for molecular functions involving ion channels and trans-membrane transporters. 

% % % latex table generated in R 3.6.3 by xtable 1.8-4 package
% % % Sun Oct  4 12:19:53 2020

% %   \begin{table}[ht]
% % \centering
% % \begin{adjustbox}{width=\textwidth}
% % \setlength{\extrarowheight}{2pt}
% % \begin{tabular}{@{}clllcl@{}}
% %   \toprule
% %   ID & Molecular Function & $n$ & $n$ PSP & $p$ & $q$ FDR B H \\ 

% %   \midrule
% % GO:0005215 & transporter activity & 66 & 419 & $2.40 \times 10^{-5}$ & $1.02 \times 10^{-2}$ \\ 
% %   GO:0022857 & transmembrane transporter activity & 61 & 378 & $2.44 \times 10^{-5}$ & $1.02 \times 10^{-2}$ \\ 
% %   GO:0004930 & G protein-coupled receptor activity & 10 & 24 & $3.91 \times 10^{-5}$ & $1.09 \times 10^{-2}$ \\ 
% %   GO:0043176 & amine binding & 4 & 4 & $1.01 \times 10^{-4}$ & $2.10 \times 10^{-2}$ \\ 
% %   GO:0046873 & metal ion transmembrane transporter activity & 36 & 203 & $2.12 \times 10^{-4}$ & $3.19 \times 10^{-2}$ \\ 
% %   GO:0015075 & ion transmembrane transporter activity & 53 & 340 & $2.34 \times 10^{-4}$ & $3.19 \times 10^{-2}$ \\ 
% %   GO:0022803 & passive transmembrane transporter activity & 39 & 230 & $3.05 \times 10^{-4}$ & $3.19 \times 10^{-2}$ \\ 
% %   GO:0015267 & channel activity & 39 & 230 & $3.05 \times 10^{-4}$ & $3.19 \times 10^{-2}$ \\ 
% %   GO:0008324 & cation transmembrane transporter activity & 46 & 288 & $3.61 \times 10^{-4}$ & $3.36 \times 10^{-2}$ \\ 
% %   GO:0015318 & inorganic molecular entity transmembrane transporter activity & 50 & 324 & $4.60 \times 10^{-4}$ & $3.86 \times 10^{-2}$ \\ 
% %   \bottomrule
% % \end{tabular}
% % \end{adjustbox}
% % \caption[Gene ontology enrichment Low Closeness genes Molecular Function of genes above 90th centile of distribution]{\textbf{Low Closeness genes. Molecular Function - Gene ontology enrichment analysis} using the ToppGene web client and all PSP genes as the background set.  Biological process tested gene set is below 10th centile of degree distribution.  p value; q FDR B H = q adjusted significance level False Discovery Rate using Benjamini and Hochberg adjustment; $n$= $n$ genes annotated in test group; $n$ PSP= n genes annotated in PSP. $n$ ontology terms significantly (q FDR$>=0.05$) enriched for this centrality measure and ontology category (e.g. MF,CC,BP): \textcolor{red}{n}} 
 
% % \label{tab:ToppGENE GO: Molecular Function. clo 10 centile cwpsp.txtp = p value; q FDR B H = q adjusted significance level False Discovery Rate using Benjamini and Hochberg adjustment; n= n genes annotated in test group; n PSP= n genes annotated in PSP}
% % \end{table}


% % \paragraph{Cellular component}
% % Cellular component table~\ref{tab:ToppGENE GO: Cellular Component. clo 10 centile cwpsp.txtp = p value; q FDR B H = q adjusted significance level False Discovery Rate using Benjamini and Hochberg adjustment; n= n genes annotated in test group; n PSP= n genes annotated in PSP}. Again shows transmembrane transporters also integral part of synaptic and post synaptic membrane. The proteins encoded by these genes the ion channels and synaptic membrane components are also on the periphery of the network in the sense that they are more distant from most other nodes.




% %   \begin{table}[ht]
% % \centering
% % \begin{adjustbox}{width=\textwidth}
% % \setlength{\extrarowheight}{2pt}
% % \begin{tabular}{@{}clllcl@{}}
% %   \toprule
% %   ID & Cellular Component & $n$ & $n$ PSP & $p$ & $q$ FDR B H \\ 

% %   \midrule
% % GO:0031226 & intrinsic component of plasma membrane & 72 & 331 & $8.90 \times 10^{-12}$ & $4.95 \times 10^{-9}$ \\ 
% %   GO:0005887 & integral component of plasma membrane & 66 & 307 & $1.41 \times 10^{-10}$ & $3.91 \times 10^{-8}$ \\ 
% %   GO:1990351 & transporter complex & 30 & 132 & $7.50 \times 10^{-6}$ & $1.35 \times 10^{-3}$ \\ 
% %   GO:1902495 & transmembrane transporter complex & 29 & 127 & $9.72 \times 10^{-6}$ & $1.35 \times 10^{-3}$ \\ 
% %   GO:0034702 & ion channel complex & 27 & 117 & $1.62 \times 10^{-5}$ & $1.57 \times 10^{-3}$ \\ 
% %   GO:0034703 & cation channel complex & 24 & 98 & $1.69 \times 10^{-5}$ & $1.57 \times 10^{-3}$ \\ 
% %   GO:0099055 & integral component of postsynaptic membrane & 19 & 72 & $4.33 \times 10^{-5}$ & $3.22 \times 10^{-3}$ \\ 
% %   GO:0072534 & perineuronal net & 5 & 6 & $5.12 \times 10^{-5}$ & $3.22 \times 10^{-3}$ \\ 
% %   GO:0098936 & intrinsic component of postsynaptic membrane & 20 & 79 & $5.22 \times 10^{-5}$ & $3.22 \times 10^{-3}$ \\ 
% %   GO:0099699 & integral component of synaptic membrane & 22 & 96 & $1.13 \times 10^{-4}$ & $6.28 \times 10^{-3}$ \\ 
% %   \bottomrule
% % \end{tabular}
% % \end{adjustbox}
% % \caption[Gene ontology enrichment Low Closeness genes Cellular Component of genes above 90th centile of distribution]{\textbf{Low Closeness genes. Cellular Component - Gene ontology enrichment analysis} using the ToppGene web client and all PSP genes as the background set.  Biological process tested gene set is below 10th centile of degree distribution.  p value; q FDR B H = q adjusted significance level False Discovery Rate using Benjamini and Hochberg adjustment; $n$= $n$ genes annotated in test group; $n$ PSP= n genes annotated in PSP. $n$ ontology terms significantly (q FDR$>=0.05$) enriched for this centrality measure and ontology category (e.g. MF,CC,BP): \textcolor{red}{n}} 

% % \label{tab:ToppGENE GO: Cellular Component. clo 10 centile cwpsp.txtp = p value; q FDR B H = q adjusted significance level False Discovery Rate using Benjamini and Hochberg adjustment; n= n genes annotated in test group; n PSP= n genes annotated in PSP}
% % \end{table}


% % \clearpage


% % \subsection{Eigenvector centrality}
% % \subsubsection{High eigenvector centrality}

% % \paragraph{Molecular function}

% % Molecular function table~\ref{tab:ToppGENE GO: Molecular Function. 90 centile cw psp eig.txtp = p value; q FDR B H = q adjusted significance level False Discovery Rate using Benjamini and Hochberg adjustment; n= n genes annotated in test group; n PSP= n genes annotated in PSP} shows over representation of ubiquitin, RNA binding and ribosomal components. Similar to other centrality measurs but very strong over representation\footnote{\url{source('~/RProjects/chapter3/R/sig_magma_genes/central_genes_toppgene/eigenvector/5_0_clip_centrality_eig_mfunc_n_10.R')}}.


% % % latex table generated in R 3.6.3 by xtable 1.8-4 package
% % % Sun Oct  4 12:32:37 2020
% %   \begin{table}[ht]
% % \centering
% % \begin{adjustbox}{width=\textwidth}
% % \setlength{\extrarowheight}{2pt}
% % \begin{tabular}{@{}clllcl@{}}
% %   \toprule
% %   ID & Molecular Function & $n$ & $n$ PSP & $p$ & $q$ FDR B H \\ 

% %   \midrule
% % GO:0003723 & RNA binding & 198 & 554 & $7.00 \times 10^{-79}$ & $7.01 \times 10^{-76}$ \\ 
% %   GO:0003735 & structural constituent of ribosome & 54 & 103 & $2.82 \times 10^{-28}$ & $1.41 \times 10^{-25}$ \\ 
% %   GO:0031625 & ubiquitin protein ligase binding & 56 & 117 & $1.15 \times 10^{-26}$ & $3.83 \times 10^{-24}$ \\ 
% %   GO:0044389 & ubiquitin-like protein ligase binding & 57 & 123 & $3.21 \times 10^{-26}$ & $8.03 \times 10^{-24}$ \\ 
% %   GO:0003729 & mRNA binding & 51 & 100 & $5.76 \times 10^{-26}$ & $1.15 \times 10^{-23}$ \\ 
% %   GO:0005198 & structural molecule activity & 82 & 303 & $5.13 \times 10^{-19}$ & $8.55 \times 10^{-17}$ \\ 
% %   GO:0019843 & rRNA binding & 25 & 34 & $1.26 \times 10^{-18}$ & $1.80 \times 10^{-16}$ \\ 
% %   GO:0003727 & single-stranded RNA binding & 21 & 31 & $1.19 \times 10^{-14}$ & $1.49 \times 10^{-12}$ \\ 
% %   GO:0045296 & cadherin binding & 59 & 224 & $4.31 \times 10^{-13}$ & $4.80 \times 10^{-11}$ \\ 
% %   GO:0003730 & mRNA 3'-UTR binding & 20 & 32 & $5.22 \times 10^{-13}$ & $5.22 \times 10^{-11}$ \\ 
% %   \hline
% % \end{tabular}
% % \end{adjustbox}

% % \caption[Gene ontology enrichment High Eigenvector centrality genes Molecular Function of genes above 90th centile of distribution]{\textbf{High Eigenvector centrality genes. Molecular Function - Gene ontology enrichment analysis} using the ToppGene web client and all PSP genes as the background set.  Molecular function tested gene set is 90th centile of degree distribution.  p value; q FDR B H = q adjusted significance level False Discovery Rate using Benjamini and Hochberg adjustment; $n$= $n$ genes annotated in test group; $n$ PSP= n genes annotated in PSP. $n$ ontology terms significantly (q FDR$>=0.05$) enriched for this centrality measure and ontology category (e.g. MF,CC,BP): \textcolor{red}{n}} 
% % \label{tab:ToppGENE GO: Molecular Function. 90 centile cw psp eig.txtp = p value; q FDR B H = q adjusted significance level False Discovery Rate using Benjamini and Hochberg adjustment; n= n genes annotated in test group; n PSP= n genes annotated in PSP}
% % \end{table}

% % \paragraph{Biological process}
% % Biological process table~\ref{tab:ToppGENE GO: Biological Process. 90 centile cw psp eig.txtp = p value; q FDR B H = q adjusted significance level False Discovery Rate using Benjamini and Hochberg adjustment; n= n genes annotated in test group; n PSP= n genes annotated in PSP}

% % % latex table generated in R 3.6.3 by xtable 1.8-4 package
% % % Sun Oct  4 12:34:28 2020
% %   \begin{table}[ht]
% % \centering
% % \begin{adjustbox}{width=\textwidth}
% % \setlength{\extrarowheight}{2pt}
% % \begin{tabular}{@{}clllcl@{}}
% %   \toprule
% %   ID & Biological Process & $n$ & $n$ PSP & $p$ & $q$ FDR B H \\ 

% %   \midrule
% % GO:0016071 & mRNA metabolic process & 143 & 260 & $1.30 \times 10^{-83}$ & $7.64 \times 10^{-80}$ \\ 
% %   GO:0009057 & macromolecule catabolic process & 159 & 443 & $1.09 \times 10^{-59}$ & $3.20 \times 10^{-56}$ \\ 
% %   GO:0044265 & cellular macromolecule catabolic process & 144 & 372 & $5.80 \times 10^{-58}$ & $9.34 \times 10^{-55}$ \\ 
% %   GO:0006402 & mRNA catabolic process & 101 & 180 & $6.34 \times 10^{-58}$ & $9.34 \times 10^{-55}$ \\ 
% %   GO:0006401 & RNA catabolic process & 102 & 186 & $3.28 \times 10^{-57}$ & $3.87 \times 10^{-54}$ \\ 
% %   GO:0044403 & symbiotic process & 139 & 364 & $1.05 \times 10^{-54}$ & $1.03 \times 10^{-51}$ \\ 
% %   GO:0044419 & interspecies interaction between organisms & 139 & 368 & $5.35 \times 10^{-54}$ & $4.50 \times 10^{-51}$ \\ 
% %   GO:0016032 & viral process & 134 & 348 & $6.30 \times 10^{-53}$ & $4.64 \times 10^{-50}$ \\ 
% %   GO:0034655 & nucleobase-containing compound catabolic process & 104 & 217 & $7.76 \times 10^{-51}$ & $5.08 \times 10^{-48}$ \\ 
% %   GO:0019439 & aromatic compound catabolic process & 104 & 232 & $2.67 \times 10^{-47}$ & $1.58 \times 10^{-44}$ \\ 
% %   \bottomrule
% % \end{tabular}
% % \end{adjustbox}
% % \caption[Gene ontology enrichment High Eigenvector centrality genes Biological Process of genes above 90th centile of distribution]{\textbf{High Eigenvector centrality genes. Biological Process - Gene ontology enrichment analysis} using the ToppGene web client and all PSP genes as the background set.  Molecular function tested gene set is 90th centile of degree distribution.  p value; q FDR B H = q adjusted significance level False Discovery Rate using Benjamini and Hochberg adjustment; $n$= $n$ genes annotated in test group; $n$ PSP= n genes annotated in PSP. $n$ ontology terms significantly (q FDR$>=0.05$) enriched for this centrality measure and ontology category (e.g. MF,CC,BP): \textcolor{red}{n}} 
% % \label{tab:ToppGENE GO: Biological Process. 90 centile cw psp eig.txtp = p value; q FDR B H = q adjusted significance level False Discovery Rate using Benjamini and Hochberg adjustment; n= n genes annotated in test group; n PSP= n genes annotated in PSP}
% % \end{table}


% % \paragraph{Cellular component}
% % Cellular component table~\ref{tab:ToppGENE GO: Cellular Component. 90 centile cw psp eig.txtp = p value; q FDR B H = q adjusted significance level False Discovery Rate using Benjamini and Hochberg adjustment; n= n genes annotated in test group; n PSP= n genes annotated in PSP}. PRP is pre-ribosomal rna processing complx 19.snRNP are small ribonuclear protein. Enrichment for ribosomes and for focal adhesion. U5 snRNP associates with Prp8 (ref wikipedia). The coverage of some of these ontology terms for cellular component (to coin a phrase where i am referring to how much n in the group represents of the whole annotation). It finds 28 of 34 U5 snRNP genes. 

% % % latex table generated in R 3.6.3 by xtable 1.8-4 package
% % % Sun Oct  4 12:38:30 2020
% %  \begin{table}[ht]
% % \centering
% % \begin{adjustbox}{width=\textwidth}
% % \setlength{\extrarowheight}{2pt}
% % \begin{tabular}{@{}clllcl@{}}
% %   \toprule
% %   ID & Cellular Component & $n$ & $n$ PSP & $p$ & $q$ FDR B H \\ 

% %   \midrule
% % GO:1990904 & ribonucleoprotein complex & 144 & 303 & $4.68 \times 10^{-73}$ & $3.68 \times 10^{-70}$ \\ 
% %   GO:0022626 & cytosolic ribosome & 60 & 97 & $3.21 \times 10^{-37}$ & $1.26 \times 10^{-34}$ \\ 
% %   GO:0044391 & ribosomal subunit & 61 & 124 & $4.99 \times 10^{-30}$ & $1.31 \times 10^{-27}$ \\ 
% %   GO:0005840 & ribosome & 65 & 141 & $6.93 \times 10^{-30}$ & $1.36 \times 10^{-27}$ \\ 
% %   GO:0000974 & Prp19 complex & 29 & 31 & $1.48 \times 10^{-27}$ & $2.33 \times 10^{-25}$ \\ 
% %   GO:0005681 & spliceosomal complex & 32 & 40 & $1.11 \times 10^{-25}$ & $1.46 \times 10^{-23}$ \\ 
% %   GO:0071013 & catalytic step 2 spliceosome & 27 & 29 & $1.49 \times 10^{-25}$ & $1.67 \times 10^{-23}$ \\ 
% %   GO:0097525 & spliceosomal snRNP complex & 29 & 35 & $3.56 \times 10^{-24}$ & $3.49 \times 10^{-22}$ \\ 
% %   GO:0005925 & focal adhesion & 78 & 239 & $7.67 \times 10^{-24}$ & $6.70 \times 10^{-22}$ \\ 
% %   GO:0005682 & U5 snRNP & 28 & 34 & $3.16 \times 10^{-23}$ & $2.48 \times 10^{-21}$ \\ 
% %   \bottomrule
% % \end{tabular}
% % \end{adjustbox}
% % \caption[Gene ontology enrichment High Eigenvector centrality genes Cellular Component of genes above 90th centile of distribution]{\textbf{High Eigenvector centrality genes. Cellular Component - Gene ontology enrichment analysis} using the ToppGene web client and all PSP genes as the background set.  Molecular function tested gene set is 90th centile of degree distribution.  p value; q FDR B H = q adjusted significance level False Discovery Rate using Benjamini and Hochberg adjustment; $n$= $n$ genes annotated in test group; $n$ PSP= n genes annotated in PSP. $n$ ontology terms significantly (q FDR$>=0.05$) enriched for this centrality measure and ontology category (e.g. MF,CC,BP): \textcolor{red}{n}} 

% % \label{tab:ToppGENE GO: Cellular Component. 90 centile cw psp eig.txtp = p value; q FDR B H = q adjusted significance level False Discovery Rate using Benjamini and Hochberg adjustment; n= n genes annotated in test group; n PSP= n genes annotated in PSP}
% % \end{table}

% % \clearpage
% % \subsubsection{Low eigenvector centrality}
% % \paragraph{Molecular function}

% % Molecular function \ref{tab:ToppGENE GO: Molecular Function. 10 centile cw psp eig.txtp = p value; q FDR B H = q adjusted significance level False Discovery Rate using Benjamini and Hochberg adjustment; n= n genes annotated in test group; n PSP= n genes annotated in PSP}. Low eigenvector genes are enriched for molecular function terms related to transmembrane transporter activity and voltage gated ion channels. 

% % No significant enrichment for biological process. 

% % % latex table generated in R 3.6.3 by xtable 1.8-4 package
% % % Sun Oct  4 14:17:27 2020
% %  \begin{table}[ht]
% % \centering
% % \begin{adjustbox}{width=\textwidth}
% % \setlength{\extrarowheight}{2pt}
% % \begin{tabular}{@{}clllcl@{}}
% %   \toprule
% %   ID & Molecular Function & $n$ & $n$ PSP & $p$ & $q$ FDR B H \\ 

% %   \midrule
% % GO:0022857 & transmembrane transporter activity & 69 & 378 & $5.33 \times 10^{-8}$ & $2.75 \times 10^{-5}$ \\ 
% %   GO:0005215 & transporter activity & 74 & 419 & $6.69 \times 10^{-8}$ & $2.75 \times 10^{-5}$ \\ 
% %   GO:0022803 & passive transmembrane transporter activity & 47 & 230 & $3.54 \times 10^{-7}$ & $7.28 \times 10^{-5}$ \\ 
% %   GO:0015267 & channel activity & 47 & 230 & $3.54 \times 10^{-7}$ & $7.28 \times 10^{-5}$ \\ 
% %   GO:0015075 & ion transmembrane transporter activity & 61 & 340 & $7.16 \times 10^{-7}$ & $1.18 \times 10^{-4}$ \\ 
% %   GO:0015318 & inorganic molecular entity transmembrane transporter activity & 58 & 324 & $1.54 \times 10^{-6}$ & $1.67 \times 10^{-4}$ \\ 
% %   GO:0005244 & voltage-gated ion channel activity & 29 & 118 & $1.62 \times 10^{-6}$ & $1.67 \times 10^{-4}$ \\ 
% %   GO:0022832 & voltage-gated channel activity & 29 & 118 & $1.62 \times 10^{-6}$ & $1.67 \times 10^{-4}$ \\ 
% %   GO:0008324 & cation transmembrane transporter activity & 53 & 288 & $1.96 \times 10^{-6}$ & $1.79 \times 10^{-4}$ \\ 
% %   GO:0046873 & metal ion transmembrane transporter activity & 41 & 203 & $2.97 \times 10^{-6}$ & $2.44 \times 10^{-4}$ \\ 
% %   \bottomrule
% % \end{tabular}
% % \end{adjustbox}
% % \caption[Gene ontology enrichment Low Eigenvector centrality genes Molecular Function of genes above 10th centile of distribution]{\textbf{Low Eigenvector centrality genes. Molecular Function - Gene ontology enrichment analysis} using the ToppGene web client and all PSP genes as the background set.  Molecular function tested gene set is 10th centile of degree distribution.  p value; q FDR B H = q adjusted significance level False Discovery Rate using Benjamini and Hochberg adjustment; $n$= $n$ genes annotated in test group; $n$ PSP= n genes annotated in PSP. $n$ ontology terms significantly (q FDR$>=0.05$) enriched for this centrality measure and ontology category (e.g. MF,CC,BP): \textcolor{red}{n}} 

% % \label{tab:ToppGENE GO: Molecular Function. 10 centile cw psp eig.txtp = p value; q FDR B H = q adjusted significance level False Discovery Rate using Benjamini and Hochberg adjustment; n= n genes annotated in test group; n PSP= n genes annotated in PSP}
% % \end{table}

% % \paragraph{Cellular component}

% % Cellular component table~\ref{tab:ToppGENE GO: Cellular Component. 10 centile cw psp eig.txtp = p value; q FDR B H = q adjusted significance level False Discovery Rate using Benjamini and Hochberg adjustment; n= n genes annotated in test group; n PSP= n genes annotated in PSP} is enriched for transmembrane transporter complexes, ion channels and integral and intrinsic part of synaptic and post synaptic membrane. 
% %  \begin{table}[ht]
% % \centering
% % \begin{adjustbox}{width=\textwidth}
% % \setlength{\extrarowheight}{2pt}
% % \begin{tabular}{@{}clllcl@{}}
% %   \toprule
% %   ID & Cellular Component & $n$ & $n$ PSP & $p$ & $q$ FDR B H \\ 

% %   \midrule
% % GO:0031226 & intrinsic component of plasma membrane & 75 & 331 & $3.06 \times 10^{-13}$ & $1.69 \times 10^{-10}$ \\ 
% %   GO:0005887 & integral component of plasma membrane & 69 & 307 & $5.24 \times 10^{-12}$ & $1.45 \times 10^{-9}$ \\ 
% %   GO:1990351 & transporter complex & 36 & 132 & $5.47 \times 10^{-9}$ & $9.30 \times 10^{-7}$ \\ 
% %   GO:1902495 & transmembrane transporter complex & 35 & 127 & $6.71 \times 10^{-9}$ & $9.30 \times 10^{-7}$ \\ 
% %   GO:0034702 & ion channel complex & 33 & 117 & $9.91 \times 10^{-9}$ & $1.10 \times 10^{-6}$ \\ 
% %   GO:0034703 & cation channel complex & 29 & 98 & $2.53 \times 10^{-8}$ & $2.33 \times 10^{-6}$ \\ 
% %   GO:0099055 & integral component of postsynaptic membrane & 21 & 72 & $3.01 \times 10^{-6}$ & $2.37 \times 10^{-4}$ \\ 
% %   GO:0099699 & integral component of synaptic membrane & 25 & 96 & $3.42 \times 10^{-6}$ & $2.37 \times 10^{-4}$ \\ 
% %   GO:0098936 & intrinsic component of postsynaptic membrane & 22 & 79 & $4.08 \times 10^{-6}$ & $2.51 \times 10^{-4}$ \\ 
% %   GO:0099240 & intrinsic component of synaptic membrane & 26 & 108 & $1.04 \times 10^{-5}$ & $5.78 \times 10^{-4}$ \\ 
% %   \bottomrule
% % \end{tabular}
% % \end{adjustbox}
% % \caption[Gene ontology enrichment Low Eigenvector centrality genes Molecular Function of genes above 10th centile of distribution]{\textbf{Low Eigenvector centrality genes. Molecular Function - Gene ontology enrichment analysis} using the ToppGene web client and all PSP genes as the background set.  Molecular function tested gene set is 10th centile of degree distribution.  p value; q FDR B H = q adjusted significance level False Discovery Rate using Benjamini and Hochberg adjustment; $n$= $n$ genes annotated in test group; $n$ PSP= n genes annotated in PSP. $n$ ontology terms significantly (q FDR$>=0.05$) enriched for this centrality measure and ontology category (e.g. MF,CC,BP): \textcolor{red}{n}} 

% % \label{tab:ToppGENE GO: Cellular Component. 10 centile cw psp eig.txtp = p value; q FDR B H = q adjusted significance level False Discovery Rate using Benjamini and Hochberg adjustment; n= n genes annotated in test group; n PSP= n genes annotated in PSP}
% % \end{table}


% % \paragraph{Other tables}
% % Pathway neuronal system compared with rest PSP  34,189 q = 0.0348

% % Human phenotype epilepsies 

% % Diseases top 10 epilepsies  
% % % latex table generated in R 3.6.3 by xtable 1.8-4 package
% % % Sun Oct  4 14:20:15 2020



% % \clearpage


% % \subsection{Transitivity}
% % Transitivity with isolates set to NA and NA removed. 

% % \textcolor{red}{Key findings}

% % Significance score lower

% % smaller groups
% % +

% % \subsubsection{High transitivity}
% % \paragraph{Molecular function}
% % Molecular function table~\ref{tab:ToppGENE GO: Molecular Function. tra 90 centile NA narm cwpsp.txtp = p value; q FDR B H = q adjusted significance level False Discovery Rate using Benjamini and Hochberg adjustment; n= n genes annotated in test group; n PSP= n genes annotated in PSP. n significant in category 2}. Only two significant terms, compare with other high centralities. Over representation of constituent of ribosome. 

% % % latex table generated in R 3.6.3 by xtable 1.8-4 package
% % % Sun Oct  4 14:35:45 2020
% %   \begin{table}[ht]
% % \centering
% % \begin{adjustbox}{width=\textwidth}
% % \setlength{\extrarowheight}{2pt}
% % \begin{tabular}{@{}clllcl@{}}
% %   \toprule
% %   ID &Molecular Function & $n$ & $n$ PSP & $p$ & $q$ FDR B H \\ 

% %   \midrule
% % GO:0003735 & structural constituent of ribosome & 29 & 103 & $1.45 \times 10^{-8}$ & $1.06 \times 10^{-5}$ \\ 
% %   GO:0005198 & structural molecule activity & 48 & 303 & $6.32 \times 10^{-5}$ & $2.31 \times 10^{-2}$ \\ 
% %   \hline
% % \end{tabular}
% % \end{adjustbox}
% % \caption[Gene ontology enrichment High Transitivity NaN genes. Molecular Function of genes above 90th centile of distribution]{\textbf{High Transitivity NaN genes. Molecular Function - Gene ontology enrichment analysis} using the ToppGene web client and all PSP genes as the background set.  Molecular function tested gene set is 90th centile of degree distribution.  p value; q FDR B H = q adjusted significance level False Discovery Rate using Benjamini and Hochberg adjustment; $n$= $n$ genes annotated in test group; $n$ PSP= n genes annotated in PSP. $n$ ontology terms significantly (q FDR$>=0.05$) enriched for this centrality measure and ontology category (e.g. MF,CC,BP): 2} 

% % \label{tab:ToppGENE GO: Molecular Function. tra 90 centile NA narm cwpsp.txtp = p value; q FDR B H = q adjusted significance level False Discovery Rate using Benjamini and Hochberg adjustment; n= n genes annotated in test group; n PSP= n genes annotated in PSP. n significant in category 2}
% % \end{table}
% % \paragraph{Biological process}
% % Biological process table~\ref{tab:ToppGENE GO: Biological Process. tra 90 centile NA narm cwpsp.txtp = p value; q FDR B H = q adjusted significance level False Discovery Rate using Benjamini and Hochberg adjustment; n= n genes annotated in test group; n PSP= n genes annotated in PSP. n significant in category 22}

% % Mitochondrial translation is enriched. 



% % % latex table generated in R 3.6.3 by xtable 1.8-4 package
% % % Sun Oct  4 14:37:31 2020
% % % latex table generated in R 3.6.3 by xtable 1.8-4 package
% % % Sun Oct  4 12:34:28 2020
% %   \begin{table}[ht]
% % \centering
% % \begin{adjustbox}{width=\textwidth}
% % \setlength{\extrarowheight}{2pt}
% % \begin{tabular}{@{}clllcl@{}}
% %   \toprule
% %   ID & Biological Process & $n$ & $n$ PSP & $p$ & $q$ FDR B H \\ 

% %   \midrule
% % GO:0006415 & translational termination & 14 & 35 & $8.46 \times 10^{-7}$ & $2.13 \times 10^{-3}$ \\ 
% %   GO:0070126 & mitochondrial translational termination & 13 & 32 & $1.71 \times 10^{-6}$ & $2.13 \times 10^{-3}$ \\ 
% %   GO:0070125 & mitochondrial translational elongation & 13 & 32 & $1.71 \times 10^{-6}$ & $2.13 \times 10^{-3}$ \\ 
% %   GO:0043624 & cellular protein complex disassembly & 25 & 108 & $7.84 \times 10^{-6}$ & $6.98 \times 10^{-3}$ \\ 
% %   GO:0032543 & mitochondrial translation & 15 & 47 & $9.37 \times 10^{-6}$ & $6.98 \times 10^{-3}$ \\ 
% %   GO:0006414 & translational elongation & 46 & 279 & $3.13 \times 10^{-5}$ & $1.66 \times 10^{-2}$ \\ 
% %   GO:0006412 & translation & 46 & 279 & $3.13 \times 10^{-5}$ & $1.66 \times 10^{-2}$ \\ 
% %   GO:0140053 & mitochondrial gene expression & 15 & 52 & $3.65 \times 10^{-5}$ & $1.70 \times 10^{-2}$ \\ 
% %   GO:0043043 & peptide biosynthetic process & 46 & 283 & $4.54 \times 10^{-5}$ & $1.72 \times 10^{-2}$ \\ 
% %   GO:0006521 & regulation of cellular amino acid metabolic process & 12 & 36 & $4.62 \times 10^{-5}$ & $1.72 \times 10^{-2}$ \\ 
% %   \bottomrule
% % \end{tabular}
% % \end{adjustbox}
% % \caption[Gene ontology enrichment High Transitivity NaN genes Biological Process of genes above 90th centile of distribution]{\textbf{High Transitivity NaN genes. Biological Process - Gene ontology enrichment analysis} using the ToppGene web client and all PSP genes as the background set.  Molecular function tested gene set is 90th centile of degree distribution.  p value; q FDR B H = q adjusted significance level False Discovery Rate using Benjamini and Hochberg adjustment; $n$= $n$ genes annotated in test group; $n$ PSP= n genes annotated in PSP. $n$ ontology terms significantly (q FDR$>=0.05$) enriched for this centrality measure and ontology category (e.g. MF,CC,BP): 22} 
% % \label{tab:ToppGENE GO: Biological Process. tra 90 centile NA narm cwpsp.txtp = p value; q FDR B H = q adjusted significance level False Discovery Rate using Benjamini and Hochberg adjustment; n= n genes annotated in test group; n PSP= n genes annotated in PSP. n significant in category 22}
% % \end{table}

% % \paragraph{Cellular component}

% % Cellular component table~\ref{tab:ToppGENE GO: Cellular Component. tra 90 centile NA narm cwpsp.txtp = p value; q FDR B H = q adjusted significance level False Discovery Rate using Benjamini and Hochberg adjustment; n= n genes annotated in test group; n PSP= n genes annotated in PSP. n significant in category 26}. Mitochondrial protein complex and ribosomal subunit are enriched consisent with findings of translation being enriched in biological process.

% % % latex table generated in R 3.6.3 by xtable 1.8-4 package
% % % Sun Oct  4 14:38:37 2020
% %   \begin{table}[ht]
% % \centering
% % \begin{adjustbox}{width=\textwidth}
% % \setlength{\extrarowheight}{2pt}
% % \begin{tabular}{@{}clllcl@{}}
% %   \toprule
% %   ID & Cellular Component & $n$ & $n$ PSP & $p$ & $q$ FDR B H \\ 

% %   \midrule
% % GO:0044391 & ribosomal subunit & 31 & 124 & $8.04 \times 10^{-8}$ & $3.86 \times 10^{-5}$ \\ 
% %   GO:0098798 & mitochondrial protein complex & 33 & 139 & $1.15 \times 10^{-7}$ & $3.86 \times 10^{-5}$ \\ 
% %   GO:0005840 & ribosome & 32 & 141 & $5.48 \times 10^{-7}$ & $1.23 \times 10^{-4}$ \\ 
% %   GO:0005761 & mitochondrial ribosome & 13 & 31 & $1.03 \times 10^{-6}$ & $1.39 \times 10^{-4}$ \\ 
% %   GO:0000313 & organellar ribosome & 13 & 31 & $1.03 \times 10^{-6}$ & $1.39 \times 10^{-4}$ \\ 
% %   GO:0005762 & mitochondrial large ribosomal subunit & 9 & 16 & $2.44 \times 10^{-6}$ & $2.34 \times 10^{-4}$ \\ 
% %   GO:0000315 & organellar large ribosomal subunit & 9 & 16 & $2.44 \times 10^{-6}$ & $2.34 \times 10^{-4}$ \\ 
% %   GO:0015934 & large ribosomal subunit & 18 & 64 & $8.18 \times 10^{-6}$ & $6.88 \times 10^{-4}$ \\ 
% %   GO:0000502 & proteasome complex & 13 & 38 & $1.50 \times 10^{-5}$ & $1.01 \times 10^{-3}$ \\ 
% %   GO:1905369 & endopeptidase complex & 13 & 38 & $1.50 \times 10^{-5}$ & $1.01 \times 10^{-3}$ \\ 
% %   \hline
% % \bottomrule

% % \end{tabular}
% % \end{adjustbox}
% % \caption[Gene ontology enrichment High Transitivity NaN genes Cellular Component of genes above 90th centile of distribution]{\textbf{High Transitivity NaN genes. Cellular Component - Gene ontology enrichment analysis} using the ToppGene web client and all PSP genes as the background set.  Molecular function tested gene set is 90th centile of degree distribution.  p value; q FDR B H = q adjusted significance level False Discovery Rate using Benjamini and Hochberg adjustment; $n$= $n$ genes annotated in test group; $n$ PSP= n genes annotated in PSP. $n$ ontology terms significantly (q FDR$>=0.05$) enriched for this centrality measure and ontology category (e.g. MF,CC,BP): 26} 
% % \label{tab:ToppGENE GO: Cellular Component. tra 90 centile NA narm cwpsp.txtp = p value; q FDR B H = q adjusted significance level False Discovery Rate using Benjamini and Hochberg adjustment; n= n genes annotated in test group; n PSP= n genes annotated in PSP. n significant in category 26}
% % \end{table}




% % \paragraph{Other}
% % Pathway related to mitochondria. No disease enrichment.




% % \subsubsection{Low transitivity}
% % No enrichment



% % \clearpage
% % \subsection{kcoreness}


% % \subsubsection{High kcoreness}
% % \paragraph{Molecular function}

% % Molecular function table~\ref{tab:ToppGENE GO: Molecular Function. kco 90 centile cwpsp.txtp = p value; q FDR B H = q adjusted significance level False Discovery Rate using Benjamini and Hochberg adjustment; n= n genes annotated in test group; n PSP= n genes annotated in PSP. n significant in category 180}
% % Ubiquitin and RNA binding. Low adjusted p for RNA binding comparable to eigenvector\footnote{\url{source('~/RProjects/chapter3/R/sig_magma_genes/central_genes_toppgene/kcoreness/5_0_clip_centrality_kco_high_mf_n_10.R')}}.

% %   \begin{table}[ht]
% % \centering
% % \begin{adjustbox}{width=\textwidth}
% % \setlength{\extrarowheight}{2pt}
% % \begin{tabular}{@{}clllcl@{}}
% %   \toprule
% %   ID & Molecular Function & $n$ & $n$ PSP & $p$ & $q$ FDR B H \\ 

% %   \midrule
% % GO:0003723 & RNA binding & 188 & 554 & $2.64 \times 10^{-58}$ & $2.68 \times 10^{-55}$ \\ 
% %   GO:0044389 & ubiquitin-like protein ligase binding & 60 & 123 & $2.82 \times 10^{-26}$ & $1.43 \times 10^{-23}$ \\ 
% %   GO:0031625 & ubiquitin protein ligase binding & 58 & 117 & $7.13 \times 10^{-26}$ & $2.42 \times 10^{-23}$ \\ 
% %   GO:0003729 & mRNA binding & 48 & 100 & $1.05 \times 10^{-20}$ & $2.66 \times 10^{-18}$ \\ 
% %   GO:0003735 & structural constituent of ribosome & 43 & 103 & $8.30 \times 10^{-16}$ & $1.69 \times 10^{-13}$ \\ 
% %   GO:0101005 & ubiquitinyl hydrolase activity & 36 & 77 & $2.84 \times 10^{-15}$ & $4.12 \times 10^{-13}$ \\ 
% %   GO:0004843 & thiol-dependent ubiquitin-specific protease activity & 36 & 77 & $2.84 \times 10^{-15}$ & $4.12 \times 10^{-13}$ \\ 
% %   GO:0008134 & transcription factor binding & 53 & 156 & $1.02 \times 10^{-14}$ & $1.30 \times 10^{-12}$ \\ 
% %   GO:0044877 & protein-containing complex binding & 112 & 513 & $4.46 \times 10^{-14}$ & $5.04 \times 10^{-12}$ \\ 
% %   GO:0019843 & rRNA binding & 22 & 34 & $1.33 \times 10^{-13}$ & $1.35 \times 10^{-11}$ \\ 
% %   \bottomrule
% % \end{tabular}
% % \end{adjustbox}
% % \caption[Gene ontology enrichment High kcoreness genes Molecular Function of genes above 90th centile of distribution]{\textbf{High kcoreness genes. Molecular Function - Gene ontology enrichment analysis} using the ToppGene web client and all PSP genes as the background set.  Molecular function tested gene set is 90th centile of degree distribution.  p value; q FDR B H = q adjusted significance level False Discovery Rate using Benjamini and Hochberg adjustment; $n$= $n$ genes annotated in test group; $n$ PSP= n genes annotated in PSP. $n$ ontology terms significantly (q FDR$>=0.05$) enriched for this centrality measure and ontology category (e.g. MF,CC,BP): 180} 
% % \label{tab:ToppGENE GO: Molecular Function. kco 90 centile cwpsp.txtp = p value; q FDR B H = q adjusted significance level False Discovery Rate using Benjamini and Hochberg adjustment; n= n genes annotated in test group; n PSP= n genes annotated in PSP. n significant in category 180}
% % \end{table}

% % \paragraph{Biological process}
% % Biological process table~\ref{tab:ToppGENE GO: Biological Process. kco 90 centile cwpsp.txtp = p value; q FDR B H = q adjusted significance level False Discovery Rate using Benjamini and Hochberg adjustment; n= n genes annotated in test group; n PSP= n genes annotated in PSP. n significant in category 1141}.

% % Again mRNA and also viral processes and symbiotic showing that these are common and probably preserved across different organisms. Very low adjusted value and high coverage for terms.


% % % latex table generated in R 3.6.3 by xtable 1.8-4 package
% % % Sun Oct  4 15:02:29 2020
% %   \begin{table}[ht]
% % \centering
% % \begin{adjustbox}{width=\textwidth}
% % \setlength{\extrarowheight}{2pt}
% % \begin{tabular}{@{}clllcl@{}}
% %   \toprule
% %   ID & Biological Process & $n$ & $n$ PSP & $p$ & $q$ FDR B H \\ 

% %   \midrule
% % GO:0016071 & mRNA metabolic process & 142 & 260 & $3.73 \times 10^{-74}$ & $2.26 \times 10^{-70}$ \\ 
% %   GO:0009057 & macromolecule catabolic process & 166 & 443 & $4.75 \times 10^{-57}$ & $1.44 \times 10^{-53}$ \\ 
% %   GO:0044265 & cellular macromolecule catabolic process & 150 & 372 & $1.43 \times 10^{-55}$ & $2.88 \times 10^{-52}$ \\ 
% %   GO:0006402 & mRNA catabolic process & 100 & 180 & $2.06 \times 10^{-51}$ & $3.12 \times 10^{-48}$ \\ 
% %   GO:0006401 & RNA catabolic process & 100 & 186 & $1.34 \times 10^{-49}$ & $1.62 \times 10^{-46}$ \\ 
% %   GO:0044403 & symbiotic process & 140 & 364 & $2.98 \times 10^{-48}$ & $3.01 \times 10^{-45}$ \\ 
% %   GO:0044419 & interspecies interaction between organisms & 140 & 368 & $1.46 \times 10^{-47}$ & $1.27 \times 10^{-44}$ \\ 
% %   GO:0016032 & viral process & 135 & 348 & $8.48 \times 10^{-47}$ & $6.43 \times 10^{-44}$ \\ 
% %   GO:0034655 & nucleobase-containing compound catabolic process & 102 & 217 & $1.67 \times 10^{-43}$ & $1.13 \times 10^{-40}$ \\ 
% %   GO:0019439 & aromatic compound catabolic process & 102 & 232 & $3.61 \times 10^{-40}$ & $2.19 \times 10^{-37}$ \\ 
% %   \bottomrule
% % \end{tabular}
% % \end{adjustbox}

% % \caption[Gene ontology enrichment High kcoreness genes Biological Process of genes above 90th centile of distribution]{\textbf{High kcoreness genes. Biological Process - Gene ontology enrichment analysis} using the ToppGene web client and all PSP genes as the background set.  Biological Process tested gene set is 90th centile of degree distribution.  p value; q FDR B H = q adjusted significance level False Discovery Rate using Benjamini and Hochberg adjustment; $n$= $n$ genes annotated in test group; $n$ PSP= n genes annotated in PSP. $n$ ontology terms significantly (q FDR$>=0.05$) enriched for this centrality measure and ontology category (e.g. MF,CC,BP): 1141} 
% % \label{tab:ToppGENE GO: Biological Process. kco 90 centile cwpsp.txtp = p value; q FDR B H = q adjusted significance level False Discovery Rate using Benjamini and Hochberg adjustment; n= n genes annotated in test group; n PSP= n genes annotated in PSP. n significant in category 1141}
% % \end{table}

% % \paragraph{Cellular component}

% % Cellular component table~\ref{tab:ToppGENE GO: Cellular Component. kco 90 centile cwpsp.txtp = p value; q FDR B H = q adjusted significance level False Discovery Rate using Benjamini and Hochberg adjustment; n= n genes annotated in test group; n PSP= n genes annotated in PSP. n significant in category 181} again ribonucleoprotein terms and splicasome terms as per eigenvector. Low FDR q values. Adherens junction which contains proteins at cell to cell junction including cadherins. 

% % % latex table generated in R 3.6.3 by xtable 1.8-4 package
% % % Sun Oct  4 15:05:27 2020
% %   \begin{table}[ht]
% % \centering
% % \begin{adjustbox}{width=\textwidth}
% % \setlength{\extrarowheight}{2pt}
% % \begin{tabular}{@{}clllcl@{}}
% %   \toprule
% %   ID & Cellular Component & $n$ & $n$ PSP & $p$ & $q$ FDR B H \\ 

% %   \midrule
% % GO:1990904 & ribonucleoprotein complex & 137 & 303 & $6.81 \times 10^{-58}$ & $5.48 \times 10^{-55}$ \\ 
% %   GO:0005925 & focal adhesion & 87 & 239 & $9.15 \times 10^{-27}$ & $3.68 \times 10^{-24}$ \\ 
% %   GO:0022626 & cytosolic ribosome & 53 & 97 & $1.86 \times 10^{-26}$ & $4.98 \times 10^{-24}$ \\ 
% %   GO:0000974 & Prp19 complex & 29 & 31 & $4.58 \times 10^{-26}$ & $8.40 \times 10^{-24}$ \\ 
% %   GO:0030055 & cell-substrate junction & 87 & 244 & $5.22 \times 10^{-26}$ & $8.40 \times 10^{-24}$ \\ 
% %   GO:0071013 & catalytic step 2 spliceosome & 27 & 29 & $3.59 \times 10^{-24}$ & $4.81 \times 10^{-22}$ \\ 
% %   GO:0005681 & spliceosomal complex & 32 & 40 & $4.61 \times 10^{-24}$ & $5.29 \times 10^{-22}$ \\ 
% %   GO:0097525 & spliceosomal snRNP complex & 29 & 35 & $1.04 \times 10^{-22}$ & $1.05 \times 10^{-20}$ \\ 
% %   GO:0005682 & U5 snRNP & 28 & 34 & $8.18 \times 10^{-22}$ & $7.31 \times 10^{-20}$ \\ 
% %   GO:0005912 & adherens junction & 94 & 314 & $1.07 \times 10^{-21}$ & $8.59 \times 10^{-20}$ \\ 
% %   \hline
% % \end{tabular}
% % \end{adjustbox}

% % \caption[Gene ontology enrichment High kcoreness genes Cellular Component of genes above 90th centile of distribution]{\textbf{High kcoreness genes. Cellular Component - Gene ontology enrichment analysis} using the ToppGene web client and all PSP genes as the background set.  Cellular Component tested gene set is 90th centile of degree distribution.  p value; q FDR B H = q adjusted significance level False Discovery Rate using Benjamini and Hochberg adjustment; $n$= $n$ genes annotated in test group; $n$ PSP= n genes annotated in PSP. $n$ ontology terms significantly (q FDR$>=0.05$) enriched for this centrality measure and ontology category (e.g. MF,CC,BP): 181} 
% % \label{tab:ToppGENE GO: Cellular Component. kco 90 centile cwpsp.txtp = p value; q FDR B H = q adjusted significance level False Discovery Rate using Benjamini and Hochberg adjustment; n= n genes annotated in test group; n PSP= n genes annotated in PSP. n significant in category 181}
% % \end{table}

% % \paragraph{Others}
% % Mouse phenotype embroynic lethality

% % Human phenotype neoplasm and limb abnormalities. 

% % Diseases all very severe viral infections and neoplasms top neurological is number 7 C0338451  Frontotemporal dementia n in high coreness 44 n in annotation in PSP 119 p $3.75 \times 10^{-13}$ FDR q $3.87 \times 10^{-10}$  

% % Pub med integrin and beta catenin in neoplasms

% % Pathway strong enrichment gene expression
% % \clearpage
% % \subsubsection{Low kcoreness}

% % \paragraph{Cellular component}
% % Cellular component table~\ref{tab:ToppGENE GO: Cellular Component. kco 10 centile cwpsp.txtp = p value; q FDR B H = q adjusted significance level False Discovery Rate using Benjamini and Hochberg adjustment; n= n genes annotated in test group; n PSP= n genes annotated in PSP. n significant in category 2}. Over representation of integral part of plasma membrane similar to low closeness the low kcore is also the periphery (although possibly a different periphery - ie core periphery rather than far away)


% % % latex table generated in R 3.6.3 by xtable 1.8-4 package
% % % Sun Oct  4 15:15:03 2020
% %  \begin{table}[ht]
% % \centering
% % \begin{adjustbox}{width=\textwidth}
% % \setlength{\extrarowheight}{2pt}
% % \begin{tabular}{@{}clllcl@{}}
% %   \toprule
% %   ID & Cellular Component & $n$ & $n$ PSP & $p$ & $q$ FDR B H \\ 

% %   \midrule
% % GO:0031226 & intrinsic component of plasma membrane & 98 & 331 & $8.26 \times 10^{-9}$ & $5.35 \times 10^{-6}$ \\ 
% %   GO:0005887 & integral component of plasma membrane & 91 & 307 & $2.94 \times 10^{-8}$ & $9.51 \times 10^{-6}$ \\ 
% %   \hline
% % \end{tabular}
% % \end{adjustbox}
% % % \caption{ToppGene GO: Cellular Component. kco 10 centile cwpsp.txtp = p value; q FDR B H = q adjusted significance level False Discovery Rate using Benjamini and Hochberg adjustment; n= n genes annotated in test group; n PSP= n genes annotated in PSP. n significant in category 2} 
% % \caption[Gene ontology enrichment Low kcoreness genes Cellular Component of genes above 90th centile of distribution]{\textbf{Low kcoreness genes. Cellular Component - Gene ontology enrichment analysis} using the ToppGene web client and all PSP genes as the background set.  Cellular Component tested gene set is 10th centile of degree distribution.  p value; q FDR B H = q adjusted significance level False Discovery Rate using Benjamini and Hochberg adjustment; $n$= $n$ genes annotated in test group; $n$ PSP= n genes annotated in PSP. $n$ ontology terms significantly (q FDR$>=0.05$) enriched for this centrality measure and ontology category (e.g. MF,CC,BP): 2} 
% % \label{tab:ToppGENE GO: Cellular Component. kco 10 centile cwpsp.txtp = p value; q FDR B H = q adjusted significance level False Discovery Rate using Benjamini and Hochberg adjustment; n= n genes annotated in test group; n PSP= n genes annotated in PSP. n significant in category 2}
% % \end{table}

% % \paragraph{Pathway}

% % Pathway shown in table~\ref{tab:ToppGENE Pathway. kco 10 centile cwpsp.txtp = p value; q FDR B H = q adjusted significance level False Discovery Rate using Benjamini and Hochberg adjustment; n= n genes annotated in test group; n PSP= n genes annotated in PSP. n significant in category 8} is enriched for extra cellular matrix and matrix associated proteins and post translational modification of GPI. Good coverage (contains a lot of those found in the PSP by selecting this centrality measure also high for a low measure . 
% %  \begin{table}[ht]
% % \centering
% % \begin{adjustbox}{width=\textwidth}
% % \setlength{\extrarowheight}{2pt}
% % \begin{tabular}{@{}clllcl@{}}
% %   \toprule
% %   ID & Pathway & $n$ & $n$ PSP & $p$ & $q$ FDR B H \\ 

% %   \midrule
% % M5889 & \makecell{Ensemble of genes encoding extracellular matrix\\ and extracellular matrix-associated proteins}  & 37 & 91 & $1.50 \times 10^{-10}$ & $2.24 \times 10^{-7}$ \\ 
% %   M5884 & \makecell{Ensemble of genes encoding core extracellular matrix\\ including ECM glycoproteins, collagens and proteoglycans} & 17 & 31 & $8.53 \times 10^{-8}$ & $6.38 \times 10^{-5}$ \\ 
% %   M3008 & Genes encoding structural ECM glycoproteins & 13 & 25 & $6.70 \times 10^{-6}$ & $3.34 \times 10^{-3}$ \\ 
% %   132956 & Metabolic pathways & 74 & 353 & $6.99 \times 10^{-5}$ & $2.05 \times 10^{-2}$ \\ 
% %   82989 & Glycerophospholipid metabolism & 12 & 26 & $7.07 \times 10^{-5}$ & $2.05 \times 10^{-2}$ \\ 
% %   M5885 & \makecell{Ensemble of genes encoding ECM-associated proteins including\\ ECM-affilaited proteins, ECM regulators and secreted factors} & 20 & 60 & $9.51 \times 10^{-5}$ & $2.05 \times 10^{-2}$ \\ 
% %   M9131 & Glycerophospholipid metabolism & 11 & 23 & $9.57 \times 10^{-5}$ & $2.05 \times 10^{-2}$ \\ 
% %   1268710 & Post-translational modification: synthesis of GPI-anchored proteins & 7 & 11 & $1.96 \times 10^{-4}$ & $3.66 \times 10^{-2}$ \\ 
% %   \hline
% % \end{tabular}
% % \end{adjustbox}
% % \caption[Gene ontology enrichment Low kcoreness genes Pathway of genes above 90th centile of distribution]{\textbf{Low kcoreness genes. Pathway - Gene ontology enrichment analysis} using the ToppGene web client and all PSP genes as the background set.  Pathway tested gene set is 10th centile of degree distribution.  p value; q FDR B H = q adjusted significance level False Discovery Rate using Benjamini and Hochberg adjustment; $n$= $n$ genes annotated in test group; $n$ PSP= n genes annotated in PSP. $n$ ontology terms significantly (q FDR$>=0.05$) enriched for this centrality measure and ontology category (e.g. MF,CC,BP): 8} 
% % % \caption{ToppGene Pathway. kco 10 centile cwpsp.txtp = p value; q FDR B H = q adjusted significance level False Discovery Rate using Benjamini and Hochberg adjustment; n= n genes annotated in test group; n PSP= n genes annotated in PSP. n significant in category 8} 
% % \label{tab:ToppGENE Pathway. kco 10 centile cwpsp.txtp = p value; q FDR B H = q adjusted significance level False Discovery Rate using Benjamini and Hochberg adjustment; n= n genes annotated in test group; n PSP= n genes annotated in PSP. n significant in category 8}
% % \end{table}
% % \paragraph{Others}
% % Pub med pathway for regulation of habenula development 19906978  Brn3a and Nurr1 mediate a gene regulatory pathway for habenula development. n 22 n in PSP 44 p $8.18 \times 10^{-7}$ FDR q $2.36 \times 10^{-2}$
% % % latex table generated in R 3.6.3 by xtable 1.8-4 package
% % % Sun Oct  4 15:17:32 2020


% % Gene family is shown in table~\ref{tab:ToppGENE Gene Family. kco 10 centile cwpsp.txtp = p value; q FDR B H = q adjusted significance level False Discovery Rate using Benjamini and Hochberg adjustment; n= n genes annotated in test group; n PSP= n genes annotated in PSP. n significant in category 5}. It is enriched for calcium voltage gate channel subunits, fibronectin and solute carriers.

% % % latex table generated in R 3.6.3 by xtable 1.8-4 package
% % % Sun Oct  4 15:24:07 2020
% %  \begin{table}[ht]
% % \centering
% % \begin{adjustbox}{width=\textwidth}
% % \setlength{\extrarowheight}{2pt}
% % \begin{tabular}{@{}clllcl@{}}
% %   \toprule
% %   ID & Gene Family & $n$ & $n$ PSP & $p$ & $q$ FDR B H \\ 

% %   \midrule
% % 752 & Solute carriers & 21 & 45 & $7.61 \times 10^{-7}$ & $1.58 \times 10^{-4}$ \\ 
% %   253 & Calcium voltage-gated channel subunits & 8 & 14 & $4.36 \times 10^{-4}$ & $4.53 \times 10^{-2}$ \\ 
% %   1210 & ATPase phospholipid transporting & 4 & 4 & $7.39 \times 10^{-4}$ & $4.77 \times 10^{-2}$ \\ 
% %   593 & \makecell{Fibronectin type III domain containing$|$I-set domain containing$|$\\Immunoglobulin like domain containing} & 15 & 42 & $1.11 \times 10^{-3}$ & $4.77 \times 10^{-2}$ \\ 
% %   594 & \makecell{Immunoglobulin like domain containing$|$Interleukin receptors$|$\\TIR domain containing} & 13 & 34 & $1.15 \times 10^{-3}$ & $4.77 \times 10^{-2}$ \\ 
% %   \hline
% % \end{tabular}
% % \end{adjustbox}
% % \caption[Gene ontology enrichment Low kcoreness genes Gene Family of genes above 90th centile of distribution]{\textbf{Low kcoreness genes. Gene Family - Gene ontology enrichment analysis} using the ToppGene web client and all PSP genes as the background set.  Gene Family tested gene set is 10th centile of degree distribution.  p value; q FDR B H = q adjusted significance level False Discovery Rate using Benjamini and Hochberg adjustment; $n$= $n$ genes annotated in test group; $n$ PSP= n genes annotated in PSP. $n$ ontology terms significantly (q FDR$>=0.05$) enriched for this centrality measure and ontology category (e.g. MF,CC,BP): 5} 
% % % \caption{ToppGene Gene Family. kco 10 centile cwpsp.txtp = p value; q FDR B H = q adjusted significance level False Discovery Rate using Benjamini and Hochberg adjustment; n= n genes annotated in test group; n PSP= n genes annotated in PSP. n significant in category 5} 
% % \label{tab:ToppGENE Gene Family. kco 10 centile cwpsp.txtp = p value; q FDR B H = q adjusted significance level False Discovery Rate using Benjamini and Hochberg adjustment; n= n genes annotated in test group; n PSP= n genes annotated in PSP. n significant in category 5}
% % \end{table}

% % \clearpage
% % \subsection{Ontology overlap}
% % \label{sec: ontology overlap}
% % The source for this is chapter3 mac

% % Top twenty terms

% % Jaccard between and degree 0.6

% % \subsubsection{High centrality}

% % % latex table generated in R 3.6.1 by xtable 1.8-4 package
% % % Tue Dec 15 08:14:34 2020
% % \begin{table}[ht]
% % \centering
% % \begin{tabular}{rrrrrrrr}
% %   \hline
% %  & degree & betweenness & eigenvector & closeness & kcoreness & tra0 & traNA \\ 
% %   \hline
% % degree & 0.00 & 0.60 & 0.14 & 0.38 & 0.25 & 0.05 & 0.08 \\ 
% %   betweenness & 0.00 & 0.00 & 0.08 & 0.29 & 0.18 & 0.05 & 0.08 \\ 
% %   eigenvector & 0.00 & 0.00 & 0.00 & 0.29 & 0.54 & 0.25 & 0.21 \\ 
% %   closeness & 0.00 & 0.00 & 0.00 & 0.00 & 0.60 & 0.11 & 0.11 \\ 
% %   kcoreness & 0.00 & 0.00 & 0.00 & 0.00 & 0.00 & 0.11 & 0.11 \\ 
% %   tra0 & 0.00 & 0.00 & 0.00 & 0.00 & 0.00 & 0.00 & 0.82 \\ 
% %   traNA & 0.00 & 0.00 & 0.00 & 0.00 & 0.00 & 0.00 & 0.00 \\ 
% %   \hline
% % \end{tabular}
% % \caption{BP 0.9 top 0.9 centile of each centrality measure. Top 20 enriched GO terms considered. Result is overlap of GO terms Jaccard index. Both forms of transitivity show high overlap and a similar modest overlap with other centrality terms. tra0 is transitivity with isolates (degree 0 or 1) set to 0. traNA is transitivity with isolates set to NA and removed from calculation of mean.} 
% % \end{table}

% % % latex table generated in R 3.6.1 by xtable 1.8-4 package
% % % Tue Dec 15 08:27:44 2020
% % \begin{table}[ht]
% % \centering
% % \begin{tabular}{rrrrrrrr}
% %   \hline
% %  & degree & betweenness & eigenvector & closeness & kcoreness & tra0 & traNA \\ 
% %   \hline
% % degree & 0.00 & 0.60 & 0.48 & 0.74 & 0.67 & 0.11 & 0.08 \\ 
% %   betweenness & 0.00 & 0.00 & 0.43 & 0.60 & 0.48 & 0.14 & 0.11 \\ 
% %   eigenvector & 0.00 & 0.00 & 0.00 & 0.67 & 0.60 & 0.18 & 0.14 \\ 
% %   closeness & 0.00 & 0.00 & 0.00 & 0.00 & 0.74 & 0.11 & 0.08 \\ 
% %   kcoreness & 0.00 & 0.00 & 0.00 & 0.00 & 0.00 & 0.18 & 0.14 \\ 
% %   tra0 & 0.00 & 0.00 & 0.00 & 0.00 & 0.00 & 0.00 & 0.82 \\ 
% %   traNA & 0.00 & 0.00 & 0.00 & 0.00 & 0.00 & 0.00 & 0.00 \\ 
% %   \hline
% % \end{tabular}
% % \caption{MF 0.9 top 0.9 centile of each centrality measure. Top 20 enriched GO terms considered. Result is overlap of GO terms Jaccard index. Both forms of transitivity show high overlap and a similar modest overlap with other centrality terms. tra0 is transitivity with isolates (degree 0 or 1) set to 0. traNA is transitivity with isolates set to NA and removed from calculation of mean.} 
% % \end{table}



% % % Tue Dec 15 08:48:44 2020
% % \begin{table}[ht]
% % \centering
% % \begin{tabular}{rrrrrrrr}
% %   \hline
% %  & degree & betweenness & eigenvector & closeness & kcoreness & tra0 & traNA \\ 
% %   \hline
% % degree &  & 0.67 & 0.48 & 0.54 & 0.54 & 0.11 & 0.11 \\ 
% %   betweenness &  &  & 0.33 & 0.43 & 0.33 & 0.14 & 0.14 \\ 
% %   eigenvector &  &  &  & 0.60 & 0.67 & 0.18 & 0.18 \\ 
% %   closeness &  &  &  &  & 0.67 & 0.14 & 0.14 \\ 
% %   kcoreness &  &  &  &  &  & 0.11 & 0.11 \\ 
% %   tra0 &  &  &  &  &  &  & 0.90 \\ 
% %   traNA &  &  &  &  &  &  &  \\ 
% %   \hline
% % \end{tabular}
% % \caption{Cellular component top 0.9 centile of each centrality measure. Top 20 enriched GO terms considered. Result is overlap of GO terms Jaccard index. Both forms of transitivity show high overlap and a similar modest overlap with other centrality terms. tra0 is transitivity with isolates (degree 0 or 1) set to 0. traNA is transitivity with isolates set to NA and removed from calculation of mean.\url{source('~/RProjects/chapter3_mac/R/overlap_centralities/refactor/mod_GO_refactor_runjaccard_CC_modA_point9.R')} }

% % \end{table}

% % \subsubsection{Low centrality}



% % \clearpage
% % %\section{Panther gene ontology redo again}
% % % commented out below


% % % \subsection{Results betweenness centrality Statistics and GO enrichment}
% % % The log transform of betweenness centrality is more nearly normal. The histogram is slightly right skewed.
% % % Although the qq plot looks relatively straight the Shapiro Wilk test for normality after removing infinite values and NA of the log transform  is 
% % % W = 0.997 p = $1.1 \times 10^{-5}$

% % % Top 142 betweeness
% % % Panther 2016

% % % Parkinson’s disease p 2.48 x 10-22
% % % Dopamine mediated signalling pathway

% % % Mammalian phenotype – lethality
% % % BP – regulation of mRNA stability 
% % % MF kinase binding and ubiquitin
% % % CC cell adhesion
% % % 	]
% % \subsubsection{Results Edge betweenness results Statistics and GO enrichment}Table:Test for normality (Shapiro-Wilk) for centrality measures of PSP
% % The highest edge betweeness (21221.7 Median 321,4 mean 583.7) is between APP (351) and EGFR 1956.

% % The second highest is GRB2 (2885) and APP.
% % APP is on 12 of the top 20 edges. 
% % ELAV1 is on 5 of the top 20 edges.see \url{source('~/RProjects/centrality/R/other_centrality/eccentricity_and_distance.R')} EGFR (1956), 2885 and 7514 appear three times.



% % \clearpage

% % \paragraph{Move to method}
% % We can also look at the assortativity between nodes for degree.

% % Methods section~\ref{sec:Assortativity methods}

% % Increase in assortativity leads to increased branching, internal clustering or internal communications between clusters\cite{estrada2011combinatorial}.
% % \section{===tmp===}
% % \begin{figure}
% %     \centering
% %     \includegraphics[width=\textwidth]{images/chapter3/poweRlaw/Rplot_gamma_theme.png}
% %     \caption[$\gamma$ as function of $x_{min}$]{$\gamma$ as a function of $x_{min}$. }
% %     \tiny\url{source('~/RProjects/PhD_graphs/calculate_gamma_theme.R')}
% %     \label{fig:gamma_main}
% % \end{figure}

% % \section{===Removed===}

% % \subsection{Introduction to this chapter}




% % Although network science is a new discipline, it has been the subject of detailed popular science books \cite{barabasi2002linked} and one of its principles\footnote{the small world property} has been the subject of a play\cite{guare1990six},film and parlour game \footnote{Six Degrees of Kevin Bacon}\cite{collins1998s}. Many educated persons would \textit{now} be able to give an accurate answer to Milgram's question. 

% %  Hindsight bias \cite{fischhoff1975hindsight},\cite{fischhoff2007early}, the phenomenon where the probability of an event or finding seems greater after it has occurred, can make these results seem trivial because one can find plausible explanations for an effect or lack of effect from gene centrality on diseases and traits in general and for any specific entity. If one knew the association between centrality measures and intelligence, or any other complex trait, one could construct a convincing argument that this finding is to be expected. However the specific effect of proteins' position in the synaptic proteome network on synaptic function and complex traits remain unknown. Even if they were established for one trait it is not the case that they would hold for all. Even effects believed to be well established on simple model organisms, such as yeast, are more complicated that they first appeared as will be discussed in section~\ref{sec:Degree and essentialness}. The approach to assessing the role of central genes and gene products for a disorder or trait must therefore be both empirical and make use of the most accurate data we have to date. 
 
% %  A variety of biological networks exist and the significance of their topologies may differ. The existing literature on network centrality measures in molecular biology include both \todo{rephrase} gene regulatory networks and protein-protein interaction networks. Network analyses of gene function and regulation are often carried out upon co-expression networks, where an edge occurs when the expression of two genes are correlated\cite{zhang2005general}. The implications of modular structure or centrality can, therefore, be different from a protein protein interaction network. A hub node is co-expressed with a variety of different genes and may represent a driver of a biological process such as a transcriptional regulator \cite{parikshak2015systems}.
% % %  \todo{ DONE cite parikshak}
% % \todo{or footnote they have to be proximate in protein protein interaction networks}
% %  A hub node in a protein-protein interaction network has promiscuous interaction partners but may not drive a single process but be essential to several different processes\cite{patil2010hub}\footnote{p1931}. In addition it may interact with different proteins in different sites\cite{han2004evidence}\footnote{party and date hubs}.
 
% % % \textcolor{green}{? remove  this bitor move to earlier bit}While the measures of vertex importance are precisely defined how they will map onto animal models or complex traits is not and one can construct several plausible hypotheses of effect, effect of specific measures or no effect (hence the quotation at the start of the chapter\todo{?move this down into a fuller introduction section}.
% % \subsection{Goals of this chapter}

% % In this section I will test the hypothesis that genes with genetic variants associated with differences in educational attainment and intelligence are more likely to be central in the synaptic proteome network.

% % I will also test the properties the PSP has at a network level that may impact its function. I will test to see whether the degree distribution is scale free, whether the network has a small world configuration and whether its nodes are more likely to link together if they share certain properties (assortativity), and the implications this may have for its formation. 

% % I will calculate measures of the centrality or importance of individual nodes and test to see if these important nodes share common properties, support the centrality lethality hypothesis, and are disproportionately involved in animal models of cognition, in addition to determining if these genes are more significant in the samples of intelligence and educational attainment. I will determine if there is evidence to support that genes associated with intelligence are evolutionarily preserved using both gene list methods and population scale genomic data, and whether this is more or less true of synaptic genes. I will also use literature databases and genomic data to test whether central nodes are under purifying selection pressure. Finally I will determine whether animal model of cognition genes and genes found in human studies are more likely to form cohesive groups in the network consistent with the disease module hypothesis. 

% % In order to introduce these methods and results I will need to explain the main network statistics and centrality measures, explain what is known of the effects of central nodes in networks and in biology, introduce the concept of centrality lethality and of disease modules. 
% % \subsection{C(k)}
% % \begin{equation}
% %     log(C(k) = log(b) - \beta log(k).
% % \end{equation}


% %  in figure~\ref{tab:log_linear_model_c_k}
 
 
% %  \subsubsection{Q SUPPPLEMENTAL Linear model}
% %  A linear model of the log10 transformation of $C(k)$ and $k$, $log10(C(k))$ and $log10(k)$ is

% % \begin{equation}
% % C(k) = 10^{a + m \textrm{log}_{10}(k)},
% % \end{equation}
% % if $a$ and $m$ are the coefficients of the linear model or

% % \begin{equation}
% %     C(k) = 10^a k^m.
% % \end{equation}

% % Matching terms with equation~\ref{eq:C(k) function average transitivity and degree}

% % \begin{equation}
% %     C(k) = 10^a k^m = B k^{-\beta},
% % \end{equation}

% % so $B=10^a$ and $\beta=-m$ and fitting a linear model of the log transformed variables in and then transforming back the variables R is $B=0.604$ and $\beta$ = 0.453 (table~\ref{tab:log_linear_model_c_k} for coefficients of log10 model).


% % % Table created by stargazer v.5.2.2 by Marek Hlavac, Harvard University. E-mail: hlavac at fas.harvard.edu
% % % Date and time: Wed, Mar 10, 2021 - 17:01:23
% % \begin{table}[!htbp] \centering 
  
% % \begin{tabular}{@{\extracolsep{5pt}}lc} 
% % \\[-1.8ex]\hline 
% % \hline \\[-1.8ex] 
% %  & \multicolumn{1}{c}{\textit{Dependent variable:}} \\ 
% % \cline{2-2} 
% % \\[-1.8ex] & log10($C(k)$) \\ 
% % \hline \\[-1.8ex] 
% %  log10 Degree & $-$0.454$^{***}$ \\ 
% %   & (0.035) \\ 
% %   & \\ 
% %  Constant & $-$0.219$^{***}$ \\ 
% %   & (0.066) \\ 
% %   & \\ 
% % \hline \\[-1.8ex] 
% % Observations & 150 \\ 
% % R$^{2}$ & 0.533 \\ 
% % Adjusted R$^{2}$ & 0.530 \\ 
% % Residual Std. Error & 0.192 (df = 148) \\ 
% % F Statistic & 169.020$^{***}$ (df = 1; 148) \\ 
% % \hline 
% % \hline \\[-1.8ex] 
% % \textit{Note:}  & \multicolumn{1}{r}{$^{*}$p$<$0.1; $^{**}$p$<$0.05; $^{***}$p$<$0.01} \\ 
% % \end{tabular} 
% % \caption{Linear model of the effect of degree $k$ on log $C(k)$\url{source('~/RProjects/chapter3/R/transitivity/transitivity_and_degree/stargazer_cleaner.R')}} 
% %   \label{tab:log_linear_model_c_k} 
% % \end{table} 


% % \todo{fit non transformed linear model}
% % \textcolor{red}{plot the parameters with the empirical data}
% % Fitting a linear model omitting degree 1 and with log10 degree has Adjusted R-squared 0.61 compared to 0.4594 for the non logarithmic model.

% % \footnote{remove\textcolor{red}{Actually the first three points have high leverage for this perhaps should try the powerlaw x min but is not a distribution}

% % R sq 0.65 going from degree 5 so not too much different}


% % \begin{figure}
% %     \centering
% %     \includegraphics[width=\textwidth]{images/chapter3/ggplot2/c(k)/Rplot_C_k_new_formatted.png}
% %     \caption[Log10 plot of $C(k)$ against $k$]{Plot of degree $k$ log 10 scale with $C(k)$ also log 10 scale where $C(k)$ is the mean local undirected transitivity at degree $k$. Dark blue line represents line of linear regression and grey shading 95\% confidence intervals. The plot appears to bend down above approximately $k=50$. \url{source('~/RProjects/chapter3/R/transitivity/transitivity_and_degree/mean_deg_seq.R')}}
% %     \label{fig:C(k)_doublelog}
% % \end{figure}

% % \subsubsection{Minimum degree and correlation with transitivity}

% % Like the coefficient of the power law the correlation coefficient between transitivity and degree depends on the minimum value of transitivity used and varies as a function of $C_{i min}$ and the way in which isolates have been treated.

% % The degree is not normally distributed so we have calculated Spearman's rank correlation. 


% % \subsection{results small world}

% % \paragraph{Global transitivity}


% % The expected value of the global clustering coefficient with a given degree distribution connected randomly \cite{newman2018networks}\footnote{p332} is:

% % \begin{equation}
% %     C=\frac{1}{n}\frac{[<k^2> - <k>]^2}{<k^3>}
% % \end{equation}


% % The value of this for the PSP is 0.004541763 compared to the simulated value of 0.00510. \footnote{code \url{source('~/RProjects/centrality/R/transitivity/Expected_global_transitivity.R')}}. The global clustering coefficient is therefore high. Newman states p334 \cite{newman2018networks} that it is not clear why certain non social networks have higher than expected global clustering (for social networks it is ascribed to social choice). For other networks such as the internet with very right skewed degree distribution the clustering coefficient is less than expected. For biological systems such as food webs or the world wide web it is hypothesised that this may due to community formation (\cite{newman2018networks}\footnote{p36})


% % \paragraph{Feb}
% % Walkthrough of the global transitivities at \url{/home/grant/RProjects/chapter3/R/transitivity/global_transitivity/mkdown/global_transitivity_markdown.Rmd}




% % \subsubsection{Local clustering coefficient Results}
% % Local clustering coefficient tends to be inversely proportional to degree. It is undefined for nodes of degree one. See~figure \ref{fig:log_transitivity_degree}. It is suggested that this may also be due to community formation \cite{newman2018networks} p335.

% % Mean local transitivity is 0.171 (table~\ref{tab:Different global transitivities in igraph}). The distribution is right skewed with median of 0.129 and 3rd quartile of 0.23 and first quartile of 0.043. This does not match the arithmetic mean of vertex local clustering below it may be down to how we deal with NA. Above calculated as NA.rm
% % \subsubsection{Average path length}
% % Average path length for an Erdos Renyi graph on 1000 iterations (type gnm) 3.118 sd 0.00048

% % Mean global transitivity PSP Watts and Strogatz, isolates to 0 = 0.156. If setting isolates to NA and removing NA = 0.1713.

% % ER Mean global transititivity (Watts and Strogatz, isolates to 0) 0.0051

% % ER mean global transitivity (Watts and strogatz, isolates NA and remove 0.0051 (not identical isolates to 0 0.005098806, na 0.005116117)

% % ER mean global transitivity (triangles 0.005100209 \todo{change to correct sf but keep for now for check not using equal values}) sd 0.0001625842

% % % latex table generated in R 3.6.3 by xtable 1.8-4 package
% % % Sat Jun 20 16:17:24 2020
% % \begin{table}[ht]
% % \centering
% % \begin{tabular}{rlrrr}
% %   \hline
% %  & variable & mean & sd & PSP \\ 
% %   \hline
% % 1 & mean\_dist & 3.1182075093 & 0.0004731233 & 2.9797920751 \\ 
% %   2 & global\_transitivity & 0.0051154572 & 0.0001546763 & 0.0697271768 \\ 
% %   3 & global\_transitivity\_ws & 0.0051243477 & 0.0001684787 & 0.1562997932 \\ 
% %   4 & global\_transitivity\_wsNA & 0.0051243477 & 0.0001684787 & 0.1713696115 \\ 
% %   \hline
% % \end{tabular}
% % \caption{Supplemental 10 sf to show that the values are not identical Erdos renyi n3457 m 30498 graph. \textcolor{red}{check and to supplemental}}
% % \label{tab:transitivity_erdos_renyi 10sf}
% % \end{table}
% % % latex table generated in R 3.6.3 by xtable 1.8-4 package
% % % Sat Jun 20 16:07:46 2020

% % % latex table generated in R 3.6.3 by xtable 1.8-4 package
% % % Sat Jun 20 16:19:29 2020
% % \begin{table}[ht]
% % \centering
% % \begin{tabular}{rlrrr}
% %   \hline
% %  & variable & mean & sd & PSP \\ 
% %   \hline
% % 1 & mean\_dist & 3.1182 & 0.0005 & 2.9798 \\ 
% %   2 & global\_transitivity & 0.0051 & 0.0002 & 0.0697 \\ 
% %   3 & global\_transitivity\_ws & 0.0051 & 0.0002 & 0.1563 \\ 
% %   4 & global\_transitivity\_wsNA & 0.0051 & 0.0002 & 0.1714 \\ 
% %   \hline
% % \end{tabular}
% % \caption{Erdos renyi n3457 m 30498 graph. \textcolor{red}{? add difference mean and PSP to compare to sd. ? add watts strogatz 0 to show difference with na and then add in table caption. WS and global are almost identical in random model. Cross ref to Estrada :pca; and Global clustering. WS is higher in most real world networks where it diverges}}
% % \label{tab:transitivity_erdos_renyi}
% % \end{table}

% % See table~\ref{tab:transitivity_erdos_renyi}\footnote{Code \url{source('~/RProjects/graph_sw/R/small_world_random/er_random_ws.R')}}




% % % Sat Jun 20 16:38:56 2020
% % \begin{table}[ht]
% % \centering
% % \begin{tabular}{rlrrr}
% %   \hline
% %  & variable & $\mu\textsubscript{random}$ & $s.d.\textsubscript{random}$ & PSP \\ 
% %   \hline
% % 1 & Average path length & \textbf{3.0069} & 0.0053 & 2.9798 \\ 
% %   2 & Global transitivity $C$ & 0.0584 & 0.0008 & 0.0697 \\ 
% %   3 & Global transitivity WS $C_{WS}$ &\textbf{ 0.0683} & 0.0019 & 0.1563 \\ 
% %   4 & global transitivity WS NA $C_{WSNA}$ & 0.0748 & 0.0020 & 0.1714 \\ 
% %   \hline
% % \end{tabular}
% % \caption{Degree corrected configuration model transitivity and average path length. The average path length is smaller than in the Erdos Renyi model likely due to the increased number of hubs. \textcolor{red}{WS $>$ global for real world networks per Estrada. Shorter APL per Albert (than Erdos Renyi} See small world scale free networks are ultra small. \textcolor{red}{Add mu - sd}}
% % \label{tab:cmtransitivity_configuration_model}
% % \end{table}


% % % Sat Jun 20 16:38:56 2020
% % \begin{table}[ht]
% % \centering
% % \begin{tabular}{rlrrr}
% %   \hline
% %  & variable & $\mu\textsubscript{random}$ & $s.d.\textsubscript{random}$ & PSP \\ 
% %   \hline
% % 1 & Average path length & \textbf{3.0069} & 0.0053 & 2.9798 \\ 
% %   2 & Global transitivity $C$ & 0.0584 & 0.0008 & 0.0697 \\ 
% %   3 & Global transitivity WS $C_{WS}$ &\textbf{ 0.0683} & 0.0019 & 0.1563 \\ 
% %   4 & global transitivity WS NA $C_{WSNA}$ & 0.0748 & 0.0020 & 0.1714 \\ 
% %   \hline
% % \end{tabular}
% % \caption{Degree corrected configuration model transitivity and average path length. The average path length is smaller than in the Erdos Renyi model likely due to the increased number of hubs. \textcolor{red}{WS $>$ global for real world networks per Estrada. Shorter APL per Albert (than Erdos Renyi} See small world scale free networks are ultra small. \textcolor{red}{Add mu - sd}}
% % \label{tab:cmtransitivity_configuration_model}
% % \end{table}

% % % \begin{figure}
% % %     \centering
% % %     \includegraphics[width=\textwidth]{images/Rplot01_min_degree_transitivity.png}
% % %     \caption{Plot of minumum degree included in the calculation of Spearmans rank correlation between degree and $\rho$. Although the overall coefficient is positive the correlation for nodes above degree 5 are clearly negative and this becomes more apparent at high degree. This relationship is therefore complex and the typical relationship described in the literature may only hold for the large networks in which this relationship has been studies such as the internet. Potential replacement \url{source('~/RProjects/chapter3/R/transitivity/transitivity_and_degree/mean_deg_seq.R')} plot q}
% % %     \label{fig:Plot of minumum degree included in the calculation of Spearmans rank correlation between degree and rho}
% % % \end{figure}
% % % \textcolor{red}{moved from earlier to be towards small world}

% % \paragraph{Second bit APL}

% % \subsubsection{Comparison with erdos renyi graph}
% % "In the post synaptic proteome average path length is 2.979792. Global transitivity (Newman) is 0.069727 and transitivityWS is 0.171370(NA)"\textcolor{red}{WS$>>$ than global see p254 fig3a Estrada}

% % Watts strogatz transitivity with isolates to zero and isolates to NA and excluded is as below
% % "Transitivity zero 0.156300\textcolor{red}{this is the actual WS as described ie average and isolates to zero)}   Transitivity NA 0.171370"

% % The igraph function transitivity with type option "localaverageundirected" yields transitivityNA ie the absolute difference it 1000 iterations is less than $1\times10^{-15}$.

% % % latex table generated in R 3.6.3 by xtable 1.8-4 package
% % % Tue Oct 27 16:53:10 2020
% % \begin{table}[ht]
% % \centering
% % \begin{tabular}{lrrrrlll}
% %   \hline
% % name & mean & sd & d & pKS & Z &CI & ?\% CI  \\ 
% %   \hline
% % average path length & 3.11797 (2.979) & 0.00049 & 0.019 & 0.84 & 8907 &  3.117939& 3.118000 \\ 
% %   transitivity WS & 0.00510 (0.171) & 0.00017 & 0.015 & 0.98 &  -30452 & 0.005089474 &0.005110830 \\ 
% %   transitivityNA & 0.00510 & 0.00017 & 0.015 & 0.98 \\ 
% %   transitivity0 & 0.00510 & 0.00017 & 0.015 & 0.98 \\ 
% %   globaltransitivity & 0.00510 (0.0697) & 0.00017 & 0.014 & 0.99 \\ 
% %   \hline
% % \end{tabular}
% % \caption{Mean transitivity values for 1000 iterations of Erdos Renyi gnp model. Kolmogorov Smirnov statistic D = distance p = pvalue. One-sample Kolmogorov Smirnov test for normality. alterantive hypothesis two sided. Network values are in parenthesis. Using package BSDA alternative hypothesis true mean is not equal to PSP value. Note transitivity results are not identical other than transitivityWS and NA but are to 3 s.f. p Z $2.2<\times10^{-16}$ \url{source('~/RProjects/chapter3/R/small_world/ks_tests_er_sim.R')} } 
% % \label{tab:Mean transitivity values for 1000 iterations of Erdos Renyi gnp model.}
% % \end{table}

% % If we simulate mean distance using a degree sequence model we find no shorter average path length than found in the PSP (2.9 8) in 1000 iterations.


% % \subsection{Redo of Small world average path length}

% % In the small world region the path length is small comparable  or smaller than a random graph but with a high clustering coefficient which would not be seen in a random graph. 




% % \subsubsection{Comparison with configuration model}
% % % latex table generated in R 3.6.3 by xtable 1.8-4 package
% % % Sat Oct 31 14:30:26 2020
% % \begin{table}[ht]
% % \centering
% % \begin{tabular}{lrrrrrr}
% %   \hline
% % name & mean & sd & 2.5\% CI & 97.5\% CI & d & p \\ 
% %   \hline
% % average path length & 3.007 (2.979) & 0.005 & 2.996 & 3.018 & 0.01 & 0.98 \\ 
% %   transitivity WS & 0.077 (0.171) & 0.002 & 0.073 & 0.081 & 0.02 & 0.77 \\ 
% %   transitivityNA & 0.075 & 0.002 & 0.071 & 0.079 & 0.02 & 0.78 \\ 
% %   transitivity0 & 0.068 & 0.002 & 0.065 & 0.072 & 0.02 & 0.78 \\ 
% %   globaltransitivity & 0.058 (0.0697) & 0.001 & 0.057 & 0.060 & 0.02 & 0.83 \\ 
% %   \hline
% % \end{tabular}
% % \caption{Mean transitivity values for 1000 iterations of Configuration  model. Kolmogorov Smirnov statistic D = distance p = pvalue. One-sample Kolmogorov Smirnov test for normality. alterantive hypothesis two sided. Confidence interval are for empirical simulations. \url{source('~/RProjects/chapter3/R/small_world/ks_tests_configuration_sim.R')} } 
% % \label{tab:Mean transitivity values for 1000 iterations of Configuration model.}
% % \end{table}

% % \todo{transitivity WS is not equal to NA why?}



% % \subsubsection{Degree assortativity}
% % \todo{include the peel multi-scale mixing patterns}
% % which has similar degree assortativity to that calculated by Newman for the yeast interactome \cite{jeong2001lethality}.
% % \todo{? move this to discussion} Although it is not clear why this is the case it does have implications for models of how the the PSP was generated (including gene duplication and ortholog degree) and for 


% % consistent with older nodes being connected to each other and hence in turn being more likely to be connected to others which would be accurate for














































% % % \section{Supplemental Gene Ontology Enrichment}
% % % \subsection{Betweenness ToppGene}
% % % \subsubsection{High betweenness}
% % % \paragraph{Molecular function}
% % % Molecular function see table~\ref{tab:ToppGENE GO: Molecular Function. bet 90 centile cwpsp.txtp = p value; q FDR B H = q adjusted significance level False Discovery Rate using Benjamini and Hochberg adjustment; n= n genes annotated in test group; n PSP= n genes annotated in PSP}


% % % % latex table generated in R 3.6.3 by xtable 1.8-4 package
% % % % Sun Oct  4 11:46:04 2020
% % % \begin{table}[ht]
% % % \centering
% % % \begin{adjustbox}{width=\textwidth}
% % % \setlength{\extrarowheight}{2pt}
% % % \begin{tabular}{@{}clllcl@{}}
% % %   \toprule
% % %   ID & Molecular Function & $n$ & $n$ PSP & $p$ & $q$ FDR B H \\ 

% % %   \midrule
% % % GO:0044877 & protein-containing complex binding & 120 & 513 & $1.74 \times 10^{-22}$ & $1.83 \times 10^{-19}$ \\ 
% % %   GO:0031625 & ubiquitin protein ligase binding & 51 & 117 & $5.84 \times 10^{-22}$ & $3.07 \times 10^{-19}$ \\ 
% % %   GO:0044389 & ubiquitin-like protein ligase binding & 52 & 123 & $1.24 \times 10^{-21}$ & $4.34 \times 10^{-19}$ \\ 
% % %   GO:0019904 & protein domain specific binding & 89 & 329 & $1.02 \times 10^{-20}$ & $2.69 \times 10^{-18}$ \\ 
% % %   GO:0019900 & kinase binding & 76 & 325 & $9.81 \times 10^{-14}$ & $2.06 \times 10^{-11}$ \\ 
% % %   GO:0008234 & cysteine-type peptidase activity & 54 & 189 & $1.31 \times 10^{-13}$ & $2.29 \times 10^{-11}$ \\ 
% % %   GO:0008134 & transcription factor binding & 48 & 156 & $1.64 \times 10^{-13}$ & $2.45 \times 10^{-11}$ \\ 
% % %   GO:0005102 & signaling receptor binding & 87 & 405 & $2.09 \times 10^{-13}$ & $2.74 \times 10^{-11}$ \\ 
% % %   GO:0050839 & cell adhesion molecule binding & 69 & 288 & $5.05 \times 10^{-13}$ & $5.88 \times 10^{-11}$ \\ 
% % %   GO:0019901 & protein kinase binding & 69 & 295 & $1.77 \times 10^{-12}$ & $1.86 \times 10^{-10}$ \\ 
% % %   \bottomrule
% % % \end{tabular}
% % % \end{adjustbox}
% % % \caption[Gene ontology enrichment High betweenness genes Molecular Function of genes above 90th centile of distribution]{\textbf{High betweenness genes. Cellular Component - Gene ontology enrichment analysis} using the ToppGene web client and all PSP genes as the background set.  Biological process tested gene set is 90th centile of degree distribution.  p value; q FDR B H = q adjusted significance level False Discovery Rate using Benjamini and Hochberg adjustment; $n$= $n$ genes annotated in test group; $n$ PSP= n genes annotated in PSP. $n$ ontology terms significantly (q FDR$>=0.05$) enriched for this centrality measure and ontology category (e.g. MF,CC,BP): \textcolor{red}{n}} 

% % % \label{tab:ToppGENE GO: Molecular Function. bet 90 centile cwpsp.txtp = p value; q FDR B H = q adjusted significance level False Discovery Rate using Benjamini and Hochberg adjustment; n= n genes annotated in test group; n PSP= n genes annotated in PSP}
% % % \end{table}

% % % \paragraph{Biological process}
% % % Biological process table~\ref{tab:ToppGENE GO: Biological Process. bet 90 centile cwpsp.txtp = p value; q FDR B H = q adjusted significance level False Discovery Rate using Benjamini and Hochberg adjustment; n= n genes annotated in test group; n PSP= n genes annotated in PSP}
% % % % latex table generated in R 3.6.3 by xtable 1.8-4 package
% % % % Sun Oct  4 11:49:29 2020
% % %   \begin{table}[ht]
% % % \centering
% % % \begin{adjustbox}{width=\textwidth}
% % % \setlength{\extrarowheight}{2pt}
% % % \begin{tabular}{@{}clllcl@{}}
% % %   \toprule
% % %   ID & Biological Process & $n$ & $n$ PSP & $p$ & $q$ FDR B H \\ 

% % %   \midrule
% % % GO:0007049 & cell cycle & 124 & 489 & $5.10 \times 10^{-27}$ & $3.33 \times 10^{-23}$ \\ 
% % %   GO:0051247 & positive regulation of protein metabolic process & 117 & 463 & $4.11 \times 10^{-25}$ & $1.34 \times 10^{-21}$ \\ 
% % %   GO:0042981 & regulation of apoptotic process & 108 & 416 & $5.38 \times 10^{-24}$ & $1.17 \times 10^{-20}$ \\ 
% % %   GO:0043067 & regulation of programmed cell death & 108 & 423 & $2.44 \times 10^{-23}$ & $3.99 \times 10^{-20}$ \\ 
% % %   GO:0009894 & regulation of catabolic process & 95 & 350 & $2.16 \times 10^{-22}$ & $2.82 \times 10^{-19}$ \\ 
% % %   GO:0006508 & proteolysis & 105 & 417 & $4.31 \times 10^{-22}$ & $4.29 \times 10^{-19}$ \\ 
% % %   GO:0010941 & regulation of cell death & 112 & 464 & $4.60 \times 10^{-22}$ & $4.29 \times 10^{-19}$ \\ 
% % %   GO:0032270 & positive regulation of cellular protein metabolic process & 106 & 433 & $2.89 \times 10^{-21}$ & $2.36 \times 10^{-18}$ \\ 
% % %   GO:0009057 & macromolecule catabolic process & 106 & 443 & $2.02 \times 10^{-20}$ & $1.47 \times 10^{-17}$ \\ 
% % %   GO:0043066 & negative regulation of apoptotic process & 73 & 240 & $3.94 \times 10^{-20}$ & $2.36 \times 10^{-17}$ \\ 
% % %   \bottomrule
% % % \end{tabular}
% % % \end{adjustbox}
% % % \caption[Gene ontology enrichment High betweenness genes Biological Process of genes above 90th centile of distribution]{\textbf{High betweenness genes. Biological Process - Gene ontology enrichment analysis} using the ToppGene web client and all PSP genes as the background set.  Biological process tested gene set is 90th centile of degree distribution.  p value; q FDR B H = q adjusted significance level False Discovery Rate using Benjamini and Hochberg adjustment; $n$= $n$ genes annotated in test group; $n$ PSP= n genes annotated in PSP. $n$ ontology terms significantly (q FDR$>=0.05$) enriched for this centrality measure and ontology category (e.g. MF,CC,BP): \textcolor{red}{n}\url{source('~/RProjects/chapter3/R/sig_magma_genes/central_genes_toppgene/betweenness/5_0_clip_centrality_bet_hig_bp.R')}} 
% % % \label{tab:ToppGENE GO: Biological Process. bet 90 centile cwpsp.txtp = p value; q FDR B H = q adjusted significance level False Discovery Rate using Benjamini and Hochberg adjustment; n= n genes annotated in test group; n PSP= n genes annotated in PSP}
% % % \end{table}

% % % \paragraph{Cellular component}
% % % Cellular component table~\ref{tab:ToppGENE GO: Cellular Component. bet 90 centile cwpsp.txtp = p value; q FDR B H = q adjusted significance level False Discovery Rate using Benjamini and Hochberg adjustment; n= n genes annotated in test group; n PSP= n genes annotated in PSP}
% % % % latex table generated in R 3.6.3 by xtable 1.8-4 package
% % % % Sun Oct  4 11:50:08 2020

% % %     \begin{table}[ht]
% % % \centering
% % % \begin{adjustbox}{width=\textwidth}
% % % \setlength{\extrarowheight}{2pt}
% % % \begin{tabular}{@{}clllcl@{}}
% % %   \toprule
% % %   ID & Cellular Component & $n$ & $n$ PSP & $p$ & $q$ FDR B H \\ 

% % %   \midrule
% % % GO:0015630 & microtubule cytoskeleton & 98 & 447 & $4.56 \times 10^{-16}$ & $4.02 \times 10^{-13}$ \\ 
% % %   GO:0005814 & centriole & 55 & 194 & $7.79 \times 10^{-14}$ & $3.43 \times 10^{-11}$ \\ 
% % %   GO:0005813 & centrosome & 52 & 182 & $2.92 \times 10^{-13}$ & $8.59 \times 10^{-11}$ \\ 
% % %   GO:0005815 & microtubule organizing center & 62 & 246 & $6.14 \times 10^{-13}$ & $1.35 \times 10^{-10}$ \\ 
% % %   GO:0030990 & intraciliary transport particle & 51 & 212 & $5.95 \times 10^{-10}$ & $8.73 \times 10^{-8}$ \\ 
% % %   GO:0005929 & cilium & 51 & 212 & $5.95 \times 10^{-10}$ & $8.73 \times 10^{-8}$ \\ 
% % %   GO:0098589 & membrane region & 42 & 160 & $1.37 \times 10^{-9}$ & $1.72 \times 10^{-7}$ \\ 
% % %   GO:0005819 & spindle & 35 & 120 & $1.74 \times 10^{-9}$ & $1.92 \times 10^{-7}$ \\ 
% % %   GO:0099513 & polymeric cytoskeletal fiber & 87 & 482 & $3.13 \times 10^{-9}$ & $2.46 \times 10^{-7}$ \\ 
% % %   GO:0045121 & membrane raft & 41 & 158 & $3.20 \times 10^{-9}$ & $2.46 \times 10^{-7}$ \\ 
% % %   \hline
% % % \end{tabular}
% % % \end{adjustbox}
% % % \caption[Gene ontology enrichment High betweenness genes Cellular Component of genes above 90th centile of distribution]{\textbf{High betweenness genes. Cellular Component - Gene ontology enrichment analysis} using the ToppGene web client and all PSP genes as the background set.  Biological process tested gene set is 90th centile of degree distribution.  p value; q FDR B H = q adjusted significance level False Discovery Rate using Benjamini and Hochberg adjustment; $n$= $n$ genes annotated in test group; $n$ PSP= n genes annotated in PSP. $n$ ontology terms significantly (q FDR$>=0.05$) enriched for this centrality measure and ontology category (e.g. MF,CC,BP): \textcolor{red}{n}}
% % % \label{tab:ToppGENE GO: Cellular Component. bet 90 centile cwpsp.txtp = p value; q FDR B H = q adjusted significance level False Discovery Rate using Benjamini and Hochberg adjustment; n= n genes annotated in test group; n PSP= n genes annotated in PSP}
% % % \end{table}





% % % \clearpage

% % % \subsubsection{Low betweenness}

% % % \paragraph{Cellular component}

% % % Cellular component table~\ref{tab:ToppGENE GO: Cellular Component. bet 10 centile cwpsp.txtp = p value; q FDR B H = q adjusted significance level False Discovery Rate using Benjamini and Hochberg adjustment; n= n genes annotated in test group; n PSP= n genes annotated in PSP}


% % % % latex table generated in R 3.6.3 by xtable 1.8-4 package
% % % % Sun Oct  4 11:58:29 2020
% % % \begin{table}[ht]
% % % \centering
% % % \begin{adjustbox}{width=\textwidth}
% % % \setlength{\extrarowheight}{2pt}
% % % \begin{tabular}{@{}clllcl@{}}
% % %   \toprule
% % %   ID & Cellular Component & $n$ & $n$ PSP & $p$ & $q$ FDR B H \\ 

% % %   \midrule
% % % GO:0031226 & intrinsic component of plasma membrane & 61 & 331 & $1.85 \times 10^{-6}$ & $9.76 \times 10^{-4}$ \\ 
% % %   GO:0005887 & integral component of plasma membrane & 56 & 307 & $7.36 \times 10^{-6}$ & $1.94 \times 10^{-3}$ \\ 
% % %   \bottomrule
% % % \end{tabular}
% % % \end{adjustbox}
% % % \caption[Gene ontology enrichment Low betweenness genes Cellular Component of genes below 10th centile of distribution]{\textbf{Low betweenness genes. Cellular Component - Gene ontology enrichment analysis} using the ToppGene web client and all PSP genes as the background set.  Biological process tested gene set is below 10th centile of degree distribution.  p value; q FDR B H = q adjusted significance level False Discovery Rate using Benjamini and Hochberg adjustment; $n$= $n$ genes annotated in test group; $n$ PSP= n genes annotated in PSP. $n$ ontology terms significantly (q FDR$>=0.05$) enriched for this centrality measure and ontology category (e.g. MF,CC,BP): \textcolor{red}{n}}
% % % \label{tab:ToppGENE GO: Cellular Component. bet 10 centile cwpsp.txtp = p value; q FDR B H = q adjusted significance level False Discovery Rate using Benjamini and Hochberg adjustment; n= n genes annotated in test group; n PSP= n genes annotated in PSP}
% % % \end{table}

% % % \paragraph{Pathway}

% % % Extra cellular matrix table~\ref{tab:ToppGENE Pathway. bet 10 centile cwpsp.txtp = p value; q FDR B H = q adjusted significance level False Discovery Rate using Benjamini and Hochberg adjustment; n= n genes annotated in test group; n PSP= n genes annotated in PSP}. Glycosylphosphatidylinositol proteins (GPI proteins) are directed to the endoplasmic reticulum. 




% % % % latex table generated in R 3.6.3 by xtable 1.8-4 package
% % % % Sun Oct  4 12:01:20 2020
% % % \begin{table}[ht]
% % % \centering
% % % \begin{adjustbox}{width=\textwidth}
% % % \begin{tabular}{lcllcl}
% % %   \hline
% % % ID & Pathway & n & n PSP & p & q FDR B H \\ 
% % %   \hline
% % % M5889 &\makecell{ Ensemble of genes encoding extracellular matrix\\ and extracellular matrix-associated proteins}  & 21 & 91 & $7.23 \times 10^{-6}$ & $8.28 \times 10^{-3}$ \\ 
% % %   M1268710 & \makecell{Post-translational modification: synthesis of GPI-anchored proteins} & 6 & 11 & $9.13 \times 10^{-5}$ & $4.00 \times 10^{-2}$ \\ 
% % %   M5884 & \makecell{Ensemble of genes encoding core extracellular matrix including\\ ECM glycoproteins, collagens and proteoglycans} & 10 & 31 & $1.05 \times 10^{-4}$ & $4.00 \times 10^{-2}$ \\ 
% % %   \hline
% % % \end{tabular}
% % % \end{adjustbox}
% % % \caption{ToppGene Pathway. bet 10 centile cwpsp.txtp = p value; q FDR B H = q adjusted significance level False Discovery Rate using Benjamini and Hochberg adjustment; n= n genes annotated in test group; n PSP= n genes annotated in PSP} 
% % % \label{tab:ToppGENE Pathway. bet 10 centile cwpsp.txtp = p value; q FDR B H = q adjusted significance level False Discovery Rate using Benjamini and Hochberg adjustment; n= n genes annotated in test group; n PSP= n genes annotated in PSP}
% % % \end{table}

% % % \clearpage
% % % \subsection{Closeness Centrality ToppGene}



% % % \paragraph{Molecular function}

% % % Molecular function Gene Ontology Enrichment High Closeness centrality in table \ref{tab:ToppGENE GO: Molecular Function. clo 90 centile cwpsp.txtp = p value; q FDR B H = q adjusted significance level False Discovery Rate using Benjamini and Hochberg adjustment; n= n genes annotated in test group; n PSP= n genes annotated in PSP}

% % % % latex table generated in R 3.6.3 by xtable 1.8-4 package
% % % % Sun Oct  4 12:10:43 2020

% % %   \begin{table}[ht]
% % % \centering
% % % \begin{adjustbox}{width=\textwidth}
% % % \setlength{\extrarowheight}{2pt}
% % % \begin{tabular}{@{}clllcl@{}}
% % %   \toprule
% % %   ID & Molecular Function & $n$ & $n$ PSP & $p$ & $q$ FDR B H \\ 

% % %   \midrule
% % % GO:0003723 & RNA binding & 167 & 554 & $2.11 \times 10^{-50}$ & $2.14 \times 10^{-47}$ \\ 
% % %   GO:0031625 & ubiquitin protein ligase binding & 57 & 117 & $1.38 \times 10^{-27}$ & $7.00 \times 10^{-25}$ \\ 
% % %   GO:0044389 & ubiquitin-like protein ligase binding & 58 & 123 & $4.10 \times 10^{-27}$ & $1.39 \times 10^{-24}$ \\ 
% % %   GO:0051082 & unfolded protein binding & 31 & 56 & $2.20 \times 10^{-17}$ & $5.57 \times 10^{-15}$ \\ 
% % %   GO:0044877 & protein-containing complex binding & 110 & 513 & $4.36 \times 10^{-17}$ & $8.84 \times 10^{-15}$ \\ 
% % %   GO:0019904 & protein domain specific binding & 81 & 329 & $5.59 \times 10^{-16}$ & $9.46 \times 10^{-14}$ \\ 
% % %   GO:0003729 & mRNA binding & 39 & 100 & $6.55 \times 10^{-15}$ & $9.50 \times 10^{-13}$ \\ 
% % %   GO:0008134 & transcription factor binding & 50 & 156 & $8.56 \times 10^{-15}$ & $1.09 \times 10^{-12}$ \\ 
% % %   GO:0005102 & signaling receptor binding & 86 & 405 & $7.31 \times 10^{-13}$ & $8.25 \times 10^{-11}$ \\ 
% % %   GO:0140097 & catalytic activity, acting on DNA & 31 & 76 & $1.13 \times 10^{-12}$ & $1.15 \times 10^{-10}$ \\ 
% % %   \bottomrule
% % % \end{tabular}
% % % \end{adjustbox}
% % % \caption[Gene ontology enrichment High Closeness genes Molecular Function of genes above 90th centile of distribution]{\textbf{High Closeness genes. Molecular Function - Gene ontology enrichment analysis} using the ToppGene web client and all PSP genes as the background set.  Molecular function tested gene set is 90th centile of degree distribution.  p value; q FDR B H = q adjusted significance level False Discovery Rate using Benjamini and Hochberg adjustment; $n$= $n$ genes annotated in test group; $n$ PSP= n genes annotated in PSP. $n$ ontology terms significantly (q FDR$>=0.05$) enriched for this centrality measure and ontology category (e.g. MF,CC,BP): \textcolor{red}{n}} 
% % % \label{tab:ToppGENE GO: Molecular Function. clo 90 centile cwpsp.txtp = p value; q FDR B H = q adjusted significance level False Discovery Rate using Benjamini and Hochberg adjustment; n= n genes annotated in test group; n PSP= n genes annotated in PSP}
% % % \end{table}

% % % \paragraph{Biological Process}
% % % Biological process table~\ref{tab:ToppGENE GO: Biological Process. clo 90 centile cwpsp.txtp = p value; q FDR B H = q adjusted significance level False Discovery Rate using Benjamini and Hochberg adjustment; n= n genes annotated in test group; n PSP= n genes annotated in PSP}

% % % % latex table generated in R 3.6.3 by xtable 1.8-4 package
% % % % Sun Oct  4 12:13:59 2020
% % % \begin{table}[ht]
% % % \centering
% % % \begin{adjustbox}{width=\textwidth}
% % % \setlength{\extrarowheight}{2pt}
% % % \begin{tabular}{@{}clllcl@{}}
% % %   \toprule
% % %   ID & Biological Process & $n$ & $n$ PSP & $p$ & $q$ FDR B H \\ 

% % %   \midrule
% % % GO:0044265 & cellular macromolecule catabolic process & 131 & 372 & $4.77 \times 10^{-46}$ & $2.98 \times 10^{-42}$ \\ 
% % %   GO:0009057 & macromolecule catabolic process & 143 & 443 & $1.34 \times 10^{-45}$ & $4.18 \times 10^{-42}$ \\ 
% % %   GO:0016071 & mRNA metabolic process & 106 & 260 & $2.87 \times 10^{-43}$ & $5.96 \times 10^{-40}$ \\ 
% % %   GO:0044403 & symbiotic process & 125 & 364 & $3.12 \times 10^{-42}$ & $4.86 \times 10^{-39}$ \\ 
% % %   GO:0044419 & interspecies interaction between organisms & 125 & 368 & $1.23 \times 10^{-41}$ & $1.53 \times 10^{-38}$ \\ 
% % %   GO:0016032 & viral process & 118 & 348 & $6.85 \times 10^{-39}$ & $7.13 \times 10^{-36}$ \\ 
% % %   GO:0006402 & mRNA catabolic process & 75 & 180 & $1.13 \times 10^{-30}$ & $1.00 \times 10^{-27}$ \\ 
% % %   GO:0006401 & RNA catabolic process & 76 & 186 & $2.10 \times 10^{-30}$ & $1.63 \times 10^{-27}$ \\ 
% % %   GO:0010608 & posttranscriptional regulation of gene expression & 82 & 225 & $1.11 \times 10^{-28}$ & $7.69 \times 10^{-26}$ \\ 
% % %   GO:0034655 & nucleobase-containing compound catabolic process & 80 & 217 & $2.37 \times 10^{-28}$ & $1.48 \times 10^{-25}$ \\ 
% % %   \bottomrule
% % % \end{tabular}
% % % \end{adjustbox}
% % % \caption[Gene ontology enrichment High Closeness genes Biological Process of genes above 90th centile of distribution]{\textbf{High Closeness genes. Biological Process - Gene ontology enrichment analysis} using the ToppGene web client and all PSP genes as the background set.  Biological process tested gene set is 90th centile of degree distribution.  p value; q FDR B H = q adjusted significance level False Discovery Rate using Benjamini and Hochberg adjustment; $n$= $n$ genes annotated in test group; $n$ PSP= n genes annotated in PSP. $n$ ontology terms significantly (q FDR$>=0.05$) enriched for this centrality measure and ontology category (e.g. MF,CC,BP): \textcolor{red}{n}} 

% % % \label{tab:ToppGENE GO: Biological Process. clo 90 centile cwpsp.txtp = p value; q FDR B H = q adjusted significance level False Discovery Rate using Benjamini and Hochberg adjustment; n= n genes annotated in test group; n PSP= n genes annotated in PSP}
% % % \end{table}





% % % \paragraph{Cellular component}
% % % Cellular component table~\ref{tab:ToppGENE GO: Cellular Component. clo 90 centile cwpsp.txtp = p value; q FDR B H = q adjusted significance level False Discovery Rate using Benjamini and Hochberg adjustment; n= n genes annotated in test group; n PSP= n genes annotated in PSP}

% % % % latex table generated in R 3.6.3 by xtable 1.8-4 package
% % % % Sun Oct  4 12:15:50 2020
% % % \begin{table}[ht]
% % % \centering
% % % \begin{adjustbox}{width=\textwidth}
% % % \setlength{\extrarowheight}{2pt}
% % % \begin{tabular}{@{}clllcl@{}}
% % %   \toprule
% % %   ID & Cellular Component & $n$ & $n$ PSP & $p$ & $q$ FDR B H \\ 

% % %   \midrule
% % % GO:1990904 & ribonucleoprotein complex & 102 & 303 & $5.76 \times 10^{-33}$ & $4.89 \times 10^{-30}$ \\ 
% % %   GO:0005925 & focal adhesion & 77 & 239 & $4.94 \times 10^{-23}$ & $2.10 \times 10^{-20}$ \\ 
% % %   GO:0030055 & cell-substrate junction & 77 & 244 & $2.18 \times 10^{-22}$ & $6.15 \times 10^{-20}$ \\ 
% % %   GO:0005912 & adherens junction & 85 & 314 & $8.51 \times 10^{-20}$ & $1.80 \times 10^{-17}$ \\ 
% % %   GO:0070161 & anchoring junction & 85 & 319 & $2.64 \times 10^{-19}$ & $4.47 \times 10^{-17}$ \\ 
% % %   GO:0035770 & ribonucleoprotein granule & 36 & 78 & $1.03 \times 10^{-16}$ & $1.30 \times 10^{-14}$ \\ 
% % %   GO:0036464 & cytoplasmic ribonucleoprotein granule & 35 & 74 & $1.07 \times 10^{-16}$ & $1.30 \times 10^{-14}$ \\ 
% % %   GO:0000974 & Prp19 complex & 20 & 31 & $2.08 \times 10^{-13}$ & $2.21 \times 10^{-11}$ \\ 
% % %   GO:0101031 & chaperone complex & 16 & 21 & $1.04 \times 10^{-12}$ & $9.79 \times 10^{-11}$ \\ 
% % %   GO:0005681 & spliceosomal complex & 22 & 40 & $1.30 \times 10^{-12}$ & $1.11 \times 10^{-10}$ \\ 
% % %   \bottomrule
% % % \end{tabular}
% % % \end{adjustbox}
% % % \caption[Gene ontology enrichment High Closeness genes Cellular Component of genes above 90th centile of distribution]{\textbf{High Closeness genes. Cellular Component - Gene ontology enrichment analysis} using the ToppGene web client and all PSP genes as the background set.  Biological process tested gene set is 90th centile of degree distribution.  p value; q FDR B H = q adjusted significance level False Discovery Rate using Benjamini and Hochberg adjustment; $n$= $n$ genes annotated in test group; $n$ PSP= n genes annotated in PSP. $n$ ontology terms significantly (q FDR$>=0.05$) enriched for this centrality measure and ontology category (e.g. MF,CC,BP): \textcolor{red}{n}} 

% % % \label{tab:ToppGENE GO: Cellular Component. clo 90 centile cwpsp.txtp = p value; q FDR B H = q adjusted significance level False Discovery Rate using Benjamini and Hochberg adjustment; n= n genes annotated in test group; n PSP= n genes annotated in PSP}
% % % \end{table}






% % % \subsubsection{Low closeness}

% % % Note human phenotype epilepsy and disease enriched\footnote{\url{source('~/RProjects/chapter3/R/sig_magma_genes/central_genes_toppgene/closeness/low_closeness/5_0_clip_centrality_clo_low_cc_n_10.R'}}
% % % \paragraph{Molecular function}
% % % Molecular function table~\ref{tab:ToppGENE GO: Molecular Function. clo 10 centile cwpsp.txtp = p value; q FDR B H = q adjusted significance level False Discovery Rate using Benjamini and Hochberg adjustment; n= n genes annotated in test group; n PSP= n genes annotated in PSP}. Closeness is more normally distributed it appears the low part of its distribution is also more discriminating than those with power law distribution that have a number of genes separated by small absolute values at the lower end of the distribution. Genes encoding proteins with low closeness centrality are enriched for molecular functions involving ion channels and trans-membrane transporters. 

% % % % latex table generated in R 3.6.3 by xtable 1.8-4 package
% % % % Sun Oct  4 12:19:53 2020

% % %   \begin{table}[ht]
% % % \centering
% % % \begin{adjustbox}{width=\textwidth}
% % % \setlength{\extrarowheight}{2pt}
% % % \begin{tabular}{@{}clllcl@{}}
% % %   \toprule
% % %   ID & Molecular Function & $n$ & $n$ PSP & $p$ & $q$ FDR B H \\ 

% % %   \midrule
% % % GO:0005215 & transporter activity & 66 & 419 & $2.40 \times 10^{-5}$ & $1.02 \times 10^{-2}$ \\ 
% % %   GO:0022857 & transmembrane transporter activity & 61 & 378 & $2.44 \times 10^{-5}$ & $1.02 \times 10^{-2}$ \\ 
% % %   GO:0004930 & G protein-coupled receptor activity & 10 & 24 & $3.91 \times 10^{-5}$ & $1.09 \times 10^{-2}$ \\ 
% % %   GO:0043176 & amine binding & 4 & 4 & $1.01 \times 10^{-4}$ & $2.10 \times 10^{-2}$ \\ 
% % %   GO:0046873 & metal ion transmembrane transporter activity & 36 & 203 & $2.12 \times 10^{-4}$ & $3.19 \times 10^{-2}$ \\ 
% % %   GO:0015075 & ion transmembrane transporter activity & 53 & 340 & $2.34 \times 10^{-4}$ & $3.19 \times 10^{-2}$ \\ 
% % %   GO:0022803 & passive transmembrane transporter activity & 39 & 230 & $3.05 \times 10^{-4}$ & $3.19 \times 10^{-2}$ \\ 
% % %   GO:0015267 & channel activity & 39 & 230 & $3.05 \times 10^{-4}$ & $3.19 \times 10^{-2}$ \\ 
% % %   GO:0008324 & cation transmembrane transporter activity & 46 & 288 & $3.61 \times 10^{-4}$ & $3.36 \times 10^{-2}$ \\ 
% % %   GO:0015318 & inorganic molecular entity transmembrane transporter activity & 50 & 324 & $4.60 \times 10^{-4}$ & $3.86 \times 10^{-2}$ \\ 
% % %   \bottomrule
% % % \end{tabular}
% % % \end{adjustbox}
% % % \caption[Gene ontology enrichment Low Closeness genes Molecular Function of genes above 90th centile of distribution]{\textbf{Low Closeness genes. Molecular Function - Gene ontology enrichment analysis} using the ToppGene web client and all PSP genes as the background set.  Biological process tested gene set is below 10th centile of degree distribution.  p value; q FDR B H = q adjusted significance level False Discovery Rate using Benjamini and Hochberg adjustment; $n$= $n$ genes annotated in test group; $n$ PSP= n genes annotated in PSP. $n$ ontology terms significantly (q FDR$>=0.05$) enriched for this centrality measure and ontology category (e.g. MF,CC,BP): \textcolor{red}{n}} 
 
% % % \label{tab:ToppGENE GO: Molecular Function. clo 10 centile cwpsp.txtp = p value; q FDR B H = q adjusted significance level False Discovery Rate using Benjamini and Hochberg adjustment; n= n genes annotated in test group; n PSP= n genes annotated in PSP}
% % % \end{table}


% % % \paragraph{Cellular component}
% % % Cellular component table~\ref{tab:ToppGENE GO: Cellular Component. clo 10 centile cwpsp.txtp = p value; q FDR B H = q adjusted significance level False Discovery Rate using Benjamini and Hochberg adjustment; n= n genes annotated in test group; n PSP= n genes annotated in PSP}. Again shows transmembrane transporters also integral part of synaptic and post synaptic membrane. The proteins encoded by these genes the ion channels and synaptic membrane components are also on the periphery of the network in the sense that they are more distant from most other nodes.




% % %   \begin{table}[ht]
% % % \centering
% % % \begin{adjustbox}{width=\textwidth}
% % % \setlength{\extrarowheight}{2pt}
% % % \begin{tabular}{@{}clllcl@{}}
% % %   \toprule
% % %   ID & Cellular Component & $n$ & $n$ PSP & $p$ & $q$ FDR B H \\ 

% % %   \midrule
% % % GO:0031226 & intrinsic component of plasma membrane & 72 & 331 & $8.90 \times 10^{-12}$ & $4.95 \times 10^{-9}$ \\ 
% % %   GO:0005887 & integral component of plasma membrane & 66 & 307 & $1.41 \times 10^{-10}$ & $3.91 \times 10^{-8}$ \\ 
% % %   GO:1990351 & transporter complex & 30 & 132 & $7.50 \times 10^{-6}$ & $1.35 \times 10^{-3}$ \\ 
% % %   GO:1902495 & transmembrane transporter complex & 29 & 127 & $9.72 \times 10^{-6}$ & $1.35 \times 10^{-3}$ \\ 
% % %   GO:0034702 & ion channel complex & 27 & 117 & $1.62 \times 10^{-5}$ & $1.57 \times 10^{-3}$ \\ 
% % %   GO:0034703 & cation channel complex & 24 & 98 & $1.69 \times 10^{-5}$ & $1.57 \times 10^{-3}$ \\ 
% % %   GO:0099055 & integral component of postsynaptic membrane & 19 & 72 & $4.33 \times 10^{-5}$ & $3.22 \times 10^{-3}$ \\ 
% % %   GO:0072534 & perineuronal net & 5 & 6 & $5.12 \times 10^{-5}$ & $3.22 \times 10^{-3}$ \\ 
% % %   GO:0098936 & intrinsic component of postsynaptic membrane & 20 & 79 & $5.22 \times 10^{-5}$ & $3.22 \times 10^{-3}$ \\ 
% % %   GO:0099699 & integral component of synaptic membrane & 22 & 96 & $1.13 \times 10^{-4}$ & $6.28 \times 10^{-3}$ \\ 
% % %   \bottomrule
% % % \end{tabular}
% % % \end{adjustbox}
% % % \caption[Gene ontology enrichment Low Closeness genes Cellular Component of genes above 90th centile of distribution]{\textbf{Low Closeness genes. Cellular Component - Gene ontology enrichment analysis} using the ToppGene web client and all PSP genes as the background set.  Biological process tested gene set is below 10th centile of degree distribution.  p value; q FDR B H = q adjusted significance level False Discovery Rate using Benjamini and Hochberg adjustment; $n$= $n$ genes annotated in test group; $n$ PSP= n genes annotated in PSP. $n$ ontology terms significantly (q FDR$>=0.05$) enriched for this centrality measure and ontology category (e.g. MF,CC,BP): \textcolor{red}{n}} 

% % % \label{tab:ToppGENE GO: Cellular Component. clo 10 centile cwpsp.txtp = p value; q FDR B H = q adjusted significance level False Discovery Rate using Benjamini and Hochberg adjustment; n= n genes annotated in test group; n PSP= n genes annotated in PSP}
% % % \end{table}


% % % \clearpage


% % % \subsection{Eigenvector centrality}
% % % \subsubsection{High eigenvector centrality}

% % % \paragraph{Molecular function}

% % % Molecular function table~\ref{tab:ToppGENE GO: Molecular Function. 90 centile cw psp eig.txtp = p value; q FDR B H = q adjusted significance level False Discovery Rate using Benjamini and Hochberg adjustment; n= n genes annotated in test group; n PSP= n genes annotated in PSP} shows over representation of ubiquitin, RNA binding and ribosomal components. Similar to other centrality measurs but very strong over representation\footnote{\url{source('~/RProjects/chapter3/R/sig_magma_genes/central_genes_toppgene/eigenvector/5_0_clip_centrality_eig_mfunc_n_10.R')}}.


% % % % latex table generated in R 3.6.3 by xtable 1.8-4 package
% % % % Sun Oct  4 12:32:37 2020
% % %   \begin{table}[ht]
% % % \centering
% % % \begin{adjustbox}{width=\textwidth}
% % % \setlength{\extrarowheight}{2pt}
% % % \begin{tabular}{@{}clllcl@{}}
% % %   \toprule
% % %   ID & Molecular Function & $n$ & $n$ PSP & $p$ & $q$ FDR B H \\ 

% % %   \midrule
% % % GO:0003723 & RNA binding & 198 & 554 & $7.00 \times 10^{-79}$ & $7.01 \times 10^{-76}$ \\ 
% % %   GO:0003735 & structural constituent of ribosome & 54 & 103 & $2.82 \times 10^{-28}$ & $1.41 \times 10^{-25}$ \\ 
% % %   GO:0031625 & ubiquitin protein ligase binding & 56 & 117 & $1.15 \times 10^{-26}$ & $3.83 \times 10^{-24}$ \\ 
% % %   GO:0044389 & ubiquitin-like protein ligase binding & 57 & 123 & $3.21 \times 10^{-26}$ & $8.03 \times 10^{-24}$ \\ 
% % %   GO:0003729 & mRNA binding & 51 & 100 & $5.76 \times 10^{-26}$ & $1.15 \times 10^{-23}$ \\ 
% % %   GO:0005198 & structural molecule activity & 82 & 303 & $5.13 \times 10^{-19}$ & $8.55 \times 10^{-17}$ \\ 
% % %   GO:0019843 & rRNA binding & 25 & 34 & $1.26 \times 10^{-18}$ & $1.80 \times 10^{-16}$ \\ 
% % %   GO:0003727 & single-stranded RNA binding & 21 & 31 & $1.19 \times 10^{-14}$ & $1.49 \times 10^{-12}$ \\ 
% % %   GO:0045296 & cadherin binding & 59 & 224 & $4.31 \times 10^{-13}$ & $4.80 \times 10^{-11}$ \\ 
% % %   GO:0003730 & mRNA 3'-UTR binding & 20 & 32 & $5.22 \times 10^{-13}$ & $5.22 \times 10^{-11}$ \\ 
% % %   \hline
% % % \end{tabular}
% % % \end{adjustbox}

% % % \caption[Gene ontology enrichment High Eigenvector centrality genes Molecular Function of genes above 90th centile of distribution]{\textbf{High Eigenvector centrality genes. Molecular Function - Gene ontology enrichment analysis} using the ToppGene web client and all PSP genes as the background set.  Molecular function tested gene set is 90th centile of degree distribution.  p value; q FDR B H = q adjusted significance level False Discovery Rate using Benjamini and Hochberg adjustment; $n$= $n$ genes annotated in test group; $n$ PSP= n genes annotated in PSP. $n$ ontology terms significantly (q FDR$>=0.05$) enriched for this centrality measure and ontology category (e.g. MF,CC,BP): \textcolor{red}{n}} 
% % % \label{tab:ToppGENE GO: Molecular Function. 90 centile cw psp eig.txtp = p value; q FDR B H = q adjusted significance level False Discovery Rate using Benjamini and Hochberg adjustment; n= n genes annotated in test group; n PSP= n genes annotated in PSP}
% % % \end{table}

% % % \paragraph{Biological process}
% % % Biological process table~\ref{tab:ToppGENE GO: Biological Process. 90 centile cw psp eig.txtp = p value; q FDR B H = q adjusted significance level False Discovery Rate using Benjamini and Hochberg adjustment; n= n genes annotated in test group; n PSP= n genes annotated in PSP}

% % % % latex table generated in R 3.6.3 by xtable 1.8-4 package
% % % % Sun Oct  4 12:34:28 2020
% % %   \begin{table}[ht]
% % % \centering
% % % \begin{adjustbox}{width=\textwidth}
% % % \setlength{\extrarowheight}{2pt}
% % % \begin{tabular}{@{}clllcl@{}}
% % %   \toprule
% % %   ID & Biological Process & $n$ & $n$ PSP & $p$ & $q$ FDR B H \\ 

% % %   \midrule
% % % GO:0016071 & mRNA metabolic process & 143 & 260 & $1.30 \times 10^{-83}$ & $7.64 \times 10^{-80}$ \\ 
% % %   GO:0009057 & macromolecule catabolic process & 159 & 443 & $1.09 \times 10^{-59}$ & $3.20 \times 10^{-56}$ \\ 
% % %   GO:0044265 & cellular macromolecule catabolic process & 144 & 372 & $5.80 \times 10^{-58}$ & $9.34 \times 10^{-55}$ \\ 
% % %   GO:0006402 & mRNA catabolic process & 101 & 180 & $6.34 \times 10^{-58}$ & $9.34 \times 10^{-55}$ \\ 
% % %   GO:0006401 & RNA catabolic process & 102 & 186 & $3.28 \times 10^{-57}$ & $3.87 \times 10^{-54}$ \\ 
% % %   GO:0044403 & symbiotic process & 139 & 364 & $1.05 \times 10^{-54}$ & $1.03 \times 10^{-51}$ \\ 
% % %   GO:0044419 & interspecies interaction between organisms & 139 & 368 & $5.35 \times 10^{-54}$ & $4.50 \times 10^{-51}$ \\ 
% % %   GO:0016032 & viral process & 134 & 348 & $6.30 \times 10^{-53}$ & $4.64 \times 10^{-50}$ \\ 
% % %   GO:0034655 & nucleobase-containing compound catabolic process & 104 & 217 & $7.76 \times 10^{-51}$ & $5.08 \times 10^{-48}$ \\ 
% % %   GO:0019439 & aromatic compound catabolic process & 104 & 232 & $2.67 \times 10^{-47}$ & $1.58 \times 10^{-44}$ \\ 
% % %   \bottomrule
% % % \end{tabular}
% % % \end{adjustbox}
% % % \caption[Gene ontology enrichment High Eigenvector centrality genes Biological Process of genes above 90th centile of distribution]{\textbf{High Eigenvector centrality genes. Biological Process - Gene ontology enrichment analysis} using the ToppGene web client and all PSP genes as the background set.  Molecular function tested gene set is 90th centile of degree distribution.  p value; q FDR B H = q adjusted significance level False Discovery Rate using Benjamini and Hochberg adjustment; $n$= $n$ genes annotated in test group; $n$ PSP= n genes annotated in PSP. $n$ ontology terms significantly (q FDR$>=0.05$) enriched for this centrality measure and ontology category (e.g. MF,CC,BP): \textcolor{red}{n}} 
% % % \label{tab:ToppGENE GO: Biological Process. 90 centile cw psp eig.txtp = p value; q FDR B H = q adjusted significance level False Discovery Rate using Benjamini and Hochberg adjustment; n= n genes annotated in test group; n PSP= n genes annotated in PSP}
% % % \end{table}


% % % \paragraph{Cellular component}
% % % Cellular component table~\ref{tab:ToppGENE GO: Cellular Component. 90 centile cw psp eig.txtp = p value; q FDR B H = q adjusted significance level False Discovery Rate using Benjamini and Hochberg adjustment; n= n genes annotated in test group; n PSP= n genes annotated in PSP}. PRP is pre-ribosomal rna processing complx 19.snRNP are small ribonuclear protein. Enrichment for ribosomes and for focal adhesion. U5 snRNP associates with Prp8 (ref wikipedia). The coverage of some of these ontology terms for cellular component (to coin a phrase where i am referring to how much n in the group represents of the whole annotation). It finds 28 of 34 U5 snRNP genes. 

% % % % latex table generated in R 3.6.3 by xtable 1.8-4 package
% % % % Sun Oct  4 12:38:30 2020
% % %  \begin{table}[ht]
% % % \centering
% % % \begin{adjustbox}{width=\textwidth}
% % % \setlength{\extrarowheight}{2pt}
% % % \begin{tabular}{@{}clllcl@{}}
% % %   \toprule
% % %   ID & Cellular Component & $n$ & $n$ PSP & $p$ & $q$ FDR B H \\ 

% % %   \midrule
% % % GO:1990904 & ribonucleoprotein complex & 144 & 303 & $4.68 \times 10^{-73}$ & $3.68 \times 10^{-70}$ \\ 
% % %   GO:0022626 & cytosolic ribosome & 60 & 97 & $3.21 \times 10^{-37}$ & $1.26 \times 10^{-34}$ \\ 
% % %   GO:0044391 & ribosomal subunit & 61 & 124 & $4.99 \times 10^{-30}$ & $1.31 \times 10^{-27}$ \\ 
% % %   GO:0005840 & ribosome & 65 & 141 & $6.93 \times 10^{-30}$ & $1.36 \times 10^{-27}$ \\ 
% % %   GO:0000974 & Prp19 complex & 29 & 31 & $1.48 \times 10^{-27}$ & $2.33 \times 10^{-25}$ \\ 
% % %   GO:0005681 & spliceosomal complex & 32 & 40 & $1.11 \times 10^{-25}$ & $1.46 \times 10^{-23}$ \\ 
% % %   GO:0071013 & catalytic step 2 spliceosome & 27 & 29 & $1.49 \times 10^{-25}$ & $1.67 \times 10^{-23}$ \\ 
% % %   GO:0097525 & spliceosomal snRNP complex & 29 & 35 & $3.56 \times 10^{-24}$ & $3.49 \times 10^{-22}$ \\ 
% % %   GO:0005925 & focal adhesion & 78 & 239 & $7.67 \times 10^{-24}$ & $6.70 \times 10^{-22}$ \\ 
% % %   GO:0005682 & U5 snRNP & 28 & 34 & $3.16 \times 10^{-23}$ & $2.48 \times 10^{-21}$ \\ 
% % %   \bottomrule
% % % \end{tabular}
% % % \end{adjustbox}
% % % \caption[Gene ontology enrichment High Eigenvector centrality genes Cellular Component of genes above 90th centile of distribution]{\textbf{High Eigenvector centrality genes. Cellular Component - Gene ontology enrichment analysis} using the ToppGene web client and all PSP genes as the background set.  Molecular function tested gene set is 90th centile of degree distribution.  p value; q FDR B H = q adjusted significance level False Discovery Rate using Benjamini and Hochberg adjustment; $n$= $n$ genes annotated in test group; $n$ PSP= n genes annotated in PSP. $n$ ontology terms significantly (q FDR$>=0.05$) enriched for this centrality measure and ontology category (e.g. MF,CC,BP): \textcolor{red}{n}} 

% % % \label{tab:ToppGENE GO: Cellular Component. 90 centile cw psp eig.txtp = p value; q FDR B H = q adjusted significance level False Discovery Rate using Benjamini and Hochberg adjustment; n= n genes annotated in test group; n PSP= n genes annotated in PSP}
% % % \end{table}

% % % \clearpage
% % % \subsubsection{Low eigenvector centrality}
% % % \paragraph{Molecular function}

% % % Molecular function \ref{tab:ToppGENE GO: Molecular Function. 10 centile cw psp eig.txtp = p value; q FDR B H = q adjusted significance level False Discovery Rate using Benjamini and Hochberg adjustment; n= n genes annotated in test group; n PSP= n genes annotated in PSP}. Low eigenvector genes are enriched for molecular function terms related to transmembrane transporter activity and voltage gated ion channels. 

% % % No significant enrichment for biological process. 

% % % % latex table generated in R 3.6.3 by xtable 1.8-4 package
% % % % Sun Oct  4 14:17:27 2020
% % %  \begin{table}[ht]
% % % \centering
% % % \begin{adjustbox}{width=\textwidth}
% % % \setlength{\extrarowheight}{2pt}
% % % \begin{tabular}{@{}clllcl@{}}
% % %   \toprule
% % %   ID & Molecular Function & $n$ & $n$ PSP & $p$ & $q$ FDR B H \\ 

% % %   \midrule
% % % GO:0022857 & transmembrane transporter activity & 69 & 378 & $5.33 \times 10^{-8}$ & $2.75 \times 10^{-5}$ \\ 
% % %   GO:0005215 & transporter activity & 74 & 419 & $6.69 \times 10^{-8}$ & $2.75 \times 10^{-5}$ \\ 
% % %   GO:0022803 & passive transmembrane transporter activity & 47 & 230 & $3.54 \times 10^{-7}$ & $7.28 \times 10^{-5}$ \\ 
% % %   GO:0015267 & channel activity & 47 & 230 & $3.54 \times 10^{-7}$ & $7.28 \times 10^{-5}$ \\ 
% % %   GO:0015075 & ion transmembrane transporter activity & 61 & 340 & $7.16 \times 10^{-7}$ & $1.18 \times 10^{-4}$ \\ 
% % %   GO:0015318 & inorganic molecular entity transmembrane transporter activity & 58 & 324 & $1.54 \times 10^{-6}$ & $1.67 \times 10^{-4}$ \\ 
% % %   GO:0005244 & voltage-gated ion channel activity & 29 & 118 & $1.62 \times 10^{-6}$ & $1.67 \times 10^{-4}$ \\ 
% % %   GO:0022832 & voltage-gated channel activity & 29 & 118 & $1.62 \times 10^{-6}$ & $1.67 \times 10^{-4}$ \\ 
% % %   GO:0008324 & cation transmembrane transporter activity & 53 & 288 & $1.96 \times 10^{-6}$ & $1.79 \times 10^{-4}$ \\ 
% % %   GO:0046873 & metal ion transmembrane transporter activity & 41 & 203 & $2.97 \times 10^{-6}$ & $2.44 \times 10^{-4}$ \\ 
% % %   \bottomrule
% % % \end{tabular}
% % % \end{adjustbox}
% % % \caption[Gene ontology enrichment Low Eigenvector centrality genes Molecular Function of genes above 10th centile of distribution]{\textbf{Low Eigenvector centrality genes. Molecular Function - Gene ontology enrichment analysis} using the ToppGene web client and all PSP genes as the background set.  Molecular function tested gene set is 10th centile of degree distribution.  p value; q FDR B H = q adjusted significance level False Discovery Rate using Benjamini and Hochberg adjustment; $n$= $n$ genes annotated in test group; $n$ PSP= n genes annotated in PSP. $n$ ontology terms significantly (q FDR$>=0.05$) enriched for this centrality measure and ontology category (e.g. MF,CC,BP): \textcolor{red}{n}} 

% % % \label{tab:ToppGENE GO: Molecular Function. 10 centile cw psp eig.txtp = p value; q FDR B H = q adjusted significance level False Discovery Rate using Benjamini and Hochberg adjustment; n= n genes annotated in test group; n PSP= n genes annotated in PSP}
% % % \end{table}

% % % \paragraph{Cellular component}

% % % Cellular component table~\ref{tab:ToppGENE GO: Cellular Component. 10 centile cw psp eig.txtp = p value; q FDR B H = q adjusted significance level False Discovery Rate using Benjamini and Hochberg adjustment; n= n genes annotated in test group; n PSP= n genes annotated in PSP} is enriched for transmembrane transporter complexes, ion channels and integral and intrinsic part of synaptic and post synaptic membrane. 
% % %  \begin{table}[ht]
% % % \centering
% % % \begin{adjustbox}{width=\textwidth}
% % % \setlength{\extrarowheight}{2pt}
% % % \begin{tabular}{@{}clllcl@{}}
% % %   \toprule
% % %   ID & Cellular Component & $n$ & $n$ PSP & $p$ & $q$ FDR B H \\ 

% % %   \midrule
% % % GO:0031226 & intrinsic component of plasma membrane & 75 & 331 & $3.06 \times 10^{-13}$ & $1.69 \times 10^{-10}$ \\ 
% % %   GO:0005887 & integral component of plasma membrane & 69 & 307 & $5.24 \times 10^{-12}$ & $1.45 \times 10^{-9}$ \\ 
% % %   GO:1990351 & transporter complex & 36 & 132 & $5.47 \times 10^{-9}$ & $9.30 \times 10^{-7}$ \\ 
% % %   GO:1902495 & transmembrane transporter complex & 35 & 127 & $6.71 \times 10^{-9}$ & $9.30 \times 10^{-7}$ \\ 
% % %   GO:0034702 & ion channel complex & 33 & 117 & $9.91 \times 10^{-9}$ & $1.10 \times 10^{-6}$ \\ 
% % %   GO:0034703 & cation channel complex & 29 & 98 & $2.53 \times 10^{-8}$ & $2.33 \times 10^{-6}$ \\ 
% % %   GO:0099055 & integral component of postsynaptic membrane & 21 & 72 & $3.01 \times 10^{-6}$ & $2.37 \times 10^{-4}$ \\ 
% % %   GO:0099699 & integral component of synaptic membrane & 25 & 96 & $3.42 \times 10^{-6}$ & $2.37 \times 10^{-4}$ \\ 
% % %   GO:0098936 & intrinsic component of postsynaptic membrane & 22 & 79 & $4.08 \times 10^{-6}$ & $2.51 \times 10^{-4}$ \\ 
% % %   GO:0099240 & intrinsic component of synaptic membrane & 26 & 108 & $1.04 \times 10^{-5}$ & $5.78 \times 10^{-4}$ \\ 
% % %   \bottomrule
% % % \end{tabular}
% % % \end{adjustbox}
% % % \caption[Gene ontology enrichment Low Eigenvector centrality genes Molecular Function of genes above 10th centile of distribution]{\textbf{Low Eigenvector centrality genes. Molecular Function - Gene ontology enrichment analysis} using the ToppGene web client and all PSP genes as the background set.  Molecular function tested gene set is 10th centile of degree distribution.  p value; q FDR B H = q adjusted significance level False Discovery Rate using Benjamini and Hochberg adjustment; $n$= $n$ genes annotated in test group; $n$ PSP= n genes annotated in PSP. $n$ ontology terms significantly (q FDR$>=0.05$) enriched for this centrality measure and ontology category (e.g. MF,CC,BP): \textcolor{red}{n}} 

% % % \label{tab:ToppGENE GO: Cellular Component. 10 centile cw psp eig.txtp = p value; q FDR B H = q adjusted significance level False Discovery Rate using Benjamini and Hochberg adjustment; n= n genes annotated in test group; n PSP= n genes annotated in PSP}
% % % \end{table}


% % % \paragraph{Other tables}
% % % Pathway neuronal system compared with rest PSP  34,189 q = 0.0348

% % % Human phenotype epilepsies 

% % % Diseases top 10 epilepsies  
% % % % latex table generated in R 3.6.3 by xtable 1.8-4 package
% % % % Sun Oct  4 14:20:15 2020



% % % \clearpage


% % % \subsection{Transitivity}
% % % Transitivity with isolates set to NA and NA removed. 

% % % \textcolor{red}{Key findings}

% % % Significance score lower

% % % smaller groups
% % % +

% % % \subsubsection{High transitivity}
% % % \paragraph{Molecular function}
% % % Molecular function table~\ref{tab:ToppGENE GO: Molecular Function. tra 90 centile NA narm cwpsp.txtp = p value; q FDR B H = q adjusted significance level False Discovery Rate using Benjamini and Hochberg adjustment; n= n genes annotated in test group; n PSP= n genes annotated in PSP. n significant in category 2}. Only two significant terms, compare with other high centralities. Over representation of constituent of ribosome. 

% % % % latex table generated in R 3.6.3 by xtable 1.8-4 package
% % % % Sun Oct  4 14:35:45 2020
% % %   \begin{table}[ht]
% % % \centering
% % % \begin{adjustbox}{width=\textwidth}
% % % \setlength{\extrarowheight}{2pt}
% % % \begin{tabular}{@{}clllcl@{}}
% % %   \toprule
% % %   ID &Molecular Function & $n$ & $n$ PSP & $p$ & $q$ FDR B H \\ 

% % %   \midrule
% % % GO:0003735 & structural constituent of ribosome & 29 & 103 & $1.45 \times 10^{-8}$ & $1.06 \times 10^{-5}$ \\ 
% % %   GO:0005198 & structural molecule activity & 48 & 303 & $6.32 \times 10^{-5}$ & $2.31 \times 10^{-2}$ \\ 
% % %   \hline
% % % \end{tabular}
% % % \end{adjustbox}
% % % \caption[Gene ontology enrichment High Transitivity NaN genes. Molecular Function of genes above 90th centile of distribution]{\textbf{High Transitivity NaN genes. Molecular Function - Gene ontology enrichment analysis} using the ToppGene web client and all PSP genes as the background set.  Molecular function tested gene set is 90th centile of degree distribution.  p value; q FDR B H = q adjusted significance level False Discovery Rate using Benjamini and Hochberg adjustment; $n$= $n$ genes annotated in test group; $n$ PSP= n genes annotated in PSP. $n$ ontology terms significantly (q FDR$>=0.05$) enriched for this centrality measure and ontology category (e.g. MF,CC,BP): 2} 

% % % \label{tab:ToppGENE GO: Molecular Function. tra 90 centile NA narm cwpsp.txtp = p value; q FDR B H = q adjusted significance level False Discovery Rate using Benjamini and Hochberg adjustment; n= n genes annotated in test group; n PSP= n genes annotated in PSP. n significant in category 2}
% % % \end{table}
% % % \paragraph{Biological process}
% % % Biological process table~\ref{tab:ToppGENE GO: Biological Process. tra 90 centile NA narm cwpsp.txtp = p value; q FDR B H = q adjusted significance level False Discovery Rate using Benjamini and Hochberg adjustment; n= n genes annotated in test group; n PSP= n genes annotated in PSP. n significant in category 22}

% % % Mitochondrial translation is enriched. 



% % % % latex table generated in R 3.6.3 by xtable 1.8-4 package
% % % % Sun Oct  4 14:37:31 2020
% % % % latex table generated in R 3.6.3 by xtable 1.8-4 package
% % % % Sun Oct  4 12:34:28 2020
% % %   \begin{table}[ht]
% % % \centering
% % % \begin{adjustbox}{width=\textwidth}
% % % \setlength{\extrarowheight}{2pt}
% % % \begin{tabular}{@{}clllcl@{}}
% % %   \toprule
% % %   ID & Biological Process & $n$ & $n$ PSP & $p$ & $q$ FDR B H \\ 

% % %   \midrule
% % % GO:0006415 & translational termination & 14 & 35 & $8.46 \times 10^{-7}$ & $2.13 \times 10^{-3}$ \\ 
% % %   GO:0070126 & mitochondrial translational termination & 13 & 32 & $1.71 \times 10^{-6}$ & $2.13 \times 10^{-3}$ \\ 
% % %   GO:0070125 & mitochondrial translational elongation & 13 & 32 & $1.71 \times 10^{-6}$ & $2.13 \times 10^{-3}$ \\ 
% % %   GO:0043624 & cellular protein complex disassembly & 25 & 108 & $7.84 \times 10^{-6}$ & $6.98 \times 10^{-3}$ \\ 
% % %   GO:0032543 & mitochondrial translation & 15 & 47 & $9.37 \times 10^{-6}$ & $6.98 \times 10^{-3}$ \\ 
% % %   GO:0006414 & translational elongation & 46 & 279 & $3.13 \times 10^{-5}$ & $1.66 \times 10^{-2}$ \\ 
% % %   GO:0006412 & translation & 46 & 279 & $3.13 \times 10^{-5}$ & $1.66 \times 10^{-2}$ \\ 
% % %   GO:0140053 & mitochondrial gene expression & 15 & 52 & $3.65 \times 10^{-5}$ & $1.70 \times 10^{-2}$ \\ 
% % %   GO:0043043 & peptide biosynthetic process & 46 & 283 & $4.54 \times 10^{-5}$ & $1.72 \times 10^{-2}$ \\ 
% % %   GO:0006521 & regulation of cellular amino acid metabolic process & 12 & 36 & $4.62 \times 10^{-5}$ & $1.72 \times 10^{-2}$ \\ 
% % %   \bottomrule
% % % \end{tabular}
% % % \end{adjustbox}
% % % \caption[Gene ontology enrichment High Transitivity NaN genes Biological Process of genes above 90th centile of distribution]{\textbf{High Transitivity NaN genes. Biological Process - Gene ontology enrichment analysis} using the ToppGene web client and all PSP genes as the background set.  Molecular function tested gene set is 90th centile of degree distribution.  p value; q FDR B H = q adjusted significance level False Discovery Rate using Benjamini and Hochberg adjustment; $n$= $n$ genes annotated in test group; $n$ PSP= n genes annotated in PSP. $n$ ontology terms significantly (q FDR$>=0.05$) enriched for this centrality measure and ontology category (e.g. MF,CC,BP): 22} 
% % % \label{tab:ToppGENE GO: Biological Process. tra 90 centile NA narm cwpsp.txtp = p value; q FDR B H = q adjusted significance level False Discovery Rate using Benjamini and Hochberg adjustment; n= n genes annotated in test group; n PSP= n genes annotated in PSP. n significant in category 22}
% % % \end{table}

% % % \paragraph{Cellular component}

% % % Cellular component table~\ref{tab:ToppGENE GO: Cellular Component. tra 90 centile NA narm cwpsp.txtp = p value; q FDR B H = q adjusted significance level False Discovery Rate using Benjamini and Hochberg adjustment; n= n genes annotated in test group; n PSP= n genes annotated in PSP. n significant in category 26}. Mitochondrial protein complex and ribosomal subunit are enriched consisent with findings of translation being enriched in biological process.

% % % % latex table generated in R 3.6.3 by xtable 1.8-4 package
% % % % Sun Oct  4 14:38:37 2020
% % %   \begin{table}[ht]
% % % \centering
% % % \begin{adjustbox}{width=\textwidth}
% % % \setlength{\extrarowheight}{2pt}
% % % \begin{tabular}{@{}clllcl@{}}
% % %   \toprule
% % %   ID & Cellular Component & $n$ & $n$ PSP & $p$ & $q$ FDR B H \\ 

% % %   \midrule
% % % GO:0044391 & ribosomal subunit & 31 & 124 & $8.04 \times 10^{-8}$ & $3.86 \times 10^{-5}$ \\ 
% % %   GO:0098798 & mitochondrial protein complex & 33 & 139 & $1.15 \times 10^{-7}$ & $3.86 \times 10^{-5}$ \\ 
% % %   GO:0005840 & ribosome & 32 & 141 & $5.48 \times 10^{-7}$ & $1.23 \times 10^{-4}$ \\ 
% % %   GO:0005761 & mitochondrial ribosome & 13 & 31 & $1.03 \times 10^{-6}$ & $1.39 \times 10^{-4}$ \\ 
% % %   GO:0000313 & organellar ribosome & 13 & 31 & $1.03 \times 10^{-6}$ & $1.39 \times 10^{-4}$ \\ 
% % %   GO:0005762 & mitochondrial large ribosomal subunit & 9 & 16 & $2.44 \times 10^{-6}$ & $2.34 \times 10^{-4}$ \\ 
% % %   GO:0000315 & organellar large ribosomal subunit & 9 & 16 & $2.44 \times 10^{-6}$ & $2.34 \times 10^{-4}$ \\ 
% % %   GO:0015934 & large ribosomal subunit & 18 & 64 & $8.18 \times 10^{-6}$ & $6.88 \times 10^{-4}$ \\ 
% % %   GO:0000502 & proteasome complex & 13 & 38 & $1.50 \times 10^{-5}$ & $1.01 \times 10^{-3}$ \\ 
% % %   GO:1905369 & endopeptidase complex & 13 & 38 & $1.50 \times 10^{-5}$ & $1.01 \times 10^{-3}$ \\ 
% % %   \hline
% % % \bottomrule

% % % \end{tabular}
% % % \end{adjustbox}
% % % \caption[Gene ontology enrichment High Transitivity NaN genes Cellular Component of genes above 90th centile of distribution]{\textbf{High Transitivity NaN genes. Cellular Component - Gene ontology enrichment analysis} using the ToppGene web client and all PSP genes as the background set.  Molecular function tested gene set is 90th centile of degree distribution.  p value; q FDR B H = q adjusted significance level False Discovery Rate using Benjamini and Hochberg adjustment; $n$= $n$ genes annotated in test group; $n$ PSP= n genes annotated in PSP. $n$ ontology terms significantly (q FDR$>=0.05$) enriched for this centrality measure and ontology category (e.g. MF,CC,BP): 26} 
% % % \label{tab:ToppGENE GO: Cellular Component. tra 90 centile NA narm cwpsp.txtp = p value; q FDR B H = q adjusted significance level False Discovery Rate using Benjamini and Hochberg adjustment; n= n genes annotated in test group; n PSP= n genes annotated in PSP. n significant in category 26}
% % % \end{table}




% % % \paragraph{Other}
% % % Pathway related to mitochondria. No disease enrichment.




% % % \subsubsection{Low transitivity}
% % % No enrichment



% % % \clearpage
% % % \subsection{kcoreness}


% % % \subsubsection{High kcoreness}
% % % \paragraph{Molecular function}

% % % Molecular function table~\ref{tab:ToppGENE GO: Molecular Function. kco 90 centile cwpsp.txtp = p value; q FDR B H = q adjusted significance level False Discovery Rate using Benjamini and Hochberg adjustment; n= n genes annotated in test group; n PSP= n genes annotated in PSP. n significant in category 180}
% % % Ubiquitin and RNA binding. Low adjusted p for RNA binding comparable to eigenvector\footnote{\url{source('~/RProjects/chapter3/R/sig_magma_genes/central_genes_toppgene/kcoreness/5_0_clip_centrality_kco_high_mf_n_10.R')}}.

% % %   \begin{table}[ht]
% % % \centering
% % % \begin{adjustbox}{width=\textwidth}
% % % \setlength{\extrarowheight}{2pt}
% % % \begin{tabular}{@{}clllcl@{}}
% % %   \toprule
% % %   ID & Molecular Function & $n$ & $n$ PSP & $p$ & $q$ FDR B H \\ 

% % %   \midrule
% % % GO:0003723 & RNA binding & 188 & 554 & $2.64 \times 10^{-58}$ & $2.68 \times 10^{-55}$ \\ 
% % %   GO:0044389 & ubiquitin-like protein ligase binding & 60 & 123 & $2.82 \times 10^{-26}$ & $1.43 \times 10^{-23}$ \\ 
% % %   GO:0031625 & ubiquitin protein ligase binding & 58 & 117 & $7.13 \times 10^{-26}$ & $2.42 \times 10^{-23}$ \\ 
% % %   GO:0003729 & mRNA binding & 48 & 100 & $1.05 \times 10^{-20}$ & $2.66 \times 10^{-18}$ \\ 
% % %   GO:0003735 & structural constituent of ribosome & 43 & 103 & $8.30 \times 10^{-16}$ & $1.69 \times 10^{-13}$ \\ 
% % %   GO:0101005 & ubiquitinyl hydrolase activity & 36 & 77 & $2.84 \times 10^{-15}$ & $4.12 \times 10^{-13}$ \\ 
% % %   GO:0004843 & thiol-dependent ubiquitin-specific protease activity & 36 & 77 & $2.84 \times 10^{-15}$ & $4.12 \times 10^{-13}$ \\ 
% % %   GO:0008134 & transcription factor binding & 53 & 156 & $1.02 \times 10^{-14}$ & $1.30 \times 10^{-12}$ \\ 
% % %   GO:0044877 & protein-containing complex binding & 112 & 513 & $4.46 \times 10^{-14}$ & $5.04 \times 10^{-12}$ \\ 
% % %   GO:0019843 & rRNA binding & 22 & 34 & $1.33 \times 10^{-13}$ & $1.35 \times 10^{-11}$ \\ 
% % %   \bottomrule
% % % \end{tabular}
% % % \end{adjustbox}
% % % \caption[Gene ontology enrichment High kcoreness genes Molecular Function of genes above 90th centile of distribution]{\textbf{High kcoreness genes. Molecular Function - Gene ontology enrichment analysis} using the ToppGene web client and all PSP genes as the background set.  Molecular function tested gene set is 90th centile of degree distribution.  p value; q FDR B H = q adjusted significance level False Discovery Rate using Benjamini and Hochberg adjustment; $n$= $n$ genes annotated in test group; $n$ PSP= n genes annotated in PSP. $n$ ontology terms significantly (q FDR$>=0.05$) enriched for this centrality measure and ontology category (e.g. MF,CC,BP): 180} 
% % % \label{tab:ToppGENE GO: Molecular Function. kco 90 centile cwpsp.txtp = p value; q FDR B H = q adjusted significance level False Discovery Rate using Benjamini and Hochberg adjustment; n= n genes annotated in test group; n PSP= n genes annotated in PSP. n significant in category 180}
% % % \end{table}

% % % \paragraph{Biological process}
% % % Biological process table~\ref{tab:ToppGENE GO: Biological Process. kco 90 centile cwpsp.txtp = p value; q FDR B H = q adjusted significance level False Discovery Rate using Benjamini and Hochberg adjustment; n= n genes annotated in test group; n PSP= n genes annotated in PSP. n significant in category 1141}.

% % % Again mRNA and also viral processes and symbiotic showing that these are common and probably preserved across different organisms. Very low adjusted value and high coverage for terms.


% % % % latex table generated in R 3.6.3 by xtable 1.8-4 package
% % % % Sun Oct  4 15:02:29 2020
% % %   \begin{table}[ht]
% % % \centering
% % % \begin{adjustbox}{width=\textwidth}
% % % \setlength{\extrarowheight}{2pt}
% % % \begin{tabular}{@{}clllcl@{}}
% % %   \toprule
% % %   ID & Biological Process & $n$ & $n$ PSP & $p$ & $q$ FDR B H \\ 

% % %   \midrule
% % % GO:0016071 & mRNA metabolic process & 142 & 260 & $3.73 \times 10^{-74}$ & $2.26 \times 10^{-70}$ \\ 
% % %   GO:0009057 & macromolecule catabolic process & 166 & 443 & $4.75 \times 10^{-57}$ & $1.44 \times 10^{-53}$ \\ 
% % %   GO:0044265 & cellular macromolecule catabolic process & 150 & 372 & $1.43 \times 10^{-55}$ & $2.88 \times 10^{-52}$ \\ 
% % %   GO:0006402 & mRNA catabolic process & 100 & 180 & $2.06 \times 10^{-51}$ & $3.12 \times 10^{-48}$ \\ 
% % %   GO:0006401 & RNA catabolic process & 100 & 186 & $1.34 \times 10^{-49}$ & $1.62 \times 10^{-46}$ \\ 
% % %   GO:0044403 & symbiotic process & 140 & 364 & $2.98 \times 10^{-48}$ & $3.01 \times 10^{-45}$ \\ 
% % %   GO:0044419 & interspecies interaction between organisms & 140 & 368 & $1.46 \times 10^{-47}$ & $1.27 \times 10^{-44}$ \\ 
% % %   GO:0016032 & viral process & 135 & 348 & $8.48 \times 10^{-47}$ & $6.43 \times 10^{-44}$ \\ 
% % %   GO:0034655 & nucleobase-containing compound catabolic process & 102 & 217 & $1.67 \times 10^{-43}$ & $1.13 \times 10^{-40}$ \\ 
% % %   GO:0019439 & aromatic compound catabolic process & 102 & 232 & $3.61 \times 10^{-40}$ & $2.19 \times 10^{-37}$ \\ 
% % %   \bottomrule
% % % \end{tabular}
% % % \end{adjustbox}

% % % \caption[Gene ontology enrichment High kcoreness genes Biological Process of genes above 90th centile of distribution]{\textbf{High kcoreness genes. Biological Process - Gene ontology enrichment analysis} using the ToppGene web client and all PSP genes as the background set.  Biological Process tested gene set is 90th centile of degree distribution.  p value; q FDR B H = q adjusted significance level False Discovery Rate using Benjamini and Hochberg adjustment; $n$= $n$ genes annotated in test group; $n$ PSP= n genes annotated in PSP. $n$ ontology terms significantly (q FDR$>=0.05$) enriched for this centrality measure and ontology category (e.g. MF,CC,BP): 1141} 
% % % \label{tab:ToppGENE GO: Biological Process. kco 90 centile cwpsp.txtp = p value; q FDR B H = q adjusted significance level False Discovery Rate using Benjamini and Hochberg adjustment; n= n genes annotated in test group; n PSP= n genes annotated in PSP. n significant in category 1141}
% % % \end{table}

% % % \paragraph{Cellular component}

% % % Cellular component table~\ref{tab:ToppGENE GO: Cellular Component. kco 90 centile cwpsp.txtp = p value; q FDR B H = q adjusted significance level False Discovery Rate using Benjamini and Hochberg adjustment; n= n genes annotated in test group; n PSP= n genes annotated in PSP. n significant in category 181} again ribonucleoprotein terms and splicasome terms as per eigenvector. Low FDR q values. Adherens junction which contains proteins at cell to cell junction including cadherins. 

% % % % latex table generated in R 3.6.3 by xtable 1.8-4 package
% % % % Sun Oct  4 15:05:27 2020
% % %   \begin{table}[ht]
% % % \centering
% % % \begin{adjustbox}{width=\textwidth}
% % % \setlength{\extrarowheight}{2pt}
% % % \begin{tabular}{@{}clllcl@{}}
% % %   \toprule
% % %   ID & Cellular Component & $n$ & $n$ PSP & $p$ & $q$ FDR B H \\ 

% % %   \midrule
% % % GO:1990904 & ribonucleoprotein complex & 137 & 303 & $6.81 \times 10^{-58}$ & $5.48 \times 10^{-55}$ \\ 
% % %   GO:0005925 & focal adhesion & 87 & 239 & $9.15 \times 10^{-27}$ & $3.68 \times 10^{-24}$ \\ 
% % %   GO:0022626 & cytosolic ribosome & 53 & 97 & $1.86 \times 10^{-26}$ & $4.98 \times 10^{-24}$ \\ 
% % %   GO:0000974 & Prp19 complex & 29 & 31 & $4.58 \times 10^{-26}$ & $8.40 \times 10^{-24}$ \\ 
% % %   GO:0030055 & cell-substrate junction & 87 & 244 & $5.22 \times 10^{-26}$ & $8.40 \times 10^{-24}$ \\ 
% % %   GO:0071013 & catalytic step 2 spliceosome & 27 & 29 & $3.59 \times 10^{-24}$ & $4.81 \times 10^{-22}$ \\ 
% % %   GO:0005681 & spliceosomal complex & 32 & 40 & $4.61 \times 10^{-24}$ & $5.29 \times 10^{-22}$ \\ 
% % %   GO:0097525 & spliceosomal snRNP complex & 29 & 35 & $1.04 \times 10^{-22}$ & $1.05 \times 10^{-20}$ \\ 
% % %   GO:0005682 & U5 snRNP & 28 & 34 & $8.18 \times 10^{-22}$ & $7.31 \times 10^{-20}$ \\ 
% % %   GO:0005912 & adherens junction & 94 & 314 & $1.07 \times 10^{-21}$ & $8.59 \times 10^{-20}$ \\ 
% % %   \hline
% % % \end{tabular}
% % % \end{adjustbox}

% % % \caption[Gene ontology enrichment High kcoreness genes Cellular Component of genes above 90th centile of distribution]{\textbf{High kcoreness genes. Cellular Component - Gene ontology enrichment analysis} using the ToppGene web client and all PSP genes as the background set.  Cellular Component tested gene set is 90th centile of degree distribution.  p value; q FDR B H = q adjusted significance level False Discovery Rate using Benjamini and Hochberg adjustment; $n$= $n$ genes annotated in test group; $n$ PSP= n genes annotated in PSP. $n$ ontology terms significantly (q FDR$>=0.05$) enriched for this centrality measure and ontology category (e.g. MF,CC,BP): 181} 
% % % \label{tab:ToppGENE GO: Cellular Component. kco 90 centile cwpsp.txtp = p value; q FDR B H = q adjusted significance level False Discovery Rate using Benjamini and Hochberg adjustment; n= n genes annotated in test group; n PSP= n genes annotated in PSP. n significant in category 181}
% % % \end{table}

% % % \paragraph{Others}
% % % Mouse phenotype embroynic lethality

% % % Human phenotype neoplasm and limb abnormalities. 

% % % Diseases all very severe viral infections and neoplasms top neurological is number 7 C0338451  Frontotemporal dementia n in high coreness 44 n in annotation in PSP 119 p $3.75 \times 10^{-13}$ FDR q $3.87 \times 10^{-10}$  

% % % Pub med integrin and beta catenin in neoplasms

% % % Pathway strong enrichment gene expression
% % % \clearpage
% % % \subsubsection{Low kcoreness}

% % % \paragraph{Cellular component}
% % % Cellular component table~\ref{tab:ToppGENE GO: Cellular Component. kco 10 centile cwpsp.txtp = p value; q FDR B H = q adjusted significance level False Discovery Rate using Benjamini and Hochberg adjustment; n= n genes annotated in test group; n PSP= n genes annotated in PSP. n significant in category 2}. Over representation of integral part of plasma membrane similar to low closeness the low kcore is also the periphery (although possibly a different periphery - ie core periphery rather than far away)


% % % % latex table generated in R 3.6.3 by xtable 1.8-4 package
% % % % Sun Oct  4 15:15:03 2020
% % %  \begin{table}[ht]
% % % \centering
% % % \begin{adjustbox}{width=\textwidth}
% % % \setlength{\extrarowheight}{2pt}
% % % \begin{tabular}{@{}clllcl@{}}
% % %   \toprule
% % %   ID & Cellular Component & $n$ & $n$ PSP & $p$ & $q$ FDR B H \\ 

% % %   \midrule
% % % GO:0031226 & intrinsic component of plasma membrane & 98 & 331 & $8.26 \times 10^{-9}$ & $5.35 \times 10^{-6}$ \\ 
% % %   GO:0005887 & integral component of plasma membrane & 91 & 307 & $2.94 \times 10^{-8}$ & $9.51 \times 10^{-6}$ \\ 
% % %   \hline
% % % \end{tabular}
% % % \end{adjustbox}
% % % % \caption{ToppGene GO: Cellular Component. kco 10 centile cwpsp.txtp = p value; q FDR B H = q adjusted significance level False Discovery Rate using Benjamini and Hochberg adjustment; n= n genes annotated in test group; n PSP= n genes annotated in PSP. n significant in category 2} 
% % % \caption[Gene ontology enrichment Low kcoreness genes Cellular Component of genes above 90th centile of distribution]{\textbf{Low kcoreness genes. Cellular Component - Gene ontology enrichment analysis} using the ToppGene web client and all PSP genes as the background set.  Cellular Component tested gene set is 10th centile of degree distribution.  p value; q FDR B H = q adjusted significance level False Discovery Rate using Benjamini and Hochberg adjustment; $n$= $n$ genes annotated in test group; $n$ PSP= n genes annotated in PSP. $n$ ontology terms significantly (q FDR$>=0.05$) enriched for this centrality measure and ontology category (e.g. MF,CC,BP): 2} 
% % % \label{tab:ToppGENE GO: Cellular Component. kco 10 centile cwpsp.txtp = p value; q FDR B H = q adjusted significance level False Discovery Rate using Benjamini and Hochberg adjustment; n= n genes annotated in test group; n PSP= n genes annotated in PSP. n significant in category 2}
% % % \end{table}

% % % \paragraph{Pathway}

% % % Pathway shown in table~\ref{tab:ToppGENE Pathway. kco 10 centile cwpsp.txtp = p value; q FDR B H = q adjusted significance level False Discovery Rate using Benjamini and Hochberg adjustment; n= n genes annotated in test group; n PSP= n genes annotated in PSP. n significant in category 8} is enriched for extra cellular matrix and matrix associated proteins and post translational modification of GPI. Good coverage (contains a lot of those found in the PSP by selecting this centrality measure also high for a low measure . 
% % %  \begin{table}[ht]
% % % \centering
% % % \begin{adjustbox}{width=\textwidth}
% % % \setlength{\extrarowheight}{2pt}
% % % \begin{tabular}{@{}clllcl@{}}
% % %   \toprule
% % %   ID & Pathway & $n$ & $n$ PSP & $p$ & $q$ FDR B H \\ 

% % %   \midrule
% % % M5889 & \makecell{Ensemble of genes encoding extracellular matrix\\ and extracellular matrix-associated proteins}  & 37 & 91 & $1.50 \times 10^{-10}$ & $2.24 \times 10^{-7}$ \\ 
% % %   M5884 & \makecell{Ensemble of genes encoding core extracellular matrix\\ including ECM glycoproteins, collagens and proteoglycans} & 17 & 31 & $8.53 \times 10^{-8}$ & $6.38 \times 10^{-5}$ \\ 
% % %   M3008 & Genes encoding structural ECM glycoproteins & 13 & 25 & $6.70 \times 10^{-6}$ & $3.34 \times 10^{-3}$ \\ 
% % %   132956 & Metabolic pathways & 74 & 353 & $6.99 \times 10^{-5}$ & $2.05 \times 10^{-2}$ \\ 
% % %   82989 & Glycerophospholipid metabolism & 12 & 26 & $7.07 \times 10^{-5}$ & $2.05 \times 10^{-2}$ \\ 
% % %   M5885 & \makecell{Ensemble of genes encoding ECM-associated proteins including\\ ECM-affilaited proteins, ECM regulators and secreted factors} & 20 & 60 & $9.51 \times 10^{-5}$ & $2.05 \times 10^{-2}$ \\ 
% % %   M9131 & Glycerophospholipid metabolism & 11 & 23 & $9.57 \times 10^{-5}$ & $2.05 \times 10^{-2}$ \\ 
% % %   1268710 & Post-translational modification: synthesis of GPI-anchored proteins & 7 & 11 & $1.96 \times 10^{-4}$ & $3.66 \times 10^{-2}$ \\ 
% % %   \hline
% % % \end{tabular}
% % % \end{adjustbox}
% % % \caption[Gene ontology enrichment Low kcoreness genes Pathway of genes above 90th centile of distribution]{\textbf{Low kcoreness genes. Pathway - Gene ontology enrichment analysis} using the ToppGene web client and all PSP genes as the background set.  Pathway tested gene set is 10th centile of degree distribution.  p value; q FDR B H = q adjusted significance level False Discovery Rate using Benjamini and Hochberg adjustment; $n$= $n$ genes annotated in test group; $n$ PSP= n genes annotated in PSP. $n$ ontology terms significantly (q FDR$>=0.05$) enriched for this centrality measure and ontology category (e.g. MF,CC,BP): 8} 
% % % % \caption{ToppGene Pathway. kco 10 centile cwpsp.txtp = p value; q FDR B H = q adjusted significance level False Discovery Rate using Benjamini and Hochberg adjustment; n= n genes annotated in test group; n PSP= n genes annotated in PSP. n significant in category 8} 
% % % \label{tab:ToppGENE Pathway. kco 10 centile cwpsp.txtp = p value; q FDR B H = q adjusted significance level False Discovery Rate using Benjamini and Hochberg adjustment; n= n genes annotated in test group; n PSP= n genes annotated in PSP. n significant in category 8}
% % % \end{table}
% % % \paragraph{Others}
% % % Pub med pathway for regulation of habenula development 19906978  Brn3a and Nurr1 mediate a gene regulatory pathway for habenula development. n 22 n in PSP 44 p $8.18 \times 10^{-7}$ FDR q $2.36 \times 10^{-2}$
% % % % latex table generated in R 3.6.3 by xtable 1.8-4 package
% % % % Sun Oct  4 15:17:32 2020


% % % Gene family is shown in table~\ref{tab:ToppGENE Gene Family. kco 10 centile cwpsp.txtp = p value; q FDR B H = q adjusted significance level False Discovery Rate using Benjamini and Hochberg adjustment; n= n genes annotated in test group; n PSP= n genes annotated in PSP. n significant in category 5}. It is enriched for calcium voltage gate channel subunits, fibronectin and solute carriers.

% % % % latex table generated in R 3.6.3 by xtable 1.8-4 package
% % % % Sun Oct  4 15:24:07 2020
% % %  \begin{table}[ht]
% % % \centering
% % % \begin{adjustbox}{width=\textwidth}
% % % \setlength{\extrarowheight}{2pt}
% % % \begin{tabular}{@{}clllcl@{}}
% % %   \toprule
% % %   ID & Gene Family & $n$ & $n$ PSP & $p$ & $q$ FDR B H \\ 

% % %   \midrule
% % % 752 & Solute carriers & 21 & 45 & $7.61 \times 10^{-7}$ & $1.58 \times 10^{-4}$ \\ 
% % %   253 & Calcium voltage-gated channel subunits & 8 & 14 & $4.36 \times 10^{-4}$ & $4.53 \times 10^{-2}$ \\ 
% % %   1210 & ATPase phospholipid transporting & 4 & 4 & $7.39 \times 10^{-4}$ & $4.77 \times 10^{-2}$ \\ 
% % %   593 & \makecell{Fibronectin type III domain containing$|$I-set domain containing$|$\\Immunoglobulin like domain containing} & 15 & 42 & $1.11 \times 10^{-3}$ & $4.77 \times 10^{-2}$ \\ 
% % %   594 & \makecell{Immunoglobulin like domain containing$|$Interleukin receptors$|$\\TIR domain containing} & 13 & 34 & $1.15 \times 10^{-3}$ & $4.77 \times 10^{-2}$ \\ 
% % %   \hline
% % % \end{tabular}
% % % \end{adjustbox}
% % % \caption[Gene ontology enrichment Low kcoreness genes Gene Family of genes above 90th centile of distribution]{\textbf{Low kcoreness genes. Gene Family - Gene ontology enrichment analysis} using the ToppGene web client and all PSP genes as the background set.  Gene Family tested gene set is 10th centile of degree distribution.  p value; q FDR B H = q adjusted significance level False Discovery Rate using Benjamini and Hochberg adjustment; $n$= $n$ genes annotated in test group; $n$ PSP= n genes annotated in PSP. $n$ ontology terms significantly (q FDR$>=0.05$) enriched for this centrality measure and ontology category (e.g. MF,CC,BP): 5} 
% % % % \caption{ToppGene Gene Family. kco 10 centile cwpsp.txtp = p value; q FDR B H = q adjusted significance level False Discovery Rate using Benjamini and Hochberg adjustment; n= n genes annotated in test group; n PSP= n genes annotated in PSP. n significant in category 5} 
% % % \label{tab:ToppGENE Gene Family. kco 10 centile cwpsp.txtp = p value; q FDR B H = q adjusted significance level False Discovery Rate using Benjamini and Hochberg adjustment; n= n genes annotated in test group; n PSP= n genes annotated in PSP. n significant in category 5}
% % % \end{table}

% % % \clearpage
% % % \subsection{Ontology overlap}
% % % \label{sec: ontology overlap}
% % % The source for this is chapter3 mac

% % % Top twenty terms

% % % Jaccard between and degree 0.6

% % % \subsubsection{High centrality}

% % % % latex table generated in R 3.6.1 by xtable 1.8-4 package
% % % % Tue Dec 15 08:14:34 2020
% % % \begin{table}[ht]
% % % \centering
% % % \begin{tabular}{rrrrrrrr}
% % %   \hline
% % %  & degree & betweenness & eigenvector & closeness & kcoreness & tra0 & traNA \\ 
% % %   \hline
% % % degree & 0.00 & 0.60 & 0.14 & 0.38 & 0.25 & 0.05 & 0.08 \\ 
% % %   betweenness & 0.00 & 0.00 & 0.08 & 0.29 & 0.18 & 0.05 & 0.08 \\ 
% % %   eigenvector & 0.00 & 0.00 & 0.00 & 0.29 & 0.54 & 0.25 & 0.21 \\ 
% % %   closeness & 0.00 & 0.00 & 0.00 & 0.00 & 0.60 & 0.11 & 0.11 \\ 
% % %   kcoreness & 0.00 & 0.00 & 0.00 & 0.00 & 0.00 & 0.11 & 0.11 \\ 
% % %   tra0 & 0.00 & 0.00 & 0.00 & 0.00 & 0.00 & 0.00 & 0.82 \\ 
% % %   traNA & 0.00 & 0.00 & 0.00 & 0.00 & 0.00 & 0.00 & 0.00 \\ 
% % %   \hline
% % % \end{tabular}
% % % \caption{BP 0.9 top 0.9 centile of each centrality measure. Top 20 enriched GO terms considered. Result is overlap of GO terms Jaccard index. Both forms of transitivity show high overlap and a similar modest overlap with other centrality terms. tra0 is transitivity with isolates (degree 0 or 1) set to 0. traNA is transitivity with isolates set to NA and removed from calculation of mean.} 
% % % \end{table}

% % % % latex table generated in R 3.6.1 by xtable 1.8-4 package
% % % % Tue Dec 15 08:27:44 2020
% % % \begin{table}[ht]
% % % \centering
% % % \begin{tabular}{rrrrrrrr}
% % %   \hline
% % %  & degree & betweenness & eigenvector & closeness & kcoreness & tra0 & traNA \\ 
% % %   \hline
% % % degree & 0.00 & 0.60 & 0.48 & 0.74 & 0.67 & 0.11 & 0.08 \\ 
% % %   betweenness & 0.00 & 0.00 & 0.43 & 0.60 & 0.48 & 0.14 & 0.11 \\ 
% % %   eigenvector & 0.00 & 0.00 & 0.00 & 0.67 & 0.60 & 0.18 & 0.14 \\ 
% % %   closeness & 0.00 & 0.00 & 0.00 & 0.00 & 0.74 & 0.11 & 0.08 \\ 
% % %   kcoreness & 0.00 & 0.00 & 0.00 & 0.00 & 0.00 & 0.18 & 0.14 \\ 
% % %   tra0 & 0.00 & 0.00 & 0.00 & 0.00 & 0.00 & 0.00 & 0.82 \\ 
% % %   traNA & 0.00 & 0.00 & 0.00 & 0.00 & 0.00 & 0.00 & 0.00 \\ 
% % %   \hline
% % % \end{tabular}
% % % \caption{MF 0.9 top 0.9 centile of each centrality measure. Top 20 enriched GO terms considered. Result is overlap of GO terms Jaccard index. Both forms of transitivity show high overlap and a similar modest overlap with other centrality terms. tra0 is transitivity with isolates (degree 0 or 1) set to 0. traNA is transitivity with isolates set to NA and removed from calculation of mean.} 
% % % \end{table}



% % % % Tue Dec 15 08:48:44 2020
% % % \begin{table}[ht]
% % % \centering
% % % \begin{tabular}{rrrrrrrr}
% % %   \hline
% % %  & degree & betweenness & eigenvector & closeness & kcoreness & tra0 & traNA \\ 
% % %   \hline
% % % degree &  & 0.67 & 0.48 & 0.54 & 0.54 & 0.11 & 0.11 \\ 
% % %   betweenness &  &  & 0.33 & 0.43 & 0.33 & 0.14 & 0.14 \\ 
% % %   eigenvector &  &  &  & 0.60 & 0.67 & 0.18 & 0.18 \\ 
% % %   closeness &  &  &  &  & 0.67 & 0.14 & 0.14 \\ 
% % %   kcoreness &  &  &  &  &  & 0.11 & 0.11 \\ 
% % %   tra0 &  &  &  &  &  &  & 0.90 \\ 
% % %   traNA &  &  &  &  &  &  &  \\ 
% % %   \hline
% % % \end{tabular}
% % % \caption{Cellular component top 0.9 centile of each centrality measure. Top 20 enriched GO terms considered. Result is overlap of GO terms Jaccard index. Both forms of transitivity show high overlap and a similar modest overlap with other centrality terms. tra0 is transitivity with isolates (degree 0 or 1) set to 0. traNA is transitivity with isolates set to NA and removed from calculation of mean.\url{source('~/RProjects/chapter3_mac/R/overlap_centralities/refactor/mod_GO_refactor_runjaccard_CC_modA_point9.R')} }

% % % \end{table}

% % % \subsubsection{Low centrality}



% % % \clearpage
% % % %\section{Panther gene ontology redo again}
% % % % commented out below


% % % % \subsection{Results betweenness centrality Statistics and GO enrichment}
% % % % The log transform of betweenness centrality is more nearly normal. The histogram is slightly right skewed.
% % % % Although the qq plot looks relatively straight the Shapiro Wilk test for normality after removing infinite values and NA of the log transform  is 
% % % % W = 0.997 p = $1.1 \times 10^{-5}$

% % % % Top 142 betweeness
% % % % Panther 2016

% % % % Parkinson’s disease p 2.48 x 10-22
% % % % Dopamine mediated signalling pathway

% % % % Mammalian phenotype – lethality
% % % % BP – regulation of mRNA stability 
% % % % MF kinase binding and ubiquitin
% % % % CC cell adhesion
% % % % 	]
% % % \subsubsection{Results Edge betweenness results Statistics and GO enrichment}Table:Test for normality (Shapiro-Wilk) for centrality measures of PSP
% % % The highest edge betweeness (21221.7 Median 321,4 mean 583.7) is between APP (351) and EGFR 1956.

% % % The second highest is GRB2 (2885) and APP.
% % % APP is on 12 of the top 20 edges. 
% % % ELAV1 is on 5 of the top 20 edges.see \url{source('~/RProjects/centrality/R/other_centrality/eccentricity_and_distance.R')} EGFR (1956), 2885 and 7514 appear three times.



% % % \clearpage

% % % % \paragraph{Move to method}
% % % % We can also look at the assortativity between nodes for degree.

% % % % Methods section~\ref{sec:Assortativity methods}

% % % % Increase in assortativity leads to increased branching, internal clustering or internal communications between clusters\cite{estrada2011combinatorial}.
% % % % In figure~\ref{fig:PSP degree power-law poisson} the degree distribution has overlaid the probability density plot for a Poisson distribution overlaid with mean ($\lambda$) equal to the mean degree of the PSP network (17.4) (figure~\ref{fig:PSP degree power-law poisson}. This would be the expected probability density if the edges were placed at random. The Poisson distribution assigns almost no probability mass to the existence of nodes with degree values greater than 100. However, there are several nodes in the PSP with this or greater degree\footnote{code at \url{/home/grant/RProjects/PhD_graphs/}}.

% % % % \begin{figure}
% % % %     \centering
% % % %     \includegraphics[width=\textwidth]{images/chapter3/ggplot2/theme/powerlaw/Rplot01_poissonanddegree2.png}
% % % %     \caption[Degree distribution with Poisson pdf overlaid]{Plot of degree distribution PSP with degree $k$ on x axis with log10 scaling. The degree distribution (the frequency of nodes of degree $k$ or $\frac{n_k}{k}$ is plotted on the y axis). The blue dash is the expected degree distribution given a Poisson distribution with mean equal to the mean of the empirical degree distribution. The solid vertical red line is the average degree. \url{source('~/RProjects/group_size_distribution/R/poissonanddegree_theme.R')}\textcolor{red}{check this is Expected value vs pdf}}
% % % %     \label{fig:PSP degree power-law poisson}
% % % % \end{figure}

% % % % \begin{figure}
% % % %     \centering
% % % %     \begin{subfigure}[t]{0.45\textwidth}
% % % %         \centering
% % % %         \includegraphics[width=\linewidth]{images/dependence_of_scale_coefficient_on_min_degree.png} 
% % % %         \caption{The parameter $\gamma$ as a function of the degree at which calculation of the coefficient starts in the degree sequence(e.g. all degree $>5$) from \url{RProjects/PhDGraphs/calculate/gamma}} \label{fig:gamma}
% % % %     \end{subfigure}
% % % %     \hfill
% % % %     \begin{subfigure}[t]{0.45\textwidth}
% % % %         \centering
% % % %         \includegraphics[width=\linewidth]{images/plot_log_degree_distribution.png} 
% % % %         \caption{The plot of log cumulative cdf of degree distribution log 10 scale x and y. \textcolor{red}{this is 1 -cdf Actually see Newman on cdf complement}} \label{fig:log_degree_distribution}
% % % %     \end{subfigure}
% % % %     \caption{Plots of degree distribution \textcolor{red}{may want to do this in subfigure packages rather than subcaption to line up edges}}
% % % %     \label{fig:Plots of degree distribution}
% % % % \end{figure}




% % % % \begin{table}[]
% % % %     \centering
    
% % % %     \begin{tabular}{ccccccc}
% % % %     \toprule
% % % %         Distribution & Log $L$ & bootstrap $p$ & $x_{min}$ & $\gamma$ or pars & Vuong $p$  &g.o.f.\\
% % % %         \midrule
% % % %         Power law \\
% % % %         Log-normal \\
% % % %         Exponential \\
% % % %         Poisson \\
    
% % % %   %      Weibull & -2925.3 &&& 0.230 ,0.0157 & drop weibull\\
% % % %         \bottomrule
% % % %     \end{tabular}
% % % %     \caption[Log $L$ of degree distribution under different models]{log-likelihood (Log $L$) of degree distribution under x min defined for power law. All distributions fitted from xmin of power law per Gillespie and Clauset etc. Vuong $p$ = one sided $p$ Vuong's closeness. g.o.f is Kolmogorov Smirnov distance statistic. *Need to recheck vuong as should be versus power law so not quoted with power law test\url{source('~/RProjects/chapter3/R/fit_power_law/fit_pl_setxmin.R')} }
% % % %     \tiny{Bootstrap p returns an error for Weibull and Weibull is a continuous distribution Error in checkForRemoteErrors(val) one node produced an error: non-finite finite-difference value [1] So we should probably drop the Weibull as it does not make comparisons easy and probably does not add very much to what we are saying which is can't choose between power law and log-normal}
% % % %     \label{tab:ll distributions}
% % % % \end{table}
% % % % \begin{table}[]
% % % %     \centering
    
% % % %     \begin{tabular}{ccccccc}
% % % %     \toprule
% % % %         Distribution & Log $L$ & bootstrap $p$ & $x_{min}$ & $\gamma$ or pars & Vuong $p$  &g.o.f.\\
% % % %         \midrule
% % % %         Power law &  -2943.6 & 0.575 & 24 & 2.51 & 0.906* &  0.0190\\
% % % %         Log-normal & -2941.6 & 0.4 (0.32) &&-2.34, 2.11& 0.903 &  0.01380035\\
% % % %         Exponential & -3044.7 & $<0.01$&&0.0285& $1.72\times10^{-7}$\\
% % % %         Poisson & -14376.3 & 0&&58.59 & & 0.088\\
    
% % % %   %      Weibull & -2925.3 &&& 0.230 ,0.0157 & drop weibull\\
% % % %         \bottomrule
% % % %     \end{tabular}
% % % %     \caption[Log $L$ of degree distribution under different models]{log-likelihood (Log $L$) of degree distribution under x min defined for power law. All distributions fitted from xmin of power law per Gillespie and Clauset etc. Vuong $p$ = one sided $p$ Vuong's closeness. g.o.f is Kolmogorov Smirnov distance statistic. *Need to recheck vuong as should be versus power law so not quoted with power law test\url{source('~/RProjects/chapter3/R/fit_power_law/fit_pl_setxmin.R')} }
% % % %     \tiny{Bootstrap p returns an error for Weibull and Weibull is a continuous distribution Error in checkForRemoteErrors(val) one node produced an error: non-finite finite-difference value [1] So we should probably drop the Weibull as it does not make comparisons easy and probably does not add very much to what we are saying which is can't choose between power law and log-normal}
% % % %     \label{tab:ll distributions}
% % % % \end{table}

% % % % % \section{===to supplemental===}
% % % % %   A table of the values of $\gamma$ for different values of $x_{min}$ can be found in supplementary table~\ref{table:gamma}.

% % % % % \section{Extra figures}
% % % % % \begin{figure} 
% % % % %     \centering
% % % % %     \includegraphics[width=\textwidth]{images/chapter3/cor_pairs/ggally/Rplot_corr_plot_ggally_large_size_11.png}
% % % % %     \caption{Scatter plot of centrality measures. Transitivity where isolates set to zero omitted for clarity and space considerations. Spearman correlation in upper triangular. Source \url{source('~/RProjects/disgen2/R/cor/cor.R')}}
% % % % %     \label{fig:scatter plot of multiple centralities gg ally}
% % % % % \end{figure}
% % % % % There is a broader dispersion of nodes from the trend line in where they are of high degree and the degree occurs only once in the degree sequence (figure~\ref{fig:Degree distribution of post synaptic proteome. Log10 - log10 scale.}).

% % % % % \begin{figure}
% % % % %     \includegraphics[width=12cm]{Rplot_DegreeDistribution}
% % % % %     \caption{Degree distribution of post synaptic proteome. Log10 - log10 scale. Code at \url{source('~/RProjects/gridsearch_gamma/R/plot_log_degreedist.R') }}
% % % % %     \label{fig:Degree distribution of post synaptic proteome. Log10 - log10 scale.}
% % % % % \end{figure}



% % % % % \subsection{C(k)}

% % % % % \begin{figure}
% % % % %     \centering
% % % % %     \includegraphics[width=\textwidth]{images/chapter3/ggplot2/degree_and_transitivity/Rplot_scatterplot_c_and_degree_new_format.png}
% % % % %       \caption{Scatter plot of the relationship between degree and local transitivity for the PSP nodes. Nodes with degree 1 have transitivity set to zero as transitivity in this case is undefined. Plot appears to show a negative correlation between degree and transitivity. The opacity of points has been decreased to reduce overplotting and show the increase in points at low degree. Potential replacement - \textcolor{red}{changed now} }
% % % % %       \tiny\url{source('~/RProjects/chapter3/R/transitivity/transitivity_and_degree/mean_deg_seq.R')} plot p
% % % % %     \label{fig:Scatter plot of the relationship between degree and local transitivity for the PSP nodes}
    
% % % % % \end{figure}


% % % % % \begin{figure}
% % % % %     \centering
% % % % %     \includegraphics[width=\textwidth]{images/chapter3/ggplot2/degree_and_transitivity/Rplot_rho_and_starting_degree_transitivity.png}
% % % % %     \caption{Plot of minumum degree included in the calculation of Spearmans rank correlation between degree and $\rho$. Although the overall coefficient is positive the correlation for nodes above degree 5 are clearly negative and this becomes more apparent at high degree. This relationship is therefore complex and the typical relationship described in the literature may only hold for the large networks in which this relationship has been studies such as the internet}.
% % % % %     \tiny\url{source('~/RProjects/chapter3/R/transitivity/transitivity_and_degree/plot/plot_start_degree.R')}
% % % % %     \label{fig:Plot of minumum degree included in the calculation of Spearmans rank correlation between degree and rho}
% % % % % \end{figure}



% % % % % \subsection{DEG}

% % % % % \begin{figure}
% % % % %     \centering
% % % % %     \includegraphics[width=\textwidth]{images/chapter3/ggplot2/db_essential_genes/theme/Rplot_single_boxplot_three_categories.png}
% % % % %     \caption{Boxplot of the degree for genes in PSP and in database of essential genes(DEG). On the x axis are three groups: those genes that do not appear in the DEG, those with at least one entry but less than five and those with five or more. Y axis is log10 transform of vertex (node) degree. The graph shows degree increasing across the three categories as genes become more essential. Notch is $\frac{1.58 IQR}{\sqrt{n}}$}
% % % % %     \tiny\url{source('~/RProjects/db_essential_genes/R/get_essential_genes/db_2020/plots/get_db_essential_threecategoriesboxplot_essential_centrality.R')}
% % % % %     \label{fig:boxplot_three_groups_DEG_logdegree1}
% % % % % \end{figure}




% % % % % \begin{figure}
% % % % %     \centering
% % % % %     \includegraphics[width=\textwidth]{images/chapter3/ggplot2/db_essential_genes/theme/Rplot_multiplot_with_theme_simple.png}
% % % % %     \caption{Multiple boxplot of the centrality measure for genes in PSP and in database of essential genes(DEG). On the x axis are three groups: those genes that do not appear in the DEG, those with at least one entry but less than five and those with five or more. Y axis is log10 transform of vertex (node) degree for degree, betweenness and eigenvector centrality. The graph shows degree increasing across the three categories as genes become more essential.} 
% % % % %     \tiny\url{ source('~/RProjects/db_essential_genes/R/get_essential_genes/db_2020/theme/get_db_essential_threecategoriesboxplot_multiplot_no_x_lab_essential_centrality_theme.R')}
% % % % %     \label{fig:boxplot_three_groups_DEG_multiplot1}
% % % % % \end{figure}


% % % % % \section{Extra tables}
% % % % % % latex table generated in R 3.6.1 by xtable 1.8-4 package
% % % % % % Sun Mar  7 16:19:10 2021
% % % % % \begin{table}[ht]
% % % % % \centering
% % % % % \begin{adjustbox}{width = \textwidth}
% % % % % \begin{tabular}{rrrrrrrr}
% % % % %   \hline
% % % % %  & degree & betweenness & eigenvector & closeness & transitivityNA & transitivity0 & kcoreness \\ 
% % % % %   \hline
% % % % % degree & 1.000 & 0.827 & 0.815 & 0.769 & -0.086 & -0.086 & 0.958 \\ 
% % % % %   betweenness & 0.827 & 1.000 & 0.598 & 0.650 & -0.315 & -0.315 & 0.707 \\ 
% % % % %   eigenvector & 0.815 & 0.598 & 1.000 & 0.938 & 0.174 & 0.174 & 0.881 \\ 
% % % % %   closeness & 0.769 & 0.650 & 0.938 & 1.000 & 0.120 & 0.120 & 0.821 \\ 
% % % % %   transitivityNA & -0.086 & -0.315 & 0.174 & 0.120 & 1.000 & 1.000 & 0.058 \\ 
% % % % %   transitivity0 & -0.086 & -0.315 & 0.174 & 0.120 & 1.000 & 1.000 & 0.058 \\ 
% % % % %   kcoreness & 0.958 & 0.707 & 0.881 & 0.821 & 0.058 & 0.058 & 1.000 \\ 
% % % % %   \hline
% % % % % \end{tabular}
% % % % % \end{adjustbox}
% % % % % \caption[Correlation of centrality measures for nodes $k>=5$]{Correlation of centrality measures. For nodes with degree five or more Spearman. Method for missing data include only pairwise complete. TransitivityNA treats isolates as NA and removes from calculation. Transitivity0 uses the option to set isolates to 0. Note that both measures of local transitivity are now identical and there is a clear negative correlation between betweenness and transitivity and between degree and transitivity.\url{source('~/RProjects/db_essential_genes/R/get_corr/cor_centrality_measures_ordered_cortable.R')}} 
% % % % % \label{tab:Correlation of centrality measures 2 transitvities deg five or more}
% % % % % \end{table}
% % % % % \paragraph{Pearson}

% % % % % % latex table generated in R 3.6.3 by xtable 1.8-4 package
% % % % % % Sun Mar 28 14:53:37 2021
% % % % % \begin{table}[ht]
% % % % % \centering
% % % % % \begin{tabular}{lrrrrrr}
% % % % %   \toprule
% % % % % Centrality & t & df & low & high & r & p \\ 
% % % % %   \midrule
% % % % % Degree & -0.136 & 3307 & -0.036 & 0.032 & -0.002 & 0.892 \\ 
% % % % %   Betweenness & -0.412 & 3307 & -0.041 & 0.027 & -0.007 & 0.680 \\ 
% % % % %   Eigenvector & -0.512 & 3307 & -0.043 & 0.025 & -0.009 & 0.608 \\ 
% % % % %   Closeness & -0.785 & 3307 & -0.048 & 0.020 & -0.014 & 0.433 \\ 
% % % % %   TransitivityNA & -0.880 & 3019 & -0.052 & 0.020 & -0.016 & 0.379 \\ 
% % % % %   Transitivity0 & -0.680 & 3307 & -0.046 & 0.022 & -0.012 & 0.497 \\ 
% % % % %   kcoreness & 0.380 & 3307 & -0.027 & 0.041 & 0.007 & 0.704 \\ 
% % % % %   \bottomrule
% % % % % \end{tabular}
% % % % % \caption{Correlation of centrality measures with negative log 10 transform p Pearson IntelligenceReplication} 
% % % % % \tiny\url{source('~/RProjects/chapter3/R/centrality_ch3/centrality_and_gwas_p_val/0_load_magma_pearson_centrality_ctg.R')}
% % % % % \label{tab:Correlation of centrality measures with negative log 10 transform p Pearson IntelligenceReplication}
% % % % % \end{table}

% % % % % % latex table generated in R 3.6.3 by xtable 1.8-4 package
% % % % % % Sun Mar 28 14:56:04 2021
% % % % % \begin{table}[ht]
% % % % % \centering
% % % % % \begin{tabular}{lrrrrrr}
% % % % %   \toprule
% % % % % Centrality & t & df & low & high & r & p \\ 
% % % % %   \midrule
% % % % % Degree & 1.051 & 3283 & -0.016 & 0.052 & 0.018 & 0.293 \\ 
% % % % %   Betweenness & 0.874 & 3283 & -0.019 & 0.049 & 0.015 & 0.382 \\ 
% % % % %   Eigenvector & -0.054 & 3283 & -0.035 & 0.033 & -0.001 & 0.957 \\ 
% % % % %   Closeness & -0.735 & 3283 & -0.047 & 0.021 & -0.013 & 0.462 \\ 
% % % % %   TransitivityNA & -2.009 & 2998 & -0.072 & -0.001 & -0.037 & 0.045 \\ 
% % % % %   Transitivity0 & -1.793 & 3283 & -0.065 & 0.003 & -0.031 & 0.073 \\ 
% % % % %   kcoreness & -0.372 & 3283 & -0.041 & 0.028 & -0.006 & 0.710 \\ 
% % % % %   \bottomrule
% % % % % \end{tabular}
% % % % % \caption{Correlation of centrality measures with negative log 10 transform p Pearson EducationReplication} 
% % % % % \tiny\url{source('~/RProjects/chapter3/R/centrality_ch3/centrality_and_gwas_p_val/0_load_magma_pearson_centrality_ea2.R')}
% % % % % \label{tab:Correlation of centrality measures with negative log 10 transform p Pearson EducationReplication}
% % % % % \end{table}



% % % % % % latex table generated in R 3.6.3 by xtable 1.8-4 package
% % % % % % Sun Mar 28 14:57:24 2021
% % % % % \begin{table}[ht]
% % % % % \centering
% % % % % \begin{tabular}{lrrrrrr}
% % % % %   \toprule
% % % % % Centrality & t & df & low & high & r & p \\ 
% % % % %   \midrule
% % % % % Degree & -0.836 & 3299 & -0.049 & 0.020 & -0.015 & 0.403 \\ 
% % % % %   Betweenness & -0.549 & 3299 & -0.044 & 0.025 & -0.010 & 0.583 \\ 
% % % % %   Eigenvector & -1.740 & 3299 & -0.064 & 0.004 & -0.030 & 0.082 \\ 
% % % % %   Closeness & -1.634 & 3299 & -0.062 & 0.006 & -0.028 & 0.102 \\ 
% % % % %   TransitivityNA & 0.741 & 3012 & -0.022 & 0.049 & 0.013 & 0.459 \\ 
% % % % %   Transitivity0 & 0.862 & 3299 & -0.019 & 0.049 & 0.015 & 0.389 \\ 
% % % % %   kcoreness & -1.000 & 3299 & -0.051 & 0.017 & -0.017 & 0.317 \\ 
% % % % %   \bottomrule
% % % % % \end{tabular}
% % % % % \caption{Correlation of centrality measures with negative log 10 transform p Pearson EducationDiscovery} 
% % % % % \tiny\url{source('~/RProjects/chapter3/R/centrality_ch3/centrality_and_gwas_p_val/0_load_magma_pearson_centrality_ukbbed.R')}
% % % % % \label{tab:Correlation of centrality measures with negative log 10 transform p Pearson EducationDiscovery}
% % % % % \end{table}



% % % % % % latex table generated in R 3.6.3 by xtable 1.8-4 package
% % % % % % Sun Mar 28 14:58:36 2021
% % % % % \begin{table}[ht]
% % % % % \centering
% % % % % \begin{tabular}{lrrrrrr}
% % % % %   \toprule
% % % % % Centrality & t & df & low & high & r & p \\ 
% % % % %   \midrule
% % % % % Degree & -0.396 & 3299 & -0.041 & 0.027 & -0.007 & 0.692 \\ 
% % % % %   Betweenness & 0.039 & 3299 & -0.033 & 0.035 & 0.001 & 0.969 \\ 
% % % % %   Eigenvector & -0.310 & 3299 & -0.039 & 0.029 & -0.005 & 0.757 \\ 
% % % % %   Closeness & -1.311 & 3299 & -0.057 & 0.011 & -0.023 & 0.190 \\ 
% % % % %   TransitivityNA & 0.587 & 3012 & -0.025 & 0.046 & 0.011 & 0.557 \\ 
% % % % %   Transitivity0 & 0.685 & 3299 & -0.022 & 0.046 & 0.012 & 0.493 \\ 
% % % % %   kcoreness & -1.480 & 3299 & -0.060 & 0.008 & -0.026 & 0.139 \\ 
% % % % %   \bottomrule
% % % % % \end{tabular}
% % % % % \caption{Correlation of centrality measures with negative log 10 transform p Pearson IntelligenceDiscovery} 
% % % % % \tiny\url{source('~/RProjects/chapter3/R/centrality_ch3/centrality_and_gwas_p_val/0_load_magma_pearson_centrality_ukbb_int.R')}
% % % % % \label{tab:Correlation of centrality measures with negative log 10 transform p Pearson IntelligenceDiscovery}
% % % % % \end{table}
% % % % % %\begin{figure}]h]
% % % % % %  \includegraphics[width=\linewidth]{Rplot_transitivity}
% % % % % %  \caption{Transitivity}
% % % % % %  \label{fig:transitivity}
% % % % % %\end{figure}
 
% % % % % %\begin{figure}]h]
% % % % % %  \includegraphics[width=\linewidth]{Rplot_transitivity_degree}
% % % % % %  \caption{Local clustering coefficient and degree}
% % % % % %  \label{fig:transitivity_degree}
% % % % % %\end{figure}


% % % % % %\begin{figure}
% % % % % %    \centering
% % % % % %    \includegraphics[width=\linewidth]{images/Rplot01_logdegree_log_transitivity.png}
% % % % % %    \caption{Plot of degree with mean local transitivity. Log10-log10 scale. Omitting degree $k=1$ as transitivity $C$ will be undefined.}
% % % % %  %   \label{fig:log_transitivity_degree}
% % % % % %\end{figure}
% % % % % \subsubsection{Compare diseases}

% % % % % ALS Mean degree 50.62
% % % % % PD Mean deg 30.26


% % % % % Curated diseases for Picks 2 genes BeFree (text mining) 210

% % % % % I guess the fact that the text mined data is similar in centrality to those of other neurdegenerative is in support


% % % % % ID 412 in PSP 1219 in middle in centrality 

% % % % % FTD 119 PSP 304 total  high centrality BeFree

% % % % % ASD higher degree similar eigenvector 300 PSP 1018 total

% % % % % Epilespy slightly higher PSP 380 All 1124 ? the low tail is enriched for epilepsy

% % % % % BPD Similar to PSP 863 total 240 in PSP

% % % % % SCZ be free 2217 PSP 623 Similar eigenvector higher betweenness

% % % % % Depression slightly up degree lower eig 1416 genes 340 in PSP

% % % % % Neuroticism 141 28 in PSP
% % % % % Degree similar eig lower core

% % % % % Cognitive dysfunction slightly higher deg and eig 1368 415 in PSP

% % % % % OCD almost identical 166 genes 45 in PSP

% % % % % Neurodegen double degree and betweenness similar eig 484 in PSP 1480 total

% % % % % Neurodevelopmental - double bet, similar eig, about 10 higher deg

% % % % % Psychotic - almost identical all 125 PSP 438 total

% % % % % Borderline - higher degree 3x betweenness similar eig and coreness 220 PSP 56

% % % % % Antisocial PSP 14 all 63
% % % % % 2x degree 3x between

% % % % % Huntingdon 945 271 in PSD 
% % % % % Deg almost double 3 x bet 2 x eig  increase kcore 3

% % % % % Encephalitis
% % % % % 4 x bet double deg not much change eig 68 PSP 285 curated 

% % % % % \url{source('~/RProjects/disgen2/R/disgen/specific_disorders/summaries_curated/encephalitis.R')}

% % % % % \todo{you could angle this as similar to the cognitive based disorders etc in Pocklington}



% \todo{We get ngenes and nspns so map snps to genes}