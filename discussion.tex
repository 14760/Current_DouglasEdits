\chapter{Discussion}

The two  aims of this thesis were to investigate the feasibility of combining network based analysis of the synaptic proteome with Gene Set Analysis and discover general  principles that are applicable to such analyses; the second was use this method to better understand a specific complex trait (intelligence) which seemed appropriate for such analysis (see section~\ref{sec:Introduction intelligence}).

In this chapter I will assess whether this has been achieved and summarise the findings of the thesis. In doing so I will discuss the strengths and limitations of this analysis and outline future work. I will conclude with some more speculative observations on this work informed by my experience in carrying out this analysis.

\section{Principal findings}
\subsection{Intelligence and Educational Ability}
My principal finding is that a community (community five) detected using the spectral clustering method, containing metabotropic glutamate receptors and ionotropic AMPA receptors, was found to be enriched for differences in intelligence and educational ability (section \ref{sec: spectral group 5 analysis}). The enrichment is present in the discovery and replication samples for both intelligence and educational attainment and the findings become stronger with more recent studies that overlap with our discovery and replication cohorts but provide a larger sample size. 

The part of the community responsible for the enrichment signal is not the receptors themselves which are prominent in animal models of learning but their neighbours, proteins that contribute to the macro-molecular complexes the receptors are embedded in. This finding supports the analyses and hypotheses of Grant \cite{grant2012synaptopathies} and the early work of Pocklington and co-workers \cite{pocklington2006proteomes} in synaptic proteome community detection. The results support those of Lundy et al.\cite{lundby2014annotation} who used a tissue specific network to examine the heart condition of long QT syndrome and also found that the enrichment signal was amongst the neighbours of the receptors. 

Analysis of the most recent studies (Savage et al. and EA3)\cite{savage2018genome},\cite{lee2018gene} (section~\ref{sec:results checking findings}) show enrichment both of glutamatergic receptors and their neighbours but a substantial amount of the enrichment remains in the neighbouring proteins. These are proteins would not be identified as part of a standard gene set analysis approach and support the utility of network based community gene set analysis as a method of analysing complex traits. 

The enriched proteins reveal few common biological themes using Gene Ontology enrichment analysis save for their proximity to receptor genes that are known to participate in glutamatergic neurotransmission.
This suggests that the\textit{ combination} of community based GSA and Gene Ontology analysis may be combined profitably to better interrogate ontology or topology defined areas of enrichment in complex traits. 

The second best performing algorithm (Louvain) also identified a group of proteins related to glutamatergic receptors that showed enrichment in the intelligence samples. 


\subsection{Vertex statistic and intelligence}

There was no association between a genes centrality statistics and its association with intelligence or educational attainment on MAGMA GWGAS (section~\ref{sec:correlation centrality measures and study p}). Lee et al\cite{lee2013network} had previously found that complex traits were more likely to occur in hubs and bottlenecks but this analysis was limited to genome wide significant SNPs and covered a wide range of disorders. This thesis shows no evidence of network centrality on intelligence or educational attainment using the most up to date data and a tissue specific network. This is consistent with intelligence being the product of many genes of small effect and is in contrast to the finding that certain diseases such as cancer and neuro-degenerative disorders are highly enriched in genes with high eigenvector centrality and degree centrality even compared to the rest of the PSP.

Cancer genes have previously been reported to be more central\cite{wachi2005interactome},\cite{ozgur2008identifying} which my findings support. The findings of the centrality of neurodegenerative disorders is limited to an over-representation of gene-disease associations in databases but to my knowledge the finding of an association between neurodegenerative disorders and synaptic proteome centrality is novel. Further studies would be required to confirm if these findings can be replicated in GWA studies of these conditions. Previous studies have suggested that disease genes are not central\cite{barabasi2011network}, this finding would suggest that at least some disorders may have disproportionately more essential genes and that the effects of network centrality must be tested for each condition with the best data one has available. 


There is a correlation between centrality measures and loss of function intolerance and an association between loss of function intolerance and intelligence and educational ability scores (although the association with educational attainment has been recently reported\cite{karczewski2020mutational}). It is possible that these factors may interact. 

Why was there no association between gene level GWA study p values for educational attainment or intelligence and measures of vertex importance such as degree? It may be that genetic changes in nodes of high degree result in a severe phenotype (as suggested by disease term enrichment section~\ref{sec:Disease term enrichment centrality measures}) and are not represented in population studies of complex traits. We would hypothesise that these core nodes are necessary for basic functions, and this is further supported by the finding that central nodes  are associated with an increased measure of evolutionary constraint (both by DEG and pLI). In this thesis I show that genes associated with intelligence and educational attainment are associated with increased levels of genetic constraint compared to the rest of the PSP and that high centrality nodes have high levels of selection pressure. The thesis extends the evidence for what might be termed centrality-severity to human populations using the DEG as a qualitative measure and Gnomad as a quantitative measure. The PSP is under a remarkable degree of genetic constraint compared to the rest of the genome and we provide evidence of this from large scale human genomic studies. 


Central genes are enriched for gene ontology terms associated with basic cellular functions such as the cell cycle (table~\ref{tab:overrepresentation all disorders by degree}) and it therefore seems reasonable that in turn they are associated with diseases that typically involve such processes (viral infection, cancer, etc). The neurological disorders affected by these are primarily neurodegenerative  and this enrichment is against the stringent criteria of correcting for multiple comparisons using all diseases, not just neurological ones. This suggests that these disorders may show common features in terms of network architecture. In turn the most peripheral nodes are those associated with the plasma membrane and extra cellular matrix and these in turn are associated with epilepsy. This suggests that there may exist a common network architecture for such disorders and that network architecture may be another means to subdivide neuropsychiatric traits. Caution should be applied however given my previous statements about gene disease databases. This suggests a different structure to the PSP network than that found in ecological or financial networks where an outer (expendable) shell protects an inner core\cite{burleson2020k}. The outer shell (kcore) are responsible for important day to day functions; it is just that the inner core is required for any coherent function. 

	
\section{GSA and Communities}
Our results suggest that genetic variation in the genes encoding a group of proteins forming a tightly interconnected sub-network in the post synaptic proteome have an enriched association with differences in educational attainment and intelligence. This group contains the majority of the genes encoding glutamate receptors and heterotrimeric G protein in the post synaptic proteome and so it appears to be a unit with a common function. These genes are commonly associated with animal models of learning and memory and would intuitively appear to have a role in human learning and intelligence. Glutamate mediated neurotransmission is required for initiation of long term potentiation, and blocking glutamate receptors in the hippocampus of mice causes failure of memory encoding.  More recently, differential expression of glutamate receptors has been shown to be associated with differences in problem solving in two species of wild finch\cite{audet2018divergence} which appears even more analagous to human intelligence.   


The glutamate receptors and heterotrimeric G proteins are not the principal source of the enrichment signal, it is instead their neighbouring and supporting proteins. The potential connection between the GWA study findings and these receptors is made possible by network analysis of the community structure of the synaptic proteome.


This thesis is to my knowledge the first to use GSA in association with community detection and make use of the complete data present in GWAS summary results. Mooney and Wilmot proposed using communities found in protein interaction networks as gene sets for GSA\cite{mooney2015gene} and gave as examples a modularity maximisation method and an annotation mining tool that incorporates module detection; however, the first does not use genetic data \cite{xu2010module} and the second implements an over-representation test, does not use the full amount of data from GWA\footnote{web address given in the paper is also broken}\cite{cowley2012pina}. Ghiassian et al.\cite{ghiassian2015disease} used the link community\cite{ahn2010link},Louvain\cite{blondel2008fast} and Markov Clustering\cite{dongen2000graph} algorithms to detect communities in the STRING protein database and concluded that communities were not associated with diseases. Their analysis did not however use a tissue specific network, or a competitive test of GSA, and the GWA data was limited to significant genes in GWA catalog\cite{ghiassian2015disease} appearing as part of an over-representation analysis and did not use the full polygenic signal available in GWA study summary statistics.  Investigating seventy diseases simultaneously also assumes a degree of homogeneity in the concept of disease that is lacking. 

In this thesis I have used a protein-protein interaction network specific to a cellular component believed implicated in the trait (the synapse and intelligence\cite{hill2014functional}). Goh et al. found that genes that were specific to tissues were more likely to be associated with diseases than others in the connectome \cite{goh2007human}, it seems a reasonable approach to take and similar approaches have been successful in other systems\cite{lundby2014annotation}. This technique could be extended to other disorders or complex traits that may be mediated by synaptic function. The synapse may however behave differently from other tissues in these forms of analyses, as we know it forms modular functional units from its proteins analogous to communities\cite{grant2012synaptopathies}, and our findings might not generalise to proteomes in non-synaptic tissue or metabolic pathways. At the risk of labouring the point I believe the investigation of synaptic network effects will have to be carried out for each disorder or trait with the most robust data available at the time and the current study other than showing a \textit{method} can contribute only to the discussion of intelligence and educational attainment. When a number of such studies have been completed it then it will be possible to decide whether certain diseases are on the whole peripheral or central or linked to topological communities.

This work extends that of Pocklington\cite{pocklington2006organization}, McLean\cite{mclean2016improved}, Bayes\cite{bayes2011characterization} and coworkers who have found differential enrichment of disease terms in network components in the synapse. Disease term enrichment based on disease gene association contains heterogenous data and the ability to combine the information in the summary data of GWA study with an analysis of network structure represents an improvement in these methods. Future extensions may include rare variants found in exome studies or network fusion studies incorporating information from other sources such as gene expression and eqtl. 
 
 Lamperter et al. \cite{lamparter2016fast} in their paper describing the PASCAL method pointed to desirability of methods that can generate gene sets from primary experimental data \cite{lamparter2016fast}. Community and network analysis is an example of such a technique.
 
 
The genes in the region of the affected groups could be subject to experiment, one could see if they in addition to the others had a role in animal learning or caused changes in synaptic electrophysiology.
 
\paragraph{Limitations}
\label{sec:discussion window}
There are limitations to the current study. The set of genes found in proteomic studies of the  postsynaptic areas continues to increase although the increase has reached a plateau\cite{heil2018systems} . If the network changes then the community allocations will change and a different network may lead to different community allocations. The network structure itself will have errors, in databases and in experiments and in collation. There are a number of principled approaches to network errors that could be adopted in future works\cite{newman2018networks}. 

The community allocations in addition are not a globally optimal allocation with regard to the modularity, and alternative partitions with similar modularity exist. The Spin glass model was an attractive alternative clustering method but has a number of parameters affecting the partition which I found made designing a proof of concept study which would properly account for multiple testing challenging.

The clustering also does not take consideration that nodes may belong to more than one component or cluster. Probabilistic methods exist to determine the most likely community membership\cite{yang2013overlapping} if vertices can belong to more than one community but the loss of non-overlapping gene sets may result in a great increase in the number of sets. We have also not taken into account a model where some genes are assigned to communities while others remain unassigned.

The question of the relationship between communities represent and ground truth a entity is interesting \cite{peel2017ground} and the fact of different community allocations with similar objective functions is irrefutable. I don't believe communities represent exact natural entities (they cannot overlap, they have only one of each component) but are a \textit{useful} mapping between network structure and genetic architecture. One approach to the existence of multiple plausible partitions that seems reasonable to me to choose the one that best fits nature. How would this be done rigorously? I have not used any data other than the distribution of edges to perform community detection but there are principled approaches\cite{newman2016structure}\cite{meng2018coupled} that use metadata and one could envisage jointly optimising a community measure and an objective function of biological plausibility or usefulness.   


A limitation of the study is that gene based analysis discard the approximately 50\% of the information from SNPs that do not lie between the boundaries of protein encoding genes. Future work may incorporate this information by for example using genetic variants known to affect genes such as eQTLs\cite{wang2020disease},\cite{gamazon2018using}.

`Gene windows' are one attempt to include regulatory SNPs but I have not used them as there is no consensus (although windows are small at present) and this is the more conservative choice. When I began the thesis some data was only available as the output from VEGAS which had fixed (50kB) windows so this is less of an issue now. Another area that concerned me was that some software such as MAGMA works differently on summary and raw genotype data. I am less concerned about this given how widely summary data has been used but think it would be useful if there are different methods if the authors could indicate the magnitude of the effect. In addition now that access to summary data is easier it would be of great help if the developers of software such as MAGMA and PASCAL were to include the expected results on a standard data set (rather than the `toy' example included with VEGAS 1 for example). 

The use of gene ontology terms as gene sets has been widespread as part of the secondary analysis of GWA studies however these ignore the fact that gene ontology has structure and that these sets often greatly overlap. This results in a significant multiple testing penalty which in the case of GWA studies of educational attainment and intelligence has been overcome by sample size but for other phenotypes it may be worth looking again at methods such as\cite{alexa2006improved} that use the tree structure of Gene Ontology terms. Specialised ontologies such as Neural /Immune\cite{geifman2010neural} are available but are non trivial to implement and to my knowledge there are no readily available packages that have this as a `drop in' solution.

Any enrichment software should provide allow the ability to provide ones own background set. The recent addition of this feature in ToppGene\cite{chen2009toppgene} led me to move all of my analysis to this, had it been available earlier would have saved me a considerable amount of time. 

I have said nothing about gene association networks. A small world model of genetic influence networks has been argued to support an omni-genic model of traits\cite{boyle2017expanded}. It is worth pointing out that the structure of a network may affect the genetic architecture of complex traits, even in a undirected network with unweighted edges and a small world structure, the rate of flow of information between genes (particularly over short timescales) depends on network structure.



Further directions in the analysis of the synaptic proteome would include looking at how its structure changes over time. Intelligence and educational attainment are relatively stable over the life course. One of the most striking things about neuropsychiatric disorders  that involve similar genes is that they occur at different times in the life course(compare for example schizophrenia and Autism Spectrum Disorder). There may be network features that emerge at critical periods (due to changes in gene expression for example).   We also know that the proteome varies across the brain\cite{grant2019synapse}








The study provides an outline for a network analysis of GWA studies of neuropsychiatric disorders with synaptic involvement. An analysis of vertex properties with gene level results, an analysis of the site of enrichment core or periphery, community enrichment analysis and topological subgroup analysis. 

\section{Study design}
\label{sec:study design discussion}
I have included a number of different methods of assessing enrichment (GSEA, PASCAL and MAGMA) for two reasons. One was so that the findings would not be dependent on a particular method. The other was to limit the number of `interesting' sets taken forward for replication and to ensure these were more likely to replicate. One option would have been to reduce the level of alpha but this seemed rather arbitrary. 

With increasing sample sizes in the most recent studies (see chapter 5) the level of significance increases relative to the penalty for multiple comparisons but not all disorders will have studies that include 1.1 million participants. For example a meticulous study of the genetic epidemiology of drug induced psychosis and its progression to schizophrenia between 1997 and 2015 using Swedish national registries (which are of very high quality) achieved a sample size of 7,606\cite{kendler2019prediction}. Careful study design will therefore be necessary to limit the number of potential comparisons that arise testing network communities. 




 
 
 \subsection{Benchmarks}
 
Benchmarks of clustering algorithm performance were useful but there remain outstanding issues. Yang et al \cite{yang2016comparative} provide a useful review of the performance of the igraph algorithms of the LFR benchmark but assumes the mixing parameter is known which it is not when face by a network. The authors also report NMI using the arithmetic mean and noting the similarity of the normalisation methods. However as the performance of algorithms decline many generate a large number of groups and NMI uncorrected for chance will overestimate performance (here we have used the implementation in scikit learn).

The LFR benchmark remains the best available but it is difficult to get it to fit all network structures. Ultimately I think that a probabilistic benchmark that takes into account features such as overlapping communities, assortativity and core periphery structure would be desirable but would be challenging to implement. In using the LFR benchmark to assess algorithms for a study I think it would be good practice to use parameters as similar to the network being analysed as possible. 

Nevertheless there were important differences between the performance of algorithms on the LFR benchmark and on the PSP graph. The infomap algorithm in particular was the best on the benchmark but gave rise to unrealistic community structures. Testing algorithms on ones network to ensure that they give reasonable results seems a sensible precaution\footnote{note to Douglas: the more opinionated way of saying this would be that almost all of these things have not been used as part of a specific practical analysis but in order to make general observations about the behaviour of algorithms (with the exception of Leskovec\cite{leskovec2010empirical} and others)}. 



 \paragraph{Work not included}
There are a number of promising areas of network science research that time and space have not allowed me to include\footnote{Douglas:should I say I have done work on these?}. I think core periphery structure is second only to community structure as a meso-scale property and will be very important in developing a probabilistic model of the synapse. This is closely followed by overlapping community detection.  Dynamic models are also not included but this analysis does provide a guide to areas that would benefit from the attention of dynamic models. Other areas include network error, motifs, data and graph fusion and percolation. 
 
 \textcolor{red}{add to discussion if you had more time you could work on a benchmark that generated nodes with degree from 1 to similar to spectral and benchmark infomap on it}


\subsection{speculative}
Taking the opportunity the discussion allows to include some more speculative material I have often thought in the course of the thesis about the implications of synaptic complexity. In particular the number of apparently trivial occasions (such as the difference between the two types of global transitivity) where the network structure departs from randomness where it has order where none is expected. In many human systems \cite{henriques2020nonlinear}\cite{robertson2015time}\cite{mullan2018habit} an early sign of the deterioration of a system is a loss of complexity specifically an increase in randomness. A possible testing paradigm for biological entities that do not result in loss of function (such as the proteins in group 5) would be a loss of complexity in the behaviour or electrophysiology of the organism rather than loss of function. Mutations in proteins for example not under high evolutionary pressure \textit{may} affect the complexity of features such as the spatial mapping of the proteome\cite{grant2019synapse}. 

 In addition to the synaptomic theory of Grant there is certainly a need for an explanation of all the complexity in the synapse. Until recently the synapse was believed to work using a very limited repetoire of proteins.  Very simple models of circuits can now do amazing things (for example deep neural networks) given increase in computational power. In recent years there has been an increased interest in different activation functions or specific architectures (memory cells) by analogy (since the simple model of the synapse was endlessly used for such analogies) the synapse may be selecting over a combinatorially massive range of activation functions. There is certainly an enormous amount of complexity to explain and the goal of simulating areas of the synapse will be consequently difficult. 
 

By contrast a review of systems biology \cite{parikshak2015systems} gives the example a tissue specific cardiac PPI used to investigate long QT syndrome (LQTS) by Lundby et al.\cite{lundby2014annotation} as an example of the value of tissue specificity. (neighbours)


Add conductance and correlation and NMI adj useful and test on PSP and benchmark
(figure~\ref{fig:conductance cdmlouvain} plotting group 7 in red)

